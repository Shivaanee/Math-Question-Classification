and onto, there is a function \(g\) from \(Y\) into \(X\) which satisfies (1) and (2). Furthermore, this \(g\) is unique. It is the function from \(Y\) into \(X\) defined by this rule: if \(y\) is in \(Y\), then \(g(y)\) is the one and only element \(x\) in \(X\) for which \(f(x)=y\). If \(f\) is invertible (1:1 and onto), the **inverse** of \(f\) is the unique function \(f^{\lhd 1}\) from \(Y\) into \(X\) satisfying \((1^{\prime})\)\(f^{\lhd 1}(f(x))=x\), for each \(x\) in \(X\), \((2^{\prime})\)\(f(f^{\lhd 1}(y))=y\), for each \(y\) in \(Y\).

Let us look at the functions in Example 2.

1. If \(X=Y\), the set of real numbers, and \(f(x)=x^{2}\), then \(f\) is not invertible. For \(f\) is neither 1:1 nor onto.
2. If \(X=Y\), the Euclidean plane, and \(f\) is 'rotation through \(90^{\circ}\)' then \(f\) is both 1:1 and onto. The inverse function \(f^{\lhd 1}\) is 'rotation through \(-90^{\circ}\)' or 'rotation through \(270^{\circ}\)'.
3. If \(X\) is the plane, \(Y\) the \(x_{1}\)-axis, and \(f((x_{1},x_{2}))=(x_{1},0)\), then \(f\) is not invertible. For, although \(f\) is onto, \(f\) is not 1:1.
4. If \(X\) is the set of real numbers, \(Y\) the set of positive real numbers, and\(f(x)=e^{x}\), then \(f\) is invertible. The function \(f^{\lhd 1}\) is the natural logarithm function of part (e): \(\log e^{x}=x\), 