\[f=\,(x-c_{1})^{d_{1}}\,\cdots\,(x-c_{k})^{d_{k}}.\]

If the scalar field \(F\) is algebraically closed, e.g., the field of complex numbers, every polynomial over \(F\) can be so factored (see Section 4.5); however, if \(F\) is not algebraically closed, we are citing a special property of \(T\) when we say that its characteristic polynomial has such a factorization. The second thing we see from (6-3) is that \(d_{i}\), the number of times which \(c_{i}\) is repeated as root of \(f\), is equal to the dimension of the space of characteristic vectors associated with the characteristic value \(c_{i}\). That is because the nullity of a diagonal matrix is equal to the number of zeros which it has on its main diagonal, and the matrix \([T-c_{i}I]_{0}\) has \(d_{i}\) zeros on its main diagonal. This relation between the dimension of the characteristic space and the multiplicity of the characteristic value as a root of \(f\) does not seem exciting at first; however, it will provide us with a simpler way of determining whether a given operator is diagonalizable.

_Lemma._ Suppose that \(\mathrm{T}\alpha=c\alpha\). If \(\mathrm{f}\) is any polynomial, then \(\mathrm{f}(\mathrm{T})\alpha=f(\mathrm{c})\alpha\).

Proof.: Exercise.

_Lemma._ Let \(\mathrm{T}\) be a linear operator on the finite-dimensional space \(\mathrm{V}\). Let \(c_{1}\), \(\ldots\), \(c_{k}\) be the distinct characteristic values of \(\mathrm{T}\) and let \(W_{i}\) be the space of characteristic vectors associated with the characteristic value \(c_{i}\). If \(W=W_{1}+\,\cdots\,+\,W_{k}\), then

\[dim\,W=dim\,W_{1}+\,\cdots\,+\,dim\,W_{k}.\]

In fact, if \((\delta_{1}\) is an ordered basis for \ 