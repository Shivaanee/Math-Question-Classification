relative to \(W\), and so we must verify that the sum and product above depend only upon the cosets involved. What this means is that we must show the following:

1. If \(\alpha\equiv\alpha^{\prime}\), mod \(W\), and \(\beta\equiv\beta^{\prime}\), mod \(W\), then \[\alpha+\beta\rightarrow\alpha^{\prime}+\beta^{\prime},\mod W.\]
2. If \(\alpha\equiv\alpha^{\prime}\), mod \(W\), then \(c\alpha\equiv c\alpha^{\prime}\), mod \(W\).

These facts are easy to verify. (1) If \(\alpha-\alpha^{\prime}\) is in \(W\) and \(\beta-\beta^{\prime}\) is in \(W\), then since \((\alpha+\beta)-(\alpha^{\prime}-\beta^{\prime})=(\alpha-\alpha^{\prime})+( \beta-\beta^{\prime})\), we see that \(\alpha+\beta\) is congruent to \(\alpha^{\prime}-\beta^{\prime}\) modulo \(W\). (2) If \(\alpha-\alpha^{\prime}\) is in \(W\) and \(c\) is any scalar, then \(c\alpha-c\alpha^{\prime}=c(\alpha-\alpha^{\prime})\) is in \(W\).

It is now easy to verify that \(V/W\), with the vector addition and scalar multiplication defined above, is a vector space over the field \(F\). One must directly check each of the axioms for a vector space. Each of the properties of vector addition and scalar multiplication follows from the corresponding property of the operations in \(V\). One comment should be made. The zero vector in \(V/W\) will be the coset of the zero vector in \(V\). In other words, \(W\) is the zero vector in \(V/W\).

The vector space 