

## Chapter 7 The _Rational and Jordan Forms_

### 7.1 Cyclic \(Subspaces\) and \(Annihilators\)

Once again \(V\) is a finite-dimensional vector space over the field \(F\) and \(T\) is a fixed (but arbitrary) linear operator on \(V.\) If \(\alpha\) is any vector in \(V\), there is a smallest subspace of \(V\) which is invariant under \(T\) and contains \(\alpha.\) This subspace can be defined as the intersection of all \(T\)-invariant subspaces which contain \(\alpha;\) however, it is more profitable at the moment for us to look at things this way. If \(W\) is any subspace of \(V\) which is invariant under \(T\) and contains \(\alpha,\) then \(W\) must also contain the vector \(T\alpha;\) hence \(W\) must contain \(T(T\alpha)=T^{\alpha}\alpha,\)\(T(T^{\alpha}\alpha)=T^{\alpha}\alpha,\) etc. In other words \(W\) must contain \(g(T)\alpha\) for every polynomial \(g\) over \(F.\) The set of all vectors of the form \(g(T)\alpha,\) with \(g\) in \(F[x],\) is clearly invariant under \(T,\) and is thus the smallest \(T\)-invariant subspace which contains \(\alpha.\)

_Definition. If \(\alpha\) is any vector in \(\mathrm{V},\) the \(\mathrm{T}\)-cyclic subspace generated_

**by \(\alpha\)** _is the subspace \(\mathrm{Z}(\alpha;\mathrm{T})\) of all vectors of the form \(\mathrm{g(T)}\alpha,\)\(g\) in \(\mathrm{F[x]}.\) If \(\mathrm{Z}(\alpha;\mathrm{T})=\mathrm{V},\) then \(\alpha\) is called a_ **cyclic vector** _for \(\mathrm{T}.\)_

Another way of describing the subspace \(Z(\alpha;T)\) is that \(Z(\alpha;T)\) is the subspace spanned by the vectors \(T^{k}\alpha,\)\(k\geq 0,\) and thus \(\alpha\) is a cyclic vector for \(T\) if and only if these vectors span \(V.\) We caution the reader that the general operator \(T\) has no cyclic vectors.

**Example 1**.: For any \(T,\) the \(T\)-cyclic subspace generated by the zero vector is the zero subspace. The space \(Z(\alpha;T)\) is one-dimensional if and only if \(\alpha\) is a characteristic vector for \(T.\) For the identity operator, every