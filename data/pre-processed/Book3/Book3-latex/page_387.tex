

#### Bilinear Forms

The reader should find it easy to verify that the matrix of \(f\) in the latter ordered basis has the block form

\[\left[\begin{array}{cc}0&J\\ -J&0\end{array}\right]\]

where \(J\) is the \(k\bigtimes k\) matrix

\[\left[\begin{array}{cccc}0&\cdots&0&1\\ 0&\cdots&1&0\\ \vdots&\cdots&\vdots&\vdots\\ 1&\cdots&0&0\end{array}\right].\]

#### Exercises

1. Let \(V\) be a vector space over a field \(F\). Show that the set of all skew-symmetric bilinear forms on \(V\) is a subspace of \(L(V,\,V,\,F)\).
2. Find all skew-symmetric bilinear forms on \(R^{3}\).
3. Find a basis for the space of all skew-symmetric bilinear forms on \(R^{n}\).
4. Let \(f\) be a symmetric bilinear form on \(C^{n}\) and \(g\) a skew-symmetric bilinear form on \(C^{n}\). Suppose \(f+g=0\). Show that \(f=g=0\).
5. Let \(V\) be an \(n\)-dimensional vector space over a subfield \(F\) of \(C\). Prove the following. 1. The equation \((P\!f)(\alpha,\beta)=\frac{1}{2}f(\alpha,\beta)-\frac{1}{2}f(\beta,\alpha)\) defines a linear operator \(P\) on \(L(V,\,V,\,F)\). 2. \(P^{\intercal}=P_{i}\), i.e., \(P\) is a projection. 3. rank \(P=\frac{n(n-1)}{2}\); nullity \(P=\frac{n(n+1)}{2}\). 4. If \(U\) is a linear operator on \(V\), the equation \((U\!f)(\alpha,\beta)=f(U\alpha,\,U\beta)\) defines a linear operator \(U\!t\) on \(L(V,\,V,\,F)\). 5. For every linear operator \(U\), the projection \(P\) commutes with \(U^{\intercal}\).
6. Prove an analogue of Exercise 11 in Section 10.2 for non-degenerate, skew-symmetric bilinear forms.
7. Let \(f\) be a bilinear form on a vector space \(V\). Let \(L_{f}\) and \(R_{f}\) be the mappings of \(V\) into \(V^{*}\) associated with \(f\) in Section 10.1. Prove that \(f\) is skew-symmetric if and only if \(L_{f}=-R_{f}\).
8. Prove an analogue of Exercise 17 in Section 10.2 for skew-symmetric forms.
9. Let \(V\) be a finite-dimensional vector space and \(L_{1}\), \(L_{2}\) linear functionals on \(V\). Show that the equation \[f(\alpha,\,\beta)=L_{1}(\alpha)L_{2}(\beta)=L_{1}(\beta)L_{2}(\alpha)\] defines a skew-symmetric bilinear form on \(V\). Show that \(f=0\) if and only if \(L_{1}\), \(L_{2}\) are linearly dependent.
10. Let \(V\) be a finite-dimensional vector space over a subfield of the complex numbers and \(f\) a skew-symmetric bilinear form on \(V\). Show that \(f\) has rank \(2\) if