\(Z(\alpha;T)\), this shows that these \(k\) vectors span \(Z(\alpha;T)\). These vectors are certainly linearly independent, because any non-trivial linear relation between them would give us a non-zero polynomial \(g\) such that \(g(T)\alpha=0\) and deg \((g)<\) deg \((p_{a})\), which is absurd. This proves (i) and (ii).

Let \(U\) be the linear operator on \(Z(\alpha;T)\) obtained by restricting \(T\) to that subspace. If \(g\) is any polynomial over \(F\), then

\[\eqalign{p_{a}(U)g(T)\alpha&=\ p_{a}(T)g(T)\alpha\cr&=\ g(T)p_{a}(T)\alpha\cr&= \ g(T)0\cr&=\ 0.\cr}\]

Thus the operator \(p_{a}(U)\) sends every vector in \(Z(\alpha;T)\) into \(0\) and is the zero operator on \(Z(\alpha;T)\). Furthermore, if \ 