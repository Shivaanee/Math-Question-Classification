Other extensions of the associative property of vector addition and the distributive properties 4(c) and 4(d) of scalar multiplication apply to linear combinations:

\[\sum\limits_{i=1}^{n}c_{i}\alpha_{i}+\sum\limits_{i=1}^{n}d_{i} \alpha_{i} =\sum\limits_{i=1}^{n}(c_{i}+d_{i})\alpha_{i}\] \[c\sum\limits_{i=1}^{n}c_{i}\alpha_{i} =\sum\limits_{i=1}^{n}(cc_{i})\alpha_{i}\]

Certain parts of linear algebra are intimately related to geometry. The very word 'space' suggests something geometrical, as does the word 'vector' to most people. As we proceed with our study of vector spaces, the reader will observe that much of the terminology has a geometrical connotation. Before concluding this introductory section on vector spaces, we shall consider the relation of vector spaces to geometry to an extent which will at least indicate the origin of the name 'vector space.' This will be a brief intuitive discussion.

Let us consider the vector space \(R\)'. In analytic geometry, one identifies triples \((x_{1},x_{2},x_{3})\) of real numbers with the points in three-dimensional Euclidean space. In that context, a vector is usually defined as a directed line segment \(PQ\), from a point \(P\) in the space to another point \(Q\). This amounts to a careful formulation of the idea of the 'arrow' from \(P\) to \(Q\). As vectors are used, it is intended that they should be determined by their length and direction. Thus one must identify two directed line segments if they have the same length and the same direction.

The directed line segment \(PQ\), from the point \(P=(x_{1},x_{2},x_{3})\) to the point \(Q=(y_{1},y_{2},y_{3})\), has the same length and direction as the directed line segment from the origin \(O=(0,0,0)\) to the point \((y_{1}-x_{1},y_{2}-x_{3},\)\(y_{3}-x_{3})\). Furthermore, this is the only segment emanating from the origin which has the same length and direction as \(PQ\). Thus, if one agrees to treat only vectors which emanate from the origin, there is exactly one vector associated with each given length and direction.

The vector \(OP\), from the origin to \(P=(x_{1},x_{2},x_{3})\), is completely determined by \(P\), and it is therefore possible to identify this vector with the point \(P\). In our definition of the vector space \(R\)', the vectors are simply defined to be the triples \((x_{1},x_{2},x_{3})\).

Given points \(P=(x_{1},x_{2},x_{3})\) and \(Q=(y_{1},y_{2},y_{3})\), the definition of the sum of the vectors \(OP\) and \(OQ\) can be given geometrically. If the vectors are not parallel, then the segments \(OP\) and \(OQ\) determine a plane and these segments are two of the edges of a parallelogram in that plane (see Figure 1). One diagonal of this parallelogram extends from \(O\) to a point \(S\), and the sum of \(OP\) and \(OQ\) is defined to be the vector \(OS\). The coordinates of the point \(S\) are \((x_{1}+y_{1},x_{2}+y_{2},x_{3}+y_{3})\) and hence this geometrical definition of vector addition is equivalent to the algebraic definition of Example 1.

 