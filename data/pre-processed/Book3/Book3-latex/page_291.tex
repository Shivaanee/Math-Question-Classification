by \(\alpha_{1},\ldots,\alpha_{m}\) and orthogonal to each of these vectors; hence it is \(0\) by (8-8). Conversely, if \(\alpha_{1},\ldots,\alpha_{m}\) are different from \(0\) and \(\alpha_{m+1}=0\), then \(\beta_{1},\ldots,\beta_{m+1}\) are linearly dependent.

**Example 12**: Consider the vectors

\[\beta_{1} = (3,0,4)\] \[\beta_{2} = (-1,0,7)\] \[\beta_{3} = (2,9,11)\]

in \(R^{3}\) equipped with the standard inner product. Applying the Gram-Schmidt process to \(\beta_{1}\), \(\beta_{2}\), \(\beta_{3}\), we obtain the following vectors.

\[\alpha_{1} = (3,0,4)\] \[\alpha_{2} = (-1,0,7)\,-\frac{((-1,0,7)|(3,0,4))}{25}\,(3,0,4)\] \[= (-1,0,7)\,-\,(3,0,4)\] \[= (-4,0,3)\] \[\alpha_{3} = (2,9,11)\,-\,\frac{((2,9,11)|(3,0,4) 