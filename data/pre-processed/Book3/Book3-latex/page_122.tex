of \(T^{\prime}\) is in the annihilator of \(N\); for, suppose\(f=T^{\prime}g\) for some \(g\) in \(W^{*}\); then, if \(\alpha\) is in \(N\)

\[f(\alpha)=(T^{\prime}g)(\alpha)=g(T\alpha)=g(0)=0.\]

Now the range of \(T^{\prime}\) is a subspace of the space \(N^{0}\), and

\[\dim N^{0}=n-\dim N=\operatorname{rank}(T)=\operatorname{rank}(T^{\prime})\]

so that the range of \(T^{\prime}\) must be exactly \(N^{0}\).

**Theorem 23**: _Let \(\mathrm{V}\) and \(\mathrm{W}\) be finite-dimensional vector spaces over the field \(\mathrm{F}\). Let \(\otimes\) be an ordered basis for \(\mathrm{V}\) with dual basis \(\otimes^{*}\), and let \(\otimes^{\prime}\) be an ordered basis for \(\mathrm{W}\) with dual basis \(\otimes^{\prime}{}^{*}\). Let \(\mathrm{T}\) be a linear transformation from \(\mathrm{V}\) into \(\mathrm{W}\); let \(\mathrm{A}\) be the matrix of \(\mathrm{T}\) relative to \(\otimes\), \(\otimes^{\prime}\) and let \(\mathrm{B}\) be the matrix of \(\mathrm{T}^{\prime}\) relative to \(\otimes^{\prime}{}^{*}\), \(\otimes^{*}\). Then \(\mathrm{B}_{\mathrm{i}\mathrm{i}}=\mathrm{A}_{\mathrm{i}\mathrm{i}}\)._

Let

\[\begin{array}{rcl}\otimes&=&\{\alpha_{1},\,.\,.\,.\,,\,\alpha_{n}\},&\quad \otimes^{\prime}&=&\{\beta_{1},\,.\,.\,.\,,\,\beta_{m}\},\\ \otimes^{*}&=&\{f_{1},\,.\,.\,.\,,\,f_{n}\},&\quad\otimes^{\prime}{}^{*}&=&\{g _{1},\,.\,.\,.\,,\,g_{m}\}.\end{array}

 