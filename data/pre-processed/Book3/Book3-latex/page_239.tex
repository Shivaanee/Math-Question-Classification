The matrix (7-2) is called the **companion matrix** of the monic polynomial \(p_{\alpha}\).

**Theorem 2.**_If \(U\) is a linear operator on the finite-dimensional space \(W\), then \(U\) has a cyclic vector if and only if there is some ordered basis for \(W\) in which \(U\) is represented by the companion matrix of the minimal polynomial for \(U\)._

Proof.: We have just observed that if \(U\) has a cyclic vector, then there is such an ordered basis for \(W\). Conversely, if we have some ordered basis \(\{\alpha_{1},\ldots,\alpha_{k}\}\) for \(W\) in which \(U\) is represented by the companion matrix of its minimal polynomial, it is obvious that \(\alpha_{1}\) is a cyclic vector for \(U\).

**Corollary**.: _If \(A\) is the companion matrix of a monic polynomial \(p\), then \(p\) is both the minimal and the characteristic polynomial of \(A\)._

Proof.: One way to see this is to let \(U\) be the linear operator on \(F^{k}\) which is represented by \(A\) in the standard ordered basis, and to apply Theorem 1 together with the Cayley-Hamilton theorem. Another method is to use Theorem 1 to see that \(p\) is the minimal polynomial for \(A\) and to verify by a direct calculation that \(p\) is the characteristic polynomial for \(A\).

One last comment--if \(T\) is any linear operator on the space \(V\) and \(\alpha\) is any vector in \(V\), then the operator \(U\) which \(T\) induces on the cyclic subspace \(Z(\alpha;T)\) has a cyclic vector, namely, \(\alpha\). Thus \(Z(\alpha;T)\) has an ordered basis in which \(U\) is represented by the companion matrix of \(p_{\alpha}\) the \(T\)-annihilator of \(\alpha\).

### Exercises

**1.** Let \(T\) be a linear operator on \(F^{2}\). Prove that any non-zero vector which is not a characteristic vector for \(T\) is a cyclic vector for \(T\). Hence, prove that either \(T\) has a cyclic vector or \(T\) is a scalar multiple of the identity operator.

**2.** Let \(T\) be the linear operator on \(R^{3}\) which is represented in the standard ordered basis by the matrix

\[\begin{bmatrix}2&0&0\\ 0&2&0\\ 0&0&-1\end{bmatrix}\]

Prove that \(T\) has no cyclic vector. What is the \(T\)-cyclic subspace generated by the vector (1, \(-1\), 3)?

**3.** Let \(T\) be the linear operator on \(C^{3}\) which is represented in the standard ordered basis by the matrix

\[\begin{bmatrix}1&i&0\\ -1&2&-i\\ 0&1&1\end{bmatrix}.\] 