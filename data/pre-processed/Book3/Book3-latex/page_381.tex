There are several comments we should make about the basis \(\langle\beta_{1},\ldots,\beta_{n}\rangle\) of Theorem 5 and the associated subspaces \(V^{+}\), \(V^{-}\), and \(V^{\perp}\) First, note that \(V^{\perp}\) is exactly the subspace of vectors which are 'orthogonal' to all of \(V\). We noted above that \(V^{\perp}\) is contained in this subspace; but,

\[\dim\,V^{\perp}=\dim\,V-(\dim\,V^{+}+\dim\,V^{-})=\dim\,V-\,{\rm rank}\,f\]

so every vector \(\alpha\) such that \(f(\alpha,\beta)=0\) for all \(\beta\) must be in \(V^{\perp}\). Thus, the subspace \(V^{\perp}\) is unique. The subspaces \(V^{+}\) and \(V^{-}\) are not unique; however, their dimensions are unique. The proof of Theorem 5 shows us that \(\dim\,V^{+}\) is the largest possible dimension of any subspace on which \(f\) 