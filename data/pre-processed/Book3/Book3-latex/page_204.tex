an illuminating and far from trivial application of the general theory of determinants developed in Chapter 5.

Let \(K\) be the commutative ring with identity consisting of all polynomials in \(T\). Of course, \(K\) is actually a commutative algebra with identity over the scalar field. Choose an ordered basis \(\{\alpha_{1},\,\ldots,\,\alpha_{n}\}\) for \(V\), and let \(A\) be the matrix which represents \(T\) in the given basis. Then

\[T\alpha_{i}=\sum\limits_{j=1}^{n}A_{ji}\alpha_{j},\qquad 1\leq i\leq n.\]

These equations may be written in the equivalent form

\[\sum\limits_{j=1}^{n}(\delta_{ij}T-A_{ji}I)\alpha_{j}=0,\qquad 1\leq i\leq n.\]

Let \(B\) denote the element of \(K^{n\times n}\) with entries

\ 