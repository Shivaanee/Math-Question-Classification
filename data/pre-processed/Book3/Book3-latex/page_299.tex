12. Let \(W\) be a finite-dimensional subspace of an inner product space \(V\), and let \(E\) be the orthogonal projection of \(V\) on \(W\). Prove that \((E\alpha|\beta)=(\alpha|E\beta)\) for all \(\alpha\), \(\beta\) in \(V\).
13. Let \(S\) be a subset of an inner product space \(V\). Show that \((S^{\perp})^{\perp}\) contains the subspace spanned by \(S\). When \(V\) is finite-dimensional, show that \((S^{\perp})^{\perp}\) is the subspace spanned by \(S\).
14. Let \(V\) be a finite-dimensional inner product space, and let \(\mathfrak{G}=\{\alpha_{1},\,\ldots,\,\alpha_{n}\}\) be an orthonormal basis for \(V\). Let \(T\) be a linear operator on \(V\) and \(A\) the matrix of \(T\) in the ordered basis \(\mathfrak{G}\). Prove that \[A_{ii}=(T\alpha_{i}|\alpha_{i}).\]
15. Suppose \(V=W_{1}\oplus W_{2}\) and that \(f_{1}\) and \(f_{2}\) are inner products on \(W_{1}\) and \(W_{3}\), respectively. Show that there is a unique inner product \(f\) on \(V\) such that 1. \(W_{2}=W_{1}^{\perp}\); 2. \(f(\alpha,\beta)=f_{k}(\alpha,\beta)\), when \(\alpha\), \(\beta\) are in \(W_{3}\), \(k=1\), \(2\).
16. Let \(V\) be an inner product space and \(W\) a finite-dimensional subspace of \(V\). There are (in general) many projections which have \(W\) as their range. One of these, the orthogonal projection on \(W\), has the property that \(||E\alpha||\leq||\alpha||\) for every \(\alpha\) in \(V\). Prove that if \(E\) is a projection with range \(W\), such that \(||E\alpha||\leq||\alpha||\) for all \(\alpha\) in \(V\), then \(E\) is the orthogonal projection on \(W\).
17. Let \(V\) be the real inner product space consisting of the space of real-valued continuous functions on the interval, \(-1\leq t\leq 1\), with the inner product \[(f|g)=\int_{-1}^{1}f(t)g(t)\,\,\mbox{\boldmath$\mathit{dl}$}.\] Let \(W\) be the subspace of odd functions, i.e., functions satisfying \(f(-t)=-f(t)\). Find the orthogonal complement of \(W\).

### Linear Functionals and Adjoints

The first portion of this section treats linear functionals on an inner product space and their relation to the inner product. The basic result is that any linear functional \(f\) on a finite-dimensional inner product space is 'inner product with a fixed vector in the space,' i.e., that such an \(f\) has the form \(f(\alpha)=(\alpha|\beta)\) for some fixed \(\beta\) in \(V\). We use this result to prove the existence of the 'adjoint' of a linear operator \(T\) on \(V\), this being a linear operator \(T^{*}\) such that \((T\alpha|\beta)=(\alpha|T^{*}\mbox{\boldmath$\mathit{\char 37}$})\) for all \(\alpha\) and \(\mbox{\boldmath$\mathit{\char 37}$}\) in \(V\). Through the use of an orthonormal basis, this adjoint operation on linear operators (passing from \(T\) to \(T^{*}\)) is identified with the operation of forming the conjugate transpose of a matrix. We explore slightly the analogy between the adjoint operation and conjugation on complex numbers.

Let \(V\) be any inner product space, and let \(\beta\) be some fixed vector in \(V\). We define a function \(f_{\beta}\) from \(V\) into the scalar field by 