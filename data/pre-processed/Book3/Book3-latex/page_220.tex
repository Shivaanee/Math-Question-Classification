\[\begin{array}{l}A\,=\,A_{1}\,+\,A_{2}\\ A_{1}\,=\,\frac{1}{2}(A\,+\,A\,^{\rm t})\\ A_{2}\,=\,\frac{1}{2}(A\,-\,A\,^{\rm t}).\end{array}\]

**Example 13**: Let \(T\) be any linear operator on a finite-dimensional space \(V\). Let \(c_{\rm i}\), . . , \(c_{\rm k}\) be the distinct characteristic values of \(T\), and let \(W_{i}\) be the space of characteristic vectors associated with the characteristic value \(c_{i}\). Then \(W_{1}\), . . , \(W_{k}\) are independent. See the lemma before Theorem 2. In particular, if \(T\) is diagonalizable, then \(V\,=\,W_{1}\,\oplus\,\cdots\,\oplus\,W_{k}\).

**Definition**: If \(V\) is a vector space, a **projection** of \(V\) is a linear operator \(E\) on \(V\) such that \(E^{2}\,=\,E\).

Suppose that \(E\) is a projection. Let \(R\) be the range of \(E\) and let \(N\) be the null space of \(E\).

1. The vector \(\beta\) is in the range \(R\) if and only if \(E\beta\,=\,\beta\). If \(\beta\,=\,E\alpha\), then \(E\beta\,=\,E^{2}\alpha\,=\,E\alpha\,=\,\beta\). Conversely, if \(\beta\,=\,E\beta\), then (of course) \(\beta\) is in the range of \(E\).

2. \(V\,=\,R\,\oplus\,N\).

3. The unique expression for \(\alpha\) as a sum of vectors in \(R\) and \(N\) is \(\alpha\,=\,E\alpha\,+\,(\alpha\,-\,E\alpha)\).

From (1), (2), (3) it is easy to see the following. If \(R\) and \(N\) are subspaces of \(V\) such that \(V\,=\,R\,\oplus\,N\), there is one and only one projection operator 