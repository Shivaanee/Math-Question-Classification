\[B=\left[\begin{array}{ccc}-1&0&0\\ -1&0&0\\ 0&0&0\end{array}\right].\]

In this case \(B\neq B^{*}\), and \(B^{*}\) is not the matrix of \(E^{*}=E\) in the basis \(\left\{\alpha_{1},\,\alpha_{2},\,\alpha_{3}\right\}.\) Applying the corollary, we conclude that \(\left\{\alpha_{1},\,\alpha_{2},\,\alpha_{3}\right\}\) is not an orthonormal basis. Of course this is quite obvious anyway.

**Definition**.: _Let \(\mathrm{T}\) be a linear operator on an inner product space \(\mathrm{V}\). Then we say that \(\mathrm{T}\)_**has an adjoint on \(\mathrm{V}\) if there exists a linear operator \(\mathrm{T}^{*}\) on \(\mathrm{V}\) such that \((\mathrm{T}\alpha|\beta)=(\alpha|\mathrm{T}^{*}\beta)\) for all \(\alpha\) and \(\beta\) in \(\mathrm{V}\)._

By Theorem 7 every linear operator on a finite-dimensional inner product space \(V\) has an adjoint on \(V\). In the infinite-dimensional case this is not always true. But in any case there is at most one such operator \(T^{*}\); when it exists, we call it the **adjoint** of \(T\).

Two comments should be made about the finite-dimensional case.

1. The adjoint of \(T\) depends not only on \(T\) but on the inner product as well.

2. As shown by Example 17, in an arbitrary ordered basis \(\mathbb{O}\), the relation between \([T]_{\mathbb{O}}\) and \([T^{*}]_{\mathbb{O}}\) is more complicated than that given in the corollary above.

**Example 18**.: _Let \(V\) be \(C^{*\times 1}\), the space of complex \(n\times 1\) matrices, with inner product \((X|Y)=Y^{*}X\). If \(A\) is an \(n\times n\) matrix with complex entries, the adjoint of the linear operator \(X\to AX\) is the operator \(X\to A^{*}X\). For_

\[(AX|Y)=Y^{*}AX=(A^{*}Y)^{*}X=(X|A^{*}Y).\]

_The reader should convince himself that this is really a special case of the last corollary._

**Example 19**.: _This is similar to Example 18. Let \(V\) be \(C^{*\times n}\) with the inner product \((A|B)=\mathrm{tr}\;(B^{*}A)\). Let \(M\) be a fixed \(n\times n\) matrix over \(C\). The adjoint of left multiplication by \(M\) is left multiplication by \(M^{*}\). Of course, 'left multiplication by \(M^{*}\) is the linear operator \(L_{M}\) defined by \(L_{M}(A)=MA\)._

\[\begin{array}{rl}(L_{M}(A)|B)&=\mathrm{tr}\;(B^{*}(MA))\\ &=\mathrm{tr}\;(MAB^{*})\\ &=\mathrm{tr}\;(AB^{*}M)\\ &=\mathrm{tr}\;(A\,(M^{*}B)^{*})\\ &=(A\big{|}L_{M}^{*}(B)\big{)}.\end{array}\] 