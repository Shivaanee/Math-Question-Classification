11. Let \(T\) be a linear operator on \(F^{n}\). Define \[D_{T}(\alpha_{1},\ldots,\alpha_{n})\,=\,\det\,(T\alpha_{1},\ldots,T\alpha_{n}).\] (a) Show that \(D_{T}\) is an alternating \(n\)-linear function. (b) If \[c\,=\,\det\,(T\epsilon_{1},\ldots,T\epsilon_{n})\] show that for any \(n\) vectors \(\alpha_{1},\ldots,\alpha_{n}\) we have \[\det\,(T\alpha_{1},\ldots,T\alpha_{n})\,=\,c\,\det\,(\alpha_{1},\ldots,\alpha_ {n}).\] (c) If \(\Theta\) is any ordered basis for \(F^{n}\) and \(A\) is the matrix of \(T\) in the ordered basis \(\Theta\), show that \(\det A\,=\,c\). (d) What do you think is a reasonable name for the scalar \(c\)? 12. If \(\sigma\) is a permutation of degree \(n\) and \(A\) is an \(n\times n\) matrix over the field \(F\) with row vectors \(\alpha_{1}\), \(\ldots,\alpha_{n}\), let \(\sigma(A)\) denote the \(n\times n\) matrix with row vectors \(\alpha_{\epsilon_{1}}\), \(\ldots,\alpha_{n}\). (a) Prove that \(\sigma(AB)\,=\,\sigma(A)B\), and in particular that \(\sigma(A)\,=\,\sigma(I)A\). (b) If \(T\) is the linear operator of Exercise 9, prove that the matrix of \(T\) in the standard ordered basis is \(\sigma(I)\). (c) Is \(\sigma^{-1}(I)\) the inverse matrix of \(\sigma(I)\)? (d) Is it true that \(\sigma(A)\) is similar to \(A\)? 13. Prove that the sign function on permutations is unique in the following sense. If \(f\) is any function which assigns to each permutation of degree \(n\) an integer, and if \(f(\sigma\tau)=f(\sigma)f(\tau)\), then \(f\) is identically \(0\), or \(f\) is identically \(1\), or \(f\) is the sign function.

### Additional Properties of Determinants

In this section we shall relate some of the useful properties of the determinant function on \(n\times n\) matrices. Perhaps the first thing we should point out is the following. In our discussion of \(\det A\), the rows of \(A\) have played a privileged role. Since there is no fundamental difference between rows and columns, one might very well expect that \(\det A\) is an alternating \(n\)-linear function of the columns of \(A\). This is the case, and to prove it, it suffices to show that (5-17) \[\det\,(A\,!)\,=\,\det\,(A)\] where \(A\,!\) denotes the transpose of \(A\). If \(\sigma\) is a permutation of degree \(n\), \[A\,!(i,\sigma\,i)\,=\,A\,(\sigma i,\,i).\] From the expression (5-15) one then has \[\det\,(A\,!)\,=\,\sum_{\sigma}\,(\mbox{sgn}\,\,\sigma)A\,(\sigma 1,\,1)\,\cdots\,A\,(\sigma n,\,n).\] When \(i\,=\,\sigma^{-1}\!j\), \(A\,(\sigma i,\,i)\,=\,A\,(j,\,\sigma^{-1}\!j)\). Thus \[A\,(\sigma 1,\,1)\,\cdots\,A\,(\sigma n,\,n)\,=\,A\,(1,\sigma^{-1}\!1)\, \cdots\,A\,(n,\sigma^ 