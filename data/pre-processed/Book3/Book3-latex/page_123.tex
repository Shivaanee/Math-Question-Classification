Definition: If A is an m x n matrix over the field F, the transpose of A is the n x m matrix A\({}^{\iota}\) defined by A\({}^{\iota}_{\iota}\) = A\({}_{\iota i}\).

Theorem 23 thus states that if \(T\) is a linear transformation from \(V\) into \(W\), the matrix of which in some pair of bases is \(A\), then the transpose transformation _T\({}^{\iota}\)_ is represented in the dual pair of bases by the transpose matrix _A\({}^{\iota}\)_.

Theorem 24: _Let A be any m x n matrix over the field F. Then the row rank of A is equal to the column rank of A._

Proof: Let B be the standard ordered basis for _F\({}^{n}\)_ and B\({}^{\prime}\)_ the standard ordered basis for _F\({}^{m}\)_. Let \(T\) be the linear transformation from _F\({}^{n}\)_ into _F\({}^{m}\)_ such that the matrix of \(T\) relative to the pair B, B\({}^{\prime}\) is \(A\), i.e.,

\[T(x_{1},\,.\,.\,.\,,\,x_{n})\,=\,(y_{1},\,.\,.\,.\,,\,y_{m})\]

where

\[y_{i}\,=\,\sum\limits_{j\,=\,1}^{n}\,A\,_{ij}x_{j}.\]

The column rank of \(A\) is the rank of the transformation \(T\), because the range of \(T\) consists of all _m_-tuples which are linear combinations of the column vectors of \(A\).

Relative to the dual bases B\({}^{\prime}\)* and B*, the transpose mapping _T\({}^{\iota}\)_ is represented by the matrix _A\({}^{\iota}\)_. Since the columns of _A\({}^{\iota}\)_ are the rows of \(A\), we see by the same reasoning that the row rank of \(A\) (the column rank of _A\({}^{\iota}\)_) is equal to the rank of _T\({}^{\iota}\)_. By Theorem 22, \(T\) and _T\({}^{\iota}\)_ have the same rank, and hence the row rank of \(A\) is equal to the column rank of \(A\).

Now we see that if \(A\) is an _m x n matrix over \(F\) and \(T\) is the linear transformation from _F\({}^{n}\)_ into _F\({}^{m}\)_ defined above, then

\[\text{rank }(T)\,=\,\text{row rank }(A)\,=\,\text{column rank }(A)\]

and we shall call this number simply the **rank** of \(A\).

**Example 25**: _This example will be of a general nature--more discussion than example. Let \(V\) be an _n_-dimensional vector space over the field \(F\), and let \(T\) be a linear operator on \(V\). Suppose B\({}^{\prime}\) = {\(\alpha_{1},\,.\,.\,,\,\alpha_{n}\)} is an ordered basis for \(V\). The matrix of \(T\) in the ordered basis B is defined to be the _n x n matrix \(A\) such that

\[T\alpha_{j}\,=\,\sum\limits_{j\,=\,1}^{n}\,A\,_{ij}\alpha_{i}\]

in other words, _A\({}_{ij}\)_ is the \(i\)th coordinate of the vector _T\(\alpha_{j}\)_ in the ordered basis B. If {_f_, _._ ._ ,\(f\), _n_} is the dual basis of B, this can be stated simply

\[A\,_{ij}\,=\,f_{i}(T\alpha_{j}).\] 