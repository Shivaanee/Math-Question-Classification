non-zero vector generates a one-dimensional cyclic subspace; thus, if \(\dim\,V>1\), the identity operator has no cyclic vector. An example of an operator which has a cyclic vector is the linear operator \(T\) on \(F^{2}\) which is represented in the standard ordered basis by the matrix \(\begin{bmatrix}0&0\\ 1&0\end{bmatrix}\). Here the cyclic vector (a cyclic vector) is \(\epsilon_{1}\); for, if \(\delta=(a,b)\), then with \(g=a+bx\) we have \(\beta=g(T)\epsilon_{1}\). For this same operator \(T\), the cyclic subspace generated by \(\epsilon_{2}\) is the one-dimensional space spanned by \(\epsilon_{2}\), because \(\epsilon_{2}\) is a characteristic vector of \(T\).

For any \(T\) and \(\alpha\), we shall be interested in linear relations

\[c_{\alpha}\alpha+c_{1}T\alpha+\,\cdots\,+\,c_{k}T^{k}\alpha=0\]

between the vectors \(T^{i}\alpha\), that is, we shall be interested in the polynomials \(g=c_{0}+c_{1}x+\,\cdots\,+\,\alpha x^{k}\) which have the property that \(g(T)\alpha=0\). The set of all \(g\) in \(F[x]\) such that \(g(T)\alpha 