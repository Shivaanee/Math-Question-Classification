Theorem 1. Equivalent systems of linear equations have exactly the same solutions.

If the elimination process is to be effective in finding the solutions of a system like (1-1), then one must see how, by forming linear combinations of the given equations, to produce an equivalent system of equations which is easier to solve. In the next section we shall discuss one method of doing this.
Exercises
1. Verify that the set of complex numbers described in Example 4 is a subfield of $C$.
2. Let $F$ be the field of complex numbers. Are the following two systems of linear equations equivalent? If so, express each equation in each system as a linear combination of the equations in the other system.
$$
\begin{array}{rlrl}
x_1-x_2 & =0 & 3 x_1+x_2 & =0 \\
2 x_1+x_2 & =0 & x_1+x_2 & =0
\end{array}
$$
3. Test the following systems of equations as in Exercise 2.
$$
\begin{aligned}
-x_1+x_2+4 x_3 & =0 & x_1 & -x_3=0 \\
x_1+3 x_2+8 x_3 & =0 & & x_2+3 x_3=0 \\
{ }_2^1 x_1+x_2+\frac{5}{2} x_3 & =0 & &
\end{aligned}
$$
4. Test the following systems as in Exercise 2.
$$
\begin{array}{rlr}
2 x_1+(-1+i) x_2+x_4 & =0 & \left(1+\frac{i}{2}\right) x_1+8 x_2-i x_3-x_4=0 \\
3 x_2-2 i x_3+5 x_4 & =0 & \frac{2}{3} x_1-\frac{1}{2} x_2+x_3+7 x_4=0
\end{array}
$$
5. Let $F$ be a set which contains exactly two elements, 0 and 1 . Define an addition and multiplication by the tables:
\begin{tabular}{c|ccc|cc}
+ & 0 & 1 \\
\hline 0 & 0 & 1 \\
1 & 1 & 0 &. & 0 & 1 \\
\hline 0 & 0 & 0 \\
1 & 0 & 1
\end{tabular}

Verify that the set $F$, together with these two operations, is a field.
6. Prove that if two homogeneous systems of linear equations in two unknowns have the same solutions, then they are equivalent.
7. Prove that each subfield of the field of complex numbers contains every rational number.
8. Prove that each field of characteristic zero contains a copy of the rational number field.
