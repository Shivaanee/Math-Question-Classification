Therefore

Now we have already observed that

Thus, it follows that

(5-46)

This formula simplifies a number of computations. For example, suppose we have an \(r\)-shuffle \(I=(i_{1},\,\ldots,\,i_{r})\) and \(s\)-shuffle \(J=(j_{1},\,\ldots,\,j_{s})\). To make things simple, assume, in addition, that

\[i_{1}<\,\cdots\,<\,i_{r}<j_{1}<\,\cdots\,<j_{s}.\]

Then we have the associated determinant functions

\[\begin{array}{l}D_{I}=\,\pi_{r}(E_{I})\\ D_{J}=\,\pi_{s}(E_{J})\end{array}\]

where \(E_{I}\) and \(E_{J}\) are given by (5-30). Using (5-46), we see immediately that

\[\begin{array}{l}D_{I}\,\cdot\,D_{J}=\,\pi_{r+s}[\pi_{r}(E_{I})\,\,\hbox{ \raise 1.0pt\hbox{$\circ$}\hbox to 0.0pt{$\circ$}\hbox to 0.0pt{$ 