Hence \(|(\alpha|\beta)|^{2}\leq||\alpha||^{2}||\beta||^{2}\). Now using (c) we find that

\[\begin{array}{rl}||\alpha+\beta||^{2}&=\ ||\alpha||^{2}+(\alpha|\beta)+( \beta|\alpha)+||\beta||^{2}\\ &=\ ||\alpha||^{2}+2\ \mathrm{Re}\ (\alpha|\beta)+||\beta||^{2}\\ &\leq\ ||\alpha||^{2}+2\ ||\alpha||\ ||\beta||+||\beta||^{2}\\ &=\ (||\alpha||\ +||\beta||)^{2}.\end{array}\]

Thus, \(||\alpha+\beta||\leq||\alpha||+||\beta||\).

The inequality in (iii) is called the **Cauchy-Schwarz inequality.** It has a wide variety of applications. The proof shows that if (for example) \(\alpha\) is non-zero, then \(|(\alpha|\beta)|<||\alpha||\ ||\beta||\) unless

\[\beta=\frac{(\beta|\alpha)}{||\alpha||^{2}}\,\alpha.\]

Thus, equality occurs in (iii) if and only if \(\alpha\) and \(\beta\) are linearly dependent.

If we apply the Cauchy-Schwarz inequality to the inner products given in Examples 1, 2, 3, and 5, we obtain the following:

\begin{tabular}{l c} (a) & \(|\Sigma\ x_{i}\bar{y}_{k}|\leq(\Sigma\ |x_{k}|^{2})^{1/2}(\Sigma\ |y_{k}|^{2})^{1/2}\) \\ (b) & \(|x_{1}y_{1}-x_{2}y_{1}-x_{1}y_{2}+4x_{2}y_{2}|\) \\ & \(\leq\ ((x_{1}-x_{2})^{2}+3x_{2}^{2})^{1/2}((y_{1}-y_{2})^{2}+3y_{2}^ 