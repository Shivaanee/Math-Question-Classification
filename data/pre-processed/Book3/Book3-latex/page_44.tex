

**Theorem 1**: _A non-empty subset \(W\) of \(V\) is a subspace of \(V\) if and only if for each pair of vectors \(\alpha\), \(\beta\) in \(W\) and each scalar \(c\) in \(F\) the vector \(c\alpha\)\(+\)\(\beta\) is again in \(W.\)_

Suppose that \(W\) is a non-empty subset of \(V\) such that \(c\alpha\)\(+\)\(\beta\) belongs to \(W\) for all vectors \(\alpha\), \(\beta\) in \(W\) and all scalars \(c\) in \(F.\) Since \(W\) is non-empty, there is a vector \(\rho\) in \(W\), and hence \((-1)\rho\)\(+\)\(\rho\)\(=\)\(0\) is in \(W.\) Then if \(\alpha\) is any vector in \(W\) and \(c\) any scalar, the vector \(c\alpha\)\(+\)\(\alpha\) is in \(W\). In particular, \((-1)\alpha\)\(=\)\(-\alpha\) is in \(W\). Finally, if \(\alpha\) and \(\beta\) are in \(W\), then \(\alpha\)\(+\)\(\beta\)\(=\)\(1\alpha\)\(+\)\(\beta\) is in \(W\). Thus \(W\) is a subspace of \(V\).

Conversely, if \(W\) is a subspace of \(V\), \(\alpha\) and \(\beta\) are in \(W\), and \(c\) is a scalar, certainly \(c\alpha\)\(+\)\(\beta\) is in \(W\).

Some people prefer to use the \(c\alpha\)\(+\)\(\beta\) property in Theorem 1 as the definition of a subspace. It makes little difference. The important point is that, if \(W\) is a non-empty subset of \(V\) such that \(c\alpha\)\(+\)\(\beta\) is in \(V\) for all \(\alpha\), \(\beta\) in \(W\) and all \(c\) in \(F\), then (with the operations inherited from \(V\)) \(W\) is a vector space. This provides us with many new examples of vector spaces.

**Example 6**: __

(a) If \(V\) is any vector space, \(V\) is a subspace of \(V\); the subset consisting of the zero vector alone is a subspace of \(V\), called the **zero subspace** of \(V\).

(b) In \(F\)", the set of \(n\)-tuples \((x_{1},\,.\,.\,.\,,\,x_{n})\) with \(x_{1}\)\(=\)\(0\) is a subspace; however, the set of \(n\)-tuples with \(x_{1}\)\(=\)\(1\)\(+\)\(x_{2}\) is not a subspace \((n\geq 2)\).

(c) The space of polynomial functions over the field \(F\) is a subspace of the space of all functions from \(F\) into \(F\).

(d) An \(n\)\(\times\)\(n\) (square) matrix \(A\) over the field \(F\) is **symmetric** if \(A\)\({}_{ij}\)\(=\)\(A\)\({}_{ji}\) for each \(i\) and \(j\). The symmetric matrices form a subspace of the space of all \(n\)\(\times\)\(n\) matrices over \(F\).

(e) An \(n\)\(\times\)\(n\) (square) matrix \(A\) over the field \(C\) of complex numbers is **Hermitian** (or **self-adjoint**) if

\[A_{jk}\)\(=\)\(\overline{A_{kj}}\]

for each \(j\), \(k\), the bar denoting complex conjugation. A \(2\)\(\times\)\(2\) matrix is Hermitian if and only if it has the form

\[\left[\begin{matrix}z&x+iy\\ x-iy&w\end{matrix}\right]\]

where \(x\), \(y\), \(z\), and \(w\) are real numbers. The set of all Hermitian matrices is _not_ a subspace of the space of all \(n\)\(\times\)\(n\) matrices over \(C\). For if \(A\) is Hermitian, its diagonal entries \(A\)\({}_{11}\), \(A\)\({}_{22}\), \(.\,.\,.\,.\,,\,\) are all real numbers, but the diagonal entries of \(iA\) are in general not real. On the other hand, it is easily verified that the set of \(n\)\(\times\)\(n\) complex Hermitian matrices is a vector space over the field \(R\) of real numbers (with the usual operations).

