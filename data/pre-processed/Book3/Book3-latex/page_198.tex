\[D=\left[\begin{matrix}1&0&0\\ 0&2&0\\ 0&0&2\end{matrix}\right].\]

The fact that \(T\) is diagonalizable means that the original matrix \(A\) is similar (over \(R\)) to the diagonal matrix \(D.\) The matrix \(P\) which enables us to change coordinates from the basis \(\otimes\) to the standard basis is (of course) the matrix which has the transposes of \(\alpha_{1},\)\(\alpha_{2},\)\(\alpha_{3}\) as its column vectors:

\[P=\left[\begin{matrix}3&2&2\\ -1&1&0\\ 3&0&1\end{matrix}\right].\]

Furthermore, \(AP=PD,\) so that

\[P^{-1}AP=D.\]

**1.** In each of the following cases, let \(T\) be the linear operator on \(R^{2}\) which is represented by the matrix \(A\) in the standard ordered basis for \(R^{2},\) and let \(U\) be the linear operator on \(C^{2}\) represented by \(A\) in the standard ordered basis 