obtain the result for any matrix over an algebraically closed field. Any field is a subfield of an algebraically closed field. If one knows that result, one obtains a proof of the Cayley-Hamilton theorem for matrices over any field. If we at least admit into our discussion the Fundamental Theorem of Algebra (the complex number field is algebraically closed), then Theorem 5 provides a proof of the Cayley-Hamilton theorem for complex matrices, and that proof is independent of the one which we gave earlier.

### Exercises

1. Let \(T\) be the linear operator on \(R^{2}\), the matrix of which in the standard ordered basis is \[A\ =\begin{bmatrix}1&-1\\ 2&2\end{bmatrix}.\] (a) Prove that the only subspaces of \(R^{2}\) invariant under \(T\) are \(R^{2}\) and the zero subspace. (b) If \(U\) is the linear operator on \(C^{2}\), the matrix of which in the standard ordered basis is \(A\), show that \(U\) has 1-dimensional invariant subspaces. 2. Let \(W\) be an invariant subspace for \(T\). Prove that the minimal polynomial for the restriction operator \(T_{W}\) divides the minimal polynomial for \(T\), without referring to matrices. 3. Let \(c\) be a characteristic value of \(T\) and let \(W\) be the space of characteristic vectors associated with the characteristic value \(c\). What is the restriction operator \(T_{W}\)? 4. Let \[A\ =\begin{bmatrix}0&1&0\\ 2&-2&2\\ 2&-3&2\end{bmatrix}.\] Is \(A\) similar over the field of real numbers to a triangular matrix? If so, find such a triangular matrix. 5. Every matrix \(A\) such that \(A^{2}=A\) is similar to a diagonal matrix. 6. Let \(T\) be a diagonalizable linear operator on the \(n\)-dimensional vector space \(V\), and let \(W\) be a subspace which is invariant under \(T\). Prove that the restriction operator \(T_{W}\) is diagonalizable. 7. Let \(T\) be a linear operator on a finite-dimensional vector space over the field of complex numbers. Prove that \(T\) is diagonalizable if and only if \(T\) is annihilated by some polynomial over \(C\) which has distinct roots. 8. Let \(T\) be a linear operator on \(V\). If every subspace of \(V\) is invariant under \(T\), then \(T\) is a scalar multiple of the identity operator. 9. Let \(T\) be the indefinite integral operator \[(T\!f)(x)=\int_{0}^{x}f(t)\ dt\] 