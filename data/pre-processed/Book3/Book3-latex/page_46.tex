

**Theorem 3**: _The subspace spanned by a non-empty subset \(S\) of a vector space \(V\) is the set of all linear combinations of vectors in \(S.\)_

Let \(W\) be the subspace spanned by \(S.\) Then each linear combination

\[\alpha=x_{1}\alpha_{1}+x_{2}\alpha_{2}+\cdots+x_{m}\alpha_{m}\]

of vectors \(\alpha_{1},\)\(\alpha_{2},\)\(\ldots,\)\(\alpha_{m}\) in \(S\) is clearly in \(W.\) Thus \(W\) contains the set \(L\) of all linear combinations of vectors in \(S.\) The set \(L,\) on the other hand, contains \(S\) and is non-empty. If \(\alpha,\)\(\beta\) belong to \(L\) then \(\alpha\) is a linear combination,

\[\alpha=x_{1}\alpha_{1}+x_{2}\alpha_{2}+\cdots+x_{m}\alpha_{m}\]

of vectors \(\alpha_{i}\) in \(S,\) and \(\beta\) is a linear combination,

\[\beta=y_{1}\beta_{1}+y_{2}\beta_{2}+\cdots+y_{m}\beta_{m}\]

of vectors \(\beta_{j}\) in \(S.\) For each scalar \(c,\)

\[c\alpha+\beta=\sum\