1. _any subset of_ V _which contains more than_ n _vectors is linearly dependent;_
2. _no subset of_ V _which contains fewer than_ n _vectors can span_ V_._

**Example 17**: _If \(F\) is a field, the dimension of \(F^{n}\) is \(n\), because the standard basis for \(F^{n}\) contains \(n\) vectors. The matrix space \(F^{m\times n}\) has dimension \(mn\). That should be clear by analogy with the case of \(F^{n}\), because the \(mn\) matrices which have a 1 in the \(i,j\) place with zeros elsewhere form a basis for \(F^{m\times n}\). If \(A\) is an \(m\times n\) matrix, then the solution space for \(A\) has dimension \(n-r\), where \(r\) is the number of non-zero rows in a row-reduced echelon matrix which is row-equivalent to \(A\). See Example 15._

_If_ V _is any vector space over_ \(F\)_, the zero subspace of_ V _is spanned by the vector_ \(0\)_, but_ \(\{0\}\) _is a linearly dependent set and not a basis. For this reason, we shall agree that the zero subspace has dimension_ \(0\)_. Alternatively, we could reach the same conclusion by arguing that the empty set is a basis for the zero subspace. The empty set spans_ \(\{0\}\)_, because the intersection of all subspaces containing the empty set is_ \(\{0\}\)_, and the empty set is linearly independent because it contains no vectors._

**Lemma**: _Let S be a linearly independent subset of a vector space V. Suppose \(\beta\) is a vector in V which is not in the subspace spanned by S. Then the set obtained by adjoining \(\beta\) to S is linearly independent._

Suppose \(\alpha_{1}\), \(\ldots\), \(\alpha_{m}\) are distinct vectors in \(S\) and that

\[c_{1}\alpha_{1}+\,\cdots\,+\,c_{m}\alpha_{m}+\,b\hbox{\kern-3.0pt\vrule height 6. 45pt width 0.4pt depth 0.0pt\kern-3.0pt\vrule height 6.45pt width 0.4pt depth 0.

 