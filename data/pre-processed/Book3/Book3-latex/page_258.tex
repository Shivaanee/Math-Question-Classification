Let \(c_{1}\), \(\ldots\), \(c_{k}\) be the distinct complex roots of \(p\):

\[p\,=\,\langle x-c_{1}\rangle^{r_{1}}\,\cdots\,\langle x-c_{k}\rangle^{r_{k}}.\]

Let \(V_{i}\) be the null space of \((D-c_{i}I)^{r_{i}}\), that is, the set of solutions to the differential equation

\[(D-c_{i}I)^{r_{i}}\!f=0.\]

Then as we noted in Example 15, Chapter 6 the primary decomposition theorem tells us that

\[V\,=\,V_{1}\bigoplus\,\cdots\,\bigoplus\,V_{k}.\]

Let \(N_{i}\) be the restriction of \(D-c_{i}I\) to \(V_{i}\). The Jordan form for the operator \(D\) (on \(V\)) is then determined by the rational forms for the nilpotent operators \(N_{i}\), \(\ldots\), \(N_{k}\) on the spaces \(V_{1}\), \(\ldots\), \(V_{k}\).

So, what we must know (for various values of \(c\)) is the rational form for the operator \(N=(D-cI)\) on the space \(V_{c}\), which consists of the solutions of the equation

\[(D-cI)^{r_{i}}\!f=0.\]

How many elementary nilpotent blocks will there be in the rational form for \(N\)? The number will be the nullity of \(N\), i.e., the dimension of the characteristic space associated with the characteristic value \(c\). That dimension is 1, because any function which satisfies the differential equation

\[Df=cf\]

is a scalar multiple of the exponential function \(h(x)=e^{cx}\). Therefore, the operator \(N\) (on the space \(V_{c}\)) has a cyclic vector. A good choice for a cyclic vector is \(g=x^{r-1}h\):

\[g(x)\,=\,x^{r-1}e^{cx}.\]

This gives

\[\begin{array}{c}Ng\,=\,(r-1)x^{r-2}h\\ \vdots\\ N 