\[f_{\beta}(\alpha)\,=\,(\alpha|\beta).\]

This function \(f_{\beta}\) is a linear functional on \(V\), because, by its very definition, \((\alpha|\beta)\) is linear as a function of \(\alpha\). If \(V\) is finite-dimensional, every linear functional on \(V\) arises in this way from some \(\beta\).

**Theorem 6**: _Let \(\mathrm{V}\) be a finite-dimensional inner product space, and \(\mathrm{f}\) a linear functional on \(\mathrm{V}\). Then there exists a unique vector \(\beta\) in \(\mathrm{V}\) such that \(\mathrm{f}(\alpha)\,=\,(\alpha|\beta)\) for all \(\alpha\,in\,\mathrm{V}\)._

Let \(\{\alpha_{1},\alpha_{2},\ldots,\alpha_{n}\}\) be an orthonormal basis for \(V\). Put

\[\beta\,=\,\sum_{j\,=\,1}^{n}\overline{f(\alpha_{j})}\alpha_{j}\] (8-13)

and let \(f_{\beta}\) be the linear functional defined by

\[f_{\beta}(\alpha)\,=\,(\alpha|\beta).\]

Then

\[f_{\beta}(\alpha_{k})\,=\,(\alpha_{k}|\,\sum_{j}\overline{f(\alpha_{j})} \alpha_{j})\,=\,f(\alpha_{k}).\]

Since this is true for each \(\alpha_{k}\), it follows that \(f\ 