* _There is some basis_ \(\{\alpha_{1},\ldots,\alpha_{n}\}\) _for_ \(V\) _such that_ \(\{T\alpha_{1},\ldots,T\alpha_{n}\}\) _is a basis for_ \(W\)_._

We shall give a proof of the equivalence of the five conditions which contains a different proof that (i), (ii), and (iii) are equivalent.

(i) \(\rightarrow\) (ii). If \(T\) is invertible, \(T\) is non-singular. (ii) \(\rightarrow\) (iii). Suppose \(T\) is non-singular. Let \(\{\alpha_{1},\ldots,\alpha_{n}\}\) be a basis for \(V\). By Theorem 8, \(\langle T\alpha_{1},\ldots,T\alpha_{n}\rangle\) is a linearly independent set of vectors in \(W\), and since the dimension of \(W\) is also \(n\), this set of vectors is a basis for \(W\). \(N\)ow let \(\beta\) be any vector in \(W\). There are scalars \(c_{1},\ldots,c_{n}\) such that

\[\begin{array}{l}\beta\,=\,c_{1}(T\alpha_{1})\,+\,\cdots\,+c_{n}(T\alpha_{n}) \\ \,=\,T(c_{1}\alpha_{1}+\,\cdots\,+c_{n}\alpha_{n})\end{array}\]

which shows that \(\beta\) is in the range of \(T\). (iii) \(\rightarrow\) (iv). We now assume that \(T\) is onto. If \(\{\alpha_{1},\ldots,\alpha_{n}\}\) is any basis for \(V\), the vectors \(T\alpha_{1},\ldots,T\alpha_{n}\) span the range of \(T\), which is all of \(W\) by assumption. Since the dimension of \(W\) is \(n\), these \(n\)vectors must be linearly independent, that is, must comprise a basis for \(W\). (iv) \(\rightarrow\) (v). This requires no comment. (v) \(\rightarrow\) (i). Suppose there is some basis \(\{\alpha_{i},\ldots,\alpha_{n}\}\) for \(V\) such that \(\{T\alpha_{1},\ldots,T\alpha_{n}\}\) is a basis for \(W\). Since the \(T\alpha_{i}\) span \(W\), it is clear that the range of \(T\) is all of \(W\). If \(\alpha=c_{1}\alpha_{1}\,+\,\cdots\,+c_{n}\alpha_{n}\) is in the null space of \(T\), then

\[T(c_{1}\alpha_{1}+\,\cdots\,+c_{n}\alpha_{n})\,=\,0\]

or

\[c_{1}(T\alpha_{1})\,+\,\cdots\,+c_{n}(T\alpha_{n})\,=\,0\]

and since the \(T\alpha_{i}\) are independent each 