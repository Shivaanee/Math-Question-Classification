are disjoint, suppose \(\gamma\) is in both \(W\) and \(W^{\prime}\). Since \(\gamma\) is in \(W\), we have \(Q\gamma=0\). But \(Q\) is 1:1 on \(W^{\prime}\), and so it must be that \(\gamma=0\). Thus we have \(V=W\oplus W^{\prime}\). .

What this theorem really says is that \(W^{\prime}\) is complementary to \(W\) if and only if \(W^{\prime}\) is a subspace which contains exactly one element from each coset of \(W\). It shows that when \(V=W\oplus W^{\prime}\), the quotient mapping \(Q\) 'identifies' \(W^{\prime}\) with \(V/W\). Briefly \((W\oplus W^{\prime})/W\) is isomorphic to \(W^{\prime}\) in a 'natural' way.

One rather obvious fact should be noted. If \(W\) is a subspace of the finite-dimensional vector space \(V\), then

\[\dim W\,+\,\dim\,(V/W)\,=\,\dim\,V.\]

One can see this from the above theorem. Perhaps it is easier to observe that what this dimension formula says is

\[\mbox{nullity }(Q)\,+\,\mbox{rank }(Q)\,=\,\dim\,V.\]

It is not our object here to give a detailed treatment of quotient spaces. But there is one fundamental result which we should prove.

Theorem: Let \(V\) and \(Z\) be vector spaces over the field \(F\). Suppose \(T\) is a linear transformation of \(V\) onto \(Z\). If \(W\) is the null space of \(T\), then \(Z\) is isomorphic to \(V/W\).

We define a transformation \(U\) from \(V/W\) into \(Z\) by \(U(\alpha+W)=T\alpha\). We must verify that \(U\) is well defined, i.e., that if \(\alpha+W=\beta+W\) then \(T\alpha=T\ 