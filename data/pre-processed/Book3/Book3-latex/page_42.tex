
Scalar multiplication has a simpler geometric interpretation. If \(c\) is a real number, then the product of \(c\) and the vector \(OP\) is the vector from the origin with length \(|c|\) times the length of \(OP\) and a direction which agrees with the direction of \(OP\) if \(c>0\), and which is opposite to the direction of \(OP\) if \(c<0\). This scalar multiplication just yields the vector \(OT\) where \(T=(cx_{1},cx_{2},cx_{3})\), and is therefore consistent with the algebraic definition given for \(R^{3}\).

From time to time, the reader will probably find it helpful to 'think geometrically' about vector spaces, that is, to draw pictures for his own benefit to illustrate and motivate some of the ideas. Indeed, he should do this. However, in forming such illustrations he must bear in mind that, because we are dealing with vector spaces as algebraic systems, all proofs we give will be of an algebraic nature.

### Exercises

**1.** If \(F\) is a field, verify that \(F^{n}\) (as defined in Example 1) is a vector space over the field \(F\).

**2.** If \(V\) is a vector space over the field \(F\), verify that

\[(\alpha_{1}+\alpha_{2})+(\alpha_{3}+\alpha_{4})=[\alpha_{2}+(\alpha_{3}+ \alpha_{1})]+\alpha_{4}\]

for all vectors \(\alpha_{1},\alpha_{3},\alpha_{4},\) and \(\alpha_{4}\) in \(V\).

**3.** If \(C\) is the field of complex numbers, which vectors in \(C^{3}\) are linear combinations of \((1,0,-1)\), \((0,1,1)\), and \((1,1,1)\)? 