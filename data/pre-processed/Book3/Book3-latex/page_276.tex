Let \(p\) be a prime polynomial which divides both \(f\) and \(f^{\prime}\). Then \(p=p_{i}\) for some \(i\). Now \(p_{i}\) divides \(f_{i}\) for \(j\neq i\), and since \(p_{i}\) also divides

\[f^{\prime}=\sum_{j=1}^{k}p_{j}^{\prime}f_{j}\]

we see that \(p_{i}\) must divide \(p_{i}^{\prime}f_{i}\). Therefore \(p_{i}\) divides either \(f_{i}\) or \(p_{i}^{\prime}\). But \(p_{i}\) does not divide \(f_{i}\) since \(p_{1}\), \(\ldots\), \(p_{k}\) are distinct. So \(p_{i}\) divides \(p_{i}^{\prime}\). This is not possible, since \(p_{i}^{\prime}\) has degree one less than the degree of \(p_{i}\). We conclude that no prime divides both \(f\) and \(f^{\prime}\), or that \((f,f^{\prime})=1\).

To see that statement (c) is equivalent to (a) and (b), we need only observe the following: Suppose \(f\) and \(g\) are polynomials over \(F\), a subfield of the complex numbers. We may also regard \(f\) and \(g\) as polynomials with complex coefficients. The statement that \(f\) and \(g\) are relatively prime as polynomials over \(F\) is equivalent to the statement that \(f\) and \(g\) are relatively prime as polynomials over the field of complex numbers. We leave the proof of this as an exercise. We use this fact with \(g=f^{\prime}\). Note that (c) is just (a) when \(f\) is regarded as a polynomial over the field of complex numbers. Thus (b) and (c) are equivalent, by the same argument that we used above.

We can now prove a theorem which makes the relation between semi-simple operators and diagonalizable operators even more apparent.

**Theorem 12**.: _Let \(F\) be a subfield of the field of complex numbers, let \(V\) be a finite-dimensional vector space over \(F\), and let \(T\) be a linear operator on \(V\). Let \(\otimes\) be an ordered basis for \(V\) and let \(A\) be the matrix of \(T\) in the ordered basis \(\otimes\). Then \(T\) is semi-simple if and only if the matrix \(A\) is similar over the field of complex numbers to a diagonal matrix._

Proof.: Let \(p\) be the minimal polynomial for \(T\). According to Theorem 11, \(T\) is semi-simple if and only if \(p=p_{1}\cdots\ p_{k}\) where \(p_{1}\), \(\ldots\), \(p_{k}\) are distinct irreducible polynomials over \(F\). By the last lemma, we see that \(T\) is semi-simple if and only if \(p\) has no repeated complex root.

Now \(p\) is also the minimal polynomial for the matrix \(A\). We know that \(A\) is similar over the field of complex numbers to a diagonal matrix if and only if its minimal polynomial has no repeated complex root. This proves the theorem.

**Theorem 13**.: _Let \(F\) be a subfield of the field of complex numbers, let \(V\) be a finite-dimensional vector space over \(F\), and let \(T\) be a linear operator on \(V\). There is a semi-simple operator \(S\) on \(V\) and a nilpotent operator \(N\) on \(V\) such that_

1. \(T=S+N\);
2. \(SN=NS\).

 