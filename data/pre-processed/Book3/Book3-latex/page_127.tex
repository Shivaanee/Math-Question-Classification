

**Example 2**: The space of all linear operators on a vector space, with composition as the product, is a linear algebra with identity. It is commutative if and only if the space is one-dimensional.

The reader may have had some experience with the dot product and cross product of vectors in \(R^{\ast}\). If so, he should observe that neither of these products is of the type described in the definition of a linear algebra. The dot product is a 'scalar product,' that is, it associates with a pair of vectors a scalar, and thus it is certainly not the type of product we are presently discussing. The cross product does associate a vector with each pair of vectors in \(R^{\ast}\); however, this is not an associative multiplication.

The rest of this section will be devoted to the construction of an algebra which is significantly different from the algebras in either of the preceding examples. Let \(F\) be a field and \(S\) the set of non-negative integers. By Example 3 of Chapter 2, the set of all functions from \(S\) into \(F\) is a vector space over \(F\). We shall denote this vector space by \(F^{\ast}\). The vectors in \(F^{\ast}\) are therefore infinite sequences \(f=(f_{0},f_{1},f_{2},\ldots)\) of scalars \(f_{i}\) in \(F\). If \(g=(g_{0},g_{1},g_{2},\ldots)\), \(g_{i}\) in \(F\), and \(a\), \(b\) are scalars in \(F\), \(af+bg\) is the infinite sequence given by

\[af+bg=(af_{0}+bg_{0},af_{1}+bg_{1},af_{2}+bg_{2},\ldots).\] (4.1)

We define a product in \(F^{\ast}\) by associating with each pair of vectors \(f\) and \(g\) in \(F^{\ast}\) the vector \(fg\) which is given by

\[(fg)_{n}=\sum_{i=0}^{n}f_{i}g_{n-i}\qquad n=0,1,2,\ldots.\] (4.2)

Thus

\[fg=(f_{0}g_{0},f_{0}g_{1}+f_{1}g_{0},f_{0}g_{2}+f_{1}g_{1}+f_{2}g_{0},\ldots)\]

and as

\[(gf)_{n}=\sum_{i=0}^{n}g_{i}f_{n-i}=\sum_{i=0}^{n}f_{i}g_{n-i}=(fg)_