\[p_{j}=\prod_{i\neq j}\frac{(x-c_{i})}{(c_{j}-c_{i})}.\]

We recall from Chapter 4 that \(p_{j}(c_{i})=\delta_{ij}\) and for any polynomial \(g\) of degree less than or equal to \((k-1)\) we have

\[g=g(c_{1})p_{1}+\cdots+g(c_{k})p_{k}.\]

Taking \(g\) to be the scalar polynomial \(1\) and then the polynomial \(x\), we have

\[\begin{array}{l}1=p_{1}+\cdots+p_{k}\\ x=c_{i}p_{1}+\cdots+c_{k}p_{k}.\end{array}\]

(The astute reader will note that the application to \(x\) may not be valid because \(k\) may be \(1\). But if \(k=1\), \(T\) is a scalar multiple of the identity and hence diagonalizable.) Now let \(E_{j}=p_{j}(T)\). From (6-15) we have

\[\begin{array}{l}I=E_{1}+\cdots+E_{k}\\ T=c_{1}E_{1}+\cdots+c_{k}E_{k}.\end{array}\]

Observe that if \(i\neq j\), then \(p_{i}p_{j}\) is divisible by the minimal polynomial \(p\), because \(p_{i}p_{j}\) contains every \((x-c_{r})\) as a factor. Thus

\[E_{i}E_{j}=0,\qquad i\neq j.\]

We must note one further thing, namely, that \(E_{i}\neq 0\) for each \(i\). This is because \(p\) is the minimal polynomial for \(T\) and so we cannot have 