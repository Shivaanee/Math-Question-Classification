Then \(f\) is a non-degenerate bilinear form on \(R^{\natural}\). The matrix of \(f\) in the standard ordered basis is the \(n\times n\) identity matrix: \[f(X,\,Y)\,=\,X^{\,\prime}Y.\] This \(f\) is usually called the dot (or scalar) product. The reader is probably familiar with this bilinear form, at least in the case \(n=3\). Geometrically, the number \(f(\alpha,\,\beta)\) is the product of the length of \(\alpha\), the length of \(\beta\), and the cosine of the angle between \(\alpha\) and \(\beta\). In particular, \(f(\alpha,\,\beta)=0\) if and only if the vectors \(\alpha\) and \(\beta\) are orthogonal (perpendicular).

### Exercises

**1.** Which of the following functions \(f\), defined on vectors \(\alpha=(x_{1},x_{2})\) and \(\beta=(y_{1},y_{2})\) in \(R^{\natural}\), are bilinear forms?

1. \(f(\alpha,\,\beta)\,=\,1\). 2. \(f(\alpha,\,\beta)\,=\,(x_{1}-y_{1})^{\natural}+x_{2}y_{2}\). 3. \(f(\alpha,\,\beta)\,=\,(x_{1}+y_{1})^{\natural}-(x_{1}-y_{1})^{\natural}\). 4. \(f(\alpha,\,\beta)\,=\,x_{1}y_{2}-x_{2}y_{1}\).
2. Let \(f\) be the bilinear form on \(R^{\natural}\) defined by \[f((x_{1},y_{1}),\,(x_{2},y_{2}))\,=\,x_{1}y_{1}+x_{2}y_{2}.\] Find the matrix of \(f\) in each of the following bases: \[\{(1,\,0),\,(0,\,1)\},\qquad\{(1,\,-1),\,(1,\,1)\},\qquad\{(1,\,2),\,(3,\,4)\}.\]
3. Let \(V\) be the space of all \(2\times 3\) matrices over \(R_{\natural}\) and let \(f\) be the bilinear form on \(V\) defined by \(f(X,\,Y)\,=\,\text{trace}\,\,(X^{\,\prime}AY)\), where \[A\,=\,\begin{bmatrix}1&2\\ 3&4\end{bmatrix}.\] Find the matrix of \(f\) in the ordered basis \[\{E^{\natural},E^{\natural},E^{\natural},E^{\natural},E^{\natural},E^{\natural \natural},E^{\natural\natural}\}\] where \(E^{\natural}\) is the matrix whose only non-zero entry is a \(1\) in row \(i\ 