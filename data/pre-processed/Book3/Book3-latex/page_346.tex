Sec. 9.5

Because \(E_{1}\), \(\ldots\), \(E_{k}\) are canonically associated with \(T\) and

\[I=E_{1}+\cdots+E_{k}\]

the family of projections \(\{E_{1},\,\ldots,E_{k}\}\) is called the **resolution of the identity defined by \(T\)**.

There is a comment that should be made about the proof of the spectral theorem. We derived the theorem using Theorems 18 and 22 of Chapter 8 on the diagonalization of self-adjoint and normal operators. There is another, more algebraic, proof in which it must first be shown that the minimal polynomial of a normal operator is a product of distinct prime factors. Then one proceeds as in the proof of the primary decomposition theorem (Theorem 12, Chapter 6). We shall give such a proof in the next section.

In various applications it is necessary to know whether one may compute certain functions of operators or matrices, e.g., square roots. This may be done rather simply for diagonalizable normal operators.

Definition: Let \(T\) be a diagonalizable normal operator on a finite-dimensional inner product space and

\[T=\sum\limits_{j=1}^{k}c_{j}E_{j}\]

its spectral resolution. Suppose \(f\) is a function whose domain includes the spectrum of \(T\) that has values in the field of scalars. Then the linear operator \(f(T)\) is defined by the equation

\[f(T)=\sum\limits_{j=1}^{k}f(c_{j})E_{i}.\]

Theorem 10: Let \(T\) be a diagonalizable normal operator with spectrum \(S\) on a finite-dimensional inner product space \(V\). Suppose \(f\) is a function whose domain contains \(S\) that has values in the field of scalars. Then \(f(T)\) is a diagonalizable normal operator with spectrum \(f(S)\). If \(U\) is a unitary map of \(V\) onto \(V^{\prime}\) and \(T^{\prime}=UTU^{-1}\), then \(S\) is the spectrum of \(T^{\prime}\) and

\[f(T^{\prime})=Uf(T)U^{-1}.\]

Proof: The normality of \(f(T)\) follows by a simple computation from (9-12) and the fact that

\[f(T)^{*}=\sum\limits_{j}\overline{f(c_{j})}E_{j}.\]

Moreover, it is clear that for every \(\alpha\) in \(E_{j}(V)\)

\[f(T)\alpha=f(c_{j})\alpha.

 