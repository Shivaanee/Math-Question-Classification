teristic values \(c_{1},\ldots,c_{k}\). Let \(W_{i}\) be the space of characteristic vectors associated with the characteristic value \(c_{i}\). As we have seen,

\[V\,=\,W_{1}\bigoplus\,\cdots\,\bigoplus\,W_{k}.\]

Let \(E_{1},\ldots,E_{k}\) be the projections associated with this decomposition, as in Theorem 9. Then (ii), (iii), (iv) and (v) are satisfied. To verify (i), proceed as follows. For each \(\alpha\) in \(V\),

\[\alpha\,=\,E_{1}\alpha\,+\,\cdots\,+\,E_{k}\alpha\]

and so

\[\eqalign{T\alpha&=\,TE_{1}\alpha\,+\,\cdots\,+\,TE_{k}\alpha\cr&=\,c_{k}E_{1} \alpha\,+\,\cdots\,+\,c_{k}E_{k}\alpha.\cr}\]

In other words, \(T=c_{1}E_{1}+\,\cdots\,+\,c_{k}E_{k}\).

Now suppose that we are given a linear operator \(T\) along with distinct scalars \(c_{i}\) and non-zero operators \(E_{i}\) which satisfy (i), (ii) and (iii). Since \(E_{i}E_{i}=0\) when \(i\neq j\), we multiply both sides of \(I=E_{1}+\,\cdots\,+\,E_{k}\) by \(E_{i}\) and obtain immediately \(E_{1}^{2}=E_{i}\). Multiplying \(T=c_{1}E_{1}+\,\cdots\,+\,c_{k}E_{k}\) by \(E_{i}\), we then have \(TE_{i}=c_{i}E_{i}\) which shows that any vector in the range of \(E_{i}\) is in the null space of \((T\,-\,c_{i}I)\). Since we have assumed that \(E_{i}\neq 0\), this proves that there is a non-zero vector in the null space of \((T\,-\,c_{i}I)\), i.e., that \(c_{i}\) is a characteristic value of \(T\). Furthermore, the \(c_{i}\) are 