\[f=\left(\begin{array}{c}a_{m}\\ \dot{b}_{n}\end{array}\right)x^{m-n}d=0\quad\mbox{or}\quad\deg\left[f-\left( \begin{array}{c}a_{m}\\ b_{n}\end{array}\right)x^{m-n}d\right]<\deg f.\]

Thus we may take \(g=\left(\begin{array}{c}a_{m}\\ \dot{b}_{n}\end{array}\right)x^{m-n}\).

Using this lemma we can show that the familiar process of 'long division' of polynomials with real or complex coefficients is possible over any field.

**Theorem 4**: _If \(f\), \(d\) are polynomials over a field \(F\) and \(d\) is different from \(0\) then there exist polynomials \(q\), \(r\) in \(F[x]\) such that_

1. \(f=dq+r\)_._
2. _either_ \(r=0\) _or deg_ \(r<degd\)_._

_The polynomials \(q\), \(r\) satisfying (i) and (ii) are unique._

If \(f\) is \(0\) or \(\deg f<\deg d\) we may take \(q=0\) and \(r=f\). In case \(f\neq 0\) and \(\deg f\geq\deg d\), the preceding lemma shows we may choose a polynomial \(g\) such that \(f-dg=0\) or \(\deg\ (f-dg)<\deg\ f\). If \(f-dg\neq 0\) and \(\deg\ (f-dg)\geq\deg\ d\) we choose a polynomial \(h\) such that \((f-dg)-dh=0\) or

\[\deg\left[f-d(g+h)\right]<\deg\ (f-dg).\]

Continuing this process as long as necessary, we ultimately obtain polynomials \(q\), \(r\) such that \(r=0\) or \(\deg\ r<\deg d\), and \(f=dq+r\). Now suppose we also have \(f=dq_{1}+r_{1}\) where \(r_{1}=0\) or \(\deg\ r_{1}<\deg d\). Then \(dq+r=dq_{1}+r_{1}\) and \(d(q-q_{1})=r_{1}-r\). If \(q-q_{1}\neq 0\) then \(d(q-q_{1})\neq 0\) and

\[\deg d+\deg\ (q 