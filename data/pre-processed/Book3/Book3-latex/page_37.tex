

## Chapter 2 Vector Spaces

### 2.1 Vector Spaces

In various parts of mathematics, one is confronted with a set, such that it is both meaningful and interesting to deal with 'linear combinations' of the objects in that set. For example, in our study of linear equations we found it quite natural to consider linear combinations of the rows of a matrix. It is likely that the reader has studied calculus and has dealt there with linear combinations of functions; certainly this is so if he has studied differential equations. Perhaps the reader has had some experience with vectors in three-dimensional Euclidean space, and in particular, with linear combinations of such vectors.

Loosely speaking, linear alcgbra is that branch of mathematics which treats the common properties of algebraic systems which consist of a set, together with a reasonable notion of a 'linear combination' of elements in the set. In this section we shall define the mathematical object which experience has shown to be the most useful abstraction of this type of algebraic system.

_Definition_.: _A vector space \((\)or linear space\()\) consists of the following:_

1. _a field_ F _of scalars;_
2. _a set_ V _of objects, called vectors;_
3. _a rule \((\)or operation\()\), called vector addition, which associates with each pair of vectors \(\alpha\), \(\beta\) in_ V _a vector \(\alpha\)_+_\(\beta\)_in V, called the sum of \(\alpha\)_and \(\beta\), in such a way that_ 1. _addition is commutative,_ \(\alpha\)_+_\(\beta\)_=_\(\beta\)_+_\(\alpha\)_;_ 2. _addition is associative,_ \(\alpha\)_+_\((\beta+\gamma)\)_=_\((\alpha+\beta)+\gamma\)