

## Chapter 6 Elementary

_Canonical Forms_

### 6.1 Introduction

We have mentioned earlier that our principal aim is to study linear transformations on finite-dimensional vector spaces. By this time, we have seen many specific examples of linear transformations, and we have proved a few theorems about the general linear transformation. In the finite-dimensional case we have utilized ordered bases to represent such transformations by matrices, and this representation adds to our insight into their behavior. We have explored the vector space \(L(V,\,W)\), consisting of the linear transformations from one space into another, and we have explored the linear algebra \(L(V,\,V)\), consisting of the linear transformations of a space into itself.

In the next two chapters, we shall be preoccupied with linear operators. Our program is to select a single linear operator \(T\) on a finite-dimensional vector space \(V\) and to 'take it apart to see what makes it tick.' At this early stage, it is easiest to express our goal in matrix language: Given the linear operator \(T\), find an ordered basis for \(V\) in which the matrix of \(T\) assumes an especially simple form.

Here is an illustration of what we have in mind. Perhaps the simplest matrices to work with, beyond the scalar multiples of the identity, are the diagonal matrices:

(6.1) \[D\,=\!\!\left[\matrix{c_{1}&0&0&\cdots&\vbox{\hrule height 0.0pt\hbox{\vrule width 0.0pt height 5.0pt\kern 5.0pt\vrule width 0.

