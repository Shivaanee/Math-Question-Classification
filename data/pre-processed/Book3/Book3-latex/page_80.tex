formations and matrices. We shall not pursue the particular description \(T\alpha=\alpha B\) because it has the matrix \(B\) on the right of the vector \(\alpha\), and that can lead to some confusion. The point of this example is to show that we can give an explicit and reasonably simple description of all linear transformations from \(F^{n}\) into \(F^{s}\).

If \(T\) is a linear transformation from \(V\) into \(W\), then the range of \(T\) is not only a subset of \(W\); it is a subspace of \(W\). Let \(R_{T}\) be the range of \(T\), that is, the set of all vectors \(\beta\) in \(W\) such that \(\beta=T\alpha\) for some \(\alpha\) in \(V\). Let \(\beta_{1}\) and \(\#_ 