is just another way of stating that either there is some \(i\) such that \(r_{i}\neq 0\) and \(\deg r_{i}<\deg f_{j}\) or else \(r_{i}=0\) for all \(i\) and \(\hat{J}_{j}\) is (therefore) the greatest common divisor of \(f_{1}\), \(\ldots\), \(f_{m}\).

The proof of the lemma is now quite simple. We start with the matrix \(M\) and apply the above procedure to obtain \(M^{\prime}\). Property (c) tells us that either \(M^{\prime}\) will serve as the matrix \(N\) in the lemma or \(l(M^{\prime}_{1})<l(M_{1})\). In the latter case, we apply the procedure to \(M^{\prime}\) to obtain the matrix \(M^{(2)}=(M^{\prime})^{\prime}\). If \(M^{(2)}\) is not a suitable \(N\), we form 