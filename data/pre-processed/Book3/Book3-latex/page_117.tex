This shows that the mapping \(\alpha\mathop{\hbox{\vrule height 6.0pt width 0.4pt depth -0.2pt\vrule height 6.0pt width 0.4pt depth -0.2pt} \vrule height 6.0pt width 0.4pt depth -0.2pt}\vrule height 6.0pt width 0.4pt depth -0.2pt}L_{\alpha}\) is a linear transformation from \(V\) into \(V^{**}\). This transformation is non-singular; for, according to the remarks above \(L_{\alpha}=0\) if and only if \(\alpha=0\). Now \(\alpha\mathop{\hbox{\vrule height 6.0pt width 0.4pt depth -0.2pt\vrule height 6.0pt width 0.4pt depth -0.2pt} \vrule height 6.0pt width 0.4pt depth -0.2pt}\vrule height 6.0pt width 0.4pt depth -0.2pt}L_{\alpha}\) is a non-singular linear transformation from \(V\) into \(V^{**}\), and since

\[\dim\,\,V^{**}=\dim\,\,V^{*}=\dim\,\,V\]

Theorem 9 tells us that this transformation is invertible, and is therefore an isomorphism of \(V\) onto \(V^{**}\).

**Corollary.**_Let \(V\) be a finite-dimensional vector space over the field \(F\). If \(L\) is a linear functional on the dual space \(V^{*}\) of \(V\), then there is a unique vector \(\alpha\) in \(V\) such that_

\[L(f)=f(\alpha)\]

_for every \(f\) in \(V^{*}\)._

**Corollary.**_Let \(V\) be a finite-dimensional vector space over the field \(F\). Each basis for \(V^{*}\) is the dual of some basis for \(V\)._

Let \(\mathop{ 