Of course, we call the matrix \(N\) in the last corollary the **normal form** of \(M\). The polynomials \(f_{1}\), \(\cdot\cdot\cdot\,,f_{l}\) are often called the **invariant factors** of \(M\).

Suppose that \(A\) is an \(n\times n\) matrix with entries in \(F\), and let \(p_{1}\), \(\cdot\cdot\cdot\,,p_{r}\) be the invariant factors for \(A\). We now see that the normal form of the matrix \(xI-A\) has diagonal entries \(1\), \(1\), \(\cdot\cdot\cdot\,,1\), \(p_{r}\), \(\cdot\cdot\cdot\,,p_{1}\). The last corollary tells us what \(p_{1}\), \(\cdot\cdot\cdot\,,p_{r}\) are, in terms of submatrices of \(xI-A\). The number \(n-r\) is the largest \(k\) such that \(\delta_{k}(xI-A)=1\). The minimal polynomial \(p_{1}\) is the characteristic polynomial for \(A\) divided by the greatest common divisor of the determinants of all \((n-1)\times(n-1)\) submatrices of \(xI-A\), etc.

### \(Exercises\)

**1.** True or false? Every matrix in \(F[x]^{n\times n}\) is row-equivalent to an upper-triangular matrix.

**2.** Let \(T\) be a linear operator on a finite-dimensional vector space and let \(A\) be the matrix of \(T\) in some ordered basis. Then \(T\) has a cyclic vector if and only if the determinants of the \((n-1)\times(n-1)\) submatrices of \(xI-A\) are relatively prime.

**3.** Let \(A\) be an \(n\times n\) matrix with entries in the field \(F\) and let \(f_{1}\), \(\cdot\cdot\cdot\,,f_{n}\) be the diagonal entries of the normal form of \(xI-A\). For which matrices \(A\) is \(f_{1}\neq 1\)?

**4.** Construct a linear operator \(T\) with minimal polynomial \(x^{2}(x-1)^{2}\) and characteristic polynomial \(x^{2}(x-1)^{4}\). Describe the primary decomposition of the vector space under \(T\) and find the projections on the primary components. Find a basis in which the matrix of \(T\) is in Jordan form. Also find an explicit direct sum decomposition of the space into \(T\)-cyclic subspaces as in Theorem 3 and give the invariant factors.

**5.** Let \(T\) be the linear operator on \(R^{4}\) which is represented in the standard basis by the matrix

\[A=\begin{bmatrix}1&1&1&1&1&1&1&1\\ 0&0&0&0&0&0&0&1\\ 0&0&0&0&0&0&-1\\ 0&1&1&0&0&0&1\\ 0&0&0&1&1&0&0&0\\ 0&1&1&1&1&1&0&1\\ 0&-1&-1&-1&-1&0&1&-1\\ 0&0&0&0&0&0&0&0\\ \end{bmatrix}\]

* Find the characteristic polynomial and the invariant factors.
* Find the primary decomposition of \(R^{4}\) under \(T\) and the projections on the primary components. Find cyclic decompositions of each primary component as in Theorem 3.

 