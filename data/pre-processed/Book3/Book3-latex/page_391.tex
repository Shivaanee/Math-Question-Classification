A linear operator \(T\) on \(R^{4}\) which preserves this particular bilinear (or quadratic) form is called a **Lorentz transformation,** and the group preserving \(f\) is called the **Lorentz group.** We should like to give one method of describing some Lorentz transformations.

Let \(H\) be the real vector space of all \(2\times 2\) complex matrices \(A\) which are Hermitian, \(A\,=\,A^{*}\). It is easy to verify that

\[\Phi(x,y,z,t)=\left[\begin{array}{cc}t+x&y+iz\\ y-iz&t-x\end{array}\right]\]

defines an isomorphism \(\Phi\) of \(R^{4}\) onto the space \(H\). Under this isomorphism, the quadratic form \(q\) is carried onto the determinant function, that is

\[q(x,y,z,t)=\det\left[\begin{array}{cc}t+x&y+iz\\ y-iz&t-x\end{array}\right]\]

or

\[q(\alpha)=\det\Phi(\alpha).\]

This suggests that we might study Lorentz transformations on \(R^{4}\) by studying linear operators on \(H\) which preserve determinants.

Let \(M\) be any complex \(2\times 2\) matrix and for a Hermitian matrix \(A\) define

\[U_{M}(A)\,=\,MAM^{*}.\]

Now \(MAM^{*}\) is also Hermitian. From this it is easy to see that \(U_{M}\) is a (real) linear operator on \(H\). Let us ask when it is true that \(U_{M}\) 'preserves' determinants, i.e., \(\det\,[U_{M}(A)]=\det A\) for each \(A\) in \(H\). Since the determinant of \(M^{*}\) is the complex conjugate of the determinant of \(M\), we see that

\[\det\,[U_{M}(A)]\,=\,|\det M|^{2}\det A.\]

Thus \(U_{M}\) preserves determinants exactly when \(\det M\) has absolute value \(1\).

So now let us select any \(2\times 2\) complex matrix \(M\) for which \(|\det M|=1\). Then \(U_{M}\) is a linear operator on \(H\) which preserves determinants. Define

\[T_{M}=\Phi^{-1}U_{M}\Phi.\]

Since \(\Phi\) is an isomorphism, \(T_{M}\) is a linear operator on \(R^{4}\). Also, \(T_{M}\) is a Lorentz transformation; for

\[\begin{array}{rl}q(T_{M}\alpha)&=\,q(\Phi^{-1}U_{M}\Phi\alpha)\\ &=\,\det\,(\Phi\Phi^{-1}U_{M}\Phi\alpha)\\ &=\,\det\,(U_{M}\Phi\alpha)\\ &=\,\det\,(\Phi\alpha)\\ &=\,q(\alpha)\end{array}\]

and so \(T_{M}\ 