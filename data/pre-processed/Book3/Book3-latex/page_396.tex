

### 4.1 Sets

We shall use the words 'set,' 'class,' 'collection,' and 'family' interchangeably, although we give preference to 'set.' If \(S\) is a set and \(x\) is an object in the set \(S\), we shall say that \(x\) is a **member of \(S\)**, that \(x\) is an **element of \(S\)**, that \(x\)**belongs to \(S\)**, or simply that \(x\) is in \(S\). If \(S\) has only a finite number of members, \(x_{1}\), \(\ldots\), \(x_{n}\), we shall often describe \(S\) by displaying its members inside braces:

\[S\ =\ \{x_{1},\ \ldots\,,x_{n}\}.\]

Thus, the set \(S\) of positive integers from 1 through 5 would be

\[S\ =\ \{1,2,3,4,5\}.\]

If \(S\) and \(T\) are sets, we say that \(S\) is a **subset of \(T\)**, or that \(S\) is **contained in \(T\)**, if each member of \(S\) is a member of \(T\). Each set \(S\) is a subset of itself. If \(S\) is a subset of \(T\) but \(S\) and \(T\) are not identical, we call \(S\) a **proper subset** of \(T\). In other words, \(S\) is a proper subset of \(T\) provided that \(S\) is contained in \(T\) but \(T\) is not contained in \(S\).

If \(S\) and \(T\) are sets, the **union of \(S\) and \(T\)** is the set \(S\,\cup\,T\), consisting of all objects \(x\) which are members of either \(S\) or \(T\). The **intersection of \(S\) and \(T\)** is the set \(S\,\cap\,T\), consisting of all \(x\) which are members of both \(S\) and \(T\). For any two sets, \(S\) and \(T\), the intersection \(S\,\cap\,T\) is a subset of the union \(S\,\cup\,T\). This should help to clarify the use of the word 'or' which will prevail in this book. When we say that \(x\) is either in \(S\) or in \(T\), we do not preclude the possibility that \(x\) is in both \(S\) and \(T\).

In order that the intersection of \(S\) and \(T\) should always be a set, it is necessary that one introduce the **empty set,** i.e., the set with no members. Then \(S\,\cap\,T\) is the empty set if and only if \(S\) and \(T\) have no members in common.

We shall frequently need to discuss the union or intersection of several sets. If \(S_{1}\), \(\ldots\), \(S_{n}\) are sets, their **union** is the set \(\mathop{\cup}\limits_{j\,=\,1}^{n}\,S_{j}\) consisting of all \(x\) which are members of at least one of the sets \(S_{1}\), \(\ldots\), \(S_{n}\). Their **intersection** is the set \(\mathop{\cap}\limits_{j\,=\,1}^{n}\,S_{j}\), consisting of all \(x\) which are members of each of the sets \(S_{1}\), \(\ldots\), \(S_{n}\). On a few occasions, we shall discuss the union or intersection of an infinite collection of sets. It should be clear how such unions and intersections are defined. The following example should clarify these definitions and a notation for them.

**Example 1**: Let \(R\) denote the set of all real numbers (the real line). If \(t\) is in \(R\), we associate with \(t\) a subset \(S_{t}\) of \(R\), defined as follows: \(S_{t}\) consists of all real numbers \(x\) which are not less than \(t\).

