
Find the \(T\)-annihilator of the vector \((1,0,0)\). Find the \(T\)-annihilator of \((1,0,i)\).

**4.** Prove that if \(T^{\natural}\) has a cyclic vector, then \(T\) has a cyclic vector. Is the converse true?

**5.** Let \(V\) be an \(n\)-dimensional vector space over the field \(F\), and let \(N\) be a nilpotent linear operator on \(V\). Suppose \(N^{n-1}\neq 0\), and let \(\alpha\) be any vector in \(V\) such that \(N^{n-1}\alpha\neq 0\). Prove that \(\alpha\) is a cyclic vector for \(N\). What exactly is the matrix of \(N\) in the ordered basis \(\{\alpha,N\alpha,\ldots,N^{n-1}\alpha\}\)?

**6.** Give a direct proof that if \(A\) is the companion matrix of the monic polynomial \(p\), then \(p\) is the characteristic polynomial for \(A\).

**7.** Let \(V\) be an \(n\)-dimensional vector space, and let \(T\) be a linear operator on \(V\). Suppose that \(T\) is _diagonalizable_.

(a) If \(T\) has a cyclic vector, show that \(T\) has \(n\) distinct characteristic values.

(b) If \(T\) has \(n\) distinct characteristic values, and if \(\{\alpha_{1},\ldots,\alpha_{n}\}\) is a basis of characteristic vectors for \(T\), show that \(\alpha=\alpha_{1}+\cdots+\alpha_{n}\) is a cyclic vector for \(T\).

**8.** Let \(T\) be a linear operator on the finite-dimensional vector space \(V\). Suppose \(T\) has a cyclic vector. Prove that if \(U\) is any linear operator which commutes with \(T\), then \(U\) is a polynomial in \(T\).

### Cyclic Decompositions and the Rational Form

The primary purpose of this section is to prove that if \(T\) is any linear operator on a finite-dimensional space \(V\), then there exist vectors \(\alpha_{1}\), \(\ldots,\alpha_{r}\) in \(V\) such that

\[V=Z(\alpha_{1};T)\bigoplus\cdots\bigoplus Z(\alpha_{r};T).\]

In other words, we wish to prove that \(V\) is a direct sum of \(T\)-cyclic subspaces. This will show that \(T\) is the direct sum of a finite number of linear operators, each of which has a cyclic vector. The effect of this will be to reduce many questions about the general linear operator to similar questions about an operator which has a cyclic vector. The theorem which we prove (Theorem 3) is one of the deepest results in linear algebra and has many interesting corollaries.

The cyclic decomposition theorem is closely related to the following question. Which \(T\)-invariant subspaces \(W\) have the property that there exists a \(T\)-invariant subspace \(W^{\prime}\) such that \(V=W\bigoplus W^{\prime}\)? If \(W\) is any subspace of a finite-dimensional space \(V\), then there exists a subspace \(W^{\prime}\) such that \(V=W\bigoplus W^{\prime}\). Usually there are many such subspaces \(W^{\prime}\) and each of these is called **complementary** to \(W\). We are asking when a \(T\)-invariant subspace has a complementary subspace which is also invariant under \(T\).

Let us suppose that \(V=W\bigoplus W^{\prime}\) where both \(W\) and \(W^{\prime}\) are invariant under \(T\) and then see what we can discover about the subspace \(W\). Each 