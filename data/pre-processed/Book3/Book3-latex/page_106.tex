Sec. 3.5

(d) Prove that if \(M\) and \(N\) are \(n\times n\) matrices over \(F\) such that \(M^{n}=N^{n}=0\) but \(M^{n-1}\neq 0\neq N^{n-1}\), then \(M\) and \(N\) are similar.

**13.** Let \(V\) and \(W\) be finite-dimensional vector spaces over the field \(F\) and let \(T\) be a linear transformation from \(V\) into \(W\). If

\[\mathfrak{G}=\langle\alpha_{1},\,\ldots,\,\alpha_{n}\rangle\quad\mbox{and} \quad\mathfrak{G}^{\prime}=\langle\beta_{1},\,\ldots,\,\beta_{m}\rangle\]

are ordered bases for \(V\) and \(W\), respectively, define the linear transformations \(E^{p,q}\) as in the proof of Theorem 5: \(E^{p,q}(\alpha_{i})=\delta_{iq}\beta_{p}\). Then the \(E^{p,q}\), \(1\leq p\leq m\), \(1\leq q\leq n\), form a basis for \(L(V,\,W)\), and so

\[T=\sum_{p\,=\,1}^{m}\sum_{q\,=\,1}^{n}A_{pq}E^{p,q}\]

for certain scalars \(A_{pq}\) (the coordinates of \(T\) in this basis for \(L(V,\,W)\)). Show that the matrix \(A\) with entries \(A(p,q)=A_{pq}\) is precisely the matrix of \(T\) relative to the pair \(\mathfrak{G}\), \(\mathfrak{G}^{\prime}\).

_3.5. Linear Functionals_

If \(V\) is a vector space over the field \(F\), a linear transformation \(f\) from \(V\) into the scalar field \(F\) is also called a **linear functional** on \(V\). If we start from scratch, this means that \(f\) is a function from \(V\) into \(F\) such that

\[f(\alpha+\beta)=cf(\alpha)+f(\beta)\]

for all vectors \(\alpha\) and \(\beta\) in \(V\) and all scalars \(c\) in \(F\). The concept of linear functional is important in the study of finite-dimensional spaces because it helps to organize and clarify the discussion of subspaces, linear equations, and coordinates.

Example 18: Let \(F\) be a field and let \(a_{1},\,\ldots,\,a_{n}\) be scalars in \(F\). Define a function \(f\) on \(F^{n}\) by

\[f(x_{1},\,\ldots,\,x_{n})=a_{1}x_{1}+\,\cdots+a_{n}x_{n}.\]

Then \(f\) is a linear functional on \(F^{n}\). It is the linear functional which is represented by the matrix \([a_{1}\,\cdots\,a_{n}]\) relative to the standard ordered basis for \(F^{n}\) and the basis \(\langle 1\rangle\) for 