
See 7.5

_Summary; Semi-Simple Operators_ 269 shall pass from the scalar field \(F\) to the field of complex numbers. Let \(\mathfrak{G}\) be some ordered basis for the space \(V\). Then we have

\[[T]_{\mathfrak{G}}=[S]_{\mathfrak{G}}+[N]_{\mathfrak{G}}\]

while \([S]_{\mathfrak{G}}\) is diagonalizable over the complex numbers and \([N]_{\mathfrak{G}}\) is nilpotent. This diagonalizable matrix and nilpotent matrix which commute are uniquely determined, as we have shown in Chapter 6.

_Exercises_

**1.** If \(N\) is a nilpotent linear operator on \(V\), show that for any polynomial \(f\) the semi-simple part of \(f(N)\) is a scalar multiple of the identity operator (\(F\) a subfield of \(C\)).

**2.** Let \(F\) be a subfield of the complex numbers, \(V\) a finite-dimensional vector space over \(F\), and \(T\) a semi-simple linear operator on \(V\). If \(f\) is any polynomial over \(F\), prove that \(f(T)\) is semi-simple.

**3.** Let \(T\) be a linear operator on a finite-dimensional space over a subfield of \(C\). Prove that \(T\) is semi-simple if and only if the following is true: If \(f\) is a polynomial and \(f(T)\) is nilpotent, then \(f(T)=0\).

 