teristic and minimal polynomials, respectively. What all this tells us is the following. If we consider the relation of similarity on the set of \(3\times 3\) matrices over \(F\), the equivalence classes are in one-one correspondence with ordered pairs \((f,p)\) of monic polynomials over \(F\) which satisfy (a) and (b).

### The Axiom of Choice

Loosely speaking, the Axiom of Choice is a rule (or principle) of thinking which says that, given a family of non-empty sets, we can choose one element out of each set. To be more precise, suppose that we have an index set \(A\) and for each \(\alpha\) in \(A\) we have an associated set \(S_{\alpha}\), which is non-empty. To 'choose' one member of each \(S_{\alpha}\) means to give a rule \(f\) which associates with each \(\alpha\) some element \(f(\alpha)\) in the set \(S_{\alpha}\). The axiom of choice says that this is possible, i.e., given the family of sets \(\{S_{\alpha}\}\), there exists a function \(f\) from \(A\) into

\[\bigcup_{\alpha}S_{\alpha}\]

such that\(f(\alpha)\) is in \(S_{\alpha}\) for each \(\alpha\). This principle is accepted by most mathematicians, although many situations arise in which it is far from clear how any explicit function \(f\) can be found.

The Axiom of Choice has some startling consequences. Most of them have little or no bearing on the subject matter of this book; however, one consequence is worth mentioning: Every vector space has a basis. For example, the field of real numbers has a basis, as a vector space over the field of rational numbers. In other words, there is a subset \(S\) of \(R\) which is linearly independent over the field of rationals and has the property that each real number is a rational linear combination of some finite number of elements of \(S\). We shall not stop to derive this vector space result from the Axiom of Choice. For a proof, we refer the reader to the book by Kelley in the bibliography.

 