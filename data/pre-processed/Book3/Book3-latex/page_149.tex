

## Chapter 5 Determinants

### 5.1 Commutative Rings

In this chapter we shall prove the essential facts about determinants of square matrices. We shall do this not only for matrices over a field, but also for matrices with entries which are 'scalars' of a more general type. There are two reasons for this generality. First, at certain points in the next chapter, we shall find it necessary to deal with determinants of matrices with polynomial entries. Second, in the treatment of determinants which we present, one of the axioms for a field plays no role, namely, the axiom which guarantees a multiplicative inverse for each non-zero element. For these reasons, it is appropriate to develop the theory of determinants for matrices, the entries of which are elements from a commutative ring with identity.

_Definition._ \(A\) **ring** _is a set \(K\), together with two operations \((x,y)\to x+y\) and \((x,y)\to xy\) satisfying_

* \(K\) _is a commutative group under the operation_ \((x,y)\to x+y\) (\(K\) _is a commutative group under addition_)_;_
* \((x)yz=x(yz)\)__(multiplication is associative_)_;_
* \(x(y+z)=xy+xz;\ (y+z)x=yx+zx\)__(the two distributive laws hold)_._

_If \(xy=yx\) for all \(x\) and \(y\) in \(K\), we say that the ring \(K\) is_ **commutative.** _If there is an element \(1\) in \(K\) such that \(1x=x1=x\) for each \(x,\)\(K\) is said to be a_ **ring with identity,** _and \(1\) is called the_ **identity** _for \(K\)._