The second matrix fails to satisfy condition (a), because the leading non-zero entry of the first row is not 1. The first matrix does satisfy condition (a), but fails to satisfy condition (b) in column 3.

We shall now prove that we can pass from any given matrix to a row-reduced matrix, by means of a finite number of elementary row operations. In combination with Theorem 3, this will provide us with an effective tool for solving systems of linear equations.

Every \(m\times n\) matrix over the field \(F\) is row-equivalent to a row-reduced matrix.

Let \(A\) be an \(m\times n\) matrix over \(F\). If every entry in the first row of \(A\) is 0, then condition (a) is satisfied in so far as row 1 is concerned. If row 1 has a non-zero entry, let \(k\) be the smallest positive integer \(j\) for which \(A_{1j}\neq 0\). Multiply row 1 by \(A_{1k}^{-1}\), and then condition (a) is satisfied with regard to row 1. Now for each \(i\geq 2\), add \((-A_{ik})\) times row 1 to row \(i\). Now the leading non-zero entry of row 1 occurs in column \(k\), that entry is 1, and every other entry in column \(k\) is 0.

Now consider the matrix which has resulted from above. If every entry in row 2 is 0, we do nothing to row 2. If some entry in row 2 is different from 0, we multiply row 2 by a scalar so that the leading non-zero entry is 1. In the event that row 1 had a leading non-zero entry in column \(k\), this leading non-zero 