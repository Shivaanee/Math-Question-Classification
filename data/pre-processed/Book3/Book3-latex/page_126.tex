

## Chapter 4 Polynomials

The purpose of this chapter is to establish a few of the basic properties of the algebra of polynomials over a field. The discussion will be facilitated if we first introduce the concept of a linear algebra over a field.

_Definition._ Let \(F\) be a field. A **linear algebra over the field \(F\)** is a vector space \(\mathfrak{a}\) over \(F\) with an additional operation called **multiplication of vectors** which associates with each pair of vectors \(\alpha\), \(\beta\) in \(\mathfrak{a}\) a vector \(\alpha\) in \(\mathfrak{a}\) called the **product** of \(\alpha\) and \(\beta\) in such a way that

1. _multiplication is associative,_ \[\alpha(\beta\gamma)\,=\,(\alpha\beta)\gamma\]
2. _multiplication is distributive with respect to addition,_ \[\alpha(\beta\,+\,\gamma)\,=\,\alpha\beta\,+\,\alpha\gamma\quad\text{and} \quad(\alpha\,+\,\beta)\gamma\,=\,\alpha\gamma\,+\,\beta\gamma\]
3. _for each scalar_ \(c\) _in_ \(F\)_,_ \[c(\alpha\beta)\,=\,(c\alpha)\beta\,=\,\alpha(c\beta).\]

_If there is an element \(1\) in \(\mathfrak{a}\) such that \(1\alpha\,=\,\alpha 1\,=\,\alpha\,\)for each \(\alpha\) in \(\mathfrak{a}\), we call \(\mathfrak{a}\) a **linear algebra with identity over \(F\)**, and call \(1\) the **identity** of \(\alpha\). The algebra \(\mathfrak{a}\) is called **commutative** if \(\alpha\beta\,=\,\beta\alpha\) for all \(\alpha\) and \(\beta\) in \(\mathfrak{a}\)._

Example 1. The set of \(n\times n\) matrices over a field, with the usual operations, is a linear algebra with identity; in particular the field itself is an algebra with identity. This algebra is not commutative if \(n\geq 2\). The field itself is (of course) commutative.

