Thus \(D\) is represented by the same matrix in the ordered bases \(\otimes\) and \(\otimes^{\prime}\). Of course, one can see this somewhat more directly since

\[\eqalign{Dg_{1}&=\ 0\cr Dg_{2}&=\ g_{1}\cr Dg_{3}&=\ 2g_{2}\cr Dg_{4}&=\ 3g_{3}.}\]

This example illustrates a good point. If one knows the matrix of a linear operator in some ordered basis \(\otimes\) and wishes to find the matrix in another ordered basis \(\otimes^{\prime}\), it is often most convenient to perform the coordinate change using the invertible matrix \(P\); however, it may be a much simpler task to find the representing matrix by a direct appeal to its definition.

_Definition._Let \(A\) and \(B\) be \(n\times n\) (_square_) _matrices over the field \(F\). We say that \(B\) is_ similar to \(A\) over \(F\)_if there is an invertible \(n\times n\) matrix \(P\) over \(F\) such that \(B\,=\,P^{-1}AP\)._

According to Theorem 14, we have the following: If \(V\) is an \(n\)-dimensional vector space over \(F\) and \(\otimes\) and \(\otimes^{\prime}\) are two ordered bases for \(V\), then for each linear operator \(T\) on \(V\) the matrix \(B\,=\,[T]_{\otimes^{\prime}}\) is similar to the matrix \(A\,=\,[T]_{\otimes}\). The argument also goes in the other direction. Suppose \(A\,\) and \(B\) are \(n\times n\) matrices and that \(B\) is similar to 