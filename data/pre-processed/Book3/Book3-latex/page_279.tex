

## Chapter 8 Inner Product Spaces

### 8.1 Inner Products

Throughout this chapter we consider only real or complex vector spaces, that is, vector spaces over the field of real numbers or the field of complex numbers. Our main object is to study vector spaces in which it makes sense to speak of the 'length' of a vector and of the 'angle' between two vectors. We shall do this by studying a certain type of scalar-valued function on pairs of vectors, known as an inner product. One example of an inner product is the scalar or dot product of vectors in \(R^{3}\). The scalar product of the vectors

\[\alpha\,=\,(x_{1},x_{2},x_{3})\quad\text{and}\quad\beta\,=\,(y_{1},y_{2},y_{3})\]

in \(R^{3}\) is the real number

\[(\alpha|\beta)\,=\,x_{1}y_{1}+x_{2}y_{2}+x_{3}y_{3}.\]

Geometrically, this dot product is the product of the length of \(\alpha\), the length of \(\beta\), and the cosine of the angle between \(\alpha\) and \(\beta\). It is therefore possible to define the geometric concepts of 'length' and 'angle' in \(R^{3}\) by means of the algebraically defined scalar product.

An inner product on a vector space is a function with properties similar to the dot product in \(R^{3}\), and in terms of such an inner product one can also define 'length' and 'angle.' Our comments about the general notion of angle will be restricted to the concept of perpendicularity (or orthogonality) of vectors. In this first section we shall say what an inner product is, consider some particular examples, and establish a few basic