

**Corollary**.: _Let \(T\) be a linear operator on a finite-dimensional vector space \(V\)._

1. _There exists a vector_ \(\alpha\) _in_ \(V\) _such that the_ \(T\)_-annihilator of_ \(\alpha\) _is the minimal polynomial for_ \(T\)_._
2. \(T\) _has a cyclic vector if and only if the characteristic and minimal polynomials for_ \(T\) _are identical._

Proof.: If \(V=\{0\}\), the results are trivially true. If \(V\neq\{0\}\), let

(7-15) \[V=Z(\alpha_{i};\,T)\oplus\,\cdots\,\oplus\,Z(\alpha_{r};\,T)\]

where the \(T\)-annihilators \(p_{1}\), \(\ldots\), \(p\), are such that \(p_{k+1}\) divides \(p_{k}\), \(1\leq k\leq r\,-\,1\). As we noted in the proof of Theorem 3, it follows easily that \(p_{1}\) is the minimal polynomial for \(T\), i.e., the \(T\)-conductor of \(V\) into \(\{0\}\). We have proved (a).

We saw in Section 7.1 that, if \(T\) has a cyclic vector, the minimal polynomial for \(T\) coincides with the characteristic polynomial. The content; of (b) is in the converse. Choose any \(\alpha\) as in (a). If the degree of the minimal polynomial is dim \(V\), then \(V=Z(\alpha;\,T)\). 

**Theorem 4** (Generalized Cayley-Hamilton Theorem).: _Let \(T\) be a linear operator on a finite-dimensional vector space \(V\). Let \(p\) and \(f\) be the minimal and characteristic polynomials for \(T\), respectively._

1. \(p\) _divides_ \(f\)_._
2. \(p\) _and_ \(f\) _have the same prime factors, except for multiplicities._
3. _If_ (7-16) \[p=f_{1}^{*}\cdots\,f_{k}^{*}\] _is the prime factorization of_ \(p\)_, then_ (7-17) \[f=f_{1}^{*}\cdots\,f_{k}^{*_{k}}\] _where_ \(d_{i}\) _is the nullity of_ \(f_{i}(T)^{r_{i}}\) _divided by the degree of_ \(f_{i}\)_._

Proof.: We disregard the trivial case \(V=\{0\}\). To prove (i) and (ii), consider a cyclic decomposition (7-15) of \(V\) obtained from Theorem 3. As we noted in the proof of the second corollary, \(p_{1}=p\). Let \(U_{i}\) be the restriction of \(T\) to \(Z(\alpha_{i};\,T)\). Then \(U_{i}\) has a cyclic vector and so \(p_{i}\) is both the minimal polynomial and the characteristic polynomial for \(U_{i}\). Therefore, the characteristic polynomial \(f\) is the product \(f=p_{1}\cdots\,p_{r}\). That is evident from the block form (6-14) which the matrix of \(T\) assumes in a suitable basis. Clearly \(p_{1}=p\) divides \(f\), and this proves (i). Obviously any prime divisor of \(p\) is a prime divisor of \(f\). Conversely, a prime divisor of \(f=p_{1}\cdots\,p_{r}\) must divide one of the factors \(p_{i}\), which in turn divides \(p_{1}\).

Let (7-16) be the prime factorization of \(p\). We employ the primary decomposition theorem (Theorem 12 of Chapter 6). It tells us that, if \(V_{i}\) is the null space of \(f_{i}(T)^{r_{i}}\), then