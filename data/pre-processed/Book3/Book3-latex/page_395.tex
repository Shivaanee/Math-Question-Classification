

## Appendix

This Appendix separates logically into two parts. The first part, comprising the first three sections, contains certain fundamental concepts which occur throughout the book (indeed, throughout mathematics). It is more in the nature of an introduction for the book than an appendix. The second part is more genuinely an appendix to the text.

Section 1 contains a discussion of sets, their unions and intersections. Section 2 discusses the concept of function, and the related ideas of range, domain, inverse function, and the restriction of a function to a subset of its domain. Section 3 treats equivalence relations. The material in these three sections, especially that in Sections 1 and 2, is presented in a rather concise manner. It is treated more as an agreement upon terminology than as a detailed exposition. In a strict logical sense, this material constitutes a portion of the prerequisites for reading the book; however, the reader should not be discouraged if he does not completely grasp the significance of the ideas on his first reading. These ideas are important, but the reader who is not too familiar with them should find it easier to absorb them if he reviews the discussion from time to time while reading the text proper.

Sections 4 and 5 deal with equivalence relations in the context of linear algebra. Section 4 contains a brief discussion of quotient spaces. It can be read at any time after the first two or three chapters of the book. Section 5 takes a look at some of the equivalence relations which arise in the book, attempting to indicate how some of the results in the book might be interpreted from the point of view of equivalence relations. Section 6 describes the Axiom of choice and its implications for linear algebra.

