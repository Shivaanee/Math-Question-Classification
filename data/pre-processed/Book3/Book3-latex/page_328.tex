9. Operators on Inner Product Spaces

We regard most of the topics treated in Chapter 8 as fundamental, the material that everyone should know. The present chapter is for the more advanced student or for the reader who is eager to expand his knowledge concerning operators on inner product spaces. With the exception of the Principal Axis theorem, which is essentially just another formulation of Theorem 18 on the orthogonal diagonalization of self adjoint operators, and the other results on forms in Section 9.2, the material presented here is more sophisticated and generally more involved technically. We also make more demands of the reader, just as we did in the later parts of Chapters 5 and 7. The arguments and proofs are written in a more condensed style, and there are almost no examples to smooth the way; however, we have seen to it that the reader is well supplied with generous sets of exercises.

The first three sections are devoted to results concerning forms on inner product spaces and the relation between forms and linear operators. The next section deals with spectral theory, i.e., with the implications of Theorems 18 and 22 of Chapter 8 concerning the diagonalization of self-adjoint and normal operators. In the final section, we pursue the study of normal operators treating, in particular, the real case, and in so doing we examine what the primary decomposition theorem of Chapter 6 says about normal operators.

 