

**15.** A unitary operator on \(R^{4}\) (with the standard inner product) is simply a linear operator which preserves the quadratic form

\[||(x,y,z,t)||^{2}=x^{2}+y^{2}+z^{2}+t^{2}\]

that is, a linear operator \(U\) such that \(||U\alpha||^{2}=||\alpha||^{2}\) for all \(\alpha\) in \(R^{4}\). In a certain part of the theory of relativity, it is of interest to find the linear operators \(T\) which preserve the form

\[||(x,y,z,t)||^{2}_{L}=(^{2}-x^{2}-y^{2}-z^{2}.\]

Now \(||\ \ ||^{2}_{L}\) does not come from an inner product, but from something called the 'Lorentz metric' (which we shall not go into). For that reason, a linear operator \(T\) on \(R^{4}\) such that \(||T\alpha||^{2}_{L}=||\alpha||^{2}_{L}\), for every \(\alpha\) in \(R^{4}\), is called a **Lorentz transformation.**

(a) Show that the function \(U\) defined by

\[U(x,y,z,t)=\left[\matrix{t+x&y+iz\cr y-iz&t-x\cr}\right]\]

is an isomorphism of \(R^{4}\) onto the real vector space \(H\) of all self-adjoint \(2\times 2\) complex matrices.

(b) Show that \(||\alpha||^{2}_{L}=\det(U\alpha)\).

(c) Suppose \(T\) is a (real) linear operator on the space \(H\) of \(2\times 2\) self-adjoint matrices. Show that \(L=U^{-1}TU\) is a linear operator on \(R^{4}\).

(d) Let \(M\) be any \(2\times 2\) complex matrix. Show that \(T_{M}(A)=M^{*}AM\) defines a linear operator \(T_{M}\) on \(H\). (Be sure you check that \(T_{M}\) maps \(H\) into \(H\).)

(e) If \(M\) is a \(2\times 2\) matrix such that \(|\det M|=1\), show that \(L_{M}=U^{-1}T_{M}U\) is a Lorentz transformation on \(R^{4}\).

(f) Find a Lorentz transformation which is not an \(L_{M}\).

**8.5.**_Normal Operators_

The principal objective in this section is the solution of the following problem. If \(T\) is a linear operator on a finite-dimensional inner product space \(V\), under what conditions does \(V\) have an orthonormal basis consisting of characteristic vectors for \(T\)? In other words, when is there an _orthonormal_ basis \(\otimes\) for \(V\), such that the matrix of \(T\) in the basis \(\otimes\) is diagonal?

We shall begin by deriving some necessary conditions on \(T\), which we shall subsequently show are sufficient. Suppose \(\otimes=\{\alpha_{1},\ldots,\alpha_{n}\}\) is an orthonormal basis for \(V\) with the property

\[T\alpha_{j}=c_{j}\alpha_{j},\ \ \ \ \ j=1,\ldots,n.\]

This simply says that the matrix of \(T\) in the ordered basis \(\otimes\) is the diagonal matrix with diagonal entries \(c_{1},\ldots,c_{n}\). The adjoint operator \(T^{*}\) is represented in this same ordered basis by the conjugate transpose matrix, i.e., the diagonal matrix with diagonal entries \(\bar{c}_{1},\ldots,\bar{c}_{n}\). If \(V\) is a real inner