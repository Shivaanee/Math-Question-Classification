

## 10. Bilinear Forms

### 10.1. Bilinear Forms

In this chapter, we treat bilinear forms on finite-dimensional vector spaces. The reader will probably observe a similarity between some of the material and the discussion of determinants in Chapter 5 and of inner products and forms in Chapter 8 and in Chapter 9. The relation between bilinear forms and inner products is particularly strong; however, this chapter does not presuppose any of the material in Chapter 8 or Chapter 9. The reader who is not familiar with inner products would probably profit by reading the first part of Chapter 8 as he reads the discussion of bilinear forms.

This first section treats the space of bilinear forms on a vector space of dimension \(n\). The matrix of a bilinear form in an ordered basis is introduced, and the isomorphism between the space of forms and the space of \(n\times n\) matrices is established. The rank of a bilinear form is defined, and non-degenerate bilinear forms are introduced. The second section discusses symmetric bilinear forms and their diagonalization. The third section treats skew-symmetric bilinear forms. The fourth section discusses the group preserving a non-degenerate bilinear form, with special attention given to the orthogonal groups, the pseudo-orthogonal groups, and a particular pseudo-orthogonal group--the Lorentz group.

**Definition**.: _Let \(V\) be a vector space over the field \(F\). A_ **bilinear form** _on \(V\) is a function \(f\), which assigns to each ordered pair of vectors \(\alpha\), \(\beta\) in \(V\) a scalar \(f(\alpha,\beta)\) in \(F\), and which satisfies_