3. Find the Jordan form of \(A\). 4. Find a direct-sum decomposition of \(R^{*}\) into \(T\)-cyclic subspaces as in Theorem 3. (_Hint:_ One way to do this is to use the results in (b) and an appropriate generalization of the ideas discussed in Example 4.)

### \(Summary\); \(Semi\)-\(Simple\) Operators

In the last two chapters, we have been dealing with a single linear operator \(T\) on a finite-dimensional vector space \(V\). The program has been to decompose \(T\) into a direct sum of linear operators of an elementary nature, for the purpose of gaining detailed information about how \(T\) 'operates' on the space \(V\). Let us review briefly where we stand.

We began to study \(T\) by means of characteristic values and characteristic vectors. We introduced diagonalizable operators, the operators which can be completely described in terms of characteristic values and vectors. We then observed that \(T\) might not have a single characteristic vector. Even in the case of an algebraically closed scalar field, when every linear operator does have at least one characteristic vector, we noted that the characteristic vectors of \(T\) need not span the space.

We then proved the cyclic decomposition theorem, expressing any linear operator as the direct sum of operators with a cyclic vector, with no assumption about the scalar field. If \(U\) is a linear operator with a cyclic vector, there is a basis \(\langle\alpha_{1},\ldots,\alpha_{n}\rangle\) with

\[\begin{array}{l}U\alpha_{j}\ =\ \alpha_{j+1},\qquad j=1,\ldots,n-1\\ U\alpha_{n}\ = 