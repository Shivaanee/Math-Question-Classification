be a maximal linearly independent subset of \(\mathfrak{F}\), i.e., a basis for the subspace spanned by \(\mathfrak{F}\). If \(\alpha\) is a vector such that (b) holds for each \(T_{i}\), then (b) will hold for every operator which is a linear combination of \(T_{1}\), . . , \(T_{r}\).

By the lemma before Theorem 5 (this lemma for a single operator), we can find a vector \(\beta_{1}\) (not in \(W\)) and a scalar \(c_{1}\)such that \((T_{1}-c_{1}I)\beta_{1}\)is in \(W\). I.et \(V_{1}\)be the collection of all vectors \(\beta\) in \(V\) such that \((T_{1}-c_{1}I)\beta\)is in \(W\). Then \(V_{1}\)is a subspace of \(V\) which is properly larger than \(W\). Furthermore, \(V_{1}\)is invariant under \(\mathfrak{F}\), for this reason. If \(T\)commutes with \(T_{1}\), then

\[(T_{1}-c_{1}I)(T\beta)=T(T_{1}-c_{1}I)\beta.\]

If \(\beta\)is in \(V_{1}\), then \((T_{1}-c_{1}I)\beta\)is in \(W\). Since \(W\) is invariant under each \(T\)in \(\mathfrak{F}\), we have \(T(T_{1}-c_{1}I)\beta\)in \(W\), i.e., \(T\beta\)in \(V_{1}\), for all \(\beta\)in \(V_{1}\) and all \(T\)in \(\mathfrak{F}\).

Now \(W\)is a proper subspace of \(V_{1}\). Let \(U_{2}\)be the linear operator on \(V_{1}\) obtained by restricting \(T_{2}\) to the subspace \(V_{1}\). The minimal polynomial for \(U_{2}\) divides the minimal polynomial for \(T_{2}\). Therefore, we may apply the lemma before Theorem 5 to that operator and the invariant subspace \(W\). We obtain a vector \(\boldsymbol{\mu}_{2}\)in \(V_{1}\) (not in \(W\)) and a scalar \(c_{2}\)such that \((T_{2}-c_{2}I)\beta_{2}\)is in \(W\). Note that

1. \(\beta_{2}\)is not in \(W\);
2. \((T_{1}-c_{1}I)\beta_{2}\)is in \(W\);
3. \((T_{2}-c_{2}I)\beta_{2}\)is in \(W\).

Let \(V_{2}\)be the set of all vectors \(\beta\)in \(V_{1}\)such that \((T_{2}-c_{2 