The following are easy consequences of the definition.

1. Any set which contains a linearly dependent set is linearly dependent.

2. Any subset of a linearly independent set is linearly independent.

3. Any set which contains the 0 vector is linearly dependent; for 1 \(\cdot\) 0 = 0.

4. A set \(S\) of vectors is linearly independent if and only if each finite subset of \(S\) is linearly independent, i.e., if and only if for any distinct vectors \(\alpha_{1}\), \(\cdot\) . , \(\alpha_{n}\) of \(S\), \(c_{1}\alpha_{1}\) + \(\cdot\) \(\cdot\) + \(c_{n}\alpha_{n}\) = 0 implies each \(c_{i}\) = 0.

_Definition. Let V be a vector space. A_ **basis** _for V is a linearly independent set of vectors in_ **V** _which spans the space V. The space V is_ **finite-dimensional** _if it has a finite basis._

**Example 12**.: Let \(F\) be a subfield of the complex numbers. In \(F^{z}\) the vectors

\[\begin{array}{l}\alpha_{1}\ =\ (\ \ \ 3,\ 0,\ -3)\\ \alpha_{2}\ =\ (\,-1,\ 1,\ \ \ 2)\\ \alpha_{3}\ =\ (\ \ \ 4,\ 2,\ -2)\\ \alpha_{4}\ =\ (\ \ \ 2,\ 1,\ \ \ \ 1)\end{array}\]

are linearly dependent, since

\[2\alpha_{1}\ +\ 2\alpha_{2}\ -\ \alpha_{3}\ +\ 0\ \cdot\ \alpha_{4}\ =\ 0.\]

The vectors

\[\begin{array}{l}\epsilon_{1}\ =\ (1,\ 0,\ 0)\\ \epsilon_{2}\ =\ (0,\ 1,\ 0)\\ \epsilon_{3}\ =\ (0,\ 0,\ 1)\end{array}\]

are linearly independent

**Example 13**.: Let \(F\) be a field and in \(F^{\ast}\) let \(S\) be the subset consisting of the vectors \(\epsilon_{1}\), \(\epsilon_{2}\), \(\cdot\) . , \(\epsilon_{n}\) defined by

\[\begin{array}{l}\epsilon_{1}\ =\ (1,\ 0,\ 0,\ \cdot\ \cdot\ \ \cdot,\ 0)\\ \epsilon_{2}\ =\ (0,\ 1,\ 0,\ \cdot\ \cdot\ \ \cdot,\ 0)\\ \cdot\ \cdot\ \cdot\ \cdot\ \cdot\ \ 