Proof.: Since the operations (4-1) and (4-2) are those defined in the algebra \(F^{\infty}\) and since \(F[x]\) is a subspace of \(F^{\infty}\), it suffices to prove that the product of two polynomials is again a polynomial. This is trivial when one of the factors is \(0\) and otherwise follows from (i).

_Corollary 2_.: Suppose f, g, and h are polynomials over the field F such that f \(\neq 0\) and fg = fh. Then g = h.

Proof.: Since \(fg=fh\), \(f(g-h)=0\), and as \(f\neq 0\) it follows at once from (i) that \(g-h=0\).

Certain additional facts follow rather easily from the proof of Theorem 1, and we shall mention some of these.

Suppose

\[f=\sum\limits_{i=0}^{m}f_{i}x^{i}\quad\text{and}\quad g=\sum\limits_{j=0}^{n}g_ {j}x^{j}.\]

Then from (4-7) we obtain,

(4-8) \[fg=\sum\limits_{s=0}^{m+n}\binom{s}{\sum\limits_{r=0}^{n}f_{r}g_{s-r}}x^{s}.\]

The reader should verify, in the special case \(f=cx^{m}\), \(g=dx^{n}\) with \(c\), \(d\) in \(F\), that (4-8) reduces to

(4-9) \[(cx^{m})(dx^{n})=cdx^{m+n}.\]

Now from (4-9) and the distributive laws in \(F[x]\), it follows that the product in (4-8) is also given by

(4-10) \[\sum\limits_{i,j}f_{i}g_{j}x^{i+j}\]

where the sum is extended over all integer pairs \(i,j\) such that \(0\leq i\leq m\), and \(0\leq j\leq n\).

_Definition._ Let G be a linear algebra with identity over the field F. We shall denote the identity of \(\alpha\) by \(1\) and make the convention that \(\alpha^{0}=1\) for each \(\alpha\) in G. Then to each polynomial f \(=\sum\limits_{i=0}^{n}f_{i}x^{i}\) over F and \(\alpha\) in G we associate an element f(\(\alpha\)) in G by the rule

\[f(\alpha)=\sum\limits_{i=0}^{n}f_{i}\alpha^{i}.\]

_Example 3_.: Let \(C\) be the field of complex numbers and let \(f=x^{2}+2\).

1. If \(\alpha 