Thus, \(W\) cannot have a complementary \(T\)-invariant subspace, contradicting the hypothesis that \(T\) is semi-simple.

Now suppose the prime factorization of \(p\) is \(p=p_{1}\cdots p_{k}\), where \(p_{1},\ldots,p_{k}\) are distinct irreducible (non-scalar) monic polynomials. Let \(W\) be a subspace of \(V\) which is invariant under \(T\). We shall prove that \(W\) has a complementary \(T\)-invariant subspace. Let \(V=W_{1}\bigoplus\cdots\bigoplus W_{k}\) be the primary decomposition for \(T\), i.e., let \(W_{j}\) be the null space of \(p_{j}(T)\). Let \(T_{j}\) be the linear operator induced on \(W_{j}\) by \(T\), so that the minimal polynomial for \(T_{j}\) is the prime \(p_{j}\). Now \(W\)\(\cap\)\(W_{j}\) is a subspace of \(W_{j}\) which is invariant under \(T_{j}\) (or under \(T\)). By the last lemma, there is a subspace \(V_{j}\) of \(W_{j}\) such that \(W_{j}=(W\cap W_{j})\bigoplus V_{j}\) and \(V_{j}\) is invariant under \(T_{j}\) (and hence under \(T\)). Then we have

\[\begin{array}{l}V=W_{1}\bigoplus\cdots\bigoplus W_{k}\\ =(W\cap W_{1})\bigoplus V_{1}\bigoplus\cdots\bigoplus(W\cap W_{k})\bigoplus V _{k}\\ =(W\cap W_{1})+\cdots+(W\cap W_{k})\bigoplus V_{1}\bigoplus\cdots\bigoplus V _{k}.\end{array}\]

By the first lemma above, \(W=(W\cap W_{1})\bigoplus\cdots\bigoplus(W\cap W_{k})\), so that if \(W^{\prime}=V_{1}\bigoplus\cdots\bigoplus V_{k}\ 