

**Example 5**: _The field \(C\) of complex numbers may be regarded as a vector space over the field \(R\) of real numbers. More generally, let \(F\) be the field of real numbers and let \(V\) be the set of \(n\)-tuples \(\alpha=(x_{1},\ldots,x_{n})\) where \(x_{1},\ldots,x_{n}\) are complex numbers. Define addition of vectors and scalar multiplication by (2-1) and (2-2), as in Example 1. In this way we obtain a vector space over the field \(R\) which is quite different from the space \(C^{n}\) and the space \(R^{n}\)._

There are a few simple facts which follow almost immediately from the definition of a vector space, and we proceed to derive these. If \(c\) is a scalar and \(0\) is the zero vector, then by 3(c) and 4(c)

\[c0=c(0+0)=c0+c0.\]

Adding \(-(c0)\) and using 3(d), we obtain

\[c0=0.\]

Similarly, for the scalar \(0\) and any vector \(\alpha\) we find that

\[0\alpha=0.\]

If \(c\) is a non-zero scalar and \(\alpha\) is a vector such that \(c\alpha=0\), then by (2-8), \(c^{-1}(c\alpha)=0\). But

\[c^{-1}(c\alpha)=(c^{-1}c)\alpha=1\alpha=\alpha\]

hence, \(\alpha=0\). Thus we see that if \(c\) is a scalar and \(\alpha\) a vector such that