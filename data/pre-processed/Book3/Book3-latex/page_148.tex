These polynomials \(p_{i}\) have real coefficients. We conclude that every irreducible polynomial over the real number field has degree 1 or 2. Each polynomial over \(R\) is the product of certain linear factors, obtained from the real roots of \(f\), and certain irreducible quadratic polynomials.

### Exercises

**1.** Let \(p\) be a monic polynomial over the field \(F\), and let \(f\) and \(g\) be relatively prime polynomials over \(F\). Prove that the g.e.d. of \(pf\) and \(pg\) is \(p\).

**2.** Assuming the Fundamental Theorem of Algebra, prove the following. If \(f\) and \(g\) are polynomials over the field of complex numbers, then g.e.d. \((f,g)=1\) if and only if \(f\) and \(g\) have no common root.

**3.** Let \(D\) be the differentiation operator on the space of polynomials over the field of complex numbers. Let \(f\) be a monic polynomial over the field of complex numbers. Prove that

\[f=(x-c_{k})\,\cdots\,(x-c_{k})\]

where \(c_{k}\), ..., \(c_{k}\) are _distinct_ complex numbers if and only if \(f\) and \(Df\) are relatively prime. In other words, \(f\) has no repeated root if and only if \(f\) and \(Df\) have no common root. (Assume the Fundamental Theorem of Algebra.)

**4.** Prove the following generalization of Taylor's formula. Let \(f\), \(g\), and \(h\) be polynomials over a subfield of the complex numbers, with \(\deg f\leq n\). Then

\[f(g)\,=\,\sum\limits_{h\,=\,0}^{n}\frac{1}{k!}f^{(h)}(h)(g-h)^{k}.\]

(Here \(f(g)\) denotes '\(f\) of \(g\).')

For the remaining exercises, we shall need the following definition. If \(f\), \(g\), and \(p\) are polynomials over the field \(F\) with \(p\neq 0\), we say that \(f\) is **congruent to \(g\) modulo \(p\)** if \((f-g)\) is divisible by \(p\). If \(f\) is congruent to \(g\) modulo \(p\), we write

\[f\equiv g\bmod p.\]

**5.** Prove, for any non-zero polynomial \(p\), that congruence modulo \(p\) is an equivalence relation.

1. It is reflexive; \(f\equiv f\bmod p\).
2. It is symmetric: if \(f\equiv g\bmod p\), then \(g\equiv f\bmod p\).
3. It is transitive: if \(f\equiv g\bmod p\) and \(g\equiv h\bmod p\), then \(f\equiv h\bmod p\).

**6.** Suppose \(f\equiv g\bmod p\) and \(f_{1}\equiv g_{1}\bmod p\).

1. Prove that \(f+f_{1}=g+g_{1}\bmod p\).
2. Prove that \(f\!f_{1}=gg_{1}\bmod p\).

**7.** Use Exercise 7 to prove the following. If \(f\), \(g\), \(h\), and \(p\) are polynomials over the field \(F\) and \(p\neq 0\), and if \(f\equiv g\bmod p\), then \(h(f)\equiv h(g)\bmod p\).

**8.** If \(p\) is an irreducible polynomial and \(fg\equiv 0\bmod p\), prove that either \(f\equiv 0\bmod p\) or \(g\equiv 0\bmod p\). Give an example which shows that: this is false if \(p\) is not irreducible.

 