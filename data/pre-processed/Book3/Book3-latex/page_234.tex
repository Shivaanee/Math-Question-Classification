Thus \(f\) satisfies (6-19) if and only if \(f\) has the form

\[f(t)\,=\,e^{et}(b_{0}\,+\,b_{1}t\,+\,\cdots\,+\,b_{r-1}t^{r-1}).\]

Accordingly, the 'functions' \(e^{et}\), \(te^{et}\), \(\ldots\), \(t^{r-1}e^{et}\) span the space of solutions of (6-19). Since \(1\), \(t\), \(\ldots\), \(t^{r-1}\) are linearly independent functions and the exponential function has no zeros, these \(r\) functions \(t^{ie^{t}}\), \(0\leq j\leq r-1\), form a basis for the space of solutions.

Returning to the differential equation (6-18), which is

\[p(\,\hbox{\vrule height 6.5pt width 0.4pt depth 0.0pt\kern-3.0ptD})f\,=\,0\] \[p\,=\,(x\,-\,c_{1})^{r_{1}}\,\cdots\,(x\,-\,c_{k})^{r_{k}}\]

we see that the \(n\) functions \(t^{n}e^{et}\), \(0\leq m\leq r_{j}-1\), \(1\leq j\leq k\), form a basis for the space of solutions to (6-18). In particular, the space of solutions is finite-dimensional and has dimension 