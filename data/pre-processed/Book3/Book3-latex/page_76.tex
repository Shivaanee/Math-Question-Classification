

## Chapter 3 Linear Transformations

### 3.1 Linear Transformations

We shall now introduce linear transformations, the objects which we shall study in most of the remainder of this book. The reader may find it helpful to read (or reread) the discussion of functions in the Appendix, since we shall freely use the terminology of that discussion.

_Definition. Let V and W be vector spaces over the field F. A_ **linear transformation from V into W_ is a function T from V into W such that

\[\mbox{T}(\mbox{c}\alpha+\beta)\,=\,\mbox{c}(\mbox{T}\alpha)\,+\,\mbox{T}\beta\]

for all a and \(\beta\) in V and all scalars c in F.

**Example 1**: _If \(V\) is any vector space, the identity transformation \(I\), defined by \(I\alpha=\alpha\), is a linear transformation from \(V\) into \(V\). The_ **zero transformation 0**_, defined by \(0\alpha=0\), is a linear transformation from \(V\) into \(V\)._

**Example 2**: _Let \(F\) be a field and let \(V\) be the space of polynomial functions \(f\) from \(F\) into \(F\), given by_

\[f(x)\,=\,c_{0}+\,c_{1}x\,+\,\cdots\,+\,c_{k}x^{k}.\]

_Let_

\[(Df)(x)\,=\,c_{1}+\,2c_{2}x\,+\,\cdots\,+\,kc_{k}x^{k-1}.\]