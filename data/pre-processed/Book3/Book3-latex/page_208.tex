characterizations of diagonalizable (and triangulable) operators in terms of their minimal polynomials.

**Definition**: Let \(\mathrm{V}\) be a vector space and \(\mathrm{T}\) a linear operator on \(\mathrm{V}\). If \(\mathrm{W}\) is a subspace of \(\mathrm{V}\), we say that \(\mathrm{W}\) is **invariant under \(\mathrm{T}\)** if for each vector \(\alpha\) in \(\mathrm{W}\) the vector \(\mathrm{T}\alpha\) is in \(\mathrm{W}\), i.e., if \(\mathrm{T}(\mathrm{W})\) is contained in \(\mathrm{W}\).

If \(T\) is any linear operator on \(V\), then \(V\) is invariant under \(T\), as is the zero subspace. The range of \(T\) and the null space of \(T\) are also invariant under \(T\).

Let \(F\) be a field and let \(D\) be the differentiation operator on the space \(F[x]\) of polynomials over \(F\). Let \(n\) be a positive integer and let \(W\) be the subspace of polynomials of degree not greater than \(n\). Then \(W\) is invariant under \(D\). This is just another way of saying that \(D\) is 'degree decreasing.'

Here is a very useful generalization of Example 6. Let \(T\) be a linear operator on \(V\). Let \(U\) be any linear operator on \(V\) which commutes with \(T\), i.e., \(TU=UT\). Let \(W\) be the range of \(U\) and let \(N\) be the null space of \(U\). Both \(W\) and \(N\) are invariant under \(T\). If \(\alpha\) is in the range of \(U\), say \(\alpha=U\beta\), then \(T\alpha=T(U\beta)=U(T\beta)\) so that \(T\alpha\) is in the range of \(U\). If \(\alpha\) is in \(N\), then \(U(T\alpha)=T(U\alpha)=T(0)=0\); hence, \(T\alpha\) is in \(N\).

A particular type of operator which commutes with \(T\) is an operator \(U=g(T)\), where \(g\) is a polynomial. For instance, we might have \(U=T-cI\), where \(c\) is a characteristic value of \(T\). The null space of \(U\) is familiar to us. We see that this example 