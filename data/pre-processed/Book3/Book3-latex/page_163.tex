In other words, \(\sigma\tau\) is an even permutation if \(\sigma\) and \(\tau\) are either both even or both odd, while \(\sigma\tau\) is odd if one of the two permutations is odd and the other is even. One can see this from the definition of the sign in terms of successive interchanges of pairs \((i,j)\). It may also be instructive if we point out how \(\operatorname{sgn}\,\,(\sigma\tau)=(\operatorname{sgn}\,\sigma)( \operatorname{sgn}\,\tau)\) follows from a fundamental property of determinants.

Let \(K\) be the ring of integers and let \(\sigma\) and \(\tau\) be permutations of degree \(n\). Let \(\epsilon_{1}\), \(\ldots\), \(\epsilon_{n}\) be the rows of the \(n\)\(\times\)\(n\) identity matrix over \(K\), let \(A\) be the matrix with rows \(\epsilon_{n}\), \(\ldots\), \(\epsilon_{n}\), and let \(B\) be the matrix with rows \(\epsilon_{n1}\), \(\ldots\), \(\epsilon_{n}\). The \(i\)th row of \(A\) contains exactly one non-zero entry, namely the \(1\) in column \(\tau\)i. From this it is easy to see that \(\epsilon_{\tau\tau i}\) is the \(i\)th row of the product matrix \(AB\). Now

\[\det\,(A)\,=\,\operatorname{sgn}\,\tau,\qquad\det\,(B)\,=\,\operatorname{sgn} \,\sigma,\qquad\text{and}\quad\det\,(AB)\,=\,\operatorname{sgn}\,\,(\sigma\tau).\]

So we shall have \(\operatorname{sgn}\,\,(\sigma\tau)=(\operatorname{sgn}\,\sigma)( \operatorname{sgn}\,\tau)\) as soon as we prove the following.

**Theorem 3**: _Let \(K\) be a commutative ring with identity, and let \(A\) and \(B\) be \(n\)\(\times\)\(n\) matrices over \(K\). Then_

\[\det\,(AB)\,=\,(\det\,A)(\det\,B).\]

Let \(B\) be a fixed \(n\)\(\times\)\(n\) matrix over \(K\), and for each \(n\)\(\times\)\(n\) matrix \(A\) define \(D(A)\,=\,\det(AB)\). If we denote the rows of \(A\) by \(\alpha_{1}

 