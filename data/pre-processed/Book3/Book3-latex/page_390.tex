noted \(O(n,R)\) and \(O(n,C)\). Of course, the orthogonal group is also the group which preserves the quadratic form

\[q(x_{1},\ .\ .\ .\ ,\ x_{n})\,=\,x_{1}^{2}\,+\ .\ .\ +\ x_{n}^{2}.\]

Let \(f\) be the symmetrie bilinear form on \(R^{n}\) with quadratic form

\[q(x_{1},\ .\ .\ ,\ .\ ,\ x_{n})\,=\,\sum\limits_{j\,=\,1}^{p}\,x_{j}^{2}\,-\, \sum\limits_{j\,=\,p\,+1}^{n}\,x_{j}^{2}.\]

Then \(f\) is non-degenerate and has signature \(2p-n\). The group of matrices preserving a form of this type is called a **pseudo-orthogonal group**. When \(p=n\), we obtain the orthogonal group \(O(n,R)\) as a particular type of pseudo-orthogonal group. For each of the \(n\,+\,1\) values \(p\,=\,0,\,1,\,2,\ .\ .\ .\ ,\ .

 