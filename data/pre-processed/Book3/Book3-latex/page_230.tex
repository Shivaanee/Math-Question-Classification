For each \(i\), let

\[f_{i}=\frac{p}{p_{i}^{r_{i}}}=\mathop{\Pi}\limits_{j\neq i}\ p_{j}^{r_{i}}.\]

Since \(p_{1}\), \(\ldots\), \(p_{k}\) are distinct prime polynomials, the polynomials \(f_{1}\), \(\ldots\), \(f_{k}\) are relatively prime (Theorem 10, Chapter 4). Thus there are polynomials \(g_{1}\), \(\ldots\), \(g_{k}\) such that

\[\mathop{\Sigma}\limits_{i=1}^{n}f_{i}g_{i}=1.\]

Note also that if \(i\neq j\), then \(f\!\!\!f_{j}\) is divisible by the polynomial \(p\), because \(f\!\!\!f_{j}\) contains each \(p_{m}^{r_{i}}\) as a factor. We shall show that the polynomials \(h_{i}=f\!\!\!g_{i}\) behave in the manner described in the first paragraph of the proof.

Let \(E_{i}=h_{i}(T)=f_{i}(T)g_{i}(T)\). Since \(h_{1}+\cdots+h_{k}=1\) and \(p\) divides \(f_{i}f_{j}\) for \(i\neq j\), we have

\[\begin{array}{l}E_{1}+\cdots+E_{k}=I\\ E_{i}E_{j}=0,\qquad\mbox{if}\ \ i\neq j.\end{array}\]

Thus the \(E_{i}\) are projections which correspond to some direct-sum decomposition of the space \(V\). We wish to show that the range of \(E_{i}\) is exactly the subspace \(W_{i}\). It is clear that each vector in the range of \(E_{i}\) is in \(W_{i}\), for if \(\alpha\) is in the range of \(E_{i}\), then \(\alpha=E_{i}\alpha\) and so

\[\begin{array}{l}p_{i}(T)^{r_{i}}\alpha=p_{i}(T)^{r_{i}}E_{i}\alpha\\ =p_{i}(T)^{r_{i}}f_{i}(T)g_{i}(T)\alpha\\ =0\end{array 