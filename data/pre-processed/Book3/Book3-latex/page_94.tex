from \(V\) into \(F^{n}\), as follows: If \(\alpha\) is in \(V\), let \(T\alpha\) be the \(n\)-tuple \((x_{1},\ .\ .\ .\ ,\ x_{n})\) of coordinates of \(\alpha\) relative to the ordered basis \(\mathfrak{d}\), i.e., the \(n\)-tuple such that

\[\alpha\,=\,x_{1}\alpha_{1}+\ .\ .\ .\ +\ x_{n}\alpha_{n}.\]

In our discussion of coordinates in Chapter 2, we verified that this \(T\) is linear, one-one, and maps \(V\) onto \(F^{n}\).

For many purposes one often regards isomorphic vector spaces as being 'the same,' although the vectors and operations in the spaces may be quite different, that is, one often identifies isomorphic spaces. We shall not attempt a lengthy discussion of this idea at present but shall let the understanding of isomorphism and the sense in which isomorphic spaces are 'the same' grow as we continue our study of vector spaces.

We shall make a few brief comments. Suppose \(T\) is an isomorphism of \(V\) onto \(W\). If \(S\) is a subset of \(V\), then Theorem 8 tells us that \(S\) is linearly independent if and only if the set \(T(S)\) in \(W\) is independent. Thus in deciding whether \(S\) is independent it doesn't matter whether we look at \(S\) or \(T(S)\). From this one sees that an isomorphism is 'dimension preserving,' that is, any finite-dimensional subspace of \(V\) has the same dimension as its image under \(T\). Here is a very simple illustration of this idea. Suppose \(A\) is an \(m\times n\) matrix over the field \(F\). We have really given two definitions of the solution space of the matrix \(A\). The first is the set of all \(n\)-tuples \((x_{1},\ .\ .\ .\ ,x_{n})\) in \(F^{n}\) which satisfy each of the equations in the system \(AX=0\). The second is the set of all \(n\times 1\) column matrices \(X\) such that \(AX=0\). The first solution space is thus a subspace of \(F^{n}\) and the second is a subspace of the space of all \(n\times 1\) matrices over \(F\). Now there is a completely obvious isomorphism between \(F^{n}\) and \(F^{n\times 1}\), namely,

\[(x_{1},\ .\ .\ .\ ,x_{n})\to\left[\begin{array}{c}x_{1}\\ \vdots\\ x_{n}\end{array}\right].\]

Under this isomorphism, the first solution space of \(A\) is carried onto the second solution space. These spaces have the same dimension, and so if we want to prove a theorem about the dimension of the solution space, it is immaterial which space we choose to discuss. In fact, the reader would probably not balk if we chose to identify \(F^{n}\) and the space of \(n\times 1\) matrices. We may do this when it is convenient, and when it is not convenient we shall not.

### 1.

Let \(V\) be the set of complex numbers and let \(F\) be the field of real numbers. With the usual operations, \(V\) is a vector space over \(F\).

Describe explicitly an isomorphism of this space onto \(R^{2}\).

 