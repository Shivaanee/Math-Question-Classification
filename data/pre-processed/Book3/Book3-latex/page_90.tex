function in the range of \(TU\) is in the range of \(T\), which is the space of polynomial functions\(f\) such that\(f(0)=0\).

**Example 12**: _Let \(F\) be a field and let \(T\) be the linear operator on \(F^{\sharp}\) defined by_

\[T(x_{1},x_{2})=(x_{1}+x_{2},x_{1}).\]

_Then \(T\) is non-singular, because if \(T(x_{1},x_{2})=0\) we have_

\[\begin{array}{c}x_{1}+x_{2}=0\\ x_{1}=0\end{array}\]

_so that \(x_{1}=x_{2}=0\). We also see that \(T\) is onto; for, let \((z_{1},z_{2})\) be any vector in \(F^{2}\). To show that \((z_{1},z_{2})\) is in the range of \(T\) we must find scalars \(x_{1}\) and \(x_{2}\) such that_

\[\begin{array}{c}x_{1}+x_{2}=z_{1}\\ x_{1}=z_{2 