1. \(R\) _is row-reduced;_ 2. _every row of_ \(R\) _which has all its entries_ \(0\) _occurs below every row which has a non-zero entry;_ 3. _if rows_ \(1\)_,_ \(\ldots\)_,_ \(r\) _are the non-zero rows of_ \(R\)_, and if the leading non-zero entry of_ \(raw\) _i occurs in column_ \(k_{i}\), \(i=1,\ldots,r\), _then_ \(k_{1}<k_{2}<\cdots<k_{r}\).

One can also describe an \(m\times n\) row-reduced echelon matrix \(R\) as follows. Either every entry in \(R\) is \(0\), or there exists a positive integer \(r\), \(1\leq r\leq m\), and \(r\) positive integers \(k_{1}\), \(\ldots\), \(k_{r}\) with \(1\leq k_{i}\leq n\) and

1. \(R_{ij}=0\) for \(i>r\), and \(R_{ij}=0\) if \(j<k_{i}\).
2. \(R_{ik}=\delta_{ij}\), \(1\leq i\leq r\), \(1\leq j\leq r\).
3. \(k_{1}<\cdots 