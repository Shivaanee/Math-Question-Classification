polynomial. We shall not prove this now. But, we want to emphasize the fact that the knowledge that the characteristic polynomial has the form (6-7) tells us that the minimal polynomial has the form (6-8), and it tells us nothing else about \(p\).

**Example 5**.: Let \(A\) be the \(4\times 4\) (rational) matrix

\[A\,=\begin{bmatrix}0&1&0&1\\ 1&0&1&0\\ 0&1&0&1\\ 1&0&1&0\end{bmatrix}.\]

The powers of \(A\) are easy to compute:

\[A^{\,\tt 2}=\begin{bmatrix}2&0&2&0\\ 0&2&0&2\\ 2&0&2&0\\ 0&2&0&2\end{bmatrix}\]

\[A^{\,\tt 3}=\begin{bmatrix}0&4&0&4\\ 4&0&4&0\\ 0&4&0&4\\ 4&0&4&0\end{bmatrix}.\]

Thus \(A^{\,\tt 3}=4A\), i.e., if \(p=x^{\tt 2}-4x=x(x+2)(x-2)\), then \(p(A)=0\). The minimal polynomial for \(A\) must divide \(p\). That minimal polynomial is obviously not of degree \(1\), since that would mean that \(A\) was a scalar multiple of the identity. Hence, the candidates for the minimal polynomial are: \(p\), \(x(x\,+\,2)\), \(x(x\,-\,2)\), \(x^{\,\tt 2}\,-\,4\). The three quadratic polynomials can be eliminated because it is obvious at a glance 