

**Theorem 6**: _Let \(\mathrm{V}\) be a finite-dimensional vector space over the field \(\mathrm{F}\) and let \(\mathrm{T}\) be a linear operator on \(\mathrm{V}.\) Then \(\mathrm{T}\) is diagonalizable if and only if the minimal polynomial for \(\mathrm{T}\) has the form_

\[p\,=\,(x\,-\,c_{\mathrm{i}})\,\cdots\,(x\,-\,c_{\mathrm{k}})\]

_where \(c_{\mathrm{i}},\,\ldots,\,c_{\mathrm{k}}\) are distinct elements of \(\mathrm{F}.\)_

We have noted earlier that, if \(T\) is diagonalizable, its minimal polynomial is a product of distinct linear factors (see the discussion prior to Example 4). To prove the converse, let \(W\) be the subspace spanned by all of the characteristic vectors of \(T,\) and suppose \(W\,\neq\,V.\) By the lemma used in the proof of Theorem 5, there is a vector \(\alpha\) not in \(W\) and a characteristic value \(c_{i}\) of \(T\) such that the vector

\[\beta\,=\,(T\,-\,c_{i}I)\alpha\]

lies in \(W.\) Since \(\beta\) is in \(W,\)

\[\beta\,=\,\beta_{1}\,+\,\cdots\,+\,\beta_{\mathrm{k}}\]

where \(T\beta_{i}\,=\,c_{i}\beta_{i},\,1\,\leq i\,\leq\,k,\) and therefore the vector

\[h(T)\beta\,=\,h(c_{