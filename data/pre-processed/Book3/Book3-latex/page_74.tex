\[\begin{bmatrix}b_{1}\\ b_{3}\\ b_{b_{4}}\end{bmatrix}.\]

(e) There are many ways to write the vectors in \(W\) as linear combinations of the rows of \(A\). Perhaps the easiest method is to follow the first procedure indicated before Example 21:

\[\begin{array}{l}\beta\ =\ (b_{1},\,2b_{1},\,b_{3},\,3b_{1}+4b_{3},\,b_{8})\\ \ =\ [b_{1},\,b_{3},\,b_{5},\,0,\,0]\ \cdot\,R\\ \ =\ [b_{1},\,b_{3},\,b_{5},\,0,\,0]\ \cdot\,PA\\ \ =\ [b_{1},\,b_{3},\,b_{5},\,0,\,0]\begin{bmatrix}1&0&0&0&0\\ 1&-1&0&0&0\\ 0&0&0&0&1\\ -1&1&1&0&0\\ -3&1&0&1&-1\end{bmatrix}.\ A\\ \ =\ [b_{1}+\,b_{3},\,-b_{2},\,0,\,0,\,b_{5}]\ \cdot\,A.\end{array}\]

In particular, with \(\beta\ =\ (-5,\,-10,\,1,\,-11,\,20)\) we have

\[\beta\ =\ (-4,\,-1,0,\,0,\,20)\begin{bmatrix}1&2&0&3&0\\ 1&2&-1&-1&0\\ 0&0&1&4&0\\ 2&4&1&10&1\\ 0&0&0&0&1\end{bmatrix}.

 