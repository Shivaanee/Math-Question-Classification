other words, if \(A\) is a real symmetric \(n\times n\) matrix, there is a real orthogonal matrix \(P\) such that \(P^{t}AP\) is diagonal; however, this is not at all apparent from what we did above (see Chapter 8).

Theorem 4: _Let \(V\) be a finite-dimensional vector space over the field of complex numbers. Let \(f\) be a symmetric bilinear form on \(V\) which has rank \(r\). Then there is an ordered basis \(\,\hbox{\hbox{\hbox{\hbox to 0.0pt{$\sqcap$}}\hbox{\hbox{$\sqcup$}}\hbox{$ \sqcup$}}}\,=\,\{\hbox{\hbox to 0.0pt{$\sqcap$}}\hbox{$\sqcup$}_{1},\,\ldots, \hbox{\hbox to 0.0pt{$\sqcap$}}\hbox{$\sqcup$}_{n}\}\,\hbox{for $V$ such that}\)_

1. _the matrix of_ \(f\) _in the ordered basis_ \(\,\hbox{\hbox{\hbox to 0.0pt{$\sqcap$}}\hbox{$\sqcup$}}\,\) _is diagonal;_
2. \(f(\beta_{i},\,\beta_{i})\,=\,\begin{cases}1,&j\,=\,1,\ldots,\,r\\ 0,&j\,>\,r.\end{cases}\)__

Proof: By Theorem 3, there is an ordered basis \(\{\alpha_{1},\,\ldots,\,\alpha_{n}\}\) for \(V\) such that

\[f(\alpha_{i},\,\alpha_{j})\,=\,0\quad\hbox{for}\quad i\neq j.\]

Since \(f\) has rank \(r\), so does its matrix in the ordered basis \(\{\alpha_{1},\,\ldots,\,\alpha_{n}\}\). Thus we must have \(f(\alpha_{j},\,\alpha_{i})\neq 0\) for precisely \(r\) values of \(j\). By reordering the vectors \(\alpha_{j}\), we may assume that

\[f(\alpha_{j},\,\alpha_{i})\,\neq\,0,\qquad j\,=\,1,\,\ldots,\,r.\]

Now we use the fact that the scalar field is the field of complex numbers. If \(\sqrt{f(\alpha_{j},\,\alpha_{j})}\) denotes any complex square root of \(f(\alpha_{j},\,\alpha_{j})\), and if we put

\[\beta_{j}\,=\,\begin{cases}\frac{1}{\sqrt{f(\alpha_{j},\,\alpha_{j})}}\, \alpha_{j},&j\,=\,1,\,\ldots,\,r\\ \alpha_{j},&j\,>\,r\end{cases}\]

the basis \(\{\beta_{1},\,\ldots,\,\beta_{n}\}\) satisfies conditions (i) and (ii). 

Of course, Theorem 4 is valid if the scalar field is any subfield of the complex numbers in which each element has a square root. It is not valid, for example, when the scalar field is the field of real numbers. Over the field of real numbers, we have the following substitute for Theorem 4.

Theorem 5: _Let \(V\) be an \(n\)-dimensional vector space over the field of real numbers, and let \(f\) be a symmetric bilinear form on \(V\) which has rank \(r\). Then there is an ordered basis \(\,\{\hbox{\hbox to 0.0pt{$\sqcap$}}\hbox{$\sqcup$}_{1},\,\hbox{\hbox to 0.0pt{$ \sqcap$}}\hbox{$\sqcup$}_{2},\,\ldots,\,\hbox{\hbox to 0.0pt{$\sqcap$}}\hbox{$\sqcup$}_{n}\}\,\hbox{for $V$}\) in which the matrix of \(f\) is diagonal and such that_

\[f(\beta_{i},\,\beta_{i})\,=\,\pm 1,\qquad j\,=\,1,\,\ldots 