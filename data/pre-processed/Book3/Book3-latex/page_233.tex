\[\frac{d^{n}\!f}{dt^{n}}+a_{n-1}\frac{d^{n-1}\!f}{dt^{n-1}}+\cdots+a_{1}\frac{df}{ dt}+a_{0}\!f=0\]

where \(a_{n},\ldots,a_{n-1}\) are some fixed constants. If \(C_{n}\) denotes the space of \(n\) times continuously differentiable functions, then the space \(V\) of solutions of this differential equation is a subspace of \(C_{n}\). If \(D\) denotes the differentiation operator and \(p\) is the polynomial

\[p=x^{n}+a_{n-1}x^{n-1}+\cdots+a_{1}x+a_{0}\]

then \(V\) is the null space of the operator \(p(D)\), because (6-18) simply says \(p(D)f=0\). Therefore, \(V\) is invariant under \(D\). Let us now regard \(D\) as a linear operator on the subspace \(V\). Then \(p(D)=0\).

If we are discussing differentiable complex-valued functions, then \(C_{n}\) and \(V\) are complex vector spaces, and \(a_{0},\ldots,a_{n-1}\) may be any complex numbers. We now write

\[p=(x-c_{1})^{r_{1}}\cdots(x-c_{k})^{r_{k}}\]

where \(c_{1},\ldots,c_{k}\) are distinct complex numbers. If \(W_{j}\) is the null space of \((D-c_{j}I)^{r_{j}}\), then Theorem 12 says that

\[V=W_{1}\oplus\cdots\oplus W_{k}.\]

In other words, if \(f\) satisfies the differential equation (6-18), then \(f\) is uniquely expressible in the form

\[f=f_{1}+\cdots+f_{k}\]

where \(f_{j}\) satisfies the differential equation \((D-c_{j}I)^{r_{j}}f_{j}=0\). Thus, the study of the solutions to the equation (6-18) is reduced to the study of the space of solutions of a differential equation of the form

\[(D-cI)^{r}\!f=0.\]

This reduction has been accomplished by the general methods of linear algebra, i.e., by the primary decomposition theorem.

To describe the space of solutions to (6-19 