

### 5.6 Multilinear Functions

The purpose of this section is to place our discussion of determinants in what we believe to be the proper perspective. We shall treat alternating multilinear forms on modules. These forms are the natural generalization of determinants as we presented them. The reader who has not read (or does not wish to read) the brief account of modules in Section 5.5 can still study this section profitably by consistently reading 'vector space over \(F\) of dimension \(n\)' for 'free module over \(K\) of rank \(n\).'

Let \(K\) be a commutative ring with identity and let \(V\) be a module over \(K\). If \(r\) is a positive integer, a function \(L\) from \(V^{r}=V\times V\times\cdots\times V\) into \(K\) is called **multilinear** if \(L(\alpha_{1},\ldots,\alpha_{r})\) is linear as a function of each \(\alpha_{i}\) when the other \(\alpha_{j}\)'s are held fixed, that is, if for each \(i\)

\(L(\alpha_{1},\ldots,\alpha_{r}+\beta_{i},\ldots,\alpha_{r})=cL(\alpha_{1}, \ldots,\alpha_{i},\ldots,\alpha_{r}+\)

\(L(\alpha_{1},\ldots,\beta_{i},\ldots,\alpha_{r})\).

A multilinear function on \(V^{r}\) will also be called an \(r\)**-linear form** on \(V\) or a **multilinear form of degree \(r\)** on \(V\). Such functions are sometimes called \(r\)**-tensors** on \(V\). The collection of all multilinear functions on \(V^{r}\) will be denoted by \(M^{r}(V)\). If \(L\) and \(M\) are in \(M^{r}(V)\), then the sum \(L+M\):

\((L+M)(\alpha_{1},\ldots,\alpha_{r})=L(\alpha_{1},\ldots,\alpha_{r})+M(\alpha_ {1},\ldots,\alpha_{r})\)

is also multilinear; and, if \(c\) is an element of \(K\), the product \(cL\):

\((cL)(\alpha_{1},\ldots,\alpha_{