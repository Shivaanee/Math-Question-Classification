7. There is a unique non-zero element 1 (one) in \(F\) such that \(x1=x\), for every \(x\) in \(F\).

8. To each non-zero \(x\) in \(F\) there corresponds a unique element \(x^{-1}\) (or \(1/x\)) in \(F\) such that \(xx^{-1}=1\).

9. Multiplication distributes over addition; that is, \(x(y+z)=xy+xz\), for all \(x\), \(y\), and \(z\) in \(F\).

Suppose one has a set \(F\) of objects \(x\), \(y\), \(z\), \(\ldots\) and two operations on the elements of \(F\) as follows. The first operation, called addition, associates with each pair of elements \(x\), \(y\) in \(F\) an element \((x+y)\) in \(F\); the second operation, called multiplication, associates with each pair \(x\), \(y\) an element \(xy\) in \(F\); and these two operations satisfy conditions (1)-(9) above. The set \(F\), together with these two operations, is then called a **field.** Roughly speaking, a field is a set together with some operations on the objects in that set which behave like ordinary addition, subtraction, multiplication, and division of numbers in the sense that they obey the nine rules of algebra listed above. With the usual operations of addition and multiplication, the set \(C\) of complex numbers is a field, as is the set \(R\) of real numbers.

For most of this book the 'numbers' we use may as well be the elements from any field \(F\). To allow for this generality, we shall use the word 'scalar' rather than 'number.' Not much will be lost to the reader if he always assumes that the field of scalars is a subfield of the field of complex numbers. A **subfield** of the field \(C\) is a set \(F\) of complex numbers which is itself a field under the usual operations of addition and multiplication of complex numbers. This means that 0 and 1 are in the set \(F\), and that if \(x\) and \(y\) are elements of \(F\), so are \((x+y)\), \(-x\), \(xy\), and \(x^{-1}\) (if \(x\neq 0\)). An example of such a subfield is the field \(R\) of real numbers; for, if we identify the real numbers with the complex numbers \((a+ib)\) for which \(b=0\), the 0 and 1 of the complex field are real numbers, and if \(x\) and \(y\) are real, so are \((x+y)\), \(-x\), \(xy\), and \(x^{-1}\) (if \(x\neq 0\)). We shall give other examples below. The point of our discussing subfields is essentially this: If we are working with scalars from a certain subfield of \(C\), then the performance of the operations of addition, subtraction, multiplication, or division on these scalars does not take us out of the given subfield.

**Example 1**.: The set of **positive integers:** 1, 2, 3, \(\ldots\), is not a subfield of \(C\), for a variety of reasons. For example, 0 is not a positive integer; for no positive integer \(n\) is \(-n\) a positive integer; for no positive integer \(n\) except 1 is \(1/n\) a positive integer.

**Example 2**.: The set of **integers:** \(\ldots\), \(-2\), \(-1\), 0, 1, 2, \(\ldots\), is not a subfield of \(C\), because for an integer \(n\), \(1/n\) is not an integer unless \(n\) is 1 or 