\[\begin{bmatrix}1&\frac{1}{2}&0\\ 0&1&0\\ 0&0&1\end{bmatrix},\quad\quad\begin{bmatrix}-9&60&-60\\ -36&192&-180\\ 30&-180&180\end{bmatrix}\] \[\begin{bmatrix}1&0&0\\ 0&1&0\\ 0&0&1\end{bmatrix},\quad\quad\begin{bmatrix}9&-36&30\\ -36&192&-180\\ 30&-180&180\end{bmatrix}.\]

It must have occurred to the reader that we have carried on a lengthy discussion of the rows of matrices and have said little about the columns. We focused our attention on the rows because this seemed more natural from the point of view of linear equations. Since there is obviously nothing sacred about rows, the discussion in the last sections could have been carried on using columns rather than rows. If one defines an elementary column operation and column-equivalence in a manner analogous to that of elementary row operation and row-equivalence, it is clear that each \(m\times n\) matrix will be column-equivalent to a 'column-reduced echelon' matrix. Also each elementary column operation will be of the form \(A\to AE\), where \(E\) is an \(n\times n\) elementary matrix--and so on.

### Exercises

1. Let \[A=\begin{bmatrix}1&2&1&0\\ -1&0&3&5\\ 1&-2&1&1\end{bmatrix}.\] Find a row-reduced echelon matrix \(R\) which is row-equivalent to \(A\) and an invertible \(3\times 3\) matrix \(P\) such that \(R=PA\).
2. Do Exercise 1, but with \[A=\begin{bmatrix}2&\quad\quad\quad\quad\quad\quad\quad\quad\quad\quad i\\ 1&-3&-i\\ i&1&1\end{bmatrix}.\]
3. For each of the two matrices \[\begin{bmatrix}2&5&-1\\ 4&-1&2\\ 6&4&1\end{bmatrix},\quad\quad\begin{bmatrix}1&-1&2\\ 3&2&4\\ 0&1&-2\end{bmatrix}\] use elementary row operations to discover whether it is invertible, and to find the inverse in case it is.
4. Let \[A=\begin{bmatrix}5&0&0\\ 1&5&0\\ 0&1&5\end{bmatrix}.\] 