But \(D(I,\,0,\,I)=1\), so \[D(A,B,C) = (\det C)D(A,B,I)\] \[= (\det C)D(A,0,I)\] \[= (\det C)(\det A).\] By the same sort of argument, or by taking transposes (5-20) \[\det\begin{bmatrix}A&0\\ B&C\end{bmatrix}=\,(\det A)(\det C).\]

Suppose \(K\) is the field of rational numbers and we wish to compute the determinant of the \(4\bigtimes 4\) matrix

\[A=\begin{bmatrix}1&-1&2&3\\ 2&2&0&2\\ 4&1&-1&-1\\ 1&2&3&0\end{bmatrix}.\]

By subtracting suitable multiples of row 1 from rows 2, 3, and 4, we obtain the matrix

\[\begin{bmatrix}1&-1&2&3\\ 0&4&-4&-4\\ 0&5&-9&-13\\ 0&3&1&-3\end{bmatrix}\]

which we know by (5-18) will have the same determinant as \(A\). If we subtract \(\frac{5}{4}\) of row 2 from row 3 and then subtract \(\frac{3}{4}\) of row 2 from row 4, we obtain

\[B=\begin{bmatrix}1&-1&2&3\\ 0&4&-4&-4\\ 0&0&-4&-8\\ 0&0&4&0\end{bmatrix}\]

and again \(\det B=\det A\). The block form of \(B\) tells us that

\[\det A=\det B=\begin{vmatrix}1&-1\\ 0&4\end{vmatrix}\begin{vmatrix}-4&-8\\ 4&0\end{vmatrix}=\,4(32)=128.\]

Now let \(n>1\) and let \(A\) be an \(n\bigtimes n\) matrix over \(K\). In Theorem 1, we showed how to construct a determinant function on \(n\bigtimes n\) matrices, given one on \((n-1)\bigtimes(n-1)\) matrices. Now that we have proved the uniqueness of the determinant function, the formula (5-4) tells us the following. If we fix any column index \(j\),

\[\det A=\sum\limits_{i=1}^{n}(-1)^{i+j}A_{ij}\det A(i|j).\]

The scalar \((-1)^{i+j}\det A(i|j)\) is usually called the \(i,\,j\)**cofactor** of \(A\) or the cofactor of the \(i,\,j\) entry of \(A\). The above formula for \(\det A\) is then 