Suppose \(R\) is an equivalence relation on the set \(X\). If \(x\) is an element of \(X\), we let \(E(x;R)\) denote the set of all elements \(y\) in \(X\) such that \(xRy\). This set \(E(x;R)\) is called the **equivalence class** of \(x\) (for the equivalence relation \(R\)). Since \(R\) is an equivalence relation, the equivalence classes have the following properties:

1. Each \(E(x;R)\) is non-empty; for, since \(xRx\), the element \(x\) belongs to \(E(x;R)\).
2. Let \(x\) and \(y\) be elements of \(X\). Since \(R\) is symmetric, \(y\) belongs to \(E(x;R)\) if and only if \(x\) belongs to \(E(y;R)\).
3. If \(x\) and \(y\) are elements of \(X\), the equivalence classes \(E(x;R)\) and \(E(y;R)\) are either identical or they have no members in common. First, suppose \(xRy\). Let \(z\) be any element of \(E(x;R)\) i.e., an element of \(X\) such that \(xRz\). Since \(R\) is symmetric, we also have \(zRx\). By assumption \(xRy\), and because \(R\) is transitive, we obtain \(zRy\) or \(yRz\). This shows that any member of \(E(x;R)\) is a member of \(E(y;E)\). By the symmetry of \(R\), we likewise see that any member of \(E(y;R)\) is a member of \(E(x;R)\); hence \(E(x;R)=E(y;R)\). Now we argue that if the relation \(xRy\) does not hold, then \(E(x;R)\cap E(y;R)\) is empty. For, if \(z\) is in both these equivalence classes, we have \(xRz\) and \(yRz\), thus \(xRz\) and \(zRy\), thus \(xRy\).

If we let \(\mathfrak{F}\) be the family of equivalence classes for the equivalence relation \(R\), we see that (1) each set in the family \(\mathfrak{F}\) is non-empty, (2) each element \(x\) of \(X\) belongs to one and only one of the sets in the family \(\mathfrak{F}\), (3) \(xRy\) if and only if \(x\) and \(y\) belong to the same set in the family \(\mathfrak{F}\). Briefly, the equivalence relation \(R\) subdivides \(X\) into the union of a family of non-overlapping (non-empty) subsets. The argument also goes in the other direction. Suppose \(\mathfrak{F}\) is any family of subsets of \(X\) which satisfies conditions (1) and (2) immediately above. If we define a relation \(R\) by (3), then \(R\) is an equivalence relation on \(X\) and \(\mathfrak{F}\) is the family of equivalence classes for \(\mathfrak{R}\).

**Example 6**: Let us see what the equivalence classes are for the equivalence relations in Example 5.

1. If \(R\) is equality on the set \(X\), then the equivalence class of the element \(x\) is simply the set \(\langle x\rangle\), whose only member is \(x\).
2. If \(X\) is the set of all triangles in a plane, and \(R\) is the congruence relation, about all one can say at the outset is that the equivalence class of the triangle \(T\) consists of all triangles which are congruent to \(T\). One of the tasks of plane geometry is to give other descriptions of these equivalence classes.
3. If \(X\) is the set of integers and \(R_{n}\) is the relation 'congruenee modulo \(n\),' then there are precisely \(n\) equivalence classes. Each integer \(x\) is uniquely expressible in the form \(x\,=\,qn\,+\,r\), where \(q\) and \(r\) are integers and \(0\leq r\leq n-1\). This shows that each \(x\) is congruent modulo \(n\) to 