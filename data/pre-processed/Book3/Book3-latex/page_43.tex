4. Let \(V\) be the set of all pairs \((x,y)\) of real numbers, and let \(F\) be the field of real numbers. Define \[(x,y)\,+\,(x_{1},y_{1})\,=\,(x+x_{1},y+y_{1})\] \[c(x,y)\,=\,(cx,y).\] Is \(V\), with these operations, a vector space over the field of real numbers?
5. On \(R^{*}\), define two operations \[\begin{array}{rcl}\alpha\,\oplus\,\beta\,=\,\alpha-\beta\\ c\cdot\alpha\,=\,-c\alpha.\end{array}\] The operations on the right are the usual ones. Which of the axioms for a vector space are satisfied by \((R^{*}\), \(\oplus\), \(\cdot\))?
6. Let \(V\) be the set of all complex-valued functions \(f\) on the real line such that (for all \(t\) in \(R\)) \[f(-t)\,=\,\overline{f(t)}.\] The bar denotes complex conjugation. Show that \(V\), with the operations \[\begin{array}{rcl}(f+g)(t)\,=\,f(t)\,+\,g(t)\\ (cf)(t)\,=\,cf(t)\end{array}\] is a vector space over the field of _real_ numbers. Give an example of a function in \(V\) which is not real-valued.
7. Let \(V\) be the set of pairs \((x,y)\) of real numbers and let \(F\) be the field of real numbers. Define \[\begin{array}{rcl}(x,y)\,+\,(x_{1},y_{1})\,=\,(x+x_{1},0)\\ c(x,y)\,=\,(cx,0).\end{array}\] Is \(V\), with these operations, a vector space?

### 2.2. \(Subspaces\)

In this section we shall introduce some of the basic concepts in the study of vector spaces.

_Definition._ Let \(V\) be a vector space over the field \(F\). \(A\) **subspace** _of \(V\) is a subset \(W\) of \(V\) which is itself a vector space over \(F\) with the operations of vector addition and scalar multiplication on \(V\)._

A direct check of the axioms for a vector space shows that the subset \(W\) of \(V\) is a subspace if for each \(\alpha\) and \(\beta\) in \(W\) the vector \(\alpha+\beta\) is again in \(W\); the \(0\) vector is in \(W\); for each \(\alpha\) in \(W\) the vector \((-\alpha)\) is in \(W\); for each \(\alpha\) in \(W\) and each scalar \(c\) the vector \(c\alpha\) is in \(W\). The commutativity and associativity of vector addition, and the properties (4)(a), (b), (c), and (d) of scalar multiplication do not need to be checked, since these are properties of the operations on \(V\). One can simplify things still further.

 