for \(n=0\), \(1\), \(2\), \(\ldots\), so that

\[(fg)h=f(gh).\] (4-3)

We leave it to the reader to verify that the multiplication defined by (4-2) satisfies (b) and (c) in the definition of a linear algebra, and that the vector \(1=(1,0,0,\ldots)\) serves as an identity for \(F^{\infty}\). Then \(F^{\infty}\), with the operations defined above, is a commutative linear algebra with identity over the field \(F\).

The vector \((0,1,0,\ldots,0,\ldots)\) plays a distinguished role in what follows and we shall consistently denote it by \(x\). Throughout this chapter \(x\) will never be used to denote an element of the field \(F\). The product of \(x\) with itself \(n\) times will be denoted by \(x^{n}\) and we shall put \(x^{0}=1\). Then

\[x^{2}=(0,0,1,0,\ldots),\qquad x^{3}=(0,0,0,1,0,\ldots)\]

and in general for each integer \(k\geq 0\), \((x^{k})_{k}=1\) and \((x^{k})_{n}=0\) for all non-negative integers \(n\neq k\). In concluding this section we observe that the set consisting of \(1\), \(x\), \(x^{2}\), \(\ldots\) is both independent and infinite. Thus the algebra \(F^{\infty}\) is not finite-dimensional.

The algebra \(F^{\infty 