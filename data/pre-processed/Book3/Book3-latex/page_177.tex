It is clear that \(L\not\subseteq M\) is multilinear on \(V^{r+*}\). The function \(L\not\subseteq M\) is called the **tensor product** of \(L\) and \(M\). The tensor product is not commutative. In fact, \(M\not\subseteq L\not\subseteq M\) unless \(L=0\) or \(M=0\); however, the tensor product does relate nicely to the module operations in \(M^{*}\) and \(M^{*}\).

_Lemma._ _Let \(L\), \(L_{l}\) be \(r\)-linear forms on \(V\), let \(M\), \(M_{l}\) be \(s\)-linear forms on \(V\) and let \(c\) be an element of \(K\)._

1. \((cL+L_{l})\not\subseteq M=c(L\not\subseteq M)+L_{l}\not\subseteq M\);
2. \(L\not\subseteq(cM+M_{l})=c(L\not\subseteq M)+L\not\subseteq M_{l}\).

_Proof._ Exercise.

Tensoring is associative, i.e., if \(L\), \(M\) and \(N\) are (respectively) \(r\)-, \(s\)- and \(t\)-linear forms on \(V\), then

\[(L\not\subseteq M)\not\subseteq N=L\not\subseteq(M\not\subseteq N).\]

This is immediate from the fact that the multiplication in \(K\) is associative. Therefore, if \(L_{1}\), \(L_{2}\), \(\ldots\), \(L_{k}\) are multilinear functions on \(Vn\), \(\ldots\), \(Vn\), then the tensor product

\[L=L_{1}\not\subseteq\cdots\not\subseteq L_{k}\]

is unambiguously defined as a multilinear function on \(V\), where \(r=r_{1}+\cdots+r_{k}\). We mentioned a particular case of this earlier. If \(f_{1}\), \(\ldots\), \(f_{r}\) are linear functions on \(V\), then the tensor product

\[L=f_{1}\not\subseteq\cdots\not\subseteq f_{r}\]

is given by

\[L(\alpha_{1},\ldots,\alpha_{r})=f_{1}(\alpha_{1})\,\cdots f_{r}(\alpha_{r}).\]

_Theorem 6._ _Let \(K\) be a commutative ring with identity. If \(V\) is a free \(K\)-module of rank \(n\) then \(M^{r}(V)\) is a free \(K\)-module of rank \(n^{r}\); in fact, if \(\{f_{1},\ldots,f_{n}\}\) is a 