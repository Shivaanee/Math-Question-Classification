product space, the scalars \(c_{1},\ldots,c_{n}\) are (of course) real, and so it must be that \(T=T^{*}\) .In other words, if \(V\) is a finite-dimensional _real_ inner product space and \(T\) is a linear operator for which there is an orthonormal basis of characteristic vectors, then \(T\) must be self-adjoint. If \(V\) is a complex inner product space, the scalars \(c_{1},\ldots,c_{n}\) need not be real, i.e., \(T\) need not be self-adjoint. But notice that \(T\) must satisfy

\[TT^{*}=T^{*}T.\] (8-17)

For, any two diagonal matrices commute, and since \(T\) and \(T^{*}\) are both represented by diagonal matrices in the ordered basis \(\mathbb{R}\), we have (8-17). It is a rather remarkable fact that in the complex case this condition is also sufficient to imply the existence of an orthonormal basis of characteristic vectors.

**Definition**.: _Let \(V\) be a finite-dimensional inner product space and \(T\) a linear operator on \(V\). We say that \(T\) is_ **normal** _if it commutes with its adjoint i.e.,_ TT* = T*T.__

Any self-adjoint operator is normal, as is any unitary operator. Any scalar multiple of a normal operator is normal; however, sums and products of normal operators are not generally normal. Although it is by no means necessary, we shall begin our study of normal operators by considering self-adjoint operators.

**Theorem 15**.: _Let \(V\) be an inner product space and \(T\) a self-adjoint linear operator on \(V\). Then each characteristic value of \(T\) is real, and characteristic vectors of \(T\) associated with distinct characteristic values are orthogonal._

Proof.: Suppose \(c\) is a characteristic value of \(T\), i.e., that \(T\alpha=c\alpha\) for some non-zero vector \(\alpha\). Then

\[\begin{array}{rl}c(\alpha|\alpha)&=&(c\alpha|\alpha)\\ &=&(T\alpha|\alpha)\\ &=&(\alpha|T\alpha)\\ &=&(\alpha|c\alpha)\\ &=&\bar{c}(\alpha|\alpha).\end{array}\]

Since \((\alpha|\alpha)\neq 0\), we must have \(c=\bar{c}\ 