older than \(y\).' If \(X\) is a set, what does it take to determine a relation between pairs of elements of \(X\)? What it takes, evidently, is a rule for determining whether, for any two given elements \(x\) and \(y\) in \(X\), \(x\) stands in the given relationship to \(y\) or not. Such a rule \(R\), we shall call a (binary) **relation** on \(X\). If we wish to be slightly more precise, we may proceed as follows. Let \(X\bigtimes X\) denote the set of all ordered pairs \((x,y)\) of elements of \(X\). A binary relation on \(X\) is a function \(R\) from \(X\bigtimes X\) into the set \(\{0,\,1\}\). In other words, \(R\) assigns to each ordered pair \((x,y)\) either a \(1\) or a \(0\). The idea is that if \(R(x,y)\,=\,1\), then \(x\) stands in the given relationship to \(y\), and if \(R(x,y)\,=\,0\), it does not.

If \(R\) is a binary relation on the set \(X\), it is convenient to write \(xRy\) when \(R(x,y)\,=\,1\). A binary relation \(R\) is called

1. **reflexive,** if \(xRx\) for each \(x\) in \(X\);
2. **symmetric,** if \(yRx\) whenever \(xRy\);
3. **transitive,** if \(xRx\) whenever \(xRy\) and \(yRz\).

An **equivalence relation** on \(X\) is a reflexive, symmetric, and transitive binary relation on \(X\).

**Example 5**: (a) On any set, equality is an equivalence relation. In other words, if \(xRy\) means \(x=y\), then \(R\) is an equivalence relation. For, \(x=x\), if \(x=y\) then \(y=x\), if \(x=y\) and \(y=z\) then \(x=z\). The relation '\(x\neq y\)' is symmetric, but neither reflexive nor transitive.

1. Let \(X\) be the set of real numbers, and suppose \(xRy\) means \(x<y\). Then \(R\) is not an equivalence relation. It is transitive, but it is neither reflexive nor symmetric. The relation '\(x\leq y\)' is reflexive and transitive, but not symmetric.

1. Let \(E\) be the Euclidean plane, and let \(X\) be the set of all triangles in the plane \(E\). Then congruence is an equivalence relation on \(X\), that is, '\(T_{1}\cong T_{2}\)' (\(T_{1}\) is congruent to \(T_{2}\)) is an equivalence relation on the set of all triangles in a plane.

1. Let \(X\) be the set of all integers: \(\ldots\) , \(-2\), \(-1\), \(0\), \(1\), \(2\), \(\ldots\) .

Let \(n\) be a fixed positive integer. Define a relation \(R_{n}\) on \(X\) by: \(xR_{n}y\) if and only if \((x-y)\) is divisible by \(n\). The relation \(R_{n}\) is called **congruence modulo**\(n\). Instead of \(xR_{n}y\), one usually writes

\[x\equiv y,\,\mbox{mod}\,\,n\qquad(x\mbox{ is congruent to }y\mbox{ modulo }n)\]

when \((x-y)\) is divisible by \(n\). For each positive integer \(n\), congruence modulo \(n\) is an equivalence relation on the set of integers.

1. Let \(X\) and \(Y\) be sets and \(f\) a function from \(X\) into \(Y\). We define a relation \(R\) on \(X\) by: \(x_{1}Rx_{2}\) if and only if \(f(x_{1})\,=\,f(x_{2})\). It is easy to verify that \(R\) is an equivalence relation on the set \(X\). As we shall see, this one example actually encompasses all equivalence relations.

 