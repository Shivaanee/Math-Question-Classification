(iii) if \(A\sim B\) and \(B\sim C\), then \(A\sim C\). What do we know about this equivalence relation? Actually, we know a great deal. For example, we know that \(A\sim B\) if and only if \(A=PB\) for some invertible \(m\times m\) matrix \(P\); or, \(A\sim B\) if and only if the homogeneous systems of linear equations \(AX=0\) and \(BX=0\) have the same solutions. We also have very explicit information about the equivalence classes for this relation. Each \(m\times n\) matrix \(A\) is row-equivalent to one and only one row-reduced echelon matrix. What this says is that each equivalence class for this relation contains precisely one row-reduced echelon matrix \(R\); the equivalence class determined by \(R\) consists of all matrices \(A=PR\), where \(P\) is an invertible \(m\times m\) matrix. One can also think of this description of the equivalence classes in the following way. Given an \(m\times n\) matrix \(A\), we have a rule (function) \(f\) which associates with \(A\) the row-reduced echelon matrix \(f(A)\) which is row-equivalent to \(A\). Row-equivalence is completely determined by \(f\). For, \(A\sim B\) if and only if \(f(A)=f(B)\), i.e., if and only if \(A\) and \(B\) have the same row-reduced echelon form.

Let \(n\) be a positive integer and \(F\) a field. Let \(X\) be the set of all \(n\times n\) matrices over \(F\). Then similarity is an equivalence relation on \(X\); each \(n\times n\) matrix \(A\) is similar to itself; if \(A\) is similar to \(B\), then \(B\) is similar to \(A\); if \(A\) is similar to \(B\) and \(B\) is similar to \(C\), then \(A\) is similar to \(C\). We know quite a bit about this equivalence relation too. For example, \(A\) is similar to \(B\) if and only if \(A\) and 