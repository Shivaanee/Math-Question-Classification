that one obtains by formulating the matrix analogues of the preceding theorems.

**Theorem 11**.: _Let \(A\) be a normal matrix and \(c_{i}\), \(\ldots\), \(c_{k}\) the distinct complex roots of \(det\ (xI\ -\ A)\). Let_

\[e_{i}=\prod\limits_{j\neq i}\left(\frac{x-c_{j}}{c_{i}-c_{i}}\right)\]

_and \(E_{i}=e_{i}(A)\ (1\leq i\leq k)\). Then \(E_{i}E_{j}=0\) when \(i\neq j\), \(E_{i}^{2}=E_{i}\), \(E_{i}^{*}=E_{i}\), and_

\[I\ =E_{i}+\ \cdots\ +\ E_{k}.\]

_If \(f\) is a complex-valued function whose domain includes \(c_{i}\), \(\ldots\), \(c_{k}\), then_

\[f(A)\ =f(c_{1})E_{i}+\ \cdots\ +f(c_{k})E_{k};\]

_in particular, \(A\ =\ c_{1}E_{1}+\ \cdots\ +c_{k}E_{k}\)._

We recall that an operator on an inner product space \(V\) is non-negative if \(T\) is self-adjoint and \((T\alpha|\alpha)\geq 0\) for every \(\alpha\) in \(V\).

**Theorem 12**.: _Let \(T\) be a diagonalizable normal operator on a finite-dimensional inner product space \(V\). Then \(T\) is self-adjoint, non-negative, or unitary according as each characteristic value of \(T\) is real, non-negative, or of absolute value \(1\)._

Proof.: Suppose \(T\) has the spectral resolution \(T=c_{1}E_{1}+\ \cdots\ +c_{k}E_{k}\), then \(T^{*}=\bar{c}_{1}E_{1}+\ \cdots\ +\bar{c}_{k}E_{k}\). To say \(T\) is self-adjoint is to say \(T\ =\ T^{*}\), or

\[(c_{1}-\bar{c}_{1})E_{1}+\ \cdots\ +\ (c_{k}-\bar{c}_{k})E_{k}=0.\]

Using the fact that \(E_{i}E_{j}=0\) for \(i\neq j\), and the fact that no \(E_{j}\) is the zero operator, we see that \(T\) is self-adjoint if and only if \(c_{j}=\bar{c}_{j}\), \(j=1\), \(\ldots\), \(k\). To distinguish the normal operators which are non-negative, let us look at

\[\begin{array}{rcl}(T\alpha|\alpha)&=&\left(\sum\limits_{j=1}^{k}c_{j}E_{j} \alpha|\sum\limits_{i=1}^{k}E_{i}\alpha\right)\\ &=&\sum\limits_{i}\sum\limits_{j}c_{j}(E_{j}\alpha|E_{i}\alpha)\\ &=&\sum\limits_{j}c_{j}||E_{j}\alpha||^{2}.\end{array}\]

We have used the fact that \((E_{j}\alpha|E_{i}\alpha)=0\) for \(i\neq j\). From this it is clear that the condition 