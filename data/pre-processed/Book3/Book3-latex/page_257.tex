The characteristic polynomial for \(A\) is obviously \((x-2)^{2}(x+1)\). Either this is the minimal polynomial, in which case \(A\) is similar to \[\begin{bmatrix}2&0&0\\ 1&2&0\\ 0&0&-1\end{bmatrix}\] or the minimal polynomial is \((x-2)(x+1)\), in which case \(A\) is similar to \[\begin{bmatrix}2&0&0\\ 0&2&0\\ 0&0&-1\end{bmatrix}.\] Now \[(A-2I)(A+I)=\begin{bmatrix}0&0&0\\ 3a&0&0\\ ac&0&0\end{bmatrix}\] and thus \(A\) is similar to a diagonal matrix if and only if \(a=0\).

Let

\[A\ =\begin{bmatrix}2&0&0&0\\ 1&2&0&0\\ 0&0&2&0\\ 0&0&a&2\end{bmatrix}.\]

The characteristic polynomial for \(A\) is \((x-2)^{4}\). Since \(A\) is the direct sum of two \(2\times 2\) matrices, it is clear that the minimal polynomial for \(A\) is \((x-2)^{2}\). Now if \(a=0\) or if \(a=1\), then the matrix \(A\) is in Jordan form. Notice that the two matrices we obtain for \(a=0\) and \(a=1\) have the same characteristic polynomial and the same minimal polynomial, but are not similar. They are not similar because for the 