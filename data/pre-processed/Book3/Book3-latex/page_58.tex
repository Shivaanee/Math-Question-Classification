1. Prove that \(W_{1}\) and \(W_{2}\) are subspaces of \(V\). 2. Find the dimensions of \(W_{1}\), \(W_{2}\), \(W_{1}+W_{2}\), and \(W_{1}\cap W_{2}\).

8. Again let \(V\) be the space of \(2\times 2\) matrices over \(F\). Find a basis \(\{A_{1}\), \(A_{2}\), \(A_{3}\), \(A_{4}\}\) for \(V\) such that \(A_{i}^{2}=A_{j}\) for each \(j\).

9. Let \(V\) be a vector space over a subfield \(F\) of the complex numbers. Suppose \(\alpha\), \(\beta\), and \(\gamma\) are linearly independent vectors in \(V\). Prove that \((\alpha+\beta)\), \((\beta+\gamma)\), and \((\gamma+\alpha)\) are linearly independent.

10. Let \(V\) be a vector space over the field \(F\). Suppose there are a finite number of vectors \(\alpha_{1}\), \(\ldots\), \(\alpha\), in \(V\) which span \(V\). Prove that \(V\) is finite-dimensional.

11. Let \(V\) be the set of all \(2\times 2\) matrices \(A\) with _complex_ entries which satisfy \(A_{11}+A_{22}=0\).

(a) Show that \(V\) is a vector space over the field of _real_ numbers, with the usual operations of matrix addition and multiplication of a matrix by a scalar.

(b) Find a basis for this vector space.

(c) Let \(W\) be the set of all matrices \(A\) in \(V\) such that \(A_{21}=-\overline{A}_{12}\) (the bar denotes complex conjugation). Prove that \(W\) is a subspace of \(V\) and find a basis for \(W\).

12. Prove that the space of all \(m\times n\) matrices over the field \(F\) has dimension \(mn\), by exhibiting a basis for this space.

13. Discuss Exercise 9, when \(V\) is a vector space over the field with two elements described in Exercise 5, Section 1.1.

14. Let \(V\) be the set of real numbers. Regard \(V\) as a vector space over the field of _rational_ numbers, with the usual operations. Prove that this vector space is _not_ finite-dimensional.

### Coordinates

One of the useful features of a basis \(\otimes\) in an \(n\)-dimensional space \(V\) is that it essentially enables one to introduce coordinates in \(V\) analogous to the 'natural coordinates' \(x_{i}\) of a vector \(\alpha=(x_{1},\ldots,x_{n})\) in the space \(F^{n}\). In this scheme, the coordinates of a vector \(\alpha\) in \(V\) relative to the basis \(\otimes\) will be the scalars which serve to express \(\alpha\) as a linear combination of the vectors in the basis. Thus, we should like to regard the natural coordinates of a vector \(\alpha\) in \(F^{n}\) as being defined by \(\alpha\) and the standard basis for \(F^{n}\); however, in adopting this point of view we must exercise a certain amount of care. If

\[\alpha=(x_{1},\ldots,x_{n})=\Sigma\ x_{i\in i}\]

and \(\otimes\) is the standard basis for \(F^{n}\), just how are the coordinates of \(\alpha\) determined by \(\otimes\) and \(\alpha\)? One way to phrase the answer is this. A given vector \(\alpha\) has a unique expression as a linear combination of the standard basis vectors, and the \(i\)th coordinate \(x_{i}\) of \(\alpha\) is the coefficient of \(\epsilon_{i}\) in this expression. From this point of view we are able to say which is the \(i\)th coordinate 