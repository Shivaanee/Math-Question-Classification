

**21.**: Let \(A\) be an \(n\times n\) matrix with _real_ entries. Let \(T\) be the linear operator on \(R^{n}\) which is represented by \(A\) in the standard ordered basis, and let \(U\) be the linear operator on \(C^{n}\) which is represented by \(A\) in the standard ordered basis. Use the result of Exercise 20 to prove the following: If the only subspaces invariant under \(T\) are \(R^{n}\) and the zero subspace, then \(U\) is diagonalizable.

### The Jordan Form

Suppose that \(N\) is a nilpotent linear operator on the finite-dimensional space \(V\). Let us look at the cyclic decomposition for \(N\) which we obtain from Theorem 3. We have a positive integer \(r\) and \(r\) non-zero vectors \(\alpha_{1}\), \(\ldots\), \(\alpha_{r}\) in \(V\) with \(N\)-annihilators \(p_{1}\), \(\ldots\), \(p_{r}\), such that

\[V\,=\,Z(\alpha_{1};N)\oplus\,\cdots\oplus\,Z(\alpha_{r};N)\]

and \(p_{i+1}\) divides \(p_{i}\) for \(i\,=\,1\), \(\ldots\), \(r\,-\,1\). Since \(N\) is nilpotent, the minimal polynomial is \(x^{k}\) for some \(k\leq n\). Thus each \(p_{i}\) is of the form \(p_{i}=x^{k}\), and the divisibility condition simply says that

\[k_{1}\geq k_{2}\geq\,\cdots\,\geq k_{r}.\]

Of course, \(k_{1}=k\) and \(k_{r}\geq 1\). The companion matrix of \(x^{ki}\) is the \(k_{i}\times k_{i}\) matrix

\[A_{i}=\begin{bmatrix}0&0&\cdots&0&0\\ 1&0&\cdots&0&0\\ 0&1&\cdots&0&0\\ \vdots&\vdots&&\vdots&\vdots\\ 0&0&\cdots&1&0\end{bmatrix}.\]

Thus Theorem 3 gives us an ordered basis for \(V\) in which the matrix of \(N\) is the direct sum of the elementary nilpotent matrices (7-24), the sizes of which decrease as \(i\) increases. One sees from this that associated with a nilpotent \(n\times n\) matrix is a positive integer \(r\) and \(r\) positive integers \(k_{1}\), \(\ldots\), \(k_{r}\) such that \(k_{1}+\,\cdots\,+\,k_{r}=n\) and \(k_{i}\geq k_{i+1}\), and these positive integers determine the rational form of the matrix, i.e., determine the matrix up to similarity.

Here is one thing we should like to point out about the nilpotent operator \(N\) above. The positive integer \(r\) is precisely the nullity of \(N\); in fact, the null space has as a basis the \(r\) vectors

\[N^{k_{i}-1}\alpha_{i}.\]

For, let \(\alpha\) be in the null space of \(N\). We write \(\alpha\) in the form

\[\alpha=f_{1}\alpha_{1}+\,\cdots\,+\,f_{r}\alpha_{r}\]

where \(f_{i}\) is a polynomial, the degree of which we may assume is less than \(k_{i}\). Since \(N\alpha=0\), for each \(i\) we have