\(V\) and \(c\) an arbitrary scalar. Then, by Theorem 4, \(\alpha-E\alpha\) and \(\beta-E\beta\) are each orthogonal to every vector in \(W\). Hence the vector

\[c(\alpha-E\alpha)+(\beta-E\beta)=(c\alpha+\beta)-(cE\alpha+E\beta)\]

also belongs to \(W^{\bot}\). Since \(cE\alpha+E\beta\) is a vector in \(W\), it follows from Theorem 4 that

\[E(c\alpha+\beta)=cE\alpha+E\beta.\]

Of course, one may also prove the linearity of \(E\) by using (8-11). Again let \(\beta\) be any vector in \(V\). Then \(E\beta\) is the unique vector in \(W\) such that \(\beta-E\beta\) is in \(W^{\bot}\). Thus \(E\beta=0\) when \(\beta\) is in \(W^{\bot}\). Conversely, \(\beta\) is in \(W^{\bot}\) when \(E\beta=0\). Thus \(W^{\bot}\) is the null space of \(E\). The equation

\[\beta=E\beta+\beta-E\beta\]

shows that \(V=W+W^{\bot}\); moreover, \(W\cap W^{\bot}=\{0\}\). For if \(\alpha\) is a vector in \(W\cap W^{\bot}\), then \((\alpha|\alpha)=0\). Therefore, \(\alpha=0\), and \(V\) is the direct sum of \(W\) and \(W^{\bot}\) 