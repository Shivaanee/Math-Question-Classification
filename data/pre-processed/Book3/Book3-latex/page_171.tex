be an \(n\times n\) matrix over the field \(F\) such that \(\det A\neq 0\). If \(y_{1},\ldots,y_{n}\) are any scalars in \(F\), the unique solution \(X=A^{-1}Y\) of the system of equations \(AX=Y\) is given by \[x_{j}=\frac{\det B_{j}}{\det A},\qquad j=1,\ldots,n\] where \(B_{j}\) is the \(n\times n\) matrix obtained from \(A\) by replacing the \(j\)th column of \(A\) by \(Y\). In concluding this chapter, we should like to make some comments which serve to place determinants in what we believe to be the proper perspective. From time to time it is necessary to compute specific determinants, and this section has been partially devoted to techniques which will facilitate such work. However, the principal role of determinants in this book is theoretical. There is no disputing the beauty of facts such as Cramer's rule. But Cramer's rule is an inefficient tool for solving systems of linear equations, chiefly because it involves too many computations. So one should concentrate on what Cramer's rule says, rather than on how to compute with it. Indeed, while reflecting on this entire chapter, we hope that the reader will place more emphasis on understanding what the determinant function is and how it behaves than on how to compute determinants of specific matrices.

### Exercises

**1.** Use the classical adjoint formula to compute the inverses of each of the following \(3\times 3\) real matrices.

\[\left[\begin{array}{rrr}-2&3&2\\ 6&0&3\\ 4&1&-1\end{array}\right],\qquad\left[\begin{array}{rrr}\cos\theta&0&-\sin \theta\\ 0&1&0\\ \sin\theta&0&\cos\theta\end{array}\right]\]

**2.** Use Cramer's rule to solve each of the following systems of linear equations over the field of rational numbers.

\begin{tabular}{l l l l} (a) & \(x+\) & \(y+\) & \(z=11\) \\  & \(2x-\) & \(6y-\) & \(z=\) & \(0\) \\  & \(3x+\) & \(4y+\) & \(2z=\) & \(0\). \\  & \(3x-\) & \(2y=\) & \(7\) \\  & \(3y-\) & \(2z=\) & \(6\) \\  & \(3z-\) & \(2x=\) & \(-1\). \\ \end{tabular}

**3.** An \(n\times n\) matrix \(A\) over a field \(F\) is **skew-symmetric** if \(A^{i}=-A\). If \(A\) is a skew-symmetric \(n\times n\) matrix with complex entries and \(n\) is odd, prove that \(\det A=0\).

**4.** An \(n\times n\) matrix \(A\) over a field \(F\) is called **orthogonal** if \(AA^{i}=I\). If \(A\) is orthogonal, show that \(\det A=\pm 1\). Give an example of an orthogonal matrix for which \(\det A=-1\).

 