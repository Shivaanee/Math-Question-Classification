In thinking about the preceding discussion, it is important for one to keep in mind that the spectrum of the normal operator \(T\) is the set

\[S\,=\,\{c_{\!1},\ldots,\,c_{\!k}\}\]

of distinct characteristic values. When \(T\) is represented by a diagonal matrix in a basis of characteristic vectors, it is necessary to repeat each value \(c_{\!i}\) as many times as the dimension of the corresponding space of characteristic vectors. This is the reason for the change of notation in the following result.

**Corollary**. With the assumptions of Theorem 10, suppose that \(T\) is represented in the ordered basis \(\hbox{\goth G}\,=\,\{\alpha_{\!1},\ldots,\alpha_{\!n}\}\) by the diagonal matrix \(D\) with entries \(d_{\!1},\ldots,d_{\!n}\). Then, in the basis \(\hbox{\goth G}\), \(f(T)\) is represented by the diagonal matrix \(f(D)\) with entries \(f(d_{\!1}),\ldots,f(d_{\!n})\). If \(\hbox{\goth G}^{\prime}\,=\,\{\alpha_{\!1}^{\prime},\ldots,\alpha_{\!n}^{ \prime}\}\) is any other ordered basis and \(P\) the matrix such that

\[\alpha_{\!1}^{\prime}\,=\,\sum_{i}\,P_{\!i_{1}}\!\alpha_{\!i}\]

then \(P^{-1}\!f(D)P\) is the matrix of \(f(T)\) in the basis \(\hbox{\goth G}^{\prime}\).

Proof.: For each index \(i\), there is a unique \(j\) such that \(1\leq j\leq k\), \(\alpha_{\!i}\) belongs to \(E_{\!j}(V)\), and \(d_{\!i}\,=\,c_{\!i}\). Hence \(f(T)\alpha_{\!i}=f 