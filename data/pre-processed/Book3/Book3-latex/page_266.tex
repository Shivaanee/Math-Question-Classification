\[PAP^{-1}=\begin{bmatrix}A_{1}&0&\cdots&0\\ 0&A_{2}&\cdots&0\\ \vdots&\vdots&\ddots&\vdots\\ 0&0&\cdots&A_{r}\end{bmatrix}\]

where \(A_{i}\) is the companion matrix of the polynomial \(p_{i}\). According to Theorem 7, the matrix

\[P(xI-A)P^{-1}=xI-PAP^{-1}\]

is equivalent to \(xI-A\). Now

\[xI-PAP^{-1}=\begin{bmatrix}xI-A_{1}&0&\cdots&0\\ 0&xI-A_{2}&\cdots&0\\ \vdots&\vdots&&\vdots\\ 0&0&\cdots&xI-A_{r}\end{bmatrix}\]

where the various \(I\)'s we have used are identity matrices of appropriate sizes. At the beginning of this section, we showed that \(xI-A_{i}\) is equivalent to the matrix

\[\begin{bmatrix}p_{i}&0&\cdots&0\\ 0&1&\cdots&0\\ \vdots&\vdots&&\vdots\\ 0&0&\cdots&1\end{bmatrix}\! 