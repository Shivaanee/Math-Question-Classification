for which we wish to find the solutions. If we let \(f_{i}\), \(i=1\), \(\ldots\), \(m\), be the linear functional on \(F^{n}\) defined by

\[f_{i}(x_{1},\ldots,x_{n})\,=\,A_{\,\,\,i}x_{1}+\,\cdots\,+\,A_{\,\,\,i}x_{n}\]

then we are seeking the subspace of \(F^{n}\) of all \(\alpha\) such that

\[f_{i}(\alpha)\,=\,0,\qquad i\,=\,1,\ldots,m.\]

In other words, we are seeking the subspace annihilated by \(f_{1},\ldots,f_{m}\). Row-reduction of the coefficient matrix provides us with a systematic method of finding this subspace. The \(n\)-tuple \((A_{\,\,\,\,i\!i},\ldots,A_{\,\,\,i\!i})\) gives the coordinates of the linear functional \(f_{i}\) relative to the basis which is dual to the standard basis for \(F^{n}\). The row space of the coefficient matrix may thus be regarded as the space of linear functionals spanned by \(f_{1},\ldots,f_{m}\). The solution space is the subspace annihilated by this space of functionals.

Now one may look at the system of equations from the 'dual' point of view. That is, suppose that we are given \(m\) vectors in \(F^{n}\)

\[\alpha_{i}\,=\,(A_{\,\,\,i\!i},\ldots,A_{\,\,\,i\!i})\]

and we wish to find the annihilator of the subspace spanned by these vectors. Since a typical linear functional on \(F^{n}\) has the form

\[f(x_{1},\ldots,x_{n})\,=\,c_{1}x_{1}+\,\cdots\,+\,c_{n}x_{n}\]

the condition that \(f\) be in this annihilator is that

\[\sum_{j\,=\,1}^{n}A_{\,\,\,\,i\!j}c_{j}\,=\,0,\qquad i\,=\,1,\ldots,m\]

that is, that \((c_{1},\ldots,c_{n})\) be a solution of the system \(AX=0\). From this point of view, row-reduction gives us a systematic method of finding the annihilator of the subspace spanned by a given finite set of vectors in \(F^{n}\).

**Example 23**: _Here are three linear functionals on \(R^{4}\):_

\[\begin{array}{l}f_{1}(x_{1},x_{1},x_{3},x_{4})\,=\,x_{1}+\,2x_{2}+2x_{3}+x_{4 }\\ f_{2}(x_{1},x_{2},x_{3},x_{4})\,=\,2x_{2}+x_{4}\\ f_{3}(x_{1},x_{3},x_{4})\,=\,-2x_{1}-4x_{4}+3x_{4}.\end{array}\]

_The subspace which they annihilate may be found explicitly by finding the row-reduced echelon form of the matrix_

\[A\,=\!\left[\begin{array}{cccc}1&2&2&1\\ 0&2&0&1\\ -2&0&-4&3\end{array}\right]\!.\]

_A short calculation, or a peek at Example 21 of Chapter 2, shows that_

\[R=\!\left[\begin{array}{cccc}1&0&2&0\\ 0&1&0&0\\ 0&0&0&1\end{array}\right]\!.\] 