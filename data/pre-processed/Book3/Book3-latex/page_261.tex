\[\left[\begin{array}{cccccccc}0&0&0&\cdots&0&x^{n}+\cdots+c_{1}x+c_{0}\\ -1&0&0&\cdots&0&x^{n-1}+\cdots+c_{2}x+c_{1}\\ 0&-1&0&\cdots&0&x^{n-2}+\cdots+c_{3}x+c_{2}\\ \vdots&\vdots&\vdots&&\vdots&\vdots\\ 0&0&0&\cdots&0&x^{2}+c_{n-1}x+c_{n-2}\\ 0&0&0&\cdots&-1&x+c_{n-1}\end{array}\right]\]

which has the same determinant as \(xI-A\). The upper right-hand entry of this matrix is the polynomial \(p\). We clean up the last column by adding to it appropriate multiples of the other columns:

\[\left[\begin{array}{cccccccc}0&0&0&\cdots&0&p\\ -1&0&0&\cdots&0&0\\ 0&-1&0&\cdots&0&0\\ \vdots&\vdots&\vdots&&\vdots&\vdots\\ 0&0&0&\cdots&0&0\\ 0&0&0&\cdots&-1&0\end{array}\right]\]

Multiply each of the first \((n-1)\) columns by \(-1\) and then perform \((n-1)\) interchanges of adjacent columns to bring the present column \(n\) to the first position. The total effect of the \(2n-2\) sign changes is to leave the determinant unaltered. We obtain the matrix

\[\left[\begin{array}{cccccccc}p&0&0&\cdots&0\\ 0&1&0&\cdots&0\\ 0&0&1&\cdots&0\\ \vdots&\vdots&\vdots&&\vdots\\ 0&0&0&\cdots&1\end{array}\right].\]

It is then clear that \(p=\det\,(xI-A)\).

We are going to show that, for any \(n\times n\) matrix \(A\), there is a succession of row and column operations which will transform \(xI-A\) into a matrix much like (7-28), in which the invariant factors of \(A\) appear down the main diagonal. Let us be completely clear about the operations we shall use.

We shall be concerned with \(F[x]^{n\times n}\), the collection of \(m\times n\) matrices with entries which are polynomials over the field \(F\). If \(M\) is such a matrix, an **elementary row operation** on \(M\) is one of the following

1. multiplication of one row of \(M\) by a non-zero scalar in \(F\);
2. replacement of the \(r\)th row of \(M\) by row \(r\) plus \(f\) times row \(s\), where \(f\) is any polynomial over \(F\) and \(r\neq s\);
3. interchange of two rows of \(M\).

The inverse operation of an elementary row operation is an elementary row operation of the same type. Notice that we could not make such an assertion if we allowed non-scalar polynomials in (1). An \(m\times m\)**ele 