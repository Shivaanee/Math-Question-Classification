By the above lemma, \(D\) is a 2-linear function. The reader who has had any experience with determinants will not find this surprising, since he will recognize (5-3) as the usual definition of the determinant of a \(2\times 2\) matrix. Of course the function \(D\) we have just defined is not a typical 2-linear function. It has many special properties. Let us note some of these properties. First, if \(I\) is the \(2\times 2\) identity matrix, then \(D(I)=1\), i.e., \(D(\epsilon_{1},\,\epsilon_{2})=1\). Second, if the two rows of \(A\) are equal, then

\[D(A)\,=\,A_{11}A_{12}-A_{12}A_{11}=0.\]

Third, if \(A^{\prime}\) is the matrix obtained from a \(2\times 2\) matrix \(A\) by interchanging its rows, then \(D(A^{\prime})\,=\,-D(A)\); for

\[\begin{array}{ll}D(A^{\prime})&=\,A^{\prime}_{11}A^{\prime}_{22}-A^{\prime} _{12}A^{\prime}_{21}\\ &=\,A_{11}A_{12}-A_{22}A_{11}\\ &=\,-D(A).\end{array}\]

_Definition._ Let D be an n-linear function. We say D is **alternating** (or **alternate**) if the following two conditions are satisfied:

1. \(\mbox{\rm D}(A)\,=\,0\) whenever two rows of A are equal.
2. \(If\) A\({}^{\prime}\) is a matrix obtained from A by interchanging two rows of A, then \(\mbox{\rm D}(A^{\prime})\,=\,-\mbox{\rm D}(A)\).

We shall prove below that any \(n\)-linear function \(D\) which satisfies (a) automatically satisfies (b). We have put both properties in the definition of alternating \(n\)-linear function as a matter of convenience. The reader will probably also note that if \(D\) satisfies (b) and \(A\) is a matrix with two equal rows, then \(D(A)=-D(A)\). It is tempting to conclude that \(D\) satisfies condition (a) as well. This is true, for example, if \(K\) is a field in which \(1\,+\,1\neq 0\), but in general (a) is not a consequence of (b).

_Definition._ Let K be a commutative ring with identity, and 