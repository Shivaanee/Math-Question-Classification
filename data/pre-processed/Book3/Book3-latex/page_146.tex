factorization. This decomposition is also clearly unique, and is called the **primary decomposition** of \(f\). It is easily verified that every monic divisor of \(f\) has the form

\[p_{1}^{m_{1}}p_{2}^{m_{2}}\cdots\ p_{r}^{m_{r}},\ \ \ \ \ 0\leq m_{i}\leq n_{i}.\] (4-18)

From (4-18) it follows that the g.c.d. of a finite number of non-scalar monic polynomials \(f_{1},\ldots,f_{s}\) is obtained by combining all those monic' primes which occur simultaneously in the factorizations of \(f_{1},\ldots,f_{s}\). The exponent to which each prime is to be taken is the largest for which the corresponding prime power is a factor of each \(f_{i}\). If no (non-trivial) prime power is a factor of each \(f_{i}\), the polynomials are relatively prime.

**Example 11**: _Suppose \(F\) is a field, and let \(a\), \(b\), \(c\) be distinct elements of \(F\). Then the polynomials \(x\)\(-\)\(a\), \(x\)\(-\)\(b\), \(x\)\(-\)\(c\) are distinct monic primes in \(F[x]\). If \(m\), \(n\), and \(s\) are positive integers, \((x\)\(-\)\(c)^{*}\) is the g.c.d. of the polynomials._

\[(x\)\(-\)\(b)^{n}(x\)\(-\)\(c)^{*}\ \ \ \mbox{and}\ \ \ (x\)\(-\)\(a)^{m}(x\)\(-\)\(c)^{*}\]

_whereas the three polynomials_

\[(x\)\(-\)\(b)^{n}(x\)\(-\)\(c)^{*},\ \ \ \ \ \ (x\)\(-\)\(a)^{m}(x\)\(-\)\(c)^{*},\ \ \ \ \ (x\)\(-\)\(a)^{m}(x\)\(-\)\(b)^{n}\]

_are relatively prime._

\(\ 