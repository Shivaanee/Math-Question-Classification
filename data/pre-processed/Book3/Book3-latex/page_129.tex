not assign a degree to the 0-polynomial. If \(f\) is a non-zero polynomial of degree \(n\) it follows that

\[f=f_{0}x^{0}+f_{1}x+f_{2}x^{2}+\ \cdots+f_{n}x^{n},\ \ \ \ \ f_{n}\neq 0.\] (4-5)

The scalars \(f_{0}\), \(f_{1}\), \(\ldots\), \(f_{n}\) are sometimes called the **coefficients** of \(f\), and we may say that \(f\) is a polynomial with coefficients in \(F\). We shall call polynomials of the form \(cx^{0}\)**scalar polynomials,** and frequently write \(c\) for \(cx^{0}\). A non-zero polynomial \(f\) of degree \(n\) such that \(f_{n}=1\) is said to be a **monic** polynomial.

The reader should note that polynomials are not the same sort of objects as the polynomial functions on \(F\) which we have discussed on several occasions. If \(F\) contains an infinite number of elements, there is a natural isomorphism between \(F[x]\) and the algebra of polynomial functions on \(F\). We shall discuss that in the next section. Let us verify that \(F[x]\) is an algebra.

**Theorem 1**: _Let \(f\) and \(g\) be non-zero polynomials over \(F\). Then_

1. \(fg\) _is a non-zero polynomial;_
2. \(deg\) (\(fg\)) _=_ \(deg\)__\(f\) _+_ \(deg\)__\(g\);_
3. \(fg\) _is a monic polynomial if both_ \(f\) _and_ \(g\) _are monic polynomials;_
4. \(fg\) _is a scalar polynomial if and only if both_ \(f\) _and_ \(g\) _are scalar polynomials;_
5. _if_ \(f\) _+_ \(g\neq 0\)_,_ \[deg\ (f+g)\leq max\ (deg\ f,\,deg\ g).\]

Suppose \(f\) has degree \(m\) and that \(g\) has degree \(n\). If \(k\) is a non-negative integer,

\[(fg)_{m+n+k}=\sum\limits_{i=0}^{m+n+k}f_{i}g_{m+n+k-i}.\]

In order that \(f_{i}g_{m+n+k-i}\neq 0\), it is necessary that \(i\leq m\) and \(m+n+k-i\leq n\). Hence it is necessary that \(m+k\leq i\leq m\), which implies 