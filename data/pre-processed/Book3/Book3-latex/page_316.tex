Let \(GL(n)\) denote the set of all invertible complex \(n\times n\) matrices. Then \(GL(n)\) is also a group under matrix multiplication. This group is called the **general linear group.** Theorem 14 is equivalent to the following result.

_Corollary. For each \(B\) in \(GL(n)\) there exist unique matrices \(N\) and \(U\) such that \(N\) is in \(T^{+}(n)\), \(U\) is in \(U(n)\), and_

\[B=N\cdot U.\]

By the theorem there is a unique matrix \(M\) in \(T^{+}(n)\) such that \(MB\) is in \(U(n)\). Let \(MB=U\) and \(N=M^{-1}\). Then \(N\) is in \(T^{+}(n)\) and \(B=N\cdot U\). On the other hand, if we are given any elements \(N\) and \(U\) such that \(N\) is in \(T^{+}(n)\), \(U\) is in \(U(n)\), and \(B=N\cdot U\), then \(N^{-1}B\) is in \(U(n)\) and \(N^{-1}\) is the unique matrix \(M\) which is characterized by the theorem; furthermore \(U\) is necessarily \(N^{-1}B\).

 