roots of \(F\). Suppose \(\{T_{1},\ldots,T_{m}\}\) is a maximal linearly independent subset of \(\mathfrak{F}\), and let \[\langle E_{i1},E_{i2},\ldots.\rangle\] be the resolution of the identity defined by \(T_{i}\) (\(1\leq i\leq m\)). Then the projections \(E_{ij}\) form a commutative family. For each \(E_{ij}\) is a polynomial in \(T_{i}\) and \(T_{1}\), \(\ldots,T_{m}\) commute with one another. Since \[I\,=\,(\mathop{\Sigma}\limits_{\hat{n}}E_{1i_{\hat{n}}})\ (\mathop{\Sigma}\limits_{\hat{n}}E_{2i_{\hat{n}}})\ \cdots\ (\mathop{\Sigma}\limits_{\hat{n}}E_{m \hat{n}})\] each vector \(\alpha\) in \(V\) may be written in the form (9-13) \[\alpha\,=\,\mathop{\Sigma}\limits_{j_{1},\ldots,j_{m}}E_{1j_{1}}E_{2j_{1}} \cdots\ E_{mj_{m}}\alpha.\] Suppose \(j_{1},\ldots,j_{m}\) are indices for which \(\beta=E_{1j_{1}}E_{2j_{1}}\cdots\ E_{m\hat{n}}\alpha\neq 0\). Let \[\beta_{i}\,=\,(\mathop{\Pi}\limits_{n\neq i}\ E_{nj_{i}})\ \alpha.\] Then \(\beta\,=E_{ij}\beta_{i}\); hence there is a scalar \(c_{i}\) such that \[T_{i}\beta\,=\,c_{i}\beta,\ \ \ \ \ 1\leq i\leq m.\] For each \(T\) in \(\mathfrak{F}\), there exist unique scalars \(b_{i}\) such that \[T\,=\,\mathop{\Sigma}\limits_{i\,=\,1}^{m}\,b 