\(\gamma=0\), i.e., if the null space of \(T\) is \(\langle 0\rangle\). Evidently, \(T\) is 1:1 if and only if \(T\) is non-singular. The extension of this remark is that non-singular linear transformations are those which preserve linear independence.

Theorem 3.1: _Let \(T\) be a linear transformation from \(V\) into \(W\). Then \(T\) is non-singular if and only if \(T\) carries each linearly independent subset of \(V\) onto a linearly independent subset of \(W\)._

Proof: First suppose that \(T\) is non-singular. Let \(S\) be a linearly independent subset of \(V\). If \(\alpha_{1}\), \(\ldots\), \(\alpha_{k}\) are vectors in \(S\), then the vectors \(Ta_{1}\), \(\ldots\), \(T\alpha_{k}\) are linearly independent; for if

\[c_{1}(Ta_{1})+\cdots+c_{k}(Ta_{k})=0\]

then

\[T(c_{1}\alpha_{1}+\cdots+c_{k}\alpha_{k})=0\]

and since \(T\) is non-singular

\[c_{1}\alpha_{1}+\cdots+c_{k}\alpha_{k}=0\]

from which it follows that each \(c_{i}=0\) because \(S\) is an independent set. This argument shows that the image of \(S\) under \(T\) is independent.

Suppose that \(T 