Sec. 9.3

_Positive Forms 325_

non-degenerate if and only if the associated linear operator \(T_{f}\) (Theorem 1) is non-singular.

7. Let \(f\) be a form on a finite-dimensional vector space \(V\). Look at the definition of left non-degeneracy given in Exercise 6. Define right non-degeneracy and prove that the form \(f\) is left non-degenerate if and only if \(f\) is right non-degenerate.

8. Let \(f\) be a non-degenerate form (Exercises 6 and 7) on a finite-dimensional space \(V\). Let \(L\) be a linear functional on \(V\). Show that there exists one and only one vector \(\beta\) in \(V\) such that \(L(\alpha)=f(\alpha,\beta)\) for all \(\alpha\).

9. Let \(f\) be a non-degenerate form on a finite-dimensional space \(V\). Show that each linear operator \(S\) has an 'adjoint relative to \(f\),' i.e., an operator \(S^{\prime}\) such that \(f(S\alpha,\beta)=f(\alpha,S^{\prime}\beta)\) for all \(\alpha\), \(\beta\).

_9.3. Positive Forms_

In this section, we shall discuss non-negative (sesqui-linear) forms and their relation to a given inner product on the underlying vector space.

_Definitions._ A form \(f\) on a real or complex vector space \(V\) is **non-negative** if it is Hermitian and \(f(\alpha,\alpha)\geq 0\) for every \(\alpha\) in \(V\). The form \(f\) is **positive** if \(f\) is Hermitian and \(f(\alpha,\alpha)>0\) for all \(\alpha\neq 0\).

A positive form on \(V\) is simply an inner product on \(V\). A non-negative form satisfies all of the properties of an inner product except that some non-zero vectors may be 'orthogonal' to themselves.

Let \(f\) be a form on the finite-dimensional space \(V\). Let \(\mathfrak{G}=\{\alpha_{1},\ldots,\alpha_{n}\}\) be an ordered basis for \(V\), and let \(A\) be the matrix of \(f\) in the basis \(\mathfrak{G}\), that is, \(A_{jk}=f(\alpha_{k},\alpha_{j})\). If \(\alpha=x_{1}\alpha_{1}+\cdots+x_{n}\alpha_{n}\), then

\[\begin{array}{rl}f(\alpha,\alpha)&=f(\sum\limits_{j}x_{j}\alpha_{j},\sum \limits_{k}x_{k}\alpha_{k})\\ &=\sum\limits_{j}\sum\limits_{k}x_{j}\overline{x}_{k}f(\alpha_{j}, 