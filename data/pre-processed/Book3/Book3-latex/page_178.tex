
**Example 12**.: Earlier in this chapter, we showed that on the module \(K^{*}\) there is precisely one alternating \(n\)-linear form \(D\) with the property that \(D(\epsilon_{1},\ldots,\epsilon_{n})=1\). We also showed in Theorem 2 that if \(L\) is any form in \(\Lambda^{n}(K^{n})\) then

\[L=L(\epsilon_{1},\ldots,\epsilon_{n})D.\]

In other words, \(\Lambda^{n}(K^{n})\) is a free \(K\)-module of rank \(1\). We also developed an explicit formula (5-15) for \(D\). In terms of the notation we are now using, that formula may be written

(5-33)

where \(f_{1},\ldots,f_{n}\) are the standard coordinate functions on \(K^{n}\) and the sum is extended over the \(n!\) different permutations \(\sigma\) of the set \(\{1,\ldots,n\}\). If we write the determinant of a matrix \(A\) as

\[\det A=\sum_{\sigma}\,(\operatorname{sgn}\sigma)\,A(\sigma 1,1)\,\cdots\,A(\sigma n,n)\] 