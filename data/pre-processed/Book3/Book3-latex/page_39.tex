The product of a scalar \(c\) and the matrix \(A\) is defined by (2-4) \[(cA)_{ij}=cA_{ij}.\] Note that \(F^{1\times n}=F^{n}\).

**Example 3**.: **The space of functions from a set to a field.** Let \(F\) be any field and let \(S\) be any non-empty set. Let \(V\) be the set of all functions from the set \(S\) into \(F\). The sum of two vectors \(f\) and \(g\) in \(V\) is the vector \(f+g\), i.e., the function from \(S\) into \(F\), defined by (2-5) \[(f+g)(s)=f(s)+g(s).\] The product of the scalar \(c\) and the function \(f\) is the function \(cf\) defined by (2-6) \[(cf)(s)=cf(s).\] The preceding examples are special cases of this one. For an \(n\)-tuple of elements of \(F\) may be regarded as a function from the set \(S\) of integers 1, \(\ldots\), \(n\) into \(F\). Similarly, an \(m\times n\) matrix over the field \(F\) is a function from the set \(S\) of pairs of integers, \((i,j)\), \(1\leq i\leq m\), \(1\leq j\leq n\), into the field \(F\). For this third example we shall indicate how one verifies that the operations we have defined satisfy conditions (3) and (4). For vector addition:

* Since addition in \(F\) is commutative, \[f(s)+g(s)=g(s)+f(s)\] for each \(s\) in \(S\), so the functions \(f+g\) and \(g+f\) are identical.
* Since addition in \(F\) is associative, \[f(s)+[g(s)+h(s)]=[f(s)+g(s)]+h(s)\] for each \(s\), so \(f+(g+h)\) is the same function
 