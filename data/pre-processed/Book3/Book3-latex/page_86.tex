In the case of Theorem 6 when \(V=W=Z\), so that \(U\) and \(T\) are linear operators on the space \(V\), we see that the composition \(UT\) is again a linear operator on \(V\). Thus the space \(L(V,\,V)\) has a 'multiplication' defined on it by composition. In this case the operator \(TU\) is also defined, and one should note that in general \(UT\neq TU\), i.e., \(UT-TU\neq 0\). We should take special note of the fact that if \(T\) is a linear operator on \(V\) then we can compose \(T\) with \(T\). We shall use the notation \(T^{2}=TT\), and in general \(T^{n}=T\,\cdots\,T\) (\(n\) times) for \(n=1\), \(2\), \(3\), \(\ldots\). We define \(T^{0}=I\) if \(T\neq 0\).

Lemma: Let \(V\) be a vector space over the field \(F\); let \(U\), \(T_{1}\) and \(T_{2}\) be linear operators on \(V\); let \(c\) be an element of \(F\).

1. \(IU=UU\) ;
2. \(U(T_{1}+T_{2})=UT_{1}+UT_{2};(T_{1}+T_{2})U=T_{1}U+T_{2}U\);
3. \(c(UT_{1})=(cU)T_{1}=U(cT_{1})\).

Demonstration Proof: (a) This property of the identity function is obvious. We have stated it here merely for emphasis.

\[\eqalign{[U(T_{1}+T_{2})](\alpha)&=\,U[(T_{1}+T_{2})(\alpha)]\cr&=\,U(T_{1} \alpha+T_{2}\alpha)\cr&=\,(UT_{1})(\alpha)+(UT_{2})(\alpha)\cr}\]

so that \(U(T_{1}+T_{2})=UT_{1}+UT_{2}\). Also

\[\eqalign{[(T_{1}+T_{2})U](\alpha)&=\,(T_{1}+T_{2})(U\alpha)\cr&=\,T_{1}(U\alpha )+T_{ 