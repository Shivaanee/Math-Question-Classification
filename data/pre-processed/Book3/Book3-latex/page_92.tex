satisfies \(IT=TJ\) for each \(T\), and for an invertible \(T\) there is (by Theorem 7) an-invertible linear operator \(T^{-1}\) such that \(TT^{-1}=T^{-1}T^{-1}=I\). Thus the set of invertible linear operators on \(V\), together with this operation, is a group. The set of invertible \(n\times n\) matrices with matrix multiplication as the operation is another example of a group. A group is called **commutative** if it satisfies the condition \(xy=yx\) for each \(x\) and \(y\). The two examples we gave above are not commutative groups, in general. One often writes the operation in a commutative group as \((x,y)\to x+y\), rather than \((x,y)\to xy\), and then uses the symbol \(0\) for the 'identity' element \(e\). The set of vectors in a vector space, together with the operation of vector addition, is a commutative group. A field can be described as a set with two operations, called addition and multiplication, which is a commutative group under addition, and in which the non-zero elements form a commutative group under multiplication, with the distributive law \(x(y+z)=xy+xz\) holding.

### Exercises

1. Let \(T\) and \(U\) be the linear operators on \(R^{2}\) defined by \[T(x_{1},x_{2})=(x_{2},x_{1})\quad\text{and}\quad U(x_{1},x_{2})=(x_{1},0).\] (1) How would you describe \(T\) and \(U\) geometrically? (2) Give rules like the ones defining \(T\) and \(U\) for each of the transformations \((U+T)\), \(UT\), \(TU\), \(T^{1}\), \(U^{2}\).
2. Let \(T\) be the (unique) linear operator on \(C^{2}\) for which \[T\epsilon_{1}=(1,0,i),\qquad T\epsilon_{2}=(0,1,1),\qquad T\epsilon_{3}=(i,1,0).\]

Is \(T\) invertible?
3. Let \(T\) be the linear operator on \(R^{3}\) defined by \[T(x_{1},x_{2},x_{3})=(3x_{1},x_{1}-x_{3},2x_{1}+x_{2}+x_{3}).\]

Is \(T\) invertible? If so, find a rule for \(T^{-1}\) like the one which defines \(T\).
4. For the linear operator \(T\) of Exercise 3, prove that \[(T^{2}-I)(T-3I)=0.\]
5. Let \(C^{2x2}\) be the complex vector space of \(2\times 2\) matrices with complex entries. Let \[B=\left[\begin{array}{cc}1&-1\\ -4&4\end{array}\right]\]

and let \(T\) be the linear operator on \(C^{2x2}\) defined by \(T(A)=BA\). What is the rank of \(T\)? Can you describe \(T\)?
6. Let \(T\) be a linear transformation from \(R^{2}\) into \(R^{2}\), and let \(U\) be a linear transformation from \(R^{2}\) into \(R^{2}\). Prove that the transformation \(UT\) is not invertible. Generalize the theorem.

 