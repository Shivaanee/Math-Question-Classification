Proof: Let \(T\) be the linear operator such that\(f(\alpha,\beta)\,=\,(T\alpha|\beta)\) for all \(\alpha\) and \(\beta\) in \(V\). Then, since \(f(\alpha,\beta)\,=\,\overline{f(\beta,\alpha)}\) and \((\overline{T\beta|\alpha})\,=\,(\alpha|T\beta)\), it follows that

\[(T\alpha|\beta)\,=\,\overline{f(\beta,\,\alpha)}\,=\,(\alpha|T\beta)\]

for all \(\alpha\) and \(\beta\); hence \(T\,=\,T^{\ast}\). By Theorem 18 of Chapter 8, there is an orthonormal basis of \(V\) which consists of characteristic vectors for \(T\). Suppose \(\{\alpha_{1},\,.\,.\,.\,,\,\alpha_{n}\}\) is an orthonormal basis and that

\[T\alpha_{j}\,=\,c_{j}\alpha_{j}\]

for \(1\leq j\leq n\). Then

\[f(\alpha_{k},\,\alpha_{j})\,=\,(T\alpha_{k}|\alpha_{j})\,=\,\delta_{kj}c_{k}\]

and by Theorem 15 of Chapter 8 each \(c_{k}\) is real.

_Corollary_.: _Under the above conditions_

\[f(\sum_{j}x_{i}\alpha_{i},\,\sum_{k}\,y_{k}\alpha_{k})\,=\,\sum_{j}\,c_{1}x_{i }\vec{y}_{i}.\]

### Exercises

**1.** Which of the following functions \(f\), defined on vectors \(\alpha=(x_{1},\,x_{2})\) and \(\beta=(y_{1},\,y_{2})\) in \(C^{2}\), are (sesqui-linear) forms on \(C^{2}\)?

(a) \(f(\alpha,\,\beta)\,=\,1\).

(b) \(f(\alpha,\,\beta)\,=\,(x_{1}-\bar{y}_{1})^{2}+x_{2}\bar{y}_{2}\).

(c) \(f(\alpha,\,\beta)\,=\,(x_{1}+\bar{y}_{1})^{2}-(x_{1}-\bar{y}_{1})^{2}\).

(d) \(f(\alpha,\,\beta)\,=\,x_{1}\bar{y}_{2}-\,\bar{x}

 