Since we are operating in a three-dimensional space, there can be only one further vector, \(\alpha_{2}\). It must generate a cyclic subspace of dimension 1, i.e., it must be a characteristic vector for \(T\). Its \(T\)-annihilator \(p_{2}\) must be \((x-2)\), because we must have \(pp_{2}=f\). Notice that this tells us immediately that the matrix \(A\) is similar to the matrix

\[B=\begin{bmatrix}0&-2&0\\ 1&3&0\\ 0&0&2\end{bmatrix}\]

that is, that \(T\) is represented by \(B\) in some ordered basis. How can we find suitable vectors \(\alpha_{1}\) and \(\alpha_{2}\)? Well, we know that any vector which generates a \(T\)-cyclic subspace of dimension 2 is a suitable \(\alpha_{1}\). So let's just try \(\epsilon_{1}\). We have

\[T\epsilon_{1}=(5,\,-1,\,3)\]

which is not a scalar multiple of \(\epsilon_{1}\); hence \(Z(\epsilon_{1};T)\) has dimension 2. This space consists of all vectors \(a\epsilon_{1}+b(T\epsilon_{1})\):

\[a(1,\,0,\,0)+b(5,\,-1,\,3)=(a+5b,\,-b,\,3b)\]

or, all vectors \((x_{1},\,x_{2},\,x_{3})\) satisfying \(x_{3}=-3x_{ 