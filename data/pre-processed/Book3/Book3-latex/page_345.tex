Proof: Let \(\alpha\) be a vector in \(W_{j}\), \(\beta\) a vector in \(W_{i}\), and suppose \(i\neq j\). Then \(c_{j}(\alpha|\beta)=(T\alpha|\beta)=(\alpha|T^{*}\beta)=(\alpha|\overline{c}_{i }\beta)\). Hence \((c_{j}-c_{i})(\alpha|\beta)=0\), and since \(c_{j}-c_{i}\neq 0\), it follows that \((\alpha|\beta)=0\). Thus \(W_{j}\) is orthogonal to \(W_{i}\) when \(i\neq j\). From the fact that \(V\) has an orthonormal basis consisting of characteristic vectors (cf. Theorems 18 and 22 of Chapter 8), it follows that \(V=W_{1}+\cdots+W_{k}\). If \(\alpha_{j}\) belongs to \(V_{j}\) (\(1\leq j\leq k\)) and \(\alpha_{1}+\cdots+\alpha_{k}=0\), then

\[\begin{array}{l}0=\ (\alpha_{i}|\underset{j}{\Sigma}\ \alpha_{j})=\underset{j}{ \Sigma}\ (\alpha_{i}|\alpha_{j})\\ =\ ||\alpha_{i}||^{2}\end{array}\]

for every \(i\), so that \(V\) is the direct sum of \(W_{1}\), \(\ldots\), \(W_{k}\). Therefore \(E_{1}+\cdots+E_{k}=I\) and

\[\begin{array}{l}T=\ TE_{1}+\cdots+\ TE_{k}\\ =\ c_{1}E_{1}+\cdots+\ c_{k}E_{ 