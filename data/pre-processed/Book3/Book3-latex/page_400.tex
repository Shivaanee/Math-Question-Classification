Now \(f\) is neither 1:1 nor onto, whereas \(f_{0}\) is both 1:1 and onto. The latter statement simply says that each non-negative number is the square of exactly one non-negative number. The inverse function \(f_{0}^{-1}\) is the function from \(X_{0}\) into \(X_{0}\) defined by \(f_{0}^{-1}(x)=\sqrt{x}\).

Let \(X\) be the set of real numbers, and let \(f\) be the function from \(X\) into \(X\) defined by \(f(x)=x^{3}+x^{2}+1\). The range of \(f\) is all of \(X\), and so \(f\) is onto. The function \(f\) is certainly not 1:1, e.g., \(f(-1)=f(0)\). But \(f\) is 