

**Example 16**.: We shall now give an example of an infinite basis. Let \(F\) be a subfield of the complex numbers and let \(V\) be the space of polynomial functions over \(F\). Recall that these functions are the functions from \(F\) into \(F\) which have a rule of the form

\[f(x)=c_{0}+c_{1}x+\cdots+c_{n}x^{n}.\]

Let \(f_{k}(x)=x_{k}\), \(k=0\), \(1\), \(2\), \(\ldots\). The (infinite) set \(\langle f_{0},f_{1},f_{2},\ldots\rangle\) is a basis for \(V\). Clearly the set spans \(V\), because the function \(f\) (above) is

\[f=c_{0}f_{0}+c_{1}f_{1}+\cdots+c_{n}f_{n}.\]

The reader should see that this is virtually a repetition of the definition of polynomial function, that is, a function \(f\) from \(F\) into \(F\) is a polynomial function if and only if there exists an integer \(n\) and scalars \(c_{0}\), \(\ldots\), \(c_{n}\) such that \(f=c_{0}f_{0}+\cdots+c_{n}f_{n}\). Why are the functions independent? To show that the set \(\langle f_{0},f_{1},f_{2},\ldots\rangle\) is independent means to show that each finite subset of it is independent. It will suffice to show that, for each \(n\), the set \(\langle f_{0},\ldots,f_{n}\rangle\) is independent. Suppose that