characteristic values. It is important to point out that \(T\) may not have any characteristic values.

**Example 1**: Let \(T\) be the linear operator on \(R^{2}\) which is represented in the standard ordered basis by the matrix

\[A\,=\begin{bmatrix}0&-1\\ 1&0\end{bmatrix}.\]

The characteristic polynomial for \(T\) (or for \(A\)) is

\[\det\,(xI-A)\,=\begin{vmatrix}x&1\\ -1&x\end{vmatrix}=x^{2}+1.\]

Since this polynomial has no real roots, \(T\) has no characteristic values. If \(U\) is the linear operator on \(C^{2}\) which is represented by \(A\) in the standard ordered basis, then \(U\) has two characteristic values, \(i\) and \(-i\). Here we see a subtle point. In discussing the characteristic values of a matrix \(A\), we must be careful to stipulate the field involved. The matrix \(A\) above has no characteristic values in \(R\), but has the two characteristic values \(i\) and \(-i\) in \(C\).

**Example 2**: Let \(A\) be the (real) \(3\times 3\) matrix

\[\begin{bmatrix}3&1&-1\\ 2&2&-1\\ 2&2&0\end{bmatrix}.\]

Then the characteristic polynomial for \(A\) is

\[\begin{vmatrix}x-3&-1&1\\ -2&x-2&1\\ -2&-2&x\end{vmatrix}=x^{3}-5x^{3}+8x-4\,=\,(x-1)(x-2)^{2}.\]

Thus the characteristic values of \(A\) are \(1\) and \(2\).

Suppose that \(T\) is the linear operator on \(R^{3}\) which is represented by \(A\) in the standard 