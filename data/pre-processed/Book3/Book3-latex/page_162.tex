however, we should like to point out how it follows from the _existence_ of a determinant function on \(n\times n\) matrices.

Let us take \(K\) to be the ring of integers. Let \(D\) be a determinant function on \(n\times n\) matrices over \(K\). Let \(\sigma\) be a permutation of degree \(n\), and suppose we pass from \((1,2,\ldots,n)\) to \((\sigma 1,\sigma 2,\ldots,\sigma n)\) by \(m\) interchanges of pairs \((i,j)\), \(i\neq j\). As we showed in (5-13)

\[(-1)^{m}=D(\epsilon_{\sigma 1},\ldots,\epsilon_{\sigma n})\]

that is, the number \((-1)^{m}\) must be the value of \(D\) on the matrix with rows \(\epsilon_{\sigma 1}\), \(\ldots,\epsilon_{\sigma n}\). If

\[D(\epsilon_{\sigma 1},\ldots,\epsilon_{\sigma n})=1,\]

then \(m\) must be even. If

\[D(\epsilon_{\sigma 1},\ldots,\epsilon_{\sigma n})=-1,\]

then \(m\) must be odd.

 