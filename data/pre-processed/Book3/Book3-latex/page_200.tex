Suppose \(T\) is a linear operator on \(V\), a vector space over the field \(F\). If \(p\) is a polynomial over \(F\), then \(p(T)\) is again a linear operator on \(V\). If \(q\) is another polynomial over \(F\), then

\[\eqalign{(p\,+\,q)(T)&=\,p(T)\,+\,q(T)\cr(pq)(T)&=\,p(T)q(T).\cr}\]

Therefore, the collection of polynomials \(p\) which annihilate \(T\), in the sense that

\[p(T)\,=\,0,\]

is an ideal in the polynomial algebra \(F[x]\). It may be the zero ideal, i.e., it may be that \(T\) is not annihilated by any non-zero polynomial. But, that cannot happen if the space \(V\) is finite-dimensional.

Suppose \(T\) is a linear operator on the \(n\)-dimensional space \(V\). Look at the first \((n^{2}+\,1)\) powers of \(T\):

\[I,\,T,\,T^{2},\,\ldots,\,T^{n^{2}}.\]

This is a sequence of \(n^{2}+1\) operators in \(L(V,\,V)\), the space of linear operators on \(V\). The space \(L(V,\,V)\) has dimension \(n^{2}\). Therefore, that sequence of \(n^{2}+\,1\) operators must be linearly dependent, i.e., we have

\[c_{0}I\,+\,c_{1}T\,+\,\cdots\,+\,c_{n^{2}

 