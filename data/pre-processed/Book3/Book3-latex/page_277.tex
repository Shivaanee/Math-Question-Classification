_Furthermore, the semi-simple \(S\) and nilpotent \(N\) satisfying (i) and (ii) are unique, and each is a polynomial in \(T.\)_

Let \(p_{1}^{\prime\prime}\cdots p_{k}^{n}\) be the prime factorization of the minimal polynomial for \(T,\) and let\(f=p_{1}\cdots p_{k}.\) Let \(r\) be the greatest of the positive integers \(r_{1},\ldots,r_{k}.\) Then the polynomial \(f\) is a product of distinct primes, \(f^{\prime}\) is divisible by the minimal polynomial for \(T,\) and so

\[f(T)^{r}=0.\]

We are going to construct a sequence of polynomials: \(g_{0}\), \(g_{1}\), \(g_{2},\ldots.\) such that

\[f\!\left(x-\sum\limits_{j=0}^{n}g_{j}f^{j}\right)\]

is divisible by \(f^{n+1},n=0,1,2,\ldots.\) We take \(g_{0}=0\) and then \(f(x-g_{0}f^{0})=f(x) 