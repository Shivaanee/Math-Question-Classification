Let \(p\) be a prime polynomial which divides both \(f\) and \(f^{\prime}\). Then \(p=p_{i}\) for some \(i\). Now \(p_{i}\) divides \(f_{j}\) for \(j\neq i\), and since \(p_{i}\) also divides

\[f^{\prime}=\sum_{j=1}^{k}p_{i}^{\prime}f_{j}\]

we see that \(p_{i}\) must divide \(p_{i}^{\prime}f_{i}\). Therefore \(p_{i}\) divides either \(f_{i}\) or \(p_{i}^{\prime}\). But \(p_{i}\) does not divide \(f_{i}\) since \(p_{1},\,\cdot\,\cdot\,,\,p_{k}\) are distinct. So \(p_{i}\) divides \(p_{i}^{\prime}\). This is not possible, since \(p_{i}^{\prime}\) has degree one less than the degree of \(p_{i}\). We conclude that no prime divides both \(f\) and \(f^{\prime}\), or that, \(f\) and \(f^{\prime}\) are relatively prime.

**Definition**.: _The field \(F\) is called_ **algebraically closed** _if every prime polynomial over \(F\) has degree \(1\)._

To say that \(F\) is algebraically closed means every non-scalar irreducible monic polynomial over \(F\) is of the form \((x-c)\). We have already observed that each such polynomial is irreducible for any \(F\). Accordingly, an equivalent definition of an algebraically closed field is a field \(F\) such that each non-scalar polynomial \(f\) in \(F[x]\) can be expressed in the form

\[f=c(x-c_{1})^{m}\cdots(x-c_{k})^{m}\]

where \(c\) is a scalar, \(c_{l},\,\cdot\,\cdot\,,\,c_{k}\ 