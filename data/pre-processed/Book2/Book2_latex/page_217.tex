

### 11.3 Solving linear equations

Back substitution.We start with an algorithm for solving a set of linear equations, \(Rx=b\), where the \(n\times n\) matrix \(R\) is upper triangular with nonzero diagonal entries (hence, invertible). We write out the equations as

\[R_{11}x_{1}+R_{12}x_{2}+\cdots+R_{1,n-1}x_{n-1}+R_{1n}x_{n} = b_{1}\] \[\vdots\] \[R_{n-2,n-2}x_{n-2}+R_{n-2,n-1}x_{n-1}+R_{n-2,n}x_{n} = b_{n-2}\] \[R_{n-1,n-1}x_{n-1}+R_{n-1,n}x_{n} = b_{n-1}\] \[R_{nn}x_{n} = b_{n}.\]

From the last equation, we find that \(x_{n}=b_{n}/R_{nn}\). Now that we know \(x_{n}\), we substitute it into the second to last equation, which gives us

\[x_{n-1}=(b_{n-1}-R_{n-1,n}x_{n})/R_{n-1,n-1}.\]

We can continue this way to find \(x_{n-2},x_{n-3},\ldots,x_{1}\). This algorithm is known as _back substitution_, since the variables are found one at a time, starting from \(x_{n}\), and we substitute the ones that are known into the remaining equations.

``` given an \(n\times n\) upper triangular matrix \(R\) with nonzero diagonal entries, and an \(n\)-vector \(b\). For \(i=n,\ldots,1\), \(x_{i}=(b_{i}-R_{i,i+1}x_{i+1}-\cdots-R_{i,n}x_{n})/R_{ii}\). ```

**Algorithm 11**Back substitution

(In the first step, with \(i=n\), we have \(x_{n}=b_{n}/R_{nn}\).) The back substitution algorithm computes the solution of \(Rx=b\), _i.e._, \(x=R^{-1}b\). It cannot fail since the divisions in each step are by the diagonal entries of \(R\), which are assumed to be nonzero.

Lower triangular matrices with nonzero diagonal elements are also invertible; we can solve equations with lower triangular invertible matrices using _forward substitution_, the obvious analog of the algorithm given above. In forward substitution, we find \(x_{1}\) first, then \(x_{2}\), and so on.

Complexity of back substitution.The first step requires \(1\) flop (division by \(R_{nn}\)). The next step requires one multiply, one subtraction, and one division, for a total of \(3\) flops. The \(k\)th step requires \(k-1\) multiplies, \(k-1\) subtractions, and one division, for a total of \(2k-1\) flops. The total number of flops for back substitution is then

\[1+3+5+\cdots+(2n-1)=n^{2}\]

flops.

This formula can be obtained from the formula (5.7), or directly derived using a similar argument. Here is the argument for the case when \(n\) is even; a similar