

## Chapter 1 Vectors

In this chapter we introduce vectors and some common operations on them. We describe some settings in which vectors are used.

### 1.1 Vectors

A _vector_ is an ordered finite list of numbers. Vectors are usually written as vertical arrays, surrounded by square or curved brackets, as in

\[\left[\begin{array}{c}-1.1\\ 0.0\\ 3.6\\ -7.2\end{array}\right]\quad\mbox{or}\quad\left(\begin{array}{c}-1.1\\ 0.0\\ 3.6\\ -7.2\end{array}\right).\]

They can also be written as numbers separated by commas and surrounded by parentheses. In this notation style, the vector above is written as

\[(-1.1,0.0,3.6,-7.2).\]

The _elements_ (or _entries_, _coefficients_, _components_) of a vector are the values in the array. The _size_ (also called _dimension_ or _length_) of the vector is the number of elements it contains. The vector above, for example, has size four; its third entry is 3.6. A vector of size \(n\) is called an _\(n\)-vector_. A 1-vector is considered to be the same as a number, _i.e._, we do not distinguish between the 1-vector \([\;1.3\;]\) and the number 1.3.

We often use symbols to denote vectors. If we denote an \(n\)-vector using the symbol \(a\), the \(i\)th element of the vector \(a\) is denoted \(a_{i}\), where the subscript \(i\) is an integer index that runs from 1 to \(n\), the size of the vector.

Two vectors \(a\) and \(b\) are _equal_, which we denote \(a=b\), if they have the same size, and each of the corresponding entries is the same. If \(a\) and \(b\) are \(n\)-vectors, then \(a=b\) means \(a_{1}=b_{1}\), ..., \(a_{n}=b_{n}\).

