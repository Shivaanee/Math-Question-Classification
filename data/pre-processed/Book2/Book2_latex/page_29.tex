

### 1.4 Inner product

The (standard) _inner product_ (also called _dot product_) of two \(n\)-vectors is defined as the scalar

\[a^{T}b=a_{1}b_{1}+a_{2}b_{2}+\cdots+a_{n}b_{n},\]

the sum of the products of corresponding entries. (The origin of the superscript 'T' in the inner product notation \(a^{T}b\) will be explained in chapter 6.) Some other notations for the inner product (that we will not use in this book) are \(\langle a,b\rangle\), \(\langle a|b\rangle\), \((a,b)\), and \(a\cdot b\). (In the notation used in this book, \((a,b)\) denotes a stacked vector of length \(2n\).) As you might guess, there is also a vector _outer product_, which we will encounter later, in SS10.1. As a specific example of the inner product, we have

\[\left[\begin{array}{c}-1\\ 2\\ 2\end{array}\right]^{T}\left[\begin{array}{c}1\\ 0\\ -3\end{array}\right]=(-1)(1)+(2)(0)+(2)(-3)=-7.\]

When \(n=1\), the inner product reduces to the usual product of two numbers.

Properties.The inner product satisfies some simple properties that are easily verified from the definition. If \(a\), \(b\), and \(c\) are vectors of the same size, and \(\gamma\) is a scalar, we have the following.

* _Commutativity_. \(a^{T}b=b^{T}a\). The order of the two vector arguments in the inner product does not matter.
* _Associativity with scalar multiplication_. \((\gamma a)^{T}b=\gamma(a^{T}b)\), so we can write both as \(\gamma a^{T}b\).
* _Distributivity with vector addition_. \((a+b)^{T}c=a^{T}c+b^{T}c\). The inner product can be distributed across vector addition.

These can be combined to obtain other identities, such as \(a^{T}(\gamma b)=\gamma(a^{T}b)\), or \(a^{T}(b+\gamma c)=a^{T}b+\gamma a^{T}c\). As another useful example, we have, for any vectors \(a,b,c,d\) of the same size,

\[(a+b)^{T}(c+d)=a^{T}c+a^{T}d+b^{T}c+b^{T}d.\]

Figure 1.12: The affine combination \((1