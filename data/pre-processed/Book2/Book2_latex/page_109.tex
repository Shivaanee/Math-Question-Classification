
Analysis of Gram-Schmidt algorithm.Let us show that the following hold, for \(i=1,\ldots,k\), assuming \(a_{1},\ldots,a_{k}\) are linearly independent.

1. \(\tilde{q}_{i}\neq 0\), so the linear dependence test in step 2 is not satisfied, and we do not have a divide-by-zero error in step 3.
2. \(q_{1},\ldots,q_{i}\) are orthonormal.
3. \(a_{i}\) is a linear combination of \(q_{1},\ldots,q_{i}\).
4. \(q_{i}\) is a linear combination of \(a_{1},\ldots,a_{i}\).

We show this by induction. For \(i=1\), we have \(\tilde{q}_{1}=a_{1}\). Since \(a_{1},\ldots,a_{k}\) are linearly independent, we must have \(a_{1}\neq 0\), and therefore \(\tilde{q 