

## 11 Exercises

* _Affine combinations of left inverses._ Let \(Z\) be a tall \(m\times n\) matrix with linearly independent columns, and let \(X\) and \(Y\) be left inverses of \(Z\). Show that for any scalars \(\alpha\) and \(\beta\) satisfying \(\alpha+\beta=1\), \(\alpha X+\beta Y\) is also a left inverse of \(Z\). It follows that if a matrix has two different left inverses, it has an infinite number of different left inverses.
* _Left and right inverses of a vector._ Suppose that \(x\) is a nonzero \(n\)-vector with \(n>1\). 1. Does \(x\) have a left inverse? 2. Does \(x\) have a right inverse? In each case, if the answer is yes, give a left or right inverse; if the answer is no, give a specific nonzero vector and show that it is not left- or right-invertible.
* _Matrix cancellation._ Suppose the scalars \(a\), \(x\), and \(y\) satisfy \(ax=ay\). When \(a\neq 0\) we can conclude that \(x=y\); that is, we can cancel the \(a\) on the left of the equation. In this exercise we explore the matrix analog of cancellation, specifically, what properties of \(A\) are needed to conclude \(X=Y\) from \(AX=AY\), for matrices \(A\), \(X\), and \(Y\)? 1. Give an example showing that \(A\neq 0\) is not enough to conclude that \(X=Y\). 2. Show that if \(A\) is left-invertible, we can conclude from \(AX=AY\) that \(X=Y\). 3. Show that if \(A\) is not left-invertible, there are matrices \(X\) and \(Y\) with \(X\neq Y\), and \(AX=AY\).

_Remark_.: Parts (b) and (c) show that you can cancel a matrix on the left when, and only when, the matrix is left-invertible.

* _Transpose of orthogonal matrix._ Let \(U\) be an orthogonal \(n\times n\) matrix. Show that its transpose \(U^{T}\) is also orthogonal.
* _Inverse of a block matrix._ Consider the \((n+1)\times(n+1)\) matrix \[A=\left[\begin{array}{cc}I&a\\ a^{T}&0\end{array}\right],\] where \(a\) is an \(n\)-vector. 1. When is \(A\) invertible? Give your answer in terms of \(a\). Justify your answer. 2. Assuming the condition you found in part (a) holds, give an expression for the inverse matrix \(A^{-1}\).
* _Inverse of a block upper triangular matrix._ Let \(B\) and \(D\) be invertible matrices of sizes \(m\times m\) and \(n\times n\), respectively, and let \(C\) be any \(m\times n\) matrix. Find the inverse of \[A=\left[\begin{array}{cc}B&C\\ 0&D\end{array}\right]\] in terms of \(B^{-1}\), \(C\), and \(D^{-1}\). (The matrix \(A\) is called _block upper triangular_.) _Hints._ First get an idea of what the solution should look like by considering the case when \(B\), \(C\), and \(D\) are scalars. For the matrix case, your goal is to find matrices \(W\), \(X\), \(Y\), \(Z\) (in terms of \(B^{-1}\), \(C\), and \(D^{-1}\)) that satisfy \[A\left[\begin{array}{cc}W&X\\ Y&Z\end{array}\right]=I.\] Use block matrix multiplication to express this as a set of four matrix equations that you can then solve. The method you will find is sometimes called _block back substitution._