has first row

\[\left[\begin{array}{ccc}1&2&3\end{array}\right]\]

(which is a 3-row-vector or a \(1\times 3\) matrix), and second column

\[\left[\begin{array}{c}2\\ 5\end{array}\right]\]

(which is a 2-vector or \(2\times 1\) matrix), also written compactly as \((2,5)\).

Block matrices and submatrices.It is useful to consider matrices whose entries are themselves matrices, as in

\[A=\left[\begin{array}{cc}B&C\\ D&E\end{array}\right],\]

where \(B\), \(C\), \(D\), and \(E\) are matrices. Such matrices are called _block matrices_; the elements \(B\), \(C\), \(D\), and \(E\) are called _blocks_ or _submatrices_ of \(A\). The submatrices can be referred to by their block row and column indices; for example, \(C\) is the 1,2 block of \(A\).

Block matrices must have the right dimensions to fit together. Matrices in the same (block) row must have the same number of rows (_i.e._, the same 'height'); matrices in the same (block) column must have the same number of columns (_i.e._, the same 'width'). In the example above, \(B\) and \(C\) must have the same number of rows, and \(C\) and \(E\) must have the same number of columns. Matrix blocks placed next to each other in the same row are said to be _concatenated_; matrix blocks placed above each other are called _stacked_.

As an example, consider

\[B=\left[\begin{array}{ccc}0&2&3\end{array}\right],\qquad C=\left[\begin{array} []{ccc}-1\end{array}\right],\qquad D=\left[\begin{array}{ccc}2&2&1\\ 1&3&5\end{array}\right],\qquad E=\left[\begin{array}{c}4\\ 4\end{array}\right].\]

Then the block matrix \(A\) above is given by

\[A=\left[\begin{array}{cccc}0&2&3&-1\\ 2&2&1&4\\ 1&3&5&4\end{array}\right].\] (6.1)

(Note that we have dropped the left and right brackets that delimit the blocks. This is similar to the way we drop the brackets in a \(1\times 1\) matrix to get a scalar.)

We can also divide a larger matrix (or vector) into 'blocks'. In this context the blocks are called _submatrices_ of the big matrix. As with vectors, we can use colon notation to denote submatrices. If \(A\) is an \(m\times n\) matrix, and \(p\), \(q\), \(r\), \(s\) are integers with \(1\leq p\leq q\leq m\) and \(1\leq r\leq s\leq n\), then \(A_{p:q,r:s}\) denotes the submatrix

\[A_{p:q,r:s}=\left[\begin{array}{cccc}A_{pr}&A_{p,r+1}&\cdots&A_{ps}\\ A_{p+1,r}&A_{p+1,r+1}&\cdots&A_{p+1,s}\\ \vdots&\vdots&&\vdots\\ A_{qr}&A_{q,r+1}&\cdots&A_{qs}\end{array}\right].\] 