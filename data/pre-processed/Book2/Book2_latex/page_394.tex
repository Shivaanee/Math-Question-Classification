

#### Examples

In this section we list a few applications that reduce to solving a set of nonlinear equations, or a nonlinear least squares problem.

Computing equilibrium points.The idea of an equilibrium, where some type of consumption and generation balance each other, arises in many applications. Consumption and generation depend, often nonlinearly, on the values of some parameters, and the goal is to find values of the parameters that lead to equilibrium. These examples typically have \(m=n\), _i.e._, the system of nonlinear equations is square.

* _Equilibrium prices_. We consider \(n\) commodities or goods, with associated prices given by the \(n\)-vector \(p\). The demand for the \(n\) goods (an \(n\)-vector) is a nonlinear function of the prices, given by \(D(p)\). (In an example on page 150 we described an approximate model for demand that is accurate when the prices change from nominal values by a few percent; here we consider the demand over a large range of prices.) The supply of the goods (an \(n\)-vector) also depends on the prices, and is given by \(S(p)\). (When the price for a good is high, for example, more producers are willing to produce it, so the supply increases.) A set of commodity prices \(p\) is an _equilibrium price vector_ if it results in supply balancing demand, _i.e._, \(S(p)=D(p)\). Finding a set of equilibrium prices is the same as solving the square set of nonlinear equations \[f(p)=S(p)-D(p)=0.\] (The vector \(f(p)\) is called the excess supply, at the set of prices \(p\).) This is shown in figure 18.1 for a simple case with \(n=1\).
* _Chemical equilibrium_. We consider \(n\) chemical species in a solution. The \(n\)-vector \(c\) denotes the concentrations of the \(n\) species. Reactions among the species consume some of them (the reactants) and generate others (the products). The rate of each reaction is a function of the concentrations of its reactants (and other parameters we assume are fixed, like temperature or presence of catalysts). We let \(C(c)\) denote the vector of total consumption of the \(n\) reactants, over all the reactions, and we let \(G(c)\) denote the vector of generation of the \(n\) reactants, over all reactions. A concentration vector \(c\) is in chemical equilibrium if \(C(c)=G(c)\), _i.e._, the rate of consumption of all species balances the rate of generation. Computing a set of equilibrium concentrations is the same as solving the square set of nonlinear equations \[f(c)=C(c)-G(c)=0.\]
* _Mechanical equilibrium_. A mechanical system in 3-D with \(N\) nodes is characterized by the positions of the nodes, given by a \(3N\)-vector \(q\) of the stacked node positions, called the _generalized position_. The net force on each node is