* Find the \(2\times 4\) state feedback gain \(K\) obtained by solving the linear quadratic control problem with \(C=I\), \(\rho=100\), \(T=100\), as described in SS17.2.3. Verify that it is almost the same as the one obtained with \(T=50\).
* Simulate the motion of the Boeing 747 with initial condition \(x_{1}=e_{4}\), under state feedback control (_i.e._, with \(u_{t}=Kx_{t}\)). Plot the state and input variables over the time interval \(t=1,\ldots,120\).
* _Bio-mass estimation._ A bio-reactor is used to grow three different bacteria. We let \(x_{t}\) be the 3-vector of the bio-masses of the three bacteria, at time period (say, hour) \(t\), for \(t=1,\ldots,T\). We believe that they each grow, independently, with growth rates given by the 3-vector \(r\) (which has positive entries). This means that \((x_{t+1})_{i}\approx(1+r_{i})(x_{t})_{i}\), for \(i=1,2,3\). (These equations are approximate; the real rate is not constant.) At every time sample we measure the total bio-mass in the reactor, _i.e._, we have measurements \(y_{t}\approx 1^{T}x_{t}\), for \(t=1,\ldots,T\). (The measurements are not exactly equal to the total mass; there are small measurement errors.) We do not know the bio-masses \(x_{1},\ldots,x_{T}\), but wish to estimate them based on the measurements \(y_{1},\ldots,y_{T}\). Set this up as a linear quadratic state estimation problem as in SS17.3. Identify the matrices \(A_{t}\), \(B_{t}\), and \(C_{t}\). Explain what effect the parameter \(\lambda\) has on the estimated bio-mass trajectory \(\hat{x}_{1},\ldots,\hat{x}_{T}\).

 