* _Market clearing._ Suppose the \(n\)-vector \(q_{i}\) represents the quantities of \(n\) goods or resources produced (when positive) or consumed (when negative) by agent \(i\), for \(i=1,\ldots,N\), so \((q_{5})_{4}=-3.2\) means that agent \(5\) consumes \(3.2\) units of resource \(4\). The sum \(s=q_{1}+\cdots+q_{N}\) is the \(n\)-vector of total net surplus of the resources (or shortfall, when the entries are negative). When \(s=0\), we have a closed market, which means that the total quantity of each resource produced by the agents balances the total quantity consumed. In other words, the \(n\) resources are _exchanged_ among the agents. In this case we say that the _market clears_ (with the resource vectors \(q_{1},\ldots,q_{N}\)).
* _Audio addition._ When \(a\) and \(b\) are vectors representing audio signals over the same period of time, the sum \(a+b\) is an audio signal that is perceived as containing both audio signals combined into one. If \(a\) represents a recording of a voice, and \(b\) a recording of music (of the same length), the audio signal \(a+b\) will be perceived as containing both the voice recording and, simultaneously, the music.
* _Feature differences._ If \(f\) and \(g\) are \(n\)-vectors that give \(n\) feature values for two items, the difference vector \(d=f-g\) gives the difference in feature values for the two objects. For example, \(d_{7}=0\) means that the two objects have the same value for feature \(7\); \(d_{3}=1.67\) means that the first object's third feature value exceeds the second object's third feature value by \(1.67\).
* _Time series._ If \(a\) and \(b\) represent time series of the same quantity, such as daily profit at two different stores, then \(a+b\) represents a time series which is the total daily profit at the two stores. An example (with monthly rainfall) is shown in figure 1.9.
* _Portfolio trading._ Suppose \(s\) is an \(n\)-vector giving the number of shares of \(n\) assets in a portfolio, and \(b\) is an \(n\)-vector giving the number of shares of the assets that we buy (when \(b_{i}\) is positive) or sell (when \(b_{i}\) is negative). After the asset purchases and sales, our portfolio is given by \(s+b\), the sum of the original portfolio vector and the purchase vector \(b\), which is also called the _trade vector_ or _trade list_. (The same interpretation works when the portfolio and trade vectors are given in dollar value.)

Addition notation in computer languages.Some computer languages for manipulating vectors define the sum of a vector and a scalar as the vector obtained by adding the scalar to each element of the vector. This is not standard mathematical notation, however, so we will not use it. Even more confusing, in some computer languages the plus symbol is used to denote concatenation of arrays, which means putting one array after another, as in \((1,2)+(3,4,5)=(1,2,3,4,5)\). While this notation might give a valid expression in some computer languages, it is not standard mathematical notation, and we will not use it in this book. In general, it is very important to distinguish between mathematical notation for vectors (which we use) and the syntax of specific computer languages or software packages for manipulating vectors.

 