There is no need to expand an original feature that is Boolean (_i.e._, takes on two values). If the original Boolean feature is encoded with the values \(0\) and \(1\), and \(0\) is taken as the default value, then the one new feature value will be the same as the original feature value.

As an example of expanding categoricals, consider a model that is used to predict house prices based on various features that include the number of bedrooms, that ranges from \(1\) to \(5\) (say). In the basic regression model, we use the number of bedrooms directly as a feature. In the basic model there is one parameter value that corresponds to value per bedroom; we multiply this parameter by the number of bedrooms to get the contribution to our price prediction. In this model, the price prediction increases (or decreases) by the same amount when we change the number of bedrooms from \(1\) to \(2\) as it does when we change the number of bedrooms from \(4\) to \(5\). If we expand this categorical feature, using \(2\) bedrooms as the default, we have \(4\) Boolean features that correspond to a house having \(1\), \(3\), \(4\), and \(5\) bedrooms. We then have \(4\) parameters in our model, which assign different amounts to add to our prediction for houses with \(1\), \(3\), \(4\), and \(5\) bedrooms, respectively. This more flexible model can capture the idea that a change from \(1\) to \(2\) bedrooms is different from a change from \(4\) to \(5\) bedrooms.

Generalized additive model.We introduce new features that are nonlinear functions of the original features, such as, for each \(x_{i}\), the functions \(\min\{x_{i}+a,0\}\) and \(\max\{x_{i}-b,0\}\), where \(a\) and \(b\) are parameters. These new features are readily interpreted: \(\min\{x_{i}+a,0\}\) is the amount by which feature \(x_{i}\) is below \(-a\), and \(\max\{x_{i}-b,0\}\) is the amount by which feature \(x_{i}\) is above \(b\). A common choice, assuming that \(x_{i}\) has already been standardized, is 