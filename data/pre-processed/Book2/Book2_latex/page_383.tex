\(x_{1},\ldots,x_{T}\). State estimation is widely used in many application areas, including all guidance and navigation systems, such as the Global Positioning System (GPS).

Since we do not know the process or measurement noises, we cannot exactly deduce the state sequence. Instead we will guess or estimate the state sequence \(x_{1},\ldots,x_{T}\) and process noise sequence \(w_{1},\ldots,w_{T-1}\), subject to the requirement that they satisfy the dynamic system model (17.10). When we guess the state sequence, we implicitly guess that the measurement noise is \(v_{t}=y_{t}-C_{t}x_{t}\). We make one fundamental assumption: The process and measurement noises are both small, or at least, not too large.

Our primary objective is the sum of squares of the norms of the measurement residuals,

\[J_{\text{meas}}=\|v_{1}\|^{2}+\cdots+\|v_{T}\|^{2}=\|C_{1}x_{1}-y_{1}\|^{2}+ \cdots+\|C_{T}x_{T}-y_{T}\|^{2}.\]

If this quantity is small, it means that the proposed state sequence guess is consistent with our measurements. Note that the quantities 