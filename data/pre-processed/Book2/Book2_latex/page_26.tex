Scalar multiplication obeys several other laws that are easy to figure out from the definition. For example, it satisfies the associative property: If \(a\) is a vector and \(\beta\) and \(\gamma\) are scalars, we have

\[(\beta\gamma)a=\beta(\gamma a).\]

On the left-hand side we see scalar-scalar multiplication (\(\beta\gamma\)) and scalar-vector multiplication; on the right-hand side we see two scalar-vector products. As a consequence, we can write the vector above as \(\beta\gamma a\), since it does not matter whether we interpret this as \(\beta(\gamma a)\) or \((\beta\gamma)a\).

The associative property holds also when we denote scalar-vector multiplication with the scalar on the right. For example, we have \(\beta(\gamma a)=(\beta a)\gamma\), and consequently we can write both as \(\beta a\gamma\). As a convention, however, this vector is normally written as \(\beta\gamma a\) or as \((\beta\gamma)a\).

If \(a\) is a vector and \(\beta\), \(\gamma\) are scalars, then

\[(\beta+\gamma)a=\beta a+\gamma a.\]

(This is the left-distributive property of scalar-vector multiplication.) Scalar multiplication, like ordinary multiplication, has higher precedence in equations than vector addition, so the right-hand side here, \(\beta a+\gamma a\), means \((\beta a)+(\gamma a)\). It is useful to identify the symbols appearing in this formula above. The \(+\) symbol on the left is addition of scalars, while the \(+\) symbol on the right denotes vector addition. When scalar multiplication is written with the scalar on the right, we have the right-distributive property:

\[a(\beta+\gamma)=a\beta+a\gamma.\]

Scalar-vector multiplication also satisfies another version of the right-distributive property:

\[\beta(a+b)=\beta a+\beta b\]

for any scalar \(\beta\) and any \(n\)-vectors \(a\) and \(b\). In this equation, both of the \(+\) symbols refer to the addition of \(n\)-vectors.

Examples.
* _Displacements._ When a vector \(a\) represents a displacement, and \(\beta>0\), \(\beta a\) is a displacement in the same direction of \(a\), with its magnitude scaled by \(\beta\). When \(\beta<0\), \(\beta a\) represents a displacement in the opposite direction of \(a\), with magnitude scaled by \(|\beta|\). This is illustrated in figure 1.10.
* _Materials requirements._ Suppose the \(n\)-vector \(q\) is the bill of materials for producing one unit of some product, _i.e._, \(q_{i}\) is the amount of raw material required to produce one unit of product. To produce \(\alpha\) units of the product will then require raw materials given by \(\alpha q\). (Here we assume that \(\alpha\geq 0\).)
* _Audio scaling._ If \(a\) is a vector representing an audio signal, the scalar-vector product \(\beta a\) is perceived as the same audio signal, but changed in volume (loudness) by the factor \(|\beta|\). For example, when \(\beta=1/2\) (or \(\beta=-1/2\)), \(\beta a\) is perceived as the same audio signal, but quieter.

 