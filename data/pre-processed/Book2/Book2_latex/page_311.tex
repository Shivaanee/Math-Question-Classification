one of the classes and \(-1\) for all others. It follows that the sum of these \(K\) right-hand sides is the vector with all entries equal to \(2-K\), _i.e._, \((2-K)\mathbf{1}\). Since the mapping from the right-hand sides to the least squares approximate solutions \(\hat{\theta}_{k}\) is linear (see page 229), we have \(\hat{\theta}_{1}+\cdots+\hat{\theta}_{k}=(2-K)a\), where \(a\) is the least squares approximate solution when the right-hand side is \(\mathbf{1}\). Assuming that the first basis function is \(f_{1}(x)=1\), we have \(a=e_{1}\). So we have

\[\hat{\theta}_{1}+\cdots+\hat{\theta}_{K}=(2-K)e_{1},\]

where \(\hat{\theta}_{k}\) is the coefficient vector for distinguishing class \(k\) from the others. Once we have computed \(\hat{\ 