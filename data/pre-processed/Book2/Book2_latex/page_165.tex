\(m\) equations, where \(m\) is the number of different atoms appearing in the chemical reaction. We define the \(m\times p\) matrix \(R\) by

\[R_{ij}=\mbox{number of atoms of type $i$ in $R_{j}$},\quad i=1,\ldots,m,\quad j=1, \ldots,p.\]

(The entries of \(R\) are nonnegative integers.) The matrix \(R\) is interesting; for example, its \(j\)th column gives the chemical formula for reactant \(R_{j}\). We let \(a\) denote the \(p\)-vector with entries \(a_{1},\ldots,a_{p}\). Then, the \(m\)-vector \(Ra\) gives the total number of atoms of each type appearing in the reactants. We define an \(m\times q\) matrix \(P\) in a similar way, so the \(m\)-vector \(Pb\) gives the total number of atoms of each type that appears in the products.

We write the balance condition using vectors and matrices as \(Ra=Pb\). We can express this as

\[\left[\begin{array}{cc}R&-P\end{array}\right]\left[\begin{array}{c}a\\ b\end{array}\right]=0,\]

which is a set of \(m\) homogeneous linear equations.

A simple solution of these equations is \(a=0\), \(b=0\). But we seek a nonzero solution. We can set one of the coefficients, say \(a_{1}\), to be one. (This might cause the other quantities to be fractional-valued.) We can add the condition that \(a_{1}=1\) to our system of linear equations as

\[\left[\begin{array}{cc}R&-P\\ e_{1}^{T}&0\end{array}\right]\left[\begin{array}{c}a\\ b\end{array}\right]=e_{m+1}.\]

Finally, we have a set of \(m+1\) equations in \(p+q\) variables that expresses the requirement that the chemical reaction balances. Finding a solution of this set of equations is called _balancing_ the chemical reaction.

For the example of electrolysis of water described above, we have \(p=1\) reactant (water) and \(q=2\) products (molecular hydrogen and oxygen). The reaction involves \(m=2\) atoms, hydrogen and oxygen. The reactant and product matrices are

\[R=\left[\begin{array}{c}2\\ 1\end{array}\right],\qquad P=\left[\begin{array}{cc}2&0\\ 0&2\end{array}\right].\]

The balancing equations are then

\[\left[\begin{array}{ccc}2&-2&0\\ 1&0&-2\\ 1&0&0\end{array}\right]\left[\begin{array}{c}a_{1}\\ b_{1}\\ b_{2}\end{array}\right]=\left[\begin{array}{c}0\\ 0\\ 1\end{array}\right].\]

These equations are easily solved, and have the solution \((1,1,1/2)\). (Multiplying these coefficients by \(2\) gives the reaction given above.)

Diffusion systems.A _diffusion system_ is a common model that arises in many areas of physics to describe _flows_ and _potentials_. We start with a directed graph with \(n\) nodes and \(m\) edges. (See SS6.1.) Some quantity (like electricity, heat, energy, or mass) can flow across the edges, from one node to another.

With edge \(j\) we associate a flow (rate) \(f_{j}\), which is a scalar; the vector of all \(m\) flows is the flow \(m\)-vector \(f\). The flows \(f_{j}\) can be positive or negative: Positive 