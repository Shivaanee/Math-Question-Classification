Solving constrained least squares problems via QR factorization.We now give a method for solving the constrained least squares problem that generalizes the QR factorization method for least squares problems (algorithm 12.1). We assume that \(A\) and \(C\) satisfy the conditions (16.5).

We start by rewriting the KKT equations (16.4) as

\[2(A^{T}A+C^{T}C)\hat{x}+C^{T}w=2A^{T}b,\qquad C\hat{x}=d\] (16.7)

with a new variable \(w=\hat{z}-2d\). To obtain (16.7) we multiplied the equation \(C\hat{x}=d\) on the left by \(2C^{T}\), then added the result to the first equation of (16.4), and replaced the variable \(\hat{z}\) with \(w+2d\).

Next we use the QR factorization

\[\left[\begin{array}{c}A\\ C\end{array}\right]=QR=\left[\begin{array}{c}Q_{1}\\ Q_{2}\end{array}\right]R\] (16.8)

to simplify (16.7). This factorization exists because the stacked matrix has linearly independent columns, by our assumption (16.5). In (16.8) we also partition \(Q\) in two blocks \(Q_{1}\) and \(Q_{2}\), of size \(m\times n\) and \(p\times n\), respectively. If we make the substitutions \(A=Q_{1}R\), \(C=Q_{2}R\), and \(A^{T}A+C^{T}C=R^{T}R\) in (16.7) we obtain

\[2R^{T}R\hat{x}+R^{T}Q_{2}^{T}w=2R^{T}Q_{1}^{T}b,\qquad Q_{2}R\hat{x}=d.\]

We multiply the first equation on the left by \(R^{-T}\) (which we know exists) to get

\[R\hat{x}=Q_{1}^{T}b-(1/2)Q_{2}^{T}w.\] (16.9)

Substituting this expression into \(Q_{2}R\hat{x}=d\) gives an equation in \(w\):

\[Q_{2}Q_{2}^{T}w=2Q_{2}Q_{1}^{T}b-2d.\] (16.10)

We now use the second part of the assumption (16.5) to show that the matrix \(Q_{2}^{T}=R^{-T}C^{T}\) has linearly independent columns. Suppose \(Q_{2}^{T}z=R^{-T}C^{T}z=0\). Multiplying with \(R^{T}\) gives \(C^{T}z=0\). Since \(C\) has linearly independent rows, this implies \(z=0\), and we conclude that the columns of \(Q_{2}^{T}\) are linearly independent.

The matrix \(Q_{2}^{T}\) therefore has a QR factorization \(Q_{2}^{T}=\tilde{Q}\tilde{R}\). Substituting this into (16.10) gives

\[\tilde{R}^{T}\tilde{R}w=2\tilde{R}^{T}\tilde{Q}^{T}Q_{1}^{T}b-2d,\]

which we can write as

\[\tilde{R}w=2\tilde{Q}^{T}Q_{1}^{T}b-2\tilde{R}^{-T}d.\]

We can use this to compute \(w\), first by computing \(\tilde{R}^{-T}d\) (by forward substitution), then forming the right-hand side, and then solving for \(w\) using back substitution. Once we know \(w\), we can find \(\hat{x}\) from (16.9). The method is summarized in the following algorithm.

 