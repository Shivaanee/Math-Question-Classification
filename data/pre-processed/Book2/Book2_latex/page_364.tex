

**16.11**: _Least distance problem._ A variation on the least norm problem (16.2) is the least distance problem,

\[\begin{array}{ll}\mbox{minimize}&\left\|x-a\right\|^{2}\\ \mbox{subject to}&Cx=d,\end{array}\]

where the \(n\)-vector \(x\) is to be determined, the \(n\)-vector \(a\) is given, the \(p\times n\) matrix \(C\) is given, and the \(p\)-vector \(d\) is given. Show that the solution of this problem is

\[\hat{x}=a-C^{\dagger}(Ca-d),\]

assuming the rows of \(C\) are linearly independent. _Hint._ You can argue directly from the KKT equations for the least distance problem, or solve for the variable \(y=x-a\) instead of \(x\).
**16.12**: _Least norm polynomial interpolation._ (Continuation of exercise 8.7.) Find the polynomial of degree 4 that satisfies the interpolation conditions given in exercise 8.7, and minimizes the sum of the squares of its coefficients. Plot it, to verify that if satisfies the interpolation conditions.
**16.13**: _Steganography via least norm._ In steganography, a secret message is embedded in an image in such a way that the image looks the same, but an accomplice can decode the message. In this exercise we explore a simple approach to steganography that relies on constrained least squares. The secret message is given by a \(k\)-vector \(s\) with entries that are all either \(+1\) or \(-1\) (_i.e._, it is a Boolean vector). The original image is given by the \(n\)-vector \(x\), where \(n\) is usually much larger than \(k\). We send (or publish or transmit) the modified message \(x+z\), where \(z\) is an \(n\)-vector of modifications. We would like \(z\) to be small, so that the original image \(x\) and the modified one \(x+z\) look (almost) the same. Our accomplice decodes the message \(\hat{y}\) by multiplying the modified image by a \(k\times n\) matrix \(D\), which yields the \(k\)-vector \(y=D(x+z)\). The message is then decoded as \(\hat{s}=\mathbf{sign}(y)\). (We write \(\hat{s}\) to show that it is an estimate, and might not be the same as the original.) The matrix \(D\) must have linearly independent rows, but otherwise is arbitrary.

1. _Encoding via least norm._ Let \(\alpha\) be a positive constant. We choose \(z\) to minimize \(\left\|z\right\|^{2}\) subject to \(D(x+z)=\alpha s\). (This guarantees that the decoded message is correct, _i.e._, \(\hat{s}=s\).) Give a formula for \(z\) in terms of \(D^{\dagger}\), \(\alpha\), and \(x\).
2. _Complexity._ What is the complexity of encoding a secret message in an image? (You can assume that \(D^{\dagger}\) is already computed and saved.) What is the complexity of decoding the secret message? About how long would each of these take with a computer capable of carrying out 1 Gflop/s, for \(k=128\) and \(n=512^{2}=262144\) (a \(512\times 512\) image)?
3. _Try it out._ Choose an image \(x\), with entries between 0 (black) and 1 (white), and a secret message \(s\) with \(k\) small compared to \(n\), for example, \(k=128\) for a \(512\times 512\) image. (This corresponds to 16 bytes, which can encode 16 characters, _i.e._, letters, numbers, or punctuation marks.) Choose the entries of \(D\) randomly, and compute \(D^{\dagger}\). The modified image \(x+z\) may have entries outside the range \([0,1]\). We replace any negative values in the modified image with zero, and any values greater than one with one. Adjust \(\alpha\) until the original and modified images look the same, but the secret message is still decoded correctly. (If \(\alpha\) is too small, the clipping of the modified image values, or the round-off errors that occur in the computations, can lead to decoding error, _i.e._, \(\hat{s}\neq s\). If \(\alpha\) is too large, the modification will be visually apparent.) Once you've chosen \(\alpha\), send several different secret messages embedded in several different original images.
**16.14**: _Invertibility of matrix in sparse constrained least squares formulation._ Show that the \((m+n+p)\times(m+n+p)\) coefficient matrix appearing in equation (16.11) is invertible if and only if the KKT matrix is invertible, _i.e._, the conditions (16.5) hold.

