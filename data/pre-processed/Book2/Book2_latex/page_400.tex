

**Example.** The function

\[f(x)=\frac{e^{x}-e^{-x}}{e^{x}+e^{-x}}\] (18.10)

has a unique zero at the origin, _i.e._, the only solution of \(f(x)=0\) is \(x=0\). (This function is called the _sigmoid function_, and will make another appearance later.) The Newton iteration started at \(x^{(1)}=0.95\) converges quickly to the solution \(x=0\). With \(x^{(1)}=1.15\), however, the iterates diverge. This is shown in figures 18.3 and 18.4.

Figure 18.4: Value of \(f(x^{(k)})\) versus iteration number \(k\) for Newton’s method in the example of figure 18.3, started at \(x^{(1)}=0.95\) and \(x^{(1)}=1.15\).

Figure 18.3: The first iterations in the Newton algorithm for solving \(f(x)=0\), for two starting points: \(x^{(1)}=0.95\) and \(x^{(1)}=1.15\).

