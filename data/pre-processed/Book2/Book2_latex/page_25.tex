

### 1.3 Scalar-vector multiplication

Another operation is _scalar multiplication_ or _scalar-vector multiplication_, in which a vector is multiplied by a scalar (_i.e._, number), which is done by multiplying every element of the vector by the scalar. Scalar multiplication is denoted by juxtaposition, typically with the scalar on the left, as in

\[(-2)\left[\begin{array}{c}1\\ 9\\ 6\end{array}\right]=\left[\begin{array}{c}-2\\ -18\\ -12\end{array}\right].\]

Scalar-vector multiplication can also be written with the scalar on the right, as in

\[\left[\begin{array}{c}1\\ 9\\ 6\end{array}\right](1.5)=\left[\begin{array}{c}1.5\\ 13.5\\ 9\end{array}\right].\]

The meaning is the same: It is the vector obtained by multiplying each element by the scalar. A similar notation is \(a/2\), where \(a\) is a vector, meaning \((1/2)a\). The scalar-vector product \((-1)a\) is written simply as \(-a\). Note that \(0\,a=0\) (where the left-hand zero is the scalar zero, and the right-hand zero is a vector zero of the same size as \(a\)).

Properties.By definition, we have \(\alpha a=a\alpha\), for any scalar \(\alpha\) and any vector \(a\). This is called the _commutative property_ of scalar-vector multiplication; it means that scalar-vector multiplication can be written in either order.

Figure 1.9: Average monthly rainfall in inches measured in downtown Los Angeles and San Francisco International Airport, and their sum. Averages are 30-year averages (1981–2010).

