Now we expand the last term:

\[2(Ax-A\hat{x})^{T}(A\hat{x}-b) = 2(x-\hat{x})^{T}A^{T}(A\hat{x}-b)\] \[= -(x-\hat{x})^{T}C^{T}\hat{z}\] \[= -(C(x-\hat{x}))^{T}\hat{z}\] \[= 0,\]

where we use \(2A^{T}(A\hat{x}-b)=-C^{T}\hat{z}\) in the second line and \(Cx=C\hat{x}=d\) in the last line. So we have, exactly as in the case of unconstrained least squares,

\[\|Ax-b\|^{2}=\|A(x-\hat{x})\|^{2}+\|A\hat{x}-b\|^{2},\]

from which we conclude that \(\|Ax-b\|^{2}\geq\|A\hat{x}-b\|^{2}\). So \(\hat{x}\) minimizes \(\|Ax-b\|^{2}\) subject to \(Cx=d\).

It remains to show that for \(x\neq\hat{x}\), we have the strict inequality \(\|Ax-b\|^{2}>\|A\hat{x}-b\|^{2}\), which by the equation above is equivalent to \(\|A(x-\hat{x})\|^{2}>0\). If this is not the case, then \(A(x-\hat{x})=0\). We also have \(C(x-\hat{x})=0\), and so

\[\left[\begin{array}{c}A\\ C\end{array}\right](x-\hat{x})=0.\]

By our assumption that the matrix on the left has linearly independent columns, we conclude that \(x=\hat{x}\).

### 16.3 Solving constrained least squares problems

We can compute the solution (16.6) of the constrained least squares problem by forming and solving the KKT equations (16.4).

**Algorithm 16.1** Constrained least squares via KKT equations

**given** an \(m\times n\) matrix \(A\) and a \(p\times n\) matrix \(C\) that satisfy (16.5), an \(m\)-vector \(b\), and a \(p\)-vector \(d\).

1. _Form Gram matrix._ Compute \(A^{T}A\).
2. _Solve KKT equations._ Solve KKT equations (16.4) by QR factorization and back substitution.

**Algorithm 16.2** Constrained least squares via KKT equations

The second step cannot fail, provided the assumption (16.5) holds. Let us analyze the complexity of this algorithm. The first step, forming the Gram matrix, requires \(mn^{2}\) flops (see page 16.3). The second step requires the solution of a square system of \(n+p\) equations, which costs \(2(n+p)^{3}\) flops, so the total is

\[mn^{2}+2(n+p)^{3}\]

flops. This grows linearly in \(m\) and cubicly in \(n\) and \(p\). The assumption (16.5) implies \(p\leq n\), so in terms of order, \((n+p)^{3}\) can be replaced with \(n^{3}\).

 