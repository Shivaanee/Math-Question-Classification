a \(T\)-vector. In index tracking, the goal is for this return time series to track (or follow) as closely as possible a given target return time series \(r^{\text{tar}}\). We choose \(w\) to minimize the RMS deviation between the target return time series \(r^{\text{tar}}\) and the portfolio return time series \(r\). (Typically the target return is the return of an index, like the Dow Jones Industrial Average or the Russell 3000.) Formulate the index tracking problem as a linearly constrained least squares problem, analogous to (17.2). Give an explicit solution, analogous to (17.3).
* _Portfolio optimization with market neutral constraint._ In the portfolio optimization problem (17.2) the portfolio return time series is the \(T\)-vector \(Rw\). Let \(r^{\text{mkt}}\) denote the \(T\)-vector that gives the return of the whole market over the time periods \(t=1,\dots,T\). (This is the return associated with the total value of the market, _i.e._, the sum over the assets of asset share price times number of outstanding shares.) A portfolio is said to be _market neutral_ if \(Rw\) and \(r^{\text{mkt}}\) are uncorrelated. Explain how to formulate the portfolio optimization problem, with the additional constraint of market neutrality, as a constrained least squares problem. Give an explicit solution, analogous to (17.3).
* _State feedback control of the longitudinal motions of a Boeing 747 aircraft._ In this exercise we consider the control of the longitudinal motions of a Boeing 747 aircraft in steady level flight, at an altitude of 40000 ft, and speed 774 ft/s, which is 528 MPH or 460 knots, around Mach 0.8 at that altitude. (Longitudinal means that we consider climb rate and speed, but not turning or rolling motions.) For modest deviations from these steady state or _trim_ conditions, the dynamics is given by the linear dynamical system \(x_{t+1}=Ax_{t}+Bu_{t}\), with \[A=\left[\begin{array}{cccc}0.99&0.03&-0.02&-0.32\\ 0.01&0.47&4.70&0.00\\ 0.02&-0.06&0.40&0.00\\ 0.01&-0.04&0.72&0.99\end{array}\right],\qquad B=\left[\begin{array}{cccc}0. 01&0.99\\ -3.44&1.66\\ -0.83&0.44\\ -0.47&0.25\end{array}\right],\] with time unit one second. The state 4-vector \(x_{t}\) consists of deviations from the trim conditions of the following quantities.
* \((x_{t})_{1}\) is the velocity along the airplane body axis, in ft/s, with forward motion positive.
* \((x_{t})_{2}\) is the velocity perpendicular to the body axis, in ft/s, with positive down.
* \((x_{t})_{3}\) is the angle of the body axis above horizontal, in units of 0.01 radian (\(0.57^{\circ}\)).
* \((x_{t})_{4}\) is the derivative of the angle of the body axis, called the _pitch rate_, in units of 0.01 radian/s (\(0.57^{\circ}\)/s). The input 2-vector \(u_{t}\) (which we can control) consists of deviations from the trim conditions of the following quantities.
* \((u_{t})_{1}\) is the elevator (control surface) angle, in units of 0.01 radian.
* \((u_{t})_{2}\) is the engine thrust, in units of 10000 lbs.

You do not need to know these details; we mention them only so you know what the entries of \(x_{t}\) and \(u_{t}\) mean.

1. _Open loop trajectory._ Simulate the motion of the Boeing 747 with initial condition \(x_{1}=e_{4}\), in open-loop (_i.e._, with \(u_{t}=0\)). Plot the state variables over the time interval \(t=1,\dots,120\) (two minutes). The oscillation you will see in the open-loop simulation is well known to pilots, and called the _phugoid mode_.
2. _Linear quadratic control._ Solve the linear quadratic control problem with \(C=I\), \(\rho=100\), and \(T=100\), with initial state \(x_{1}=e_{4}\), and desired terminal state \(x^{\text{des}}=0\). Plot the state and input variables over \(t=1,\dots,120\). (For \(t=100,\dots,120\), the state and input variables are zero.) 