With a 6th degree polynomial, the relative RMS test error for both training and test sets is around 0.3. It is a good sign, in terms of generalization ability, that the training and test errors are similar. While there are no guarantees, we can guess that the 6th degree polynomial model will have a relative RMS error around 0.3 on new, unseen data, provided the new, unseen data is sufficiently similar to the test set data.

Cross-validation.Cross-validation is an extension of out-of-sample validation that can be used to get even more confidence in the generalization ability of a model, or more accurately, a choice of basis functions used to construct a model. We divide the original data set into 10 sets, called _folds_. We then fit the model using folds \(1,2,\ldots,9\) as training data, and fold 10 as test data. (So far, this is the same as out-of-sample validation.) We then fit the model using folds \(1,2,\ldots,8,10\) as training data and fold 9 as the test data. We continue, fitting a model for each choice of one of the folds as the test set. We end up with 10 (presumably different) models, and 10 assessments of these models using the fold that was not used to fit the model. (We have described 10-fold cross-validation here; 5-fold cross-validation is also commonly used.) If the test fit performance of these 10 models is similar, we can expect the same, or at least similar, performance on new unseen data. In cross-validation we can also check for _stability_ of the model coefficients. This means that the model coefficients found in the different folds are similar to each other. Stability of the model coefficients further enhances our confidence in the model.

To obtain a single number that is our guess of the prediction RMS error we can expect on new, unseen data, it is common practice to compute the RMS test set error across all 10 folds. For example, if \(\epsilon_{1},\ldots,\epsilon_{10}\) are the RMS prediction errors

Figure 13.11: RMS error versus polynomial degree for the fitting example in figures 13.6 and 13.10. Circles indicate RMS errors on the training set. Squares show RMS errors on the test set.

 