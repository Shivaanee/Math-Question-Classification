* Find a nondiagonal \(2\times 2\) matrix \(A\) that satisfies \(A^{2}=I\). This means that in general there are even more squareroots of \(I_{n}\) than you found in part (a).
* _Circular shift matrices._ Let \(A\) be the \(5\times 5\) matrix \[A=\left[\begin{array}{cccc}0&0&0&0&1\\ 1&0&0&0&0\\ 0&1&0&0&0\\ 0&0&1&0&0\\ 0&0&0&1&0\end{array}\right].\] 1. How is \(Ax\) related to \(x\)? Your answer should be in English. _Hint._ See exercise title. 2. What is \(A^{5}\)? _Hint._ The answer should make sense, given your answer to part (a).
* _Dynamics of an economy._ Let \(x_{1},x_{2},\ldots\) be \(n\)-vectors that give the level of economic activity of a country in years \(1,2,\ldots\), in \(n\) different sectors (like energy, defense, manufacturing). Specifically, \((x_{t})_{i}\) is the level of economic activity in economic sector \(i\) (say, in billions of dollars) in year \(t\). A common model that connects these economic activity vectors is \(x_{t+1}=Bx_{t}\), where \(B\) is an \(n\times n\) matrix. (See exercise 9.2.) Five a matrix expression for the total economic activity across all sectors in year \(t=6\), in terms of the matrix \(B\) and the vector of initial activity levels \(x_{1}\). Suppose you can increase economic activity in year \(t=1\) by some fixed amount (say, one billion dollars) in _one_ sector, by government spending. How should you choose which sector to stimulate so as to maximize the total economic output in year \(t=6\)?
* _Controllability matrix._ Consider the time-invariant linear dynamical system \(x_{t+1}=Ax_{t}+Bu_{t}\), with \(n\)-vector state \(x_{t}\) and \(m\)-vector input \(u_{t}\). Let \(U=(u_{1},u_{2},\ldots,u_{T-1})\) denote the sequence of inputs, stacked in one vector. Find the matrix \(C_{T}\) for which \[x_{T}=A^{T-1}x_{1}+C_{T}U\] holds. The first term is what \(x_{T}\) would be if \(u_{1}=\cdots=u_{T-1}=0\); the second term shows how the sequence of inputs \(u_{1},\ldots,u_{T-1}\) affect \(x_{T}\). The matrix \(C_{T}\) is called the _controllability matrix_ of the linear dynamical system.
* _Linear dynamical system with \(2\times\) down-sampling._ We consider a linear dynamical system with \(n\)-vector state \(x_{t}\), \(m\)-vector input \(u_{t}\), and dynamics given by \[x_{t+1}=Ax_{t}+Bu_{t},\quad t=1,2,\ldots,\] where \(A\) is \(n\times n\) matrix \(A\) and \(B\) is \(n\times m\). Define \(z_{t}=x_{2t-1}\) for \(t=1,2,\ldots\), _i.e._, \[z_{1}=x_{1},\quad z_{2}=x_{3},\quad z_{3}=x_{5},\ldots.\] (The sequence \(z_{t}\) is the original state sequence \(x_{t}\) 'down-sampled' by \(2\times\).) Define the \((2m)\)-vectors \(w_{t}\) as \(w_{t}=(u_{2t-1},u_{2t})\) for \(t=1,2,\ldots\), _i.e._, \[w_{1}=(u_{1},u_{2}),\quad w_{2}=(u_{3},u_{4}),\quad w_{3}=(u_{5},u_{6}),\ldots.\] (Each entry of the sequence \(w_{t}\) is a stack of two consecutive original inputs.) Show that \(z_{t}\), \(w_{t}\) satisfy the linear dynamics equation \(z_{t+1}=Fz_{t}+Gw_{t}\), for \(t=1,2,\ldots\). Give the matrices \(F\) and \(G\) in terms of \(A\) and \(B\).
* _Cycles in a graph._ A _cycle_ of length \(\ell\) in a directed graph is a path of length \(\ell\) that starts and ends at the same vertex. Determine the total number of cycles of length \(\ell=10\) for the directed graph given in the example on page 187. Break this number down into the number of cycles that begin (and end) at vertex 1, vertex 2, ..., vertex 5. (These should add up to the total.) _Hint._ Do not count the cycles by hand.

 