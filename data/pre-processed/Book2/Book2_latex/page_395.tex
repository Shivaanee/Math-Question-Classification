a 3-vector, which depends on \(q\), _i.e._, the node positions. We describe this as a \(3N\)-vector of forces, \(F(q)\). The system is in mechanical equilibrium if the net force on each node is zero, _i.e._, \(F(q)=0\), a set of \(3N\) nonlinear equations in \(3N\) unknowns. (A more complex mechanical equilibrium model takes into account angular displacements and torques at each node.)
* _Nash equilibrium_. We consider a simple setup for a mathematical game. Each of \(n\) competing agents or participants chooses a number \(x_{i}\). Each agent is given a (numerical) reward (say, money) that depends not only on her own choice, but also on the choice of all the other agents. The reward for agent \(i\) is given by the function \(R_{i}(x)\), called the payoff function. Each agent wishes to make a choice that maximizes her reward. This is complicated since the reward depends not only on her choice, but the choices of the other agents. A _Nash equilibrium_ (named after the mathematician John Forbes Nash, Jr.) is a set of choices given by the \(n\)-vector \(x\) where no agent can improve (increase) her reward by changing her choice. Such a choice is argued to be 'stable' since no agent is incented to change her choice. At a Nash equilibrium \(x_{i}\) maximizes \(R_{i}(x)\), so we must have \[\frac{\partial R_{i}}{\partial x_{i}}(x)=0,\quad i=1,\ldots,n.\] This necessary condition for a Nash equilibrium is a square set of nonlinear equations. The idea of a Nash equilibrium is widely used in economics, social science, and engineering. Nash was awarded the Nobel Prize in economics for this work in 1994.

Figure 18.1: Supply and demand as functions of the price, shown on the horizontal axis. They intersect at the point shown as a circle. The corresponding price is the equilibrium price.

 