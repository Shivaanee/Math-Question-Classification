holds for all \(n\)-vectors \(x\), \(y\), and all scalars \(\alpha\), \(\beta\) that satisfy \(\alpha+\beta=1\). In other words, superposition holds for affine combinations of vectors. (For linear functions, superposition holds for any linear combinations of vectors.)

The matrix \(A\) and the vector \(b\) in the representation of an affine function as \(f(x)=Ax+b\) are unique. These parameters can be obtained by evaluating \(f\) at the vectors \(0\), \(e_{1},\ldots,e_{n}\), where \(e_{k}\) is the \(k\)th unit vector in \({\bf R}^{n}\). We have

\[A=\left[\begin{array}{cccc}f(e_{1})-f(0)&f(e_{2})-f(0)&\cdots&f(e_{n})-f(0) \end{array}\right],\qquad b=f(0).\]

Just like affine scalar-valued functions, affine vector-valued functions are often called linear, even though they are linear only when the vector \(b\) is zero.

### 8.2 Linear function models

Many functions or relations between variables that arise in natural science, engineering, and social sciences can be _approximated_ as linear or affine functions. In these cases we refer to the linear function relating the two sets of variables as a _model_ or an _approximation_, to remind us that the relation is only an approximation, and not exact. We give a few examples here.

* _Price elasticity of demand._ Consider \(n\) goods or services with prices given by the \(n\)-vector \(p\), and demands for the goods given by the \(n\)-vector \(d\). A change in prices will induce a change in demands. We let \(\delta^{\rm price}\) be the \(n\)-vector that gives the fractional change in the prices, _i.e._, \(\delta^{\rm price}_{i}=(p^{\rm new}_{i}-p_{i})/p_{i}\), where \(p^{\rm new}\) is the \(n\)-vector of new (changed) prices. We let \(\delta^{\rm dem}\) be the \(n\)-vector that gives the fractional change in the product demands, _i.e._, \(\delta^{\rm dem}_{i}=(d^{\rm new}_{i}-d_{i})/d_{i}\), where \(d^{\rm new}\) is the \(n\)-vector of new demands. A linear demand elasticity model relates these vectors as \(\delta^{\rm dem}=E^{\rm d}\delta^{\rm price}\), where \(E^{\rm d}\) is the \(n\times n\)_demand elasticity matrix_. For example, suppose \(E^{\rm d}_{11}=-0.4\) and \(E^{\rm d}_{21}=0.2\). This means that a \(1\%\) increase in the price of the first good, with other prices kept the same, will cause demand for the first good to drop by \(0.4\%\), and demand for the second good to increase by \(0.2\%\). (In this example, the second good is acting as a _partial substitute_ for the first good.)
* _Elastic deformation._ Consider a steel structure like a bridge or the structural frame of a building. Let \(f\) be an \(n\)-vector that gives the forces applied to the structure at \(n\) specific places (and in \(n\) specific directions), sometimes called a _loading_. The structure will deform slightly due to the loading. Let \(d\) be an \(m\)-vector that gives the displacements (in specific directions) of \(m\) points in the structure, due to the load, _e.g._, the amount of sag at a specific point on a bridge. For small displacements, the relation between displacement and loading is well approximated as linear: \(d=Cf\), where \(C\) is the \(m\times n\)_compliance matrix_. The units of the entries of \(C\) are m/N.

 