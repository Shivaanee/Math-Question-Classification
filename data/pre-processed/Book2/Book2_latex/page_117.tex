

## Chapter 6 Matrices

In this chapter we introduce matrices and some basic operations on them. We give some applications in which they arise.

### 6.1 Matrices

A _matrix_ is a rectangular array of numbers written between rectangular brackets, as in

\[\left[\begin{array}{cccc}0&1&-2.3&0.1\\ 1.3&4&-0.1&0\\ 4.1&-1&0&1.7\end{array}\right].\]

It is also common to use large parentheses instead of rectangular brackets, as in

\[\left(\begin{array}{cccc}0&1&-2.3&0.1\\ 1.3&4&-0.1&0\\ 4.1&-1&0&1.7\end{array}\right).\]

An important attribute of a matrix is its _size_ or _dimensions_, _i.e._, the numbers of rows and columns. The matrix above has 3 rows and 4 columns, so its size is \(3\times 4\). A matrix of size \(m\times n\) is called an \(m\times n\) matrix.

The _elements_ (or _entries_ or _coefficients_) of a matrix are the values in the array. The \(i,j\) element is the value in the \(i\)th row and \(j\)th column, denoted by double subscripts: the \(i,j\) element of a matrix \(A\) is denoted \(A_{ij}\) (or \(A_{i,j}\), when \(i\) or \(j\) is more than one digit or character). The positive integers \(i\) and \(j\) are called the (row and column) _indices_. If \(A\) is an \(m\times n\) matrix, then the row index \(i\) runs from 1 to \(m\) and the column index \(j\) runs from 1 to \(n\). Row indices go from top to bottom, so row 1 is the top row and row \(m\) is the bottom row. Column indices go from left to right, so column 1 is the left column and column \(n\) is the right column.

If the matrix above is \(B\), then we have \(B_{13}=-2.3\), \(B_{32}=-1\). The row index of the bottom left element (which has value 4.1) is 3; its column index is 1.

Two matrices are equal if they have the same size, and the corresponding entries are all equal. As with vectors, we normally deal with matrices with entries that