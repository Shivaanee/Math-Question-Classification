flow is conserved at nodes 2 and 4. For this source, the flow vector

\[x=(0.6,\,0.3,\,0.6,\,-0.1,\,-0.3)\]

satisfies flow conservation, _i.e._, \(Ax+s=0\). This flow can be explained in words: The unit external flow entering node 1 splits three ways, with 0.6 flowing up, 0.3 flowing right, and 0.1 flowing diagonally up (on edge 4). The upward flow on edge 1 passes through node 2, where flow is conserved, and proceeds right on edge 3 towards node 3. The rightward flow on edge 2 passes through node 4, where flow is conserved, and proceeds up on edge 5 to node 3. The one unit of excess flow arriving at node 3 is removed as external flow.

Node potentials.A graph is also useful when we focus on the values of some quantity at each graph vertex or node. Let \(v\) be an \(n\)-vector, often interpreted as a _potential_, with \(v_{i}\) the potential value at node \(i\). We can give a simple interpretation to the matrix-vector product \(u=A^{T}v\). The \(m\)-vector \(u=A^{T}v\) gives the potential differences across the edges: \(u_{j}=v_{l}-v_{k}\), where edge \(j\) goes from node \(k\) to node \(l\).

Dirichlet energy.When the \(m\)-vector \(A^{T}v\) is small, it means that the potential differences across the edges are small. Another way to say this is that the potentials of connected vertices are near each other. A quantitative measure of this is the function of \(v\) given by

\[\mathcal{D}(v)=\|A^{T}v\|^{2}.\]

This function arises in many applications, and is called the _Dirichlet energy_ (or _Laplacian quadratic form_) associated with the graph. It can be expressed as

\[\mathcal{D}(v)=\sum_{\text{edges }(k,l)}(v_{l}-v_{k})^{2},\]

which is the sum of the squares of the potential differences of \(v\) across all edges in the graph. The Dirichlet energy is small when the potential differences across the edges of the graph are small, _i.e._, nodes that are connected by edges have similar potential values.

The Dirichlet energy is used as a measure the non-smoothness (roughness) of a set of node potentials on a graph. A set of node potentials with small Dirichlet energy can be thought of as smoothly varying across the graph. Conversely, a set of potentials with large Dirichlet energy can be thought of as non-smooth or rough. The Dirichlet energy will arise as a measure of roughness in several applications we will encounter later.

As a simple example, consider the potential vector \(v=(1,-1,2,-1)\) for the graph shown in figure 7.2. For this set of potentials, the potential differences across the edges are relatively large, with \(A^{T}v=(-2,-2,3,-1,-3)\), and the associated Dirichlet energy is \(\|A^{T}v\|^{2}=27\). Now consider the potential vector \(v=(1,2,2,1)\). The associated edge potential differences are \(A^{T}v=(1,0,0,-1,-1)\), and the Dirichlet energy has the much smaller value \(\|A^{T}v\|^{2}=3\).

 