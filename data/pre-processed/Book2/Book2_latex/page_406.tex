
Location from range measurements.We illustrate algorithm 18.3 with a small instance of the location from range measurements problem, with five points \(a_{i}\) in a plane, shown in figure 18.9. The range measurements \(\rho_{i}\) are the distances of these points to the 'true' point \((1,1)\), plus some measurement errors. Figure 18.9 also shows the level curves of \(\|f(x)\|^{2}\), and the point \((1.18,0.82)\) (marked with a star) that minimizes \(\|f(x)\|^{2}\). (This point is close to, but not equal to, the 'true' value \((1,1)\), due to the noise added to the range measurements.) Figure 18.10 shows the graph of \(\|f(x)\|\).

We run algorithm 18.3 from three different starting points,

\[x^{(1)}=(1.8,3.5),\qquad x^{(1)}=(2.2,3.5),\qquad x^{(1)}=(3.0,1.5),\]

with \(\lambda^{(1)}=0.1\). Figure 18.11 shows the iterates \(x^{(k)}\) for the three starting points. When started at \((1.8,3.5)\) (blue circles) or \((3.0,1.5)\) (brown diamonds) the algorithm converges to \((1.18,0.82)\), the point that minimizes \(\|f(x)\|^{2}\). When the algorithm is started at \((2.2,3.5)\) the algorithm converges to a non-optimal point \((2.98,2.12)\) (which gives a poor estimate of the 'true' location \((1,1)\)).

The values of \(\|f(x^{(k)})\|^{2}\) and the trust parameter \(\lambda^{(k)}\) during the iteration are shown in figure 18.12. As can be seen from this figure, in the first run of the algorithm (blue circles), \(\lambda^{(k)}\) is increased in the third iteration. Correspondingly, \(x^{(3)}=x^{(4)}\) in figure 18.12. For the second starting point (red squares) \(\lambda^{(k)}\) decreases monotonically. For the third starting point (brown diamonds) \(\lambda^{(k)}\) increases in iterations 2 and 4.

Figure 18.8: Cost function \(\|f(p^{(k)})\|^{2}\) and trust parameter \(\lambda^{(k)}\) versus iteration number \(k\) in the example of figure 18.7.

 