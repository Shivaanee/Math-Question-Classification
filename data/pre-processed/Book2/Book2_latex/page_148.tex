where \(T(b)\) is the \((n+m-1)\times n\) matrix with entries \[T(b)_{ij}=\left\{\begin{array}{ll}b_{i-j+1}&1\leq i-j+1\leq m\\ 0&\text{otherwise}\end{array}\right.\] (7.3) and similarly for \(T(a)\). For example, with \(n=4\) and \(m=3\), we have \[T(b)=\left[\begin{array}{cccc}b_{1}&0&0&0\\ b_{2}&b_{1}&0&0\\ b_{3}&b_{2}&b_{1}&0\\ 0&b_{3}&b_{2}&b_{1}\\ 0&0&b_{3}&b_{2}\\ 0&0&0&b_{3}\end{array}\right],\qquad T(a)=\left[\begin{array}{cccc}a_{1}&0&0 \\ a_{2}&a_{1}&0\\ a_{3}&a_{2}&a_{1}\\ a_{4}&a_{3}&a_{2}\\ 0&a_{4}&a_{3}\\ 0&0&a_{4}\end{array}\right].\] The matrices \(T(b)\) and \(T(a)\) are called _Toeplitz_ matrices (named after the mathematician Otto Toeplitz), which means the entries on any diagonal (_i.e._, indices with \(i-j\) constant) are the same. The columns of the Toeplitz matrix \(T(a)\) are simply shifted versions of the vector \(a\), padded with zero entries.

Variations.Several slightly different definitions of convolution are used in different applications. In one variation, \(a\) and \(b\) are infinite two-sided sequences (and not vectors) with indices ranging from \(-\infty\) to \(\infty\). In another variation, the rows of \(T(a)\) at the top and bottom that do not contain all the coefficients of \(a\) are dropped. (In this version, the rows of \(T(a)\) are shifted versions of the vector \(a\), reversed.) For consistency, we will use the one definition (7.2).

Examples.
* _Time series smoothing._ Suppose the \(n\)-vector \(x\) is a time series, and \(a=(1/3,1/3,1/3)\). Then the \((n+2)\)-vector \(y=a*x\) can be interpreted as a _smoothed_ version of the original time series: for \(i=3,\ldots,n\), \(y_{i}\) is the average of \(x_{i}\), \(x_{i-1}\), \(x_{i-2}\). The time series \(y\) is called the (3-period) _moving average_ of the time series \(x\). Figure 7.6 shows an example.
* _First order differences._ If the \(n\)-vector \(x\) is a time series and \(a=(1,-1)\), the time series \(y=a*x\) gives the first order differences in the series \(x\): \[y=(x_{1},\,x_{2}-x_{1},\,x_{3}-x_{2},\,\ldots,\,x_{n}-x_{n-1},\,-x_{n}).\] (The first and last entries here would be the first order difference if we take \(x_{0}=x_{n+1}=0\).)
* _Audio filtering._ If the \(n\)-vector \(x\) is an audio signal, and \(a\) is a vector (typically with length less than around 0.1 second of real time) the vector \(y=a*x\) is called the _filtered_ audio signal, with _filter coefficients a._ Depending on the coefficients \(a\), \(y\) will be perceived as enhancing or suppressing different frequencies, like the familiar audio tone controls.
* _Communication channel._ In a modern data communication system, a time series \(u\) is transmitted or sent over some channel (_e.g._, electrical, optical, or radio) to a receiver, which receives the time series \(y\). A very common model is that \(y\) and \(u\) are related via convolution: \(y=c*u\), where the vector \(c\) is the _channel impulse response_.

 