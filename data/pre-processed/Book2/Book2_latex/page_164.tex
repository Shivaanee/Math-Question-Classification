

#### Examples

Coefficients of linear combinations.Let \(a_{1},\ldots,a_{n}\) denote the columns of \(A\). The system of linear equations \(Ax=b\) can be expressed as

\[x_{1}a_{1}+\cdots+x_{n}a_{n}=b,\]

_i.e._, \(b\) is a linear combination of \(a_{1},\ldots,a_{n}\) with coefficients \(x_{1},\ldots,x_{n}\). So solving \(Ax=b\) is the same as finding coefficients that express \(b\) as a linear combination of the vectors \(a_{1},\ldots,a_{n}\).

Polynomial interpolation.We seek a polynomial \(p\) of degree at most \(n-1\) that interpolates a set of \(m\) given points \((t_{i},y_{i})\), \(i=1,\ldots,m\). (This means that \(p(t_{i})=y_{i}\).) We can express this as a set of \(m\) linear equations in the \(n\) unknowns \(c\), where \(c\) is the \(n\)-vector of coefficients: \(Ac=y\). Here the matrix \(A\) is the Vandermonde matrix (6.7), and the vector \(c\) is the vector of polynomial coefficients, as described in the example on page 120.

Balancing chemical reactions.A chemical reaction involves \(p\) reactants (molecules) and \(q\) products, and can be written as

\[a_{1}R_{1}+\cdots+a_{p}R_{p}\longrightarrow b_{1}P_{1}+\cdots+b_{q}P_{q}.\]

Here \(R_{1},\ldots,R_{p}\) are the reactants, \(P_{1},\ldots,P_{q}\) are the products, and the numbers \(a_{1},\ldots,a_{p}\) and \(b_{1},\ldots,b_{q}\) are positive numbers that tell us how many of each of these molecules is involved in the reaction. They are typically integers, but can be scaled arbitrarily; we could double all of these numbers, for example, and we still have the same reaction. As a simple example, we have the electrolysis of water,

\[2\mathrm{H}_{2}\mathrm{O}\longrightarrow 2\mathrm{H}_{2}+\mathrm{O}_{2},\]

which has one reactant, water (\(\mathrm{H}_{2}\mathrm{O}\)), and two products, molecular hydrogen (\(\mathrm{H}_{2}\)) and molecular oxygen (\(\mathrm{O}_{2}\)). The coefficients tell us that 2 water molecules create 2 hydrogen molecules and 1 oxygen molecule. The coefficients in a reaction can be multiplied by any nonzero numbers; for example, we could write the reaction above as \(3\mathrm{H}_{2}\mathrm{O}\longrightarrow 3\mathrm{H}_{2}+(3/2)\mathrm{O}_{2}\). By convention reactions are written with all coefficients integers, with least common divisor one.

In a chemical reaction the numbers of constituent atoms must balance. This means that for each atom appearing in any of the reactants or products, the total amount on the left-hand side must equal the total amount on the right-hand side. (If any of the reactants or products is charged, _i.e._, an ion, then the total charge must also balance.) In the simple water electrolysis reaction above, for example, we have 4 hydrogen atoms on the left (2 water molecules, each with 2 hydrogen atoms), and 4 on the right (2 hydrogen molecules, each with 2 hydrogen atoms). The oxygen atoms also balance, so this reaction is balanced.

Balancing a chemical reaction with specified reactants and products, _i.e._, finding the numbers \(a_{1},\ldots,a_{p}\) and \(b_{1},\ldots,b_{q}\), can be expressed as a system of linear equations. We can express the requirement that the reaction balances as a set of