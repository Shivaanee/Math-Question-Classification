

**10.16**: _Covariance matrix._ Consider a list of \(k\)\(n\)-vectors \(a_{1},\ldots,a_{k}\), and define the \(n\times k\) matrix \(A=[a_{1}\;\cdots\;a_{k}]\).

1. Let the \(k\)-vector \(\mu\) give the means of the columns, _i.e._, \(\mu_{i}=\mathbf{avg}(a_{i})\), \(i=1,\ldots,k\). (The symbol \(\mu\) is a traditional one to denote an average value.) Give an expression for \(\mu\) in terms of the matrix \(A\).
2. Let \(\tilde{a}_{1},\ldots,\tilde{a}_{k}\) be the de-meaned versions of \(a_{1},\ldots,a_{k}\), and define \(\tilde{A}\) as the \(n\times k\) matrix \(\tilde{A}=[\tilde{a}_{1}\;\cdots\;\tilde{a}_{k}]\). Give a matrix expression for \(\tilde{A}\) in terms of \(A\) and \(\mu\).
3. The _covariance matrix_ of the vectors \(a_{1},\ldots,a_{k}\) is the \(k\times k\) matrix \(\Sigma=(1/N)\tilde{A}^{T}\tilde{A}\), the Gram matrix of \(\tilde{A}\) multiplied with \(1/N\). Show that \[\Sigma_{ij}=\left\{\begin{array}{ll}\mathbf{std}(a_{i})^{2}&i=j\\ \mathbf{std}(a_{i})\,\mathbf{std}(a_{j})\rho_{ij}&i\neq j\end{array}\right.\] where \(\rho_{ij}\) is the correlation coefficient of \(a_{i}\) and \(a_{j}\). (The expression for \(i\neq j\) assumes that \(\rho_{ij}\) is defined, _i.e._, \(\mathbf{std}(a_{i})\) and \(\mathbf{std}(a_{j})\) are nonzero. If not, we interpret the formula as \(\Sigma_{ij}=0\).) Thus the covariance matrix encodes the standard deviations of the vectors, as well as correlations between all pairs. The correlation matrix is widely used in probability and statistics.
4. Let \(z_{1},\ldots,z_{k}\) be the standardized versions of \(a_{1},\ldots,a_{k}\). (We assume the de-meaned vectors are nonzero.) Derive a matrix expression for \(Z=[z_{1}\;\cdots\;z_{k}]\), the matrix of standardized vectors. Your expression should use \(A\), \(\mu\), and the numbers \(\mathbf{std}(a_{1}),\ldots,\mathbf{std}(a_{k})\).
**10.17**: _Patients and symptoms._ Each of a set of \(N\) patients can exhibit any number of a set of \(n\) symptoms. We express this as an \(N\times n\) matrix \(S\), with \[S_{ij}=\left\{\begin{array}{ll}1&\text{patient $i$ exhibits symptom $j$}\\ 0&\text{patient $i$ does not exhibit symptom $j$}.\end{array}\right.\] Give simple English descriptions of the following expressions. Include the dimensions, and describe the entries.

1. \(S\mathbf{1}\).
2. \(S^{T}\mathbf{1}\).
3. \(S^{T}S\).
4. \(SS^{T}\).
**10.18**: _Students, classes, and majors._ We consider \(m\) students, \(n\) classes, and \(p\) majors. Each student can be in any number of the classes (although we'd expect the number to range from 3 to 6), and can have any number of the majors (although the common values would be 0, 1, or 2). The data about the students' classes and majors are given by an \(m\times n\) matrix \(C\) and an \(m\times p\) matrix \(M\), where \[C_{ij}=\left\{\begin{array}{ll}1&\text{student $i$ is in class $j$}\\ 0&\text{student $i$ is not in class $j$},\end{array}\right.\] and \[M_{ij}=\left\{\begin{array}{ll}1&\text{student $i$ is in major $j$}\\ 0&\text{student $i$ is not in major $j$}.\end{array}\right.\] 1. Let \(E\) be the \(n\)-vector with \(E_{i}\) being the enrollment in class \(i\). Express \(E\) using matrix notation, in terms of the matrices \(C\) and \(M\). 2. Define the \(n\times p\) matrix \(S\) where \(S_{ij}\) is the total number of students in class \(i\) with major \(j\). Express \(S\) using matrix notation, in terms of the matrices \(C\) and \(M\).

