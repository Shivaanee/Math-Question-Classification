

**8.12**: _Quadrature._ Consider a function \(f:{\bf R}\to{\bf R}\). We are interested in _estimating_ the definite integral \(\alpha=\int_{-1}^{1}f(x)\;dx\) based on the value of \(f\) at some points \(t_{1},\ldots,t_{n}\). (We typically have \(-1\leq t_{1}<t_{2}<\cdots<t_{n}\leq 1\), but this is not needed here.) The standard method for estimating \(\alpha\) is to form a weighted sum of the values \(f(t_{i})\):

\[\hat{\alpha}=w_{1}f(t_{1})+\cdots+w_{n}f(t_{n}),\]

where \(\hat{\alpha}\) is our estimate of \(\alpha\), and \(w_{1},\ldots,w_{n}\) are the weights. This method of estimating the value of an integral of a function from its values at some points is a classical method in applied mathematics called _quadrature_. There are many quadrature methods (_i.e._, choices of the points \(t_{i}\) and weights \(w_{i}\)). The most famous one is due to the mathematician Carl Friedrich Gauss, and bears his name.

1. A typical requirement in quadrature is that the approximation should be exact (_i.e._, \(\hat{\alpha}=\alpha\)) when \(f\) is any polynomial up to degree \(d\), where \(d\) is given. In this case we say that the quadrature method has _order_\(d\). Express this condition as a set of linear equations on the weights, \(Aw=b\), assuming the points \(t_{1},\ldots,t_{n}\) are given. _Hint_. If \(\hat{\alpha}=\alpha\) holds for the specific cases \(f(x)=1\), \(f(x)=x\), ..., \(f(x)=x^{d}\), then it holds for any polynomial of degree up to \(d\).
2. Show that the following quadrature methods have order \(1\), \(2\), and \(3\) respectively. * _Trapezoid rule:_\(n=2\), \(t_{1}=-1\), \(t_{2}=1\), and \[w_{1}=1/2,\qquad w_{2}=1/2.\] * _Simpson's rule:_\(n=3\), \(t_{1}=-1\), \(t_{2}=0\), \(t_{3}=1\), and \[w_{1}=1/3,\qquad w_{2}=4/3,\qquad w_{3}=1/3.\] (Named after the mathematician Thomas Simpson.) * _Simpson's_\(3/8\)_rule:_\(n=4\), \(t_{1}=-1\), \(t_{2}=-1/3\), \(t_{3}=1/3\), \(t_{4}=1\), \[w_{1}=1/4,\qquad w_{2}=3/4,\qquad w_{3}=3/4,\qquad w_{4}=1/4.\]
**8.13**: _Portfolio sector exposures._ (See exercise 1.14.) The \(n\)-vector \(h\) denotes a portfolio of investments in \(n\) assets, with \(h_{i}\) the dollar value invested in asset \(i\). We consider a set of \(m\) industry sectors, such as pharmaceuticals or consumer electronics. Each asset is assigned to one of these sectors. (More complex models allow for an asset to be assigned to more than one sector.) The _exposure_ of the portfolio to sector \(i\) is defined as the sum of investments in the assets in that sector. We denote the sector exposures using the \(m\)-vector \(s\), where \(s_{i}\) is the portfolio exposure to sector \(i\). (When \(s_{i}=0\), the portfolio is said to be _neutral_ to sector \(i\).) An investment advisor specifies a set of desired sector exposures, given as the \(m\)-vector \(s^{\rm des}\). Express the requirement \(s=s^{\rm des}\) as a set of linear equations of the form \(Ah=b\). (You must describe the matrix \(A\) and the vector \(b\).) _Remark._ A typical practical case involves \(n=1000\) assets and \(m=50\) sectors. An advisor might specify \(s_{i}^{\rm des}=0\) if she does not have an opinion as how companies in that sector will do in the future; she might specify a positive value for \(s_{i}^{\rm des}\) if she thinks the companies in that sector will do well (_i.e._, generate positive returns) in the future, and a negative value if she thinks they will do poorly.
**8.14**: _Affine combinations of solutions of linear equations._ Consider the set of \(m\) linear equations in \(n\) variables \(Ax=b\), where \(A\) is an \(m\times n\) matrix, \(b\) is an \(m\)-vector, and \(x\) is the \(n\)-vector of variables. Suppose that the \(n\)-vectors \(z_{1},\ldots,z_{k}\) are solutions of this set of equations, _i.e._, satisfy \(Az_{i}=b\). Show that if the coefficients \(\alpha_{1},\ldots,\alpha_{k}\) satisfy \(\alpha_{1}+\cdots+\alpha_{k}=1\), then the affine combination

\[w=\alpha_{1}z_{1}+\cdots+\alpha_{k}z_{k}\]

is a solution of the linear equations, _i.e._, satisfies \(Aw=b\). In words: Any affine combination of solutions of a set of linear equations is also a solution of the equations.

