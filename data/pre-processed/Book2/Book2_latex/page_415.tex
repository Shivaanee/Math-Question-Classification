
\begin{table}
\begin{tabular}{c c c c c c} \hline \hline  & \multicolumn{3}{c}{Prediction} & \multicolumn{3}{c}{Prediction} \\ \cline{2-5} Outcome & \(\hat{y}=+1\) & \(\hat{y}=-1\) & Total \\ \hline \(y=+1\) & 5627 & 296 & 5923 \\ \(y=-1\) & 148 & 53929 & 54077 \\ All & 5775 & 54225 & 60000 \\ \hline \hline \end{tabular} 
\begin{tabular}{c c c c c} \hline \hline  & \multicolumn{3}{c}{Prediction} \\ \cline{2-5} Outcome & \(\hat{y}=+1\) & \(\hat{y}=-1\) & Total \\ \hline \(y=+1\) & 945 & 35 & 980 \\ \(y=-1\) & 40 & 8980 & 9020 \\ All & 985 & 9015 & 10000 \\ \hline \hline \end{tabular}
\end{table}
Table 18.1: Confusion matrices for a Boolean classifier to recognize the digit zero. The table on the left is for the training set. The table on the right is for the test set.

Figure 18.18: The distribution of the values of \(\tilde{f}(x^{(i)})\) used in the Boolean classifier (14.1) for recognizing the digit zero. The function \(\tilde{f}\) was computed by solving the nonlinear least squares problem (18.17).

 