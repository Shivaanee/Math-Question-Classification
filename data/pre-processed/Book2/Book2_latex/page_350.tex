(find \(x\) that minimizes \(\|Ax-b\|^{2}\)). Indeed each of these problems can be considered a special case of the constrained least squares problem (16.1).

The constrained least squares problem can also be thought of as a limit of a bi-objective least squares problem, with primary objective \(\|Ax-b\|^{2}\) and secondary objective \(\|Cx-d\|^{2}\). Roughly speaking, we put infinite weight on the second objective, so that any nonzero value is unacceptable (which forces \(x\) to satisfy \(Cx=d\)). So we would expect (and it can be verified) that minimizing the weighted objective

\[\|Ax-b\|^{2}+\lambda\|Cx-d\|^{2},\]

for a very large value of \(\lambda\) yields a vector close to a solution of the constrained least squares problem (16.1). We will encounter this idea again in chapter 19, when we consider the nonlinear constrained least squares problem.

Example.In figure 16.1 we fit a _piecewise-polynomial_ function \(\hat{f}(x)\) to a set of \(N=140\) points \((x_{i},y_{i})\) in the plane. The function \(\hat{f}(x)\) is defined as

\[\hat{f}(x)=\left\{\begin{array}{ll}p(x)&x\leq a\\ q(x)&x>a,\end{array}\right.\]

with \(a\) given, and \(p(x)\) and \(q(x)\) polynomials of degree three or less,

\[p(x)=\theta_{1}+\theta_{2}x+\theta_{3}x^{2}+\theta_{4}x^{3},\qquad q(x)=\theta _{5}+\theta_{6}x+\theta_{7}x^{2}+\theta_{8}x^{3}.\]

We also impose the condition that \(p(a)=q(a)\) and \(p^{\prime}(a)=q^{\prime}(a)\), so that \(\hat{f}(x 