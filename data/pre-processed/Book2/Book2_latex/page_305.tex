the true positive rate on the vertical axis and false positive rate on the horizontal axis. The name comes from radar systems deployed during World War II, where \(y=+1\) means that an enemy vehicle (or ship or airplane) is present, and \(\hat{y}=+1\) means that an enemy vehicle is detected.

Example.We examine the skewed threshold least squares classifier (14.3) for the example described above, where we attempt to detect whether or not a handwritten digit is zero. Figure 14.5 shows how the error, true positive, and false positive rates depend on the decision threshold \(\alpha\), for the training set data. We can see that as \(\alpha\) increases, the true positive rate decreases, as does the false positive rate. We can see that for this particular case the total error rate is minimized by choosing \(\alpha=-0.1\), which gives error rate \(1.4\%\), slightly lower than the basic least squares classifier. The limiting cases when \(\alpha\) is negative enough, or positive enough, are readily understood. When \(\alpha\) is very negative, the prediction is always \(\hat{y}=+1\); our error rate is then the fraction of the data set with \(y=-1\). When \(\alpha\) is very positive, the prediction is always \( 