* _ZIP code clustering._ Suppose that \(x_{i}\) is a vector giving \(n\) quantities or statistics for the residents of ZIP code \(i\), such as numbers of residents in various age groups, household size, education statistics, and income statistics. (In this example \(N\) is around 40000.) A clustering algorithm might be used to cluster the 40000 ZIP codes into, say, \(k=100\) groups of ZIP codes with similar statistics.
* _Student clustering._ Suppose the vector \(x_{i}\) gives the detailed grading record of student \(i\) in a course, _i.e._, her grades on each question in the quizzes, homework assignments, and exams. A clustering algorithm might be used to cluster the students into \(k=10\) groups of students who performed similarly.
* _Survey response clustering._ A group of \(N\) people respond to a survey with \(n\) questions. Each question contains a statement, such as 'The movie was too long', followed by some ordered options such as Strongly Disagree, Disagree, Neutral, Agree, Strongly Agree. (This is called a _Likert scale_, named after the psychologist Rensis Likert.) Suppose the \(n\)-vector \(x_{i}\) encodes the selections of respondent \(i\) on the \(n\) questions, using the numerical coding \(-2\), \(-1\), \(0\), \(+1\), \(+2\) for the responses above. A clustering algorithm can be used to cluster the respondents into \(k\) groups, each with similar responses to the survey.
* _Weather zones._ For each of \(N\) counties we have a 24-vector \(x_{i}\) that gives the average monthly temperature in the first 12 entries and the average monthly rainfall in the last 12 entries. (We can standardize all the temperatures, and all the rainfall data, so they have a typical range between \(-1\) and \(+1\).) The vector \(x_{i}\) summarizes the annual weather pattern in county \(i\). A clustering algorithm can be used to cluster the counties into \(k\) groups that have similar weather patterns, called _weather zones_. This clustering can be shown on a map, and used to recommend landscape plantings depending on zone.
* _Daily energy use patterns._ The 24-vectors \(x_{i}\) give the average (electric) energy use for \(N\) customers over some period (say, a month) for each hour of the day. A clustering algorithm partitions customers into groups, each with similar patterns of daily energy consumption. We might expect a clustering algorithm to 'discover' which customers have a swimming pool, an electric water heater, or solar panels.
* _Financial sectors._ For each of \(N\) companies we have an \(n\)-vector whose components are financial and business attributes such as total capitalization, quarterly returns and risks, trading volume, profit and loss, or dividends paid. (These quantities would typically be scaled so as to have similar ranges of values.) A clustering algorithm would group companies into _sectors_, _i.e._, groups of companies with similar attributes.

In each of these examples, it would be quite informative to know that the vectors can be well clustered into, say, \(k=5\) or \(k=37\) groups. This can be used to develop insight into the data. By examining the clusters we can often understand them, and assign labels or descriptions to them.

 