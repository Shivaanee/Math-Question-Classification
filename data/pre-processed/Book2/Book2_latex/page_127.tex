* _Commutativity._\(A+B=B+A\)_._
* _Associativity._\((A+B)+C=A+(B+C)\). We therefore write both as \(A+B+C\).
* _Addition with zero matrix._\(A+0=0+A=A\). Adding the zero matrix to a matrix has no effect.
* _Transpose of sum._\((A+B)^{T}=A^{T}+B^{T}\). The transpose of a sum of two matrices is the sum of their transposes.

#### Scalar-matrix multiplication

Scalar multiplication of matrices is defined in a similar way as for vectors, and is done by multiplying every element of the matrix by the scalar. For example

\[(-2)\left[\begin{array}{cc}1&6\\ 9&3\\ 6&0\end{array}\right]=\left[\begin{array}{cc}-2&-12\\ -18&-6\\ -12&0\end{array}\right].\]

As with scalar-vector multiplication, the scalar can also appear on the right. Note that \(0\,A=0\) (where the left-hand zero is the scalar zero, and the right-hand \(0\) is the zero matrix).

Several useful properties of scalar multiplication follow directly from the definition. For example, \((\beta A)^{T}=\beta(A^{T})\) for a scalar \(\beta\) and a matrix \(A\). If \(A\) is a matrix and \(\beta\), \(\gamma\) are scalars, then

\[(\beta+\gamma)A=\beta A+\gamma A,\qquad(\beta\gamma)A=\beta(\gamma A).\]

It is useful to identify the symbols appearing in these two equations. The \(+\) symbol on the left of the left-hand equation is addition of scalars, while the \(+\) symbol on the right of the left-hand equation denotes matrix addition.

 