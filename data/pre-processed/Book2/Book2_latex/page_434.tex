

**Algorithm 19.2** Augmented Lagrangian algorithm

**given** differentiable functions \(f:\mathbf{R}^{n}\to\mathbf{R}^{m}\) and \(g:\mathbf{R}^{n}\to\mathbf{R}^{p}\), and an initial point \(x^{(1)}\). Set \(z^{(1)}=0\), \(\mu^{(1)}=1\).

For \(k=1,2,\ldots,k^{\max}\)

1. _Solve unconstrained nonlinear least squares problem._ Set \(x^{(k+1)}\) to be the (approximate) minimizer of \[\left\|f(x)\right\|^{2}+\mu^{(k)}\|g(x)+z^{(k)}/(2\mu^{(k)})\|^{2}\] using Levenberg-Marquardt algorithm, starting from initial point \(x^{(k)}\).
2. _Update \(z^{(k)}\)._ \[z^{(k+1)}=z^{(k)}+2\mu^{(k)}g(x^{(k+1)}).\]
3. _Update \(\mu^{(k)}\)._ \[\mu^{(k+1)}=\left\{\begin{array}{ll}\mu^{(k)}&\|g(x^{(k+1)})\|<0.25\|g(x^{(k )})\|\\ 2\mu^{(k)}&\|g(x^{(k+1)})\|\geq 0.25\|g(x^{(k)})\|.\end{array}\right.\]

The augmented Lagrangian algorithm is stopped early if \(g(x^{(k)})\) is very small. Note that due to our particular choice of how \(z^{(k)}\) is updated, the iterate \(x^{(k+1)}\) (almost) satisfies the optimality condition (19.4) with \(z^{(k+1)}\).

The augmented Lagrangian algorithm is not much more complicated than the penalty algorithm, but it works much better in practice. In part this is because the penalty parameter \(\mu^{(k)}\) does not need to increase as much as the algorithm proceeds.

**Example.** We consider an example with two variables and

\[f(x_{1},x_{2})=\left[\begin{array}{c}x_{1}+\exp(-x_{2})\\ x_{1}^{2}+2x_{2}+1\end{array}\right],\qquad g(x_{1},x_{2})=x_{1}+x_{1}^{3}+x_{ 2}+x_{2}^{2}.\]

Figure 19.1 shows the contour lines of the cost function \(\|f(x)\|^{2}\) (solid lines) and the constraint function \(g(x)\) (dashed lines). The point \(\hat{x}=(0,0)\) is optimal with corresponding Lagrange multiplier \(\hat{z}=-2\). One can verify that \(g(\hat{x})=0\) and

\[2Df(\hat{x})^{T}f(\hat{x})+Dg(\hat{x})^{T}\hat{z}=2\left[\begin{array}{cc} 1&0\\ -1&2\end{array}\right]\left[\begin{array}{c}1\\ 1\end{array}\right]-2\left[\begin{array}{c}1\\ 1\end{array}\right]=0.\]

The circle at \(x=(-0.666,-0.407)\) indicates the position of the unconstrained minimizer of \(\|f(x)\|^{2}\).

The augmented Lagrangian algorithm is started from the point \(x^{(1)}=(0.5,-0.5)\). Figure 19.2 illustrates the first six iterations. The solid lines are the contour lines for \(L_{\mu}(x,z^{(k)})\), the augmented Lagrangian with the current value of the Lagrange multiplier. For comparison, we also show in figure 19.3 the first six iterations of the penalty algorithm, started from the same point. The solid lines are the contour lines of \(\|f(x)\|^{2}+\mu^{(k)}\|g(x)\|^{2}\).

In figure 19.4 we show how the algorithms converge. The horizontal axis is the cumulative number of Levenberg-Marquardt iterations. Each of these requires