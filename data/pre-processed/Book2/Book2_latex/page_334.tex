
Line integral measurements.For simplicity we will assume that each beam is a single line, and that the received value \(y_{i}\) is the integral of the quantity over the region, plus some measurement noise. (The same method can be used when more complex beam shapes are used.) We consider the 2-D case.

Let \(d(x,y)\) denote the density (say) at the position \((x,y)\) in the region. (Here \(x\) and \(y\) are the scalar 2-D coordinates, not the vectors \(x\) and \(y\) in the estimation problem.) We assume that \(d(x,y)=0\) outside the region of interest. A line through the region is defined by the set of points

\[p(t)=(x_{0},y_{0})+t(\cos\theta,\sin\theta),\]

where \((x_{0},y_{0})\) denotes a (base) point on the line, and \(\theta\) is the angle of the line with respect to the \(x\)-axis. The parameter \(t\) gives the distance along the line from the point \((x_{0},y_{0})\). The _line integral_ of \(d\) is given by

\[\int_{-\infty}^{\infty}d(p(t))\;dt.\]

We assume that \(m\) lines are specified (_e.g._, by their base points and angles), and the measurement \(y_{i}\) is the line integral of \(d\), plus some noise, which is presumed small.

We divide the region of interest into \(n\) pixels (or voxels in the 3-D case), and assume that the density has the constant value \(x_{i}\) over pixel \(i\). Figure 15.7 illustrates this for a simple example with \(n=25\) pixels. (In applications the number of pixels or voxels is in the thousands or millions.) The line integral is then given by the sum of \(x_{i}\) (the density in pixel \(i\)) times the length of the intersection of the line with pixel \(i\). In figure 15.7, with the pixels numbered row-wise starting at the top left corner, with width and height one, the line integral for the line shown is

\[1.06x_{16}+0.80x_{17}+0.27x_{12}+1.06x_{13}+1.06x_{14}+0.53x_{15}+0.54x_{10}.\]

The coefficient of \(x_{i}\) is the length of the intersection of the line with pixel \(i\).

Measurement model.We can express the vector of \(m\) line integral measurements, without the noise, as \(Ax\), where the \(m\times n\) matrix \(A\) has entries

\[A_{ij}=\text{length of line $i$ in pixel $j$},\quad i=1,\ldots,m,\quad j=1,\ldots,n,\]

with \(A_{ij}=0\) if line \(i\) does not intersect voxel \(j\).

Tomographic reconstruction.In tomography, estimation or inversion is often called _tomographic reconstruction_ or _tomographic inversion_.

The objective term \(\|Ax-y\|^{2}\) is the sum of squares of the residual between the predicted (noise-free) line integrals \(Ax\) and the actual measured line integrals \(y\). Regularization terms capture prior information or assumptions about the voxel values, for example, that they vary smoothly over the region. A simple regularizer commonly used is the Dirichlet energy (see page 135) associated with the graph that connects each voxel to its 6 neighbors (in the 3-D case) or its 4 neighbors (in the 2-D case). Using the Dirichlet energy as a regularizer is also called Laplacian regularization.

 