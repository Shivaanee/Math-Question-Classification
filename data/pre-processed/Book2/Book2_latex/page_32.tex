* _Discounted total._ Let \(c\) be an \(n\)-vector representing a cash flow, with \(c_{i}\) the cash received (when \(c_{i}>0\)) in period \(i\). Let \(d\) be the \(n\)-vector defined as \[d=(1,1/(1+r),\ldots,1/(1+r)^{n-1}),\] where \(r\geq 0\) is an interest rate. Then \[d^{T}c=c_{1}+c_{2}/(1+r)+\cdots+c_{n}/(1+r)^{n-1}\] is the discounted total of the cash flow, _i.e._, its _net present value_ (NPV), with interest rate \(r\).
* _Portfolio value._ Suppose \(s\) is an \(n\)-vector representing the holdings in shares of a portfolio of \(n\) different assets, with negative values meaning short positions. If \(p\) is an \(n\)-vector giving the prices of the assets, then \(p^{T}s\) is the total (or net) value of the portfolio.
* _Portfolio return._ Suppose \(r\) is the vector of (fractional) returns of \(n\) assets over some time period, _i.e._, the asset relative price changes \[r_{i}=\frac{p_{i}^{\text{final}}-p_{i}^{\text{initial}}}{p_{i}^{\text{initial}} },\quad i=1,\ldots,n,\] where \(p_{i}^{\text{initial}}\) and \(p_{i}^{\text{final}}\) are the (positive) prices of asset \(i\) at the beginning and end of the investment period. If \(h\) is an \(n\)-vector giving our portfolio, with \(h_{i}\) denoting the dollar value of asset \(i\) held, then the inner product \(r^{T}h\) is the total return of the portfolio, in dollars, over the period. If \(w\) represents the fractional (dollar) holdings of our portfolio, then \(r^{T}w\) gives the total return of the portfolio. For example, if \(r^{T}w=0.09\), then our portfolio return is 9%. If we had invested $10000 initially, we would have earned $900.
* _Document sentiment analysis._ Suppose the \(n\)-vector \(x\) represents the histogram of word occurrences in a document, from a dictionary of \(n\) words. Each word in the dictionary is assigned to one of three sentiment categories: _Positive_, _Negative_, and _Neutral_. The list of positive words might include 'nice' and 'superb'; the list of negative words might include 'bad' and 'terrible'. Neutral words are those that are neither positive nor negative. We encode the word categories as an \(n\)-vector \(w\), with \(w_{i}=1\) if word \(i\) is positive, with \(w_{i}=-1\) if word \(i\) is negative, and \(w_{i}=0\) if word \(i\) is neutral. The number \(w^{T}x\) gives a (crude) measure of the sentiment of the document.

### 1.5 Complexity of vector computations

Computer representation of numbers and vectors.Real numbers are stored in computers using _floating point format_, which represents a real number using a block of 64 _bits_ (0s and 1s), or 8 _bytes_ (groups of 8 bits). Each of the \(2^{64}\) possible sequences of bits corresponds to a specific real number. The floating point numbers 