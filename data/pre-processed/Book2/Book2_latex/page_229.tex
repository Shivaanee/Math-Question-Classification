

**11.12**: _Combinations of invertible matrices._ Suppose the \(n\times n\) matrices \(A\) and \(B\) are both invertible. Determine whether each of the matrices given below is invertible, without any further assumptions about \(A\) and \(B\).

1. \(A+B\).
2. \(\left[\begin{array}{cc}A&0\\ 0&B\end{array}\right]\).
3. \(\left[\begin{array}{cc}A&A+B\\ 0&B\end{array}\right]\).
4. \(ABA\).
**11.13**: _Another left inverse._ Suppose the \(m\times n\) matrix \(A\) is tall and has linearly independent columns. One left inverse of \(A\) is the pseudo-inverse \(A^{\dagger}\). In this problem we explore another one. Write \(A\) as the block matrix

\[A=\left[\begin{array}{c}A_{1}\\ A_{2}\end{array}\right],\]

where \(A_{1}\) is \(n\times n\). We assume that \(A_{1}\) is invertible (which need not happen in general). Show that the following matrix is a left inverse of \(A\):

\[\tilde{A}=\left[\begin{array}{cc}A_{1}^{-1}&0_{n\times(m-n)}\end{array} \right].\]
**11.14**: _Middle inverse._ Suppose \(A\) is an \(n\times p\) matrix and \(B\) is a \(q\times n\) matrix. If a \(p\times q\) matrix \(X\) exists that satisfies \(AXB=I\), we call it a _middle inverse_ of the pair \(A\), \(B\). (This is not a standard concept.) Note that when \(A\) or \(B\) is an identity matrix, the middle inverse reduces to the right or left inverse, respectively.

1. Describe the conditions on \(A\) and \(B\) under which a middle inverse \(X\) exists. Give your answer using only the following four concepts: Linear independence of the rows or columns of \(A\), and linear independence of the rows or columns of \(B\). You must justify your answer.
2. Give an expression for a middle inverse, assuming the conditions in part (a) hold.
**11.15**: _Invertibility of population dynamics matrix._ Consider the population dynamics matrix

\[A=\left[\begin{array}{cccccc}b_{1}&b_{2}&\cdots&b_{99}&b_{100}\\ 1-d_{1}&0&\cdots&0&0\\ 0&1-d_{2}&\cdots&0&0\\ \vdots&\vdots&\ddots&\vdots&\vdots\\ 0&0&\cdots&1-d_{99}&0\end{array}\right],\]

where \(b_{i}\geq 0\) are the birth rates and \(0\leq d_{i}\leq 1\) are death rates. What are the conditions on \(b_{i}\) and \(d_{i}\) under which \(A\) is invertible? (If the matrix is never invertible or always invertible, say so.) Justify your answer.
**11.16**: _Inverse of running sum matrix._ Find the inverse of the \(n\times n\) running sum matrix,

\[S=\left[\begin{array}{cccccc}1&0&\cdots&0&0\\ 1&1&\cdots&0&0\\ \vdots&\vdots&\ddots&\vdots&\vdots\\ 1&1&\cdots&1&0\\ 1&1&\cdots&1&1\end{array}\right].\]

Does your answer make sense?