a common notational hint that it is an approximation of the function \(f\). (The approximation is named after the mathematician Brook Taylor.)

The first-order Taylor approximation \(\hat{f}(x)\) is a very good approximation of \(f(x)\) when all \(x_{i}\) are near the associated \(z_{i}\). Sometimes \(\hat{f}\) is written with a second vector argument, as \(\hat{f}(x;z)\), to show the point \(z\) at which the approximation is developed. The first term in the Taylor approximation is a constant; the other terms can be interpreted as the contributions to the (approximate) change in the function value (from \(f(z)\)) due to the changes in the components of \(x\) (from \(z\)).

Evidently \(\hat{f}\) is an affine function of \(x\). (It is sometimes called the _linear approximation_ of \(f\) near \(z\), even though it is in general affine, and not linear.) It can be written compactly using inner product notation as

\[\hat{f}(x)=f(z)+\nabla f(z)^{T}(x-z),\] (2.5)

where \(\nabla f(z)\) is an \(n\)-vector, the _gradient of \(f\)_ (at the point \(z\)),

\[\nabla f(z)=\left[\begin{array}{c}\frac{\partial f}{\partial x_{1}}(z)\\ \vdots\\ \frac{\partial f}{\partial x_{n}}(z)\end{array}\right].\] (2.6)

The first term in the Taylor approximation (2.5) is the constant \(f(z)\), the value of the function when \(x=z\). The second term is the inner product of the gradient of \(f\) at \(z\) and the _deviation_ or _perturbation_ of \(x\) from \(z\), _i.e._, \(x-z\).

We can express the first-order Taylor approximation as a linear function plus a constant,

\[\hat{f}(x)=\nabla f(z)^{T}x+(f(z)-\nabla f(z)^{T}z),\]

but the form (2.5) is perhaps easier to interpret.

The first-order Taylor approximation gives us an organized way to construct an affine approximation of a function \(f:\mathbf{R}^{n}\to\mathbf{R}\), near a given point \(z\), when there is a formula or equation that describes \(f\), and it is differentiable. A simple example, for \(n=1\), is shown in figure 2.3. Over the full \(x\)-axis scale shown, the Taylor approximation \(\hat{f}\) does not give a good approximation of the function \(f\). But for \(x\) near \(z\), the Taylor approximation is very good.

Example.Consider the function \(f:\mathbf{R}^{2}\to\mathbf{R}\) given by \(f(x)=x_{1}+\exp(x_{2}-x_{1})\), which is not linear or affine. To find the Taylor approximation \(\hat{f}\) near the point \(z=(1,2)\), we take partial derivatives to obtain

\[\nabla f(z)=\left[\begin{array}{c}1-\exp(z_{2}-z_{1})\\ \exp(z_{2}-z_{1})\end{array}\right],\]

which evaluates to \((-1.7183,2.7183)\) at \(z=(1,2)\). The Taylor approximation at \(z=(1,2)\) is then

\[\hat{f}(x) = 3.7183+(-1.7183,2.7183)^{T}(x-(1,2))\] \[= 3.7183-1.7183(x_{1}-1)+2.7183(x_{2}-2).\]

Table 2.2 shows \(f(x)\) and \(\hat{f}(x)\), and the approximation error \(|\hat{f}(x)-f(x)|\), for some values of \(x\) relatively near \(z\). We can see that \(\hat{f

