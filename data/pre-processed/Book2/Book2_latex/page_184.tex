

## 9 Linear dynamical systems

1 _Compartmental system_. A _compartmental system_ is a model used to describe the movement of some material over time among a set of \(n\) compartments of a system, and the outside world. It is widely used in _pharmaco-kinetics_, the study of how the concentration of a drug varies over time in the body. In this application, the material is a drug, and the compartments are the bloodstream, lungs, heart, liver, kidneys, and so on. Comparmental systems are special cases of linear dynamical systems.

In this problem we will consider a very simple compartmental system with 3 compartments. We let \((x_{t})_{i}\) denote the amount of the material (say, a drug) in compartment \(i\) at time period \(t\). Between period \(t\) and period \(t+1\), the material moves as follows.

* 10% of the material in compartment 1 moves to compartment 2. (This decreases the amount in compartment 1 and increases the amount in compartment 2.)
* 5% of the material in compartment 2 moves to compartment 3.
* 5% of the material in compartment 3 moves to compartment 1.
* 5% of the material in compartment 3 is eliminated.

Express this compartmental system as a linear dynamical system, \(x_{t+1}=Ax_{t}\). (Give the matrix \(A\).) Be sure to account for all the material entering and leaving each compartment.

2 _Dynamics of an economy_. An economy (of a country or region) is described by an \(n\)-vector \(a_{t}\), where \((a_{t})_{i}\) is the economic output in sector \(i\) in year \(t\) (measured in billions of dollars, say). The total output of the economy in year \(t\) is \(\mathbf{1}^{T}a_{t}\). A very simple model of how the economic output changes over time is \(a_{t+1}=Ba_{t}\), where \(B\) is an \(n\times n\) matrix. (This is closely related to the Leontief input-output model described on page 157 of the book. But the Leontief model is static, _i.e._, doesn't consider how an economy changes over time.) The entries of \(a_{t}\) and \(B\) are positive in general.

In this problem we will consider the specific model with \(n=4\) sectors and

\[B=\left[\begin{array}{cccc}0.10&0.06&0.05&0.70\\ 0.48&0.44&0.10&0.04\\ 0.00&0.55&0.52&0.04\\ 0.04&0.01&0.42&0.51\end{array}\right].\]

* Briefly interpret \(B_{23}\), in English.
* _Simulation_. Suppose \(a_{1}=(0.6,0.9,1.3,0.5)\). Plot the four sector outputs (_i.e._, \((a_{t})_{i}\) for \(i=1,\ldots,4\)) and the total economic output (_i.e._, \(\mathbf{1}^{T}a_{t}\)) versus \(t\), for \(t=1,\ldots,20\).
* _Equilibrium point for linear dynamical system_. Consider a time-invariant linear dynamical system with offset, \(x_{t+1}=Ax_{t}+c\), where \(x_{t}\) is the state \(n\)-vector. We say that a vector \(z\) is an _equilibrium point_ of the linear dynamical system if \(x_{1}=z\) implies \(x_{2}=z\), \(x_{3}=z\), ... (In words: If the system starts in state \(z\), it stays in state \(z\).) Find a matrix \(F\) and vector \(g\) for which the set of linear equations \(Fz=g\) characterizes equilibrium points. (This means: If \(z\) is an equilibrium point, then \(Fz=g\); conversely if \(Fz=g\), then \(z\) is an equilibrium point.) Express \(F\) and \(g\) in terms of \(A\), \(c\), any standard matrices or vectors (_e.g._, \(I\), \(\mathbf{1}\), or \(0\)), and matrix and vector operations. _Remark_. Equilibrium points often have interesting interpretations. For example, if the linear dynamical system describes the population dynamics of a country, with the vector \(c\) denoting immigration (emigration when entries of \(c\) are negative), an equilibrium point is a population distribution that does not change, year to year. In other words, immigration exactly cancels the changes in population distribution caused by aging, births, and deaths.

