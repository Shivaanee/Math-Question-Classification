\(C=A*B\), however, in standard mathematical notation. So we will use the notation \(C=A\star B\).

The same properties that we observed for 1-D convolution hold for 2-D convolution: We have \(A\star B=B\star A\), \((A\star B)\star C=A\star(B\star C)\), and for fixed \(B\), \(A\star B\) is a linear function of \(A\).

Image blurring.If the \(m\times n\) matrix \(X\) represents an image, \(Y=X\star B\) represents the effect of _blurring_ the image by the _point spread function_ (PSF) given by the entries of the matrix \(B\). If we represent \(X\) and \(Y\) as vectors, we have \(y=T(B)x\), for some \((m+p-1)(n+q-1)\times mn\)-matrix \(T(B)\).

As an example, with

\[B=\left[\begin{array}{cc}1/4&1/4\\ 1/4&1/4\end{array}\right],\] (7.4)

\(Y=X\star B\) is an image where each pixel value is the average of a \(2\times 2\) block of 4 adjacent pixels in \(X\). The image \(Y\) would be perceived as the image \(X\), with some blurring of the fine details. This is illustrated in figure 7.7 for the \(8\times 9\) matrix

\[X=\left[\begin{array}{ccccccccc}1&1&1&1&1&1&1&1&1\\ 1&1&1&1&1&1&1&1&1\\ 1&1&0&0&0&0&0&1&1\\ 1&1&1&0&1&1&0&1&1\\ 1&1&1&0&1&1&0&1&1\\ 1&1&1&1&1&1&1&1&1\\ 1&1&1&1&1&1&1&1&1\end{array}\right]\] (7.5)

and its convolution with \(B\),

\[X\star B=\left[\begin{array}{ccccccccc}1/4&1/2&1/2&1/2&1/2&1/2&1/2&1/2&1/2&1 /4\\ 1/2&1&1&1&1&1&1&1&1&1/2\\ 1/2&1&3/4&1/2&1/2&1/2&1/2&3/4&1&1/2\\ 1/2&1&3/4&1/4&1/4&1/2&1/4&1/2&1&1/2\\ 1/2&1&1&1/2&1/2&1&1/2&1/2&1&1/2\\ 1/2&1&1&3/4&3/4&1&3/4&3/4&1&1/2\\ 1/2&1&1&1&1&1&1&1&1&1/2\\ 1/4&1/2&1/2&1/2 