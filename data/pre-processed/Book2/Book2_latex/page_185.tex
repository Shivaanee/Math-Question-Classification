* _Reducing a Markov model to a linear dynamical system._ Consider the 2-Markov model \[x_{t+1}=A_{1}x_{t}+A_{2}x_{t-1},\quad t=2,3,\ldots,\] where \(x_{t}\) is an \(n\)-vector. Define \(z_{t}=(x_{t},x_{t-1})\). Show that \(z_{t}\) satisfies the linear dynamical system equation \(z_{t+1}=Bz_{t}\), for \(t=2,3,\ldots\), where \(B\) is a \((2n)\times(2n)\) matrix. This idea can be used to express any \(K\)-Markov model as a linear dynamical system, with state \((x_{t},\ldots,x_{t-K+1})\).
* _Fibonacci sequence._ The Fibonacci sequence \(y_{0},y_{1},y_{2},\ldots\) starts with \(y_{0}=0\), \(y_{1}=1\), and for \(t=2,3,\ldots\), \(y_{t}\) is the sum of the previous two entries, _i.e._, \(y_{t-1}+y_{t-2}\). (Fibonacci is the name used by the 13th century mathematician Leonardo of Pisa.) Express this as a time-invariant linear dynamical system with state \(x_{t}=(y_{t},y_{t-1})\) and output \(y_{t}\), for \(t=1,2,\ldots\). Use your linear dynamical system to simulate (compute) the Fibonacci sequence up to \(t=20\). Also simulate a modified Fibonacci sequence \(z_{0},z_{1},z_{2},\ldots\), which starts with the same values \(z_{0}=0\) and \(z_{1}=1\), but for \(t=2,3,\ldots\), \(z_{t}\) is the difference of the two previous values, _i.e._, \(z_{t-1}-z_{t-2}\).
* _Recursive averaging._ Suppose that \(u_{1},u_{2},\ldots\) is a sequence of \(n\)-vectors. Let \(x_{1}=0\), and for \(t=2,3,\ldots\), let \(x_{t}\) be the average of \(u_{1},\ldots,u_{t-1}\), _i.e._, \(x_{t}=(u_{1}+\cdots+u_{t-1})/(t-1)\). Express this as a linear dynamical system with input, _i.e._, \(x_{t+1}=A_{t}x_{t}+B_{t}u_{t}\), \(t=1,2,\ldots\) (with initial state \(x_{1}=0\)). _Remark._ This can be used to compute the average of an extremely large collection of vectors, by accessing them one-by-one.
* _Complexity of linear dynamical system simulation._ Consider the time-invariant linear dynamical system with \(n\)-vector state \(x_{t}\) and \(m\)-vector input \(u_{t}\), and dynamics \(x_{t+1}=Ax_{t}+Bu_{t}\), \(t=1,2,\ldots\). You are given the matrices \(A\) and \(B\), the initial state \(x_{1}\), and the inputs \(u_{1},\ldots,u_{T-1}\). What is the complexity of carrying out a simulation, _i.e._, computing \(x_{2},x_{3},\ldots,x_{T}\)? About how long would it take to carry out a simulation with \(n=15\), \(m=5\), and \(T=10^{5}\), using a 1 Gflop/s computer? 