

## 0.2 Linear functions

### Linear or not?

Determine whether each of the following scalar-valued functions of \(n\)-vectors is linear. If it is a linear function, give its inner product representation, _i.e._, an \(n\)-vector \(a\) for which \(f(x)=a^{T}x\) for all \(x\). If it is not linear, give specific \(x\), \(y\), \(\alpha\), and \(\beta\) for which superposition fails, _i.e._,

\[f(\alpha x+\beta y)\neq\alpha f(x)+\beta f(y).\] 1. The spread of values of the vector, defined as \(f(x)=\max_{k}x_{k}-\min_{k}x_{k}\). 2. The difference of the last element and the first, \(f(x)=x_{n}-x_{1}\). 3. The median of an \(n\)-vector, where we will assume \(n=2k+1\) is odd. The median of the vector \(x\) is defined as the \((k+1)\)st largest number among the entries of \(x\). For example, the median of \((-7.1,3.2,-1.5)\) is \(-1.5\). 4. The average of the entries with odd indices, minus the average of the entries with even indices. You can assume that \(n=2k\) is even. 5. Vector extrapolation, defined as \(x_{n}+(x_{n}-x_{n-1})\), for \(n\geq 2\). (This is a simple prediction of what \(x_{n+1}\) would be, based on a straight line drawn through \(x_{n}\) and \(x_{n-1}\).) ### Processor powers and temperature

The temperature \(T\) of an electronic device containing three processors is an affine function of the power dissipated by the three processors, \(P=(P_{1},P_{2},P_{3})\). When all three processors are idling, we have \(P=(10,10,10)\), which results in a temperature \(T=30\). When the first processor operates at full power and the other two are idling, we have \(P=(100,10,10)\), and the temperature rises to \(T=60\). When the second processor operates at full power and the other two are idling, we have \(P=(10,100,10)\) and \(T=70\). When the third processor operates at full power and the other two are idling, we have \(P=(10,10,100)\) and \(T=65\). Now suppose that all three processors are operated at the same power \(P^{\text{same}}\). How large can \(P^{\text{same}}\) be, if we require that \(T\leq 85\)? _Hint._ From the given data, find the 3-vector \(a\) and number \(b\) for which \(T=a^{T}P+b\). 3. _Motion of a mass in response to applied force_. A unit mass moves on a straight line (in one dimension). The position of the mass at time \(t\) (in seconds) is denoted by \(s(t)\), and its derivatives (the velocity and acceleration) by \(s^{\prime}(t)\) and \(s^{\prime\prime}(t)\). The position as a function of time can be determined from Newton's second law \[s^{\prime\prime}(t)=F(t),\] where \(F(t)\) is the force applied at time \(t\), and the initial conditions \(s(0)\), \(s^{\prime}(0)\). We assume \(F(t)\) is piecewise-constant, and is kept constant in intervals of one second. The sequence of forces \(F(t)\), for \(0\leq t<10\), can then be represented by a 10-vector \(f\), with \[F(t)=f_{k},\quad k-1\leq t<k.\] Derive expressions for the final velocity \(s^{\prime}(10)\) and final position \(s(10)\). Show that \(s(10)\) and \(s^{\prime}(10)\) are affine functions of \(x\), and give 10-vectors \(a,c\) and constants \(b,d\) for which \[s^{\prime}(10)=a^{T}f+b,\qquad s(10)=c^{T}f+d.\] This means that the mapping from the applied force sequence to the final position and velocity is affine. _Hint._ You can use \[s^{\prime}(t)=s^{\prime}(0)+\int_{0}^{t}F(\tau)\;d\tau,\qquad s(t)=s(0)+\int_{ 0}^{t}s^{\prime}(\tau)\;d\tau.\] You will find that the mass velocity \(s^{\prime}(t)\) is piecewise-linear.

