

**11.7**: _Inverse of an upper triangular matrix_. Suppose the \(n\times n\) matrix \(R\) is upper triangular and invertible, _i.e._, its diagonal entries are all nonzero. Show that \(R^{-1}\) is also upper triangular. _Hint._ Use back substitution to solve \(Rs_{k}=e_{k}\), for \(k=1,\ldots,n\), and argue that \((s_{k})_{i}=0\) for \(i>k\).
**11.8**: _If a matrix is small, its inverse is large._ If a number \(a\) is small, its inverse \(1/a\) (assuming \(a\neq 0\)) is large. In this exercise you will explore a matrix analog of this idea. Suppose the \(n\times n\) matrix \(A\) is invertible. Show that \(\|A^{-1}\|\geq\sqrt{n}/\|A\|\). This implies that if a matrix is small, its inverse is large. _Hint._ You can use the inequality \(\|AB\|\leq\|A\|\|B\|\), which holds for any matrices for which the product makes sense. (See exercise 10.12.)
**11.9**: _Push-through identity._ Suppose \(A\) is \(m\times n\), \(B\) is \(n\times m\), and the \(m\times m\) matrix \(I+AB\) is invertible.

1. Show that the \(n\times n\) matrix \(I+BA\) is invertible. _Hint._ Show that \((I+BA)x=0\) implies \((I+AB)y=0\), where \(y=Ax\).
2. Establish the identity \[B(I+AB)^{-1}=(I+BA)^{-1}B.\] This is sometimes called the _push-through identity_ since the matrix \(B\) appearing on the left 'moves' into the inverse, and 'pushes' the \(B\) in the inverse out to the right side. _Hint._ Start with the identity \[B(I+AB)=(I+BA)B,\] and multiply on the right by \((I+AB)^{-1}\), and on the left by \((I+BA)^{-1}\).
**11.10**: _Reverse-time linear dynamical system._ A linear dynamical system has the form \[x_{t+1}=Ax_{t},\] where \(x_{t}\) in the (\(n\)-vector) state in period \(t\), and \(A\) is the \(n\times n\) dynamics matrix. This formula gives the state in the next period as a function of the current state. We want to derive a recursion of the form \[x_{t-1}=A^{\rm rev}x_{t},\] which gives the previous state as a function of the current state. We call this the _reverse time linear dynamical system_. 1. When is this possible? When it is possible, what is \(A^{\rm rev}\)? 2. For the specific linear dynamical system with dynamics matrix \[A=\left[\begin{array}{cc}3&2\\ -1&4\end{array}\right],\] find \(A^{\rm rev}\), or explain why the reverse time linear dynamical system doesn't exist.
**11.11**: _Interpolation of rational functions._ (Continuation of exercise 8.8.) Find a rational function \[f(t)=\frac{c_{1}+c_{2}t+c_{3}t^{2}}{1+d_{1}t+d_{2}t^{2}}\] that satisfies the following interpolation conditions: \[f(1)=2,\qquad f(2)=5,\qquad f(3)=9,\qquad f(4)=-1,\qquad f(5)=-4.\] In exercise 8.8 these conditions were expressed as a set of linear equations in the coefficients \(c_{1}\), \(c_{2}\), \(c_{3}\), \(d_{1}\) and \(d_{2}\); here we are asking you to form and (numerically) solve the system of equations. Plot the rational function you find over the range \(x=0\) to \(x=6\). Your plot should include markers at the interpolation points \((1,2),\ldots,(5,-4)\). (Your rational function graph should pass through these points.)