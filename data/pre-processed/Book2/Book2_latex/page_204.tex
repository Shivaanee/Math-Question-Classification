

**10.19**: _Student group membership._ Let \(G\in{\bf R}^{m\times n}\) represent a contingency matrix of \(m\) students who are members of \(n\) groups:

\[G_{ij}=\left\{\begin{array}{ll}1&\mbox{student $i$ is in group $j$}\\ 0&\mbox{student $i$ is not in group $j$.}\end{array}\right.\]

(A student can be in any number of the groups.)

1. What is the meaning of the 3rd column of \(G\)?
2. What is the meaning of the 15th row of \(G\)?
3. Give a simple formula (using matrices, vectors, etc.) for the \(n\)-vector \(M\), where \(M_{i}\) is the total membership (_i.e._, number of students) in group \(i\).
4. Interpret \((GG^{T})_{ij}\) in simple English.
5. Interpret \((G^{T}G)_{ij}\) in simple English.
**10.20**: _Products, materials, and locations._\(P\) different products each require some amounts of \(M\) different materials, and are manufactured in \(L\) different locations, which have different material costs. We let \(C_{lm}\) denote the cost of material \(m\) in location \(l\), for \(l=1,\ldots,L\) and \(m=1,\ldots,M\). We let \(Q_{mp}\) denote the amount of material \(m\) required to manufacture one unit of product \(p\), for \(m=1,\ldots,M\) and \(p=1,\ldots,P\). Let \(T_{pl}\) denote the total cost to manufacture product \(p\) in location \(l\), for \(p=1,\ldots,P\) and \(l=1,\ldots,L\). Give an expression for the matrix \(T\).
**10.21**: _Integral of product of polynomials._ Let \(p\) and \(q\) be two quadratic polynomials, given by

\[p(x)=c_{1}+c_{2}x+c_{3}x^{2},\qquad q(x)=d_{1}+d_{2}x+d_{3}x^{2}.\]

Express the integral \(J=\int_{0}^{1}p(x)q(x)\,dx\) in the form \(J=c^{T}Gd\), where \(G\) is a \(3\times 3\) matrix. Give the entries of \(G\) (as numbers).
**10.22**: _Composition of linear dynamical systems._ We consider two time-invariant linear dynamical systems with outputs. The first one is given by

\[x_{t+1}=Ax_{t}+Bu_{t},\qquad y_{t}=Cx_{t},\quad t=1,2,\ldots,\]

with state \(x_{t}\), input \(u_{t}\), and output \(y_{t}\). The second is given by

\[\tilde{x}_{t+1}=\tilde{A}\tilde{x}_{t}+\tilde{B}w_{t},\qquad v_{t}=\tilde{C} \tilde{x}_{t},\quad t=1,2,\ldots,\]

with state \(\tilde{x}_{t}\), input \(w_{t}\), and output \(v_{t}\). We now connect the output of the first linear dynamical system to the input of the second one, which means we take \(w_{t}=y_{t}\). (This is called the _composition_ of the two systems.) Show that this composition can also be expressed as a linear dynamical system with state \(z_{t}=(x_{t},\tilde{x}_{t})\), input \(u_{t}\), and output \(v_{t}\). (Give the state transition matrix, input matrix, and output matrix.)
**10.23**: Suppose \(A\) is an \(n\times n\) matrix that satisfies \(A^{2}=0\). Does this imply that \(A=0\)? (This is the case when \(n=1\).) If this is (always) true, explain why. If it is not, give a specific counterexample, _i.e._, a matrix \(A\) that is nonzero but satisfies \(A^{2}=0\).
**10.24**: _Matrix power identity._ A student says that for any square matrix \(A\),

\[(A+I)^{3}=A^{3}+3A^{2}+3A+I.\]

Is she right? If she is, explain why; if she is wrong, give a specific counterexample, _i.e._, a square matrix \(A\) for which it does not hold.
**10.25**: _Squareroots of the identity._ The number \(1\) has two squareroots (_i.e._, numbers who square is \(1\)), \(1\) and \(-1\). The \(n\times n\) identity matrix \(I_{n}\) has many more squareroots.

1. Find all diagonal squareroots of \(I_{n}\). How many are there? (For \(n=1\), you should get \(2\).)