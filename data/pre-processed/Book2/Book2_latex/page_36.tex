

**1.6**: _Vector of differences._ Suppose \(x\) is an \(n\)-vector. The associated vector of differences is the \((n-1)\)-vector \(d\) given by \(d=(x_{2}-x_{1},x_{3}-x_{2},\ldots,x_{n}-x_{n-1})\). Express \(d\) in terms of \(x\) using vector operations (_e.g._, slicing notation, sum, difference, linear combinations, inner product). The difference vector has a simple interpretation when \(x\) represents a time series. For example, if \(x\) gives the daily value of some quantity, \(d\) gives the day-to-day changes in the quantity.
**1.7**: _Transforming between two encodings for Boolean vectors._ A Boolean \(n\)-vector is one for which all entries are either 0 or 1. Such vectors are used to encode whether each of \(n\) conditions holds, with \(a_{i}=1\) meaning that condition \(i\) holds. Another common encoding of the same information uses the two values \(-1\) and \(+1\) for the entries. For example the Boolean vector \((0,1,1,0)\) would be written using this alternative encoding as \((-1,+1,+1,-1)\). Suppose that \(x\) is a Boolean vector with entries that are 0 or 1, and \(y\) is a vector encoding the same information using the values \(-1\) and \(+1\). Express \(y\) in terms of \(x\) using vector notation. Also, express \(x\) in terms of \(y\) using vector notation.
**1.8**: _Profit and sales vectors._ A company sells \(n\) different products or items. The \(n\)-vector \(p\) gives the profit, in dollars per unit, for each of the \(n\) items. (The entries of \(p\) are typically positive, but a few items might have negative entries. These items are called _loss leaders_, and are used to increase customer engagement in the hope that the customer will make other, profitable purchases.) The \(n\)-vector \(s\) gives the total sales of each of the items, over some period (such as a month), _i.e._, \(s_{i}\) is the total number of units of item \(i\) sold. (These are also typically nonnegative, but negative entries can be used to reflect items that were purchased in a previous time period and returned in this one.) Express the total profit in terms of \(p\) and \(s\) using vector notation.
**1.9**: _Symptoms vector._ A 20-vector \(s\) records whether each of 20 different symptoms is present in a medical patient, with \(s_{i}=1\) meaning the patient has the symptom and \(s_{i}=0\) meaning she does not. Express the following using vector notation.

1. The total number of symptoms the patient has.
2. The patient exhibits five out of the first ten symptoms.
**1.10**: _Total score from course record._ The record for each student in a class is given as a 10-vector \(r\), where \(r_{1},\ldots,r_{8}\) are the grades for the 8 homework assignments, each on a 0-10 scale, \(r_{9}\) is the midterm exam grade on a 0-120 scale, and \(r_{10}\) is final exam score on a 0-160 scale. The student's total course score \(s\), on a 0-100 scale, is based 25% on the homework, 35% on the midterm exam, and 40% on the final exam. Express \(s\) in the form \(s=w^{T}r\). (That is, determine the 10-vector \(w\).) You can give the coefficients of \(w\) to 4 digits after the decimal point.
**1.11**: _Word count and word count histogram vectors._ Suppose the \(n\)-vector \(w\) is the word count vector associated with a document and a dictionary of \(n\) words. For simplicity we will assume that all words in the document appear in the dictionary.

1. What is \(\mathbf{1}^{T}w\)?
2. What does \(w_{28}=0\) mean?
3. Let \(h\) be the \(n\)-vector that gives the histogram of the word counts, _i.e._, \(h_{i}\) is the fraction of the words in the document that are word \(i\). Use vector notation to express \(h\) in terms of \(w\). (You can assume that the document contains at least one word.)
**1.12**: _Total cash value._ An international company holds cash in five currencies: USD (US dollar), RMB (Chinese yuan), EUR (euro), GBP (British pound), and JPY (Japanese yen), in amounts given by the 5-vector \(c\). For example, \(c_{2}\) gives the number of RMB held. Negative entries in \(c\) represent liabilities or amounts owed. Express the total (net) value of the cash in USD, using vector notation. Be sure to give the size and define the entries of any vectors that you introduce in your solution. Your solution can refer to currency exchange rates.

