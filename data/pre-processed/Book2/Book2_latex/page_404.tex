

#### Examples

Nonlinear equation.The first example is the sigmoid function (18.10) from the example on page 390. We saw in figures 18.3 and 18.4 that the Gauss-Newton method, which reduces to Newton's method in this case, diverges when the initial value \(x^{(1)}\) is 1.15. The Levenberg-Marquardt algorithm, however, solves this problem. Figure 18.5 shows the value of the residual \(f(x^{(k)})\), and the value of \(\lambda^{(k)}\), for the Levenberg-Marquardt algorithm started from \(x^{(1)}=1.15\) and \(\lambda^{(1)}=1\). It converges to the solution \(x=0\) in around 10 iterations.

Equilibrium prices.We illustrate algorithm 18.3 with a small instance of the equilibrium price problem, with supply and demand functions

\[D(p) = \exp\left(E^{\mathrm{d}}(\log p-\log p^{\mathrm{nom}})+d^{\mathrm{ nom}}\right),\] \[S(p) = \exp\left(E^{\mathrm{s}}(\log p-\log p^{\mathrm{nom}})+s^{ \mathrm{nom}}\right),\]

where \(E^{\mathrm{d}}\) and \(E^{\mathrm{s}}\) are the demand and supply elasticity matrices, \(d^{\mathrm{nom}}\) and \(s^{\mathrm{nom}}\) are the nominal demand and supply vectors, and the log and \(\exp\) appearing in the equations apply to vectors elementwise. Figure 18.6 shows the contour lines of \(\|f(p)\|^{2}\), where \(f(p)=S(p)-D(p)\) is the excess supply, for

\[p^{\mathrm{nom}}=(2.8,10),\qquad d^{\mathrm{nom}}=(3.1,2.2),\qquad s^{\mathrm{ nom}}=(2.2,0.3)\]

and

\[E^{\mathrm{d}}=\left[\begin{array}{cc}-0.5&0.2\\ 0&-0.5\end{array}\right],\qquad E^{\mathrm{s}}=\left[\begin{array}{cc}0.5&-0. 3\\ -0.15&0.8\end{array}\right].\]

Figure 18.7 shows the iterates of the algorithm 18.3, started at \(p=(3,9)\) and \(\lambda^{(1)}=1\). The values of \(\|f(p^{(k)})\|^{2}\) and the trust parameter \(\lambda^{(k)}\) versus iteration \(k\) are shown in figure 18.8.

