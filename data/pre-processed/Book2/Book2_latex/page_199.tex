Matrix powers also come up in the analysis of a time-invariant linear dynamical system with an input. We have

\[x_{t+2}=Ax_{t+1}+Bu_{t+1}=A(Ax_{t}+Bu_{t})=A^{2}x_{t}+ABu_{t}+Bu_{t+1}.\]

Iterating this over \(\ell\) periods we obtain

\[x_{t+\ell}=A^{\ell}x_{t}+A^{\ell-1}Bu_{t}+A^{\ell-2}Bu_{t+1}+\cdots+Bu_{t+\ell -1}.\] (10.2)

(The first term agrees with the formula for \(x_{t+\ell}\) with no input.) The other terms are readily interpreted. The term \(A^{j}Bu_{t+\ell-j}\) is the contribution to the state \(x_{t+\ell}\) due to the input at time \(t+\ell-j\).

### 10.4 QR factorization

Matrices with orthonormal columns.As an application of Gram matrices, we can express the condition that the \(n\)-vectors \(a_{1},\ldots,a_{k}\) are orthonormal in a simple way using matrix notation:

\[A^{T}A=I,\]

where \(A\) is the \(n\times k\) matrix with columns \(a_{1},\ldots,a_{k}\). There is no standard term for a matrix whose columns are orthonormal: We refer to a matrix whose columns are orthonormal as 'a matrix whose columns are orthonormal'. But a _square_ matrix that satisfies \(A^{T}A=I\) is called _orthogonal_; its columns are an orthonormal basis. Orthogonal matrices have many uses, and arise in many applications.

We have already encountered some orthogonal matrices, including identity matrices, 2-D reflections and rotations (page 129), and permutation matrices (page 132).

Norm, inner product, and angle properties.Suppose the columns of the \(m\times n\) matrix \(A\) are orthonormal, and \(x\) and \(y\) are any \(n\)-vectors. We let \(f:\mathbf{R}^{n}\rightarrow\mathbf{R}^{m}\) be the function that maps \(z\) to \(Az\). Then we have the following:

* \(\|Ax\|=\|x\|\). That is, \(f\) is _norm preserving_.
* \((Ax)^{T}(Ay)=x^{T}y\). \(f\) preserves the inner product between vectors.
* \(\angle(Ax,Ay)=\angle(x,y)\). \(f\) also preserves angles between vectors.

Note that in each of the three equations above, the vectors appearing in the left- and right-hand sides have different dimensions, \(m\) on the left and \(n\) on the right.

We can verify these properties using simple matrix properties. We start with the second statement, that multiplication by \(A\) preserves the inner product. We have

\[(Ax)^{T}(Ay) = (x^{T}A^{T})(Ay)\] \[= x^{T}(A^{T}A)y\] \[= x^{T}Iy\] \[= x^{T}y.\] 