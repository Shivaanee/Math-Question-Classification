

**Examples.**

* _Average._ The _mean_ or _average_ value of an \(n\)-vector is defined as \[f(x)=(x_{1}+x_{2}+\cdots+x_{n})/n,\] and is denoted \(\mathbf{avg}(x)\) (and sometimes \(\overline{x}\)). The average of a vector is a linear function. It can be expressed as \(\mathbf{avg}(x)=a^{T}x\) with \[a=(1/n,\ldots,1/n)=\mathbf{1}/n.\]
* _Maximum._ The maximum element of an \(n\)-vector \(x\), \(f(x)=\max\{x_{1},\ldots,x_{n}\}\), is not a linear function (except when \(n=1\)). We can show this by a counterexample for \(n=2\). Take \(x=(1,-1)\), \(y=(-1,1)\), \(\alpha=1/2\), \(\beta=1/2\). Then \[f(\alpha x+\beta y)=0\neq\alpha f(x)+\beta f(y)=1.\]

##### Affine functions.

A linear function plus a constant is called an _affine_ function. A function \(f:\mathbf{R}^{n}\to\mathbf{R}\) is affine if and only if it can be expressed as \(f(x)=a^{T}x+b\) for some \(n\)-vector \(a\) and scalar \(b\), which is sometimes called the _offset_. For example, the function on 3-vectors defined by

\[f(x)=2.3-2x_{1}+1.3x_{2}-x_{3},\]

is affine, with \(b=2.3\), \(a=(-2,1.3,-1)\).

Any affine scalar-valued function satisfies the following variation on the superposition property:

\[f(\alpha x+\beta y)=\alpha f(x)+\beta f(y),\]

for all \(n\)-vectors \(x\), \(y\), and all scalars \(\alpha\), \(\beta\) that satisfy \(\alpha+\beta=1\). For linear functions, superposition holds for _any_ coefficients \(\alpha\) and \(\beta\); for affine functions, it holds _when the coefficients sum to one_ (_i.e._, when the argument is an affine combination).

To see that the restricted superposition property holds for an affine function \(f(x)=a^{T}x+b\), we note that, for any vectors \(x\), \(y\) and scalars \(\alpha\) and \(\beta\) that satisfy \(\alpha+\beta=1\),

\[f(\alpha x+\beta y) = a^{T}(\alpha x+\beta y)+b\] \[= \alpha a^{T}x+\beta a^{T}y+(\alpha+\beta)b\] \[= \alpha(a^{T}x+b)+\beta(a^{T}y+b)\] \[= \alpha f(x)+\beta f(y).\]

(In the second line we use \(\alpha+