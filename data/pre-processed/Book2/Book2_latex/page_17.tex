
Quantities.An \(n\)-vector \(q\) can represent the amounts or quantities of \(n\) different resources or products held (or produced, or required) by an entity such as a company. Negative entries mean an amount of the resource owed to another party (or consumed, or to be disposed of). For example, a _bill of materials_ is a vector that gives the amounts of \(n\) resources required to create a product or carry out a task.

Portfolio.An \(n\)-vector \(s\) can represent a stock portfolio or investment in \(n\) different assets, with \(s_{i}\) giving the number of shares of asset \(i\) held. The vector \((100,50,20)\) represents a portfolio consisting of 100 shares of asset 1, 50 shares of asset 2, and 20 shares of asset 3. Short positions (_i.e._, shares that you owe another party) are represented by negative entries in a portfolio vector. The entries of the portfolio vector can also be given in dollar values, or fractions of the total dollar amount invested.

Values across a population.An \(n\)-vector can give the values of some quantity across a population of individuals or entities. For example, an \(n\)-vector \(b\) can give the blood pressure of a collection of \(n\) patients, with \(b_{i}\) the blood pressure of patient \(i\), for \(i=1,\ldots,n\).

Proportions.A vector \(w\) can be used to give fractions or proportions out of \(n\) choices, outcomes, or options, with \(w_{i}\) the fraction with choice or outcome \(i\). In this case the entries are nonnegative and add up to one. Such vectors can also be interpreted as the recipes for a mixture of \(n\) items, an allocation across \(n\) entities, or as probability values in a probability space with \(n\) outcomes. For example, a uniform mixture of 4 outcomes is represented as the 4-vector \((1/4,1/4,1/4,1/4)\).

Time series.An \(n\)-vector can represent a _time series_ or _signal_, that is, the value of some quantity at different times. (The entries in a vector that represents a time series are sometimes called _samples_, especially when the quantity is something

Figure 1.2: Six colors and their RGB vectors.

 