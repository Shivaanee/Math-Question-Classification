

## Chapter 5 Linear independence

In this chapter we explore the concept of linear independence, which will play an important role in the sequel.

### 5.1 Linear dependence

A collection or list of \(n\)-vectors \(a_{1},\ldots,a_{k}\) (with \(k\geq 1\)) is called _linearly dependent_ if

\[\beta_{1}a_{1}+\cdots+\beta_{k}a_{k}=0\]

holds for some \(\beta_{1},\ldots,\beta_{k}\) that are not all zero. In other words, we can form the zero vector as a linear combination of the vectors, with coefficients that are not all zero. Linear dependence of a list of vectors does not depend on the ordering of the vectors in the list.

When a collection of vectors is linearly dependent, at least one of the vectors can be expressed as a linear combination of the other vectors: If \(\beta_{i}\neq 0\) in the equation above (and by definition, this must be true for at least one \(i\)), we can move the term \(\beta_{i}a_{i}\) to the other side of the equation and