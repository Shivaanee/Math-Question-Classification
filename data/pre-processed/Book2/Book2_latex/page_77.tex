

**3.25**: _Leveraging._ Consider an asset with return time series over \(T\) periods given by the \(T\)-vector \(r\). This asset has mean return \(\mu\) and risk \(\sigma\), which we assume is positive. We also consider cash as an asset, with return vector \(\mu^{\mathrm{rf}}\mathbf{1}\), where \(\mu^{\mathrm{rf}}\) is the cash interest rate per period. Thus, we model cash as an asset with return \(\mu^{\mathrm{rf}}\) and zero risk. (The superscript in \(\mu^{\mathrm{rf}}\) stands for 'risk-free'.) We will create a simple portfolio consisting of the asset and cash. If we invest a fraction \(\theta\) in the asset, and \(1-\theta\) in cash, our portfolio return is given by the time series

\[p=\theta r+(1-\theta)\mu^{\mathrm{rf}}\mathbf{1}.\]

We interpret \(\theta\) as the fraction of our portfolio we hold in the asset. We allow the choices \(\theta>1\), or \(\theta<0\). In the first case we are _borrowing_ cash and using the proceeds to buy more of the asset, which is called _leveraging_. In the second case we are _shorting_ the asset. When \(\theta\) is between \(0\) and \(1\) we are blending our investment in the asset and cash, which is a form of _hedging_.

1. Derive a formula for the return and risk of the portfolio, _i.e._, the mean and standard deviation of \(p\). These should be expressed in terms of \(\mu\), \(\sigma\), \(\mu^{\mathrm{rf}}\), and \(\theta\). Check your formulas for the special cases \(\theta=0\) and \(\theta=1\).
2. Explain how to choose \(\theta\) so the portfolio has a given target risk level \(\sigma^{\mathrm{tar}}\) (which is positive). If there are multiple values of \(\theta\) that give the target risk, choose the one that results in the highest portfolio return.
3. Assume we choose the value of \(\theta\) as in part (b). When do we use leverage? When do we short the asset? When do we hedge? Your answers should be in English.
**3.26**: _Time series auto-correlation._ Suppose the \(T\)-vector \(x\) is a non-constant time series, with \(x_{t}\) the value at time (or period) \(t\). Let \(\mu=(\mathbf{1}^{T}x)/T\) denote its mean value. The _auto-correlation_ of \(x\) is the function \(R(\tau)\), defined for \(\tau=0,1,\ldots\) as the correlation coefficient of the two vectors \((x,\mu\mathbf{1}_{\tau})\) and \((\mu\mathbf{1}_{\tau},x)\). (The subscript \(\tau\) denotes the length of the ones vector.) Both of these vectors also have mean \(\mu\). Roughly speaking, \(R(\tau)\) tells us how correlated the time series is with a version of itself lagged or shifted by \(\tau\) periods. (The argument \(\tau\) is called the lag.)

1. Explain why \(R(0)=1\), and \(R(\tau)=0\) for \(\tau\geq T\).
2. Let \(z\) denote the standardized or \(z\)-scored version of \(x\) (see page 56). Show that for \(\tau=0,\ldots,T-1\), \[R(\tau)=\frac{1}{T}\sum_{t=1}^{T-\tau}z_{t}z_{t+\tau}.\] 3. Find the auto-correlation for the time series \(x=(+1,-1,+1,-1,\ldots,+1,-1)\). You can assume that \(T\) is even.
4. Suppose \(x\) denotes the number of meals served by a restaurant on day \(\tau\). It is observed that \(R(7)\) is fairly high, and \(R(14)\) is also high, but not as high. Give an English explanation of why this might be.
**3.27**: _Another measure of the spread of the entries of a vector._ The standard deviation is a measure of how much the entries of a vector differ from their mean value. Another measure of how much the entries of an \(n\)-vector \(x\) differ from each other, called the _mean square difference_, is defined as \[\mathrm{MSD}(x)=\frac{1}{n^{2}}\sum_{i,j=1}^{n}(x_{i}-x_{j})^{2}.\] (The sum means that you should add up the \(n^{2}\) terms, as the indices \(i\) and \(j\) each range from \(1\) to \(n\).) Show that \(\mathrm{MSD}(x)=2\,\mathbf{std}(x)^{2}\). _Hint._ First observe that \(\mathrm{MSD}(\tilde{x})=\mathrm{MSD}(x)\), where \(\tilde{x}=x-\mathbf{avg}(x)\mathbf{1}\) is the de-meaned vector. Expand the sum and recall that \(\sum_{i=1}^{n}\tilde{x}_{i}=0\).

