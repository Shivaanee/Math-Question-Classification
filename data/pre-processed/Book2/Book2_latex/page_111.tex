which is indeed orthogonal to \(q_{1}\) (and \(a_{1}\)). It has norm \(\|\tilde{q}_{2}\|=2\); normalizing it gives \[q_{2}=\frac{1}{\|\tilde{q}_{2}\|}\tilde{q}_{2}=(1/2,1/2,1/2,1/2).\]
* \(i=3\). We have \(q_{1}^{T}a_{3}=2\) and \(q_{2}^{T}a_{3}=8\), so \[\tilde{q}_{3} = a_{3}-(q_{1}^{T}a_{3})q_{1}-(q_{2}^{T}a_{3})q_{2}\] \[= \left[\begin{array}{c}1\\ 3\\ 5\\ 7\end{array}\right]-2\left[\begin{array}{c}-1/2\\ 1/2\\ -1/2\\ 1/2\end{array}\right]-8\left[\begin{array}{c}1/2\\ 1/2\\ 1/2\\ 1/2\end{array}\right]\] \[= \left[\begin{array}{c}-2\\ -2\\ 2\\ 2\end{array}\right],\] which is orthogonal to \(q_{1}\) and \(q_{2}\) (and \(a_{1}\) and \(a_{2}\)). We have \(\|\tilde{q}_{3}\|=4\), so the normalized vector is \[q_{3}=\frac{1}{\|\tilde{q}_{3}\|}\tilde{q}_{3}=(-1/2,-1/2,1/2,1/2).\]

Completion of the Gram-Schmidt algorithm without early termination tells us that the vectors \(a_{1}\), \(a_{2}\), \(a_{3}\) are linearly independent.

Determining if a vector is a linear combination of linearly independent vectors.Suppose the vectors \(a_{1},\ldots,a_{k}\) are linearly independent, and we wish to determine if another vector \(b\) is a linear combination of them. (We have already noted on page 91 that if it is a linear combination of them, the coefficients are unique.) The Gram-Schmidt algorithm provides an explicit way to do this. We apply the Gram-Schmidt algorithm to the list of \(k+1\) vectors

\[a_{1},\ldots,a_{k},b.\]

These vectors are linearly dependent if \(b\) is a linear combination of \(a_{1},\ldots,a_{k}\); they are linearly independent if \(b\) is not a linear combination of \(a_{1},\ldots,a_{k}\). The Gram-Schmidt algorithm will determine which of these two cases holds. It cannot terminate in the first \(k\) steps, since we assume that \(a_{1},\ldots,a_{k}\) are linearly independent. It will terminate in the \((k+1)\)st step with \(\tilde{q}_{k+1}=0\) if \(b\) is a linear combination of \(a_{1},\ldots,a_{k}\). It will not terminate in the \((k+1)\)st step (_i.e._, \(\tilde{q}_{k+1}\neq 0\)), otherwise.

Checking if a collection of vectors is a basis.To check if the \(n\)-vectors \(a_{1},\ldots,a_{n}\) are a basis, we run the Gram-Schmidt algorithm on them. If Gram-Schmidt terminates early, they are not a basis; if it runs to completion, we know they are a basis.

 