piecewise-linear function with \(k\) knot points using the \(p=k+2\) basis functions

\[f_{1}(x)=1,\qquad f_{2}(x)=x,\qquad f_{i+2}(x)=(x-a_{i})_{+},\quad i=1,\ldots,k,\]

where \((u)_{+}=\max\{u,0\}\). These basis functions are shown in figure 13.7 for \(k=2\) knot points at \(a_{1}=-1\), \(a_{2}=1\). An example of a piecewise-linear fit with these knot points is shown in figure 13.8.

#### Regression

We now return to the general case when \(x\) is an \(n\)-vector. Recall that the regression model has the form

\[\hat{y}=x^{T}\beta+v,\]

where \(\beta\) is the weight vector and \(v\) is the offset. We can put this model in our general data fitting form using the basis functions \(f_{1}(x)=1\), and

\[f_{i}(x)=x_{i-1},\quad i=2,\ldots,n+1,\]

so \(p=n+1\). The regression model can then be expressed as

\[\hat{y}=x^{T}\theta_{2:(n+1)}+\theta_{1},\]

and we see that \(\beta=\theta_{2:n+1}\) and \(v= 