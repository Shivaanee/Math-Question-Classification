* _Weights, features, and score._ When the vector \(f\) represents a set of features of an object, and \(w\) is a vector of the same size (often called a _weight vector_), the inner product \(w^{T}f\) is the sum of the feature values, scaled (or weighted) by the weights, and is sometimes called a _score_. For example, if the features are associated with a loan applicant (_e.g._, age, income, ...), we might interpret \(s=w^{T}f\) as a credit score. In this example we can interpret \(w_{i}\) as the weight given to feature \(i\) in forming the score.
* _Price-quantity._ If \(p\) represents a vector of prices of \(n\) goods, and \(q\) is a vector of quantities of the \(n\) goods (say, the bill of materials for a product), then their inner product \(p^{T}q\) is the total cost of the goods given by the vector \(q\).
* _Speed-time._ A vehicle travels over \(n\) segments with constant speed in each segment. Suppose the \(n\)-vector \(s\) gives the speed in the segments, and the \(n\)-vector \(t\) gives the times taken to traverse the segments. Then \(s^{T}t\) is the total distance traveled.
* _Probability and expected values._ Suppose the \(n\)-vector \(p\) has nonnegative entries that sum to one, so it describes a set of proportions among \(n\) items, or a set of probabilities of \(n\) outcomes, one of which must occur. Suppose \(f\) is another \(n\)-vector, where we interpret \(f_{i}\) as the value of some quantity if outcome \(i\) occurs. Then \(f^{T}p\) gives the expected value or mean of the quantity, under the probabilities (or fractions) given by \(p\).
* _Polynomial evaluation._ Suppose the \(n\)-vector \(c\) represents the coefficients of a polynomial \(p\) of degree \(n-1\) or less: \[p(x)=c_{1}+c_{2}x+\cdots+c_{n-1}x^{n-2}+c_{n}x^{n-1}.\] Let \(t\) be a number, and let \(z=(1,t,t^{2},\ldots,t^{n-1})\) be the \(n\)-vector of powers of \(t\). Then \(c^{T}z=p(t)\), the value of the polynomial \(p\) at the point \(t\). So the inner product of a polynomial coefficient vector and vector of powers of a number evaluates the polynomial at the 