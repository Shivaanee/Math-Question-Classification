

### 9.4 Motion of a mass

Linear dynamical systems can be used to (approximately) describe the motion of many mechanical systems, for example, an airplane (that is not undergoing extreme maneuvers), or the (hopefully not too large) movement of a building during an earthquake. Here we describe the simplest example: A single mass moving in 1-D (_i.e._, a straight line), with an external force and a drag force acting on it. This is illustrated in figure 9.6. The (scalar) position of the mass at time \(\tau\) is given by \(p(\tau)\). (Here \(\tau\) is continuous, _i.e._, a real number.) The position satisfies Newton's law of motion, the differential equation

\[m\frac{d^{2}p}{d\tau^{2}}(\tau)=-\eta\frac{dp}{d\tau}(\tau)+f(\tau),\]

where \(m>0\) is the mass, \(f(\tau)\) is the external force acting on the mass at time \(\tau\), and \(\eta>0\) is the drag coefficient. The right-hand side is the total force acting on the mass; the first term is the drag force, which is proportional to the velocity and in the opposite direction.

Figure 9.5: Simulation of epidemic dynamics.

Figure 9.6: Mass moving along a line.

