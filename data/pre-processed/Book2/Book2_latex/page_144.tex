representing intersections and the edges representing road segments (one for each direction).

For a network described by the directed graph example above, the vector

\[x=(1,-1,1,0,1)\]

is a circulation, since \(Ax=0\). This flow corresponds to a unit clockwise flow on the outer edges (1, 3, 5, and 2) and no flow on the diagonal edge (4). (Visualizing this explains why such vectors are called circulations.)

Sources.In many applications it is useful to include additional flows called _source flows_ or _exogenous flows_, that enter or leave the network at the nodes, but not along the edges, as shown in figure 7.3. We denote these flows with an \(n\)-vector \(s\). We can think of \(s_{i}\) as a flow that enters the network at node \(i\) from outside, _i.e._, not from any edge. When \(s_{i}>0\) the exogenous flow is called a _source_, since it is injecting the quantity into the network at the node. When \(s_{i}<0\) the exogenous flow is called a _sink_, since it is removing the quantity from the network at the node.

Flow conservation with sources.The equation \(Ax+s=0\) means that the flow is conserved at each node, counting the source flow: The total of all incoming flow, from the incoming edges and exogenous source, minus the total outgoing flow from outgoing edges and exogenous sinks, is zero.

As an example, flow conservation with sources can be used as an approximate model of a power grid (ignoring losses), with \(x\) being the vector of power flows along the transmission lines, \(s_{i}>0\) representing a generator injecting power into the grid at node \(i\), \(s_{i}<0\) representing a load that consumes power at node \(i\), and \(s_{i}=0\) representing a substation where power is exchanged among transmission lines, with no generation or load attached.

For the example above, consider the source vector \(s=(1,0,-1,0)\), which corresponds to an injection of one unit of flow into node 1, and the removal of one unit of flow at node 3. In other words, node 1 is a source, node 3 is a sink, and

Figure 7.3: Network with four nodes and five edges, with source flows shown.

 