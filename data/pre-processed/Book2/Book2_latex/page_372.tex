

#### Example

We use daily return data for 19 stocks over a period of 2000 days (8 years). After adding a risk-free asset with a 1% annual return, we obtain a \(2000\times 20\) return matrix \(R\). The circles in figure 17.1 show the annualized risk and return for the 20 assets, _i.e._, the points

\[\left[\begin{array}{c}\sqrt{250}\,\mathbf{std}(Re_{i})\\ 250\,\mathbf{avg}(Re_{i})\end{array}\right],\quad i=1,\ldots,20.\]

It also shows the Pareto-optimal risk-return curve, and the risk and return for the uniform portfolio with equal weights \(w_{i}=1/n\). The annualized risk, return, and leverage for five portfolios (the four Pareto-optimal portfolios indicated in the figure, and the \(1/n\) portfolio) are given in table 17.1.

Figure 17.2 shows the total portfolio value (17.1) for the five portfolios. Figure 17.3 shows the portfolio values for a different test period of 500 days (two years).

Figure 17.1: The open circles show annualized risk and return for 20 assets (19 stocks and one risk-free asset with a return of 1%). The solid line shows risk and return for the Pareto optimal portfolios. The dots show risk and return for three Pareto optimal portfolios with 10%, 20%, and 40% return, and the portfolio with weights \(w_{i}=1/n\).

