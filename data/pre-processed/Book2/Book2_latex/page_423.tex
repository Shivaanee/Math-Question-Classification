

**18.4**: _Fitting an exponential to data._ Use the Levenberg-Marquardt algorithm to fit an exponential function of the form \(\hat{f}(x;\theta)=\theta_{1}e^{\theta_{2}x}\) to the data

\[0,\ 1,\ \ldots,\ 5,\qquad 5.2,\ 4.5,\ 2.7,\ 2.5,\ 2.1,\ 1.9.\]

(The first list gives \(x^{(i)}\); the second list gives \(y^{(i)}\).) Plot your model \(\hat{f}(x;\hat{\theta})\) versus \(x\), along with the data points.
**18.5**: _Mechanical equilibrium._ A mass \(m\), at position given by the 2-vector \(x\), is subject to three forces acting on it. The first force \(F^{\rm grav}\) is gravity, which has value \(F^{\rm grav}=-mg(0,1)\), where \(g=9.8\) is the acceleration of gravity. (This force points straight down, with a force that does not depend on the position \(x\).) The mass is attached to two cables, whose other ends are anchored at (2-vector) locations \(a_{1}\) and \(a_{2}\). The force \(F_{i}\) applied to the mass by cable \(i\) is given by

\[F_{i}=T_{i}(a_{i}-x)/\|a_{i}-x\|,\]

where \(T_{i}\) is the cable tension. (This means that each cable applies a force on the mass that points from the mass to the cable anchor, with magnitude given by the tension \(T_{i}\).) The cable tensions are given by

\[T_{i}=k\frac{\max\{\|a_{i}-x\|-L_{i},0\}}{L_{i}},\]

where \(k\) is a positive constant, and \(L_{i}\) is the natural or unloaded length of cable \(i\) (also positive). In words: The tension is proportional to the fractional stretch of the cable, above its natural length. (The max appearing in this formula means the tension is not a differentiable function of the position, when \(\|a_{i}-x\|=L_{i}\), but we will simply ignore this.) The mass is in equilibrium at position \(x\) if the three forces acting on it sum to zero,

\[F^{\rm grav}+F_{1}+F_{2}=0.\]

We refer to the left-hand side as the residual force. It is a function of mass position \(x\), and we write it as \(f(x)\).

Compute an equilibrium position for

\[a_{1}=(3,2),\qquad a_{2}=(-1,1),\qquad L_{1}=3,\qquad L_{2}=2,\qquad m=1, \qquad k=100,\]

by applying the Levenberg-Marquardt algorithm to the residual force \(f(x)\). Use \(x^{(1)}=(0,0)\) as starting point. (Note that it is important to start at a point where \(T_{1}>0\) and \(T_{2}>0\), because otherwise the derivative matrix \(Df(x^{(1)})\) is zero, and the Levenberg-Marquardt update gives \(x^{(2)}=x^{(1)}\).) Plot the components of the mass position and the residual force versus iterations.
**18.6**: _Fitting a simple neural network model._ A neural network is a widely used model of the form \(\hat{y}=\hat{f}(x;\theta)\), where the \(n\)-vector \(x\) is the feature vector and the \(p\)-vector \(\theta\) is the model parameter. In a neural network model, the function \(\hat{f}\) is _not_ an affine function of the parameter vector \(\theta\). In this exercise we consider a very simple neural network, with two layers, three internal nodes, and two inputs (_i.e._, \(n=2\)). This model has \(p=13\) parameters, and is given by

\[\hat{f}(x;\theta) = \theta_{1}\phi(\theta_{2}x_{1}+\theta_{3}x_{2}+\theta_{4})+ \theta_{5}\phi(\theta_{6}x_{1}+\theta_{7}x_{2}+\theta_{8})\] \[\mbox{}+\theta_{9}\phi(\theta_{10}x_{1}+\theta_{11}x_{2}+\theta_ {12})+\theta_{13}\]

where \(\phi:{\bf R}\rightarrow{\bf R}\) is the sigmoid function defined in (18.16). This function is shown as a _signal flow graph_ in figure 18.25. In this graph each edge from an input to an internal node, or from an internal node to the output node, corresponds to multiplication by one of the parameters. At each node (shown as the small filled circles) the incoming values and the constant offset are added together, then passed through the sigmoid function, to become the outgoing edge value.

