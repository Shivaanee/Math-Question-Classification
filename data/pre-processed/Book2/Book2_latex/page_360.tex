or 16.2. The simplest such methods follow these basic algorithms, replacing the QR factorizations with sparse QR factorizations (see page 190).

One potential problem with forming the KKT matrix as in algorithm 16.1 is that the Gram matrix \(A^{T}A\) can be far less sparse than the matrix \(A\). This problem can be avoided using a trick analogous to the one used on page 232 to solve sparse (unconstrained) least squares problems. We form the square set of \(m+n+p\) linear equations

\[\left[\begin{array}{ccc}0&A^{T}&C^{T}\\ A&-(1/2)I&0\\ C&0&0\end{array}\right]\left[\begin{array}{c}\hat{x}\\ \hat{y}\\ \hat{z}\end{array}\right]=\left[\begin{array}{c}0\\ b\\ d\end{array}\right].\] (16.11)

If \((\hat{x},\hat{y},\hat{z})\) satisfies these equations, it is easy to see that \((\hat{x},\hat{z})\) satisfies the KKT equations (16.4); conversely, if \((\hat{x},\hat{z})\) satisfies the KKT equations (16.4), \((\hat{x},\hat{y},\hat{z})\) satisfies the equations above, with \(\hat{y}=2(A\hat{x}-b)\). Provided \(A\) and \(C\) are sparse, the coefficient matrix above is sparse, and any method for solving a sparse system of linear equations can be used to solve it.

Solution of least norm problem.Here we specialize the solution of the general constrained least squares problem (16.1) given above to the special case of the least norm problem (16.2).

We start with the conditions (16.5). The stacked matrix is in this case

\[\left[\begin{array}{c}I\\ C\end{array}\right],\]

which always has linearly independent columns. So the conditions (16.5) reduce to: \(C\) has linearly independent rows. We make this assumption now.

For the least norm problem, the KKT equations (16.4) reduce to

\[\left[\begin{array}{cc}2I&C^{T}\\ C&0\end{array}\right]\left[\begin{array}{c}\hat{x}\\ \hat{z}\end{array}\right]=\left[\begin{array}{c}0\\ d\end{array}\right].\]

We can solve this using the methods for general constrained least squares, or derive the solution directly, which we do now. The first block row of this equation is \(2\hat{x}+C^{T}\hat{z}=0\), so

\[\hat{x}=-(1/2)C^{T}\hat{z}.\]

We substitute this into the second block equation, \(C\hat{x}=d\), to obtain

\[-(1/2)CC^{T}\hat{z}=d.\]

Since the rows of \(C\) are linearly independent, \(CC^{T}\) is invertible, so we have

\[\hat{z}=-2(CC^{T})^{-1}d.\]

Substituting this expression for \(\hat{z}\) into the formula for \(\hat{x}\) above gives

\[\hat{x}=C^{T}(CC^{T})^{-1}d.\] (16.12) 