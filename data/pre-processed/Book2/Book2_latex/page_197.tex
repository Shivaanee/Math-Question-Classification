Each term in the sum is 0 or 1, and equal to one only if there is an edge from vertex \(j\) to vertex \(k\) and an edge from vertex \(k\) to vertex \(i\), _i.e._, a path of length exactly two from vertex \(j\) to vertex \(i\) via vertex \(k\). By summing over all \(k\), we obtain the total number of paths of length two from \(j\) to \(i\).

The adjacency matrix \(A\) for the graph in figure 10.1, for example, and its square are given by

\[A=\left[\begin{array}{ccccc}0&1&0&0&1\\ 1&0&1&0&0\\ 0&0&1&1&1\\ 1&0&0&0&0\\ 0&0&0&1&0\end{array}\right],\qquad A^{2}=\left[\begin{array}{ccccc}1&0&1&1& 0\\ 0&1&1&1&2\\ 1&0&1&2&1\\ 0&1&0&0&1\\ 1&0&0&0&0\end{array}\right].\]

We can verify there is exactly one path of length two from vertex 1 to itself, _i.e._, the path \((1,2,1))\), and one path of length two from vertex 3 to vertex 1, _i.e._, the path \((3,2,1)\). There are two paths of length two from vertex 4 to vertex 3, \((4,3,3)\) and \((4,5,3)\), so \((A^{2})_{34}=2\).

The property extends to higher powers of \(A\). If \(\ell\) is a positive integer, then the \(i,j\) element of \(A^{\ell}\) is the number of paths of length \(\ell\) from vertex \(j\) to vertex \(i\). This can be proved by induction on \(\ell\). We have already shown the result for \(\ell=2\). Assume that it is true that the elements of \(A^{\ell}\) give the paths of length \(\ell\) between the different vertices. Consider the expression for the \(i,j\) element of \(A^{\ell+1}\):

\[(A^{\ell+1})_{ij}=\sum_{k=1}^{n 