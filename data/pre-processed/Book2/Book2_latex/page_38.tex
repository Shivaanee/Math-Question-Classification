

**1.19**: _Auto-regressive model._ Suppose that \(z_{1},z_{2},\ldots\) is a time series, with the number \(z_{t}\) giving the value in period or time \(t\). For example \(z_{t}\) could be the gross sales at a particular store on day \(t\). An _auto-regressive_ (AR) model is used to predict \(z_{t+1}\) from the previous \(M\) values, \(z_{t},z_{t-1},\ldots,z_{t-M+1}\):

\[\hat{z}_{t+1}=(z_{t},z_{t-1},\ldots,z_{t-M+1})^{T}\beta,\quad t=M,M+1,\ldots.\]

Here \(\hat{z}_{t+1}\) denotes the AR model's prediction of \(z_{t+1}\), \(M\) is the memory length of the AR model, and the \(M\)-vector \(\beta\) is the AR model coefficient vector. For this problem we will assume that the time period is daily, and \(M=10\). Thus, the AR model predicts tomorrow's value, given the values over the last 10 days.

For each of the following cases, give a short interpretation or description of the AR model in English, without referring to mathematical concepts like vectors, inner product, and so on. You can use words like 'vesterday' or 'today'.

1. \(\beta\approx e_{1}\).
2. \(\beta\approx 2e_{1}-e_{2}\).
3. \(\beta\approx e_{6}\).
4. \(\beta\approx 0.5e_{1}+0.5e_{2}\).

**1.20**: How many bytes does it take to store 100 vectors of length \(10^{5}\)? How many flops does it take to form a linear combination of them (with 100 nonzero coefficients)? About how long would this take on a computer capable of carrying out 1 Gflop/s?