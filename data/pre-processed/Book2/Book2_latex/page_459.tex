KKT conditions.The _KKT conditions_ (named for Karush, Kuhn, and Tucker) state that if \(\hat{x}\) is a solution of the constrained optimization problem, then there is a vector \(\hat{z}\) that satisfies

\[\frac{\partial L}{\partial x_{i}}(\hat{x},\hat{z})=0,\quad i=1,\ldots,n,\qquad \frac{\partial L}{\partial z_{i}}(\hat{x},\hat{z})=0,\quad i=1,\ldots,p.\]

(This is provided the rows of \(Dg(\hat{x})\) are linearly independent, a technical condition we ignore.) As in the unconstrained case, there can be pairs \(x,z\) that satisfy the KKT conditions but \(\hat{x}\) is not a solution of the constrained optimization problem.

The KKT conditions give us a method for solving the constrained optimization problem that is similar to the approach for the unconstrained optimization problem. We attempt to solve the KKT equations for \(\hat{x}\) and \(\hat{z}\); then we check to see if any of the points found are really solutions.

We can simplify the KKT conditions, and express them compactly using matrix notation. The last \(p\) equations can be expressed as \(g_{i}(\hat{x})=0\), which we already knew. The first \(n\) can be expressed as

\[\nabla_{x}L(\hat{x},\hat{z})=0,\]

where \(\nabla_{x}\) denotes the gradient with respect to the \(x_{i}\) arguments. This can be written as

\[\nabla h(\hat{x})+\hat{z}_{1}\nabla g_{1}(\hat{x})+\cdots+\hat{z}_{p}g_{p}( \hat{x})=\nabla h(\hat{x})+Dg(\hat{x})^{T}\hat{z}=0.\]

So the KKT conditions for the constrained optimization problem are

\[\nabla h(\hat{x})+Dg(\hat{x})^{T}\hat{z}=0,\qquad g(\hat{x})=0.\]

This is the extension of the gradient condition for unconstrained optimization to the constrained case.

Constrained nonlinear least squares.As an example, consider the constrained least squares problem

\[\begin{array}{ll}\mbox{minimize}&\|f(x)\|^{2}\\ \mbox{subject to}&g(x)=0,\end{array}\]

where \(f:\mathbf{R}^{n}\to\mathbf{R}^{m}\) and \(g:\mathbf{R}^{n}\to\mathbf{R}^{p}\). Define \(h(x)=\|f(x)\|^{2}\). Its gradient at \(\hat{x}\) is \(2Df(\hat{x})^{T}f(\hat{x})\) (see above) so the KKT conditions are

\[2Df(\hat{x})^{T}f(\hat{x})+Dg(\hat{x})^{T}\hat{z}=0,\qquad g(\hat{x})=0.\]

These conditions will hold for a solution of the problem (assuming the rows of \(Dg(\hat{x})\) are linearly independent). But there can be points that satisfy them and are not solutions.

 