vectors meet all of the \(m+n\) conditions. In other words, \(x\) lies in the intersection of \(m+n\) halfspaces. This **feasible set** has flat sides; it may be unbounded. and it may be empty.

The cost function \(cx\) brings to the problem a family of parallel planes. One plane \(cx=0\) goes through the origin. The planes \(cx=\) constant give all possible costs. As the cost varies, these planes sweep out the whole \(n\)-dimensional space. _The optimal \(x^{*}\) (lowest cost) occurs at the point where **the planes first touch the feasible set**_.

Our aim is to compute \(x^{*}\). We could do it (in principle) by finding all the corners of the feasible set, and computing their costs. In practice this is impossible. There could be billions of corners, and we cannot compute them all. Instead we turn to the _simplex method_, one of the most celebrated ideas in computational mathematics. It was developed by Dantzig as a systematic way to solve linear programs, and either by luck or genius it is an astonishing success. The steps of the simplex method are summarized later, and first we try to explain them.

### The Geometry: Movement Along Edges

I think it is the geometric explanation that gives the method away. Phase I simply locates one corner of the feasible set. _The heart of the method goes from corner to corner along the edges of the feasible set_. At a typical corner there are n edges to choose from. Some edges lead away from the optimal but unknown \(x^{*}\), and others lead gradually toward it. Dantzig chose an edge that leads to a new corner with a _lower cost_. There is no possibility of returning to anything more expensive. Eventually a special corner is reached, from which all edges go the wrong way: The cost has been minimized. That corner is the optimal vector \(x^{*}\), and the method stops.

The next problem is to turn the ideas of _corner_ and _edge_ into linear algebra. **A corner is the meeting point of \(n\) different planes**. Each plane is given by one equation--just as three planes (front wall, side wall, and floor) produce a corner in three dimensions. Each corner of the feasible set comes from turning \(n\) of the \(n+m\) inequalities \(Ax\geq b\) and \(x\geq 0\) into equations, and finding the intersection of these \(n\) planes.

One possibility is to choose the \(n\) equations \(x_{1}=0,\ldots,x_{n}=0\), and end up at the origin. Like all the other possible choices, _this intersection point will only be a genuine corner if it also satisfies the \(m\) remaining inequality constraints_. Otherwise it is not even in the feasible set, and is a complete fake. Our example with \(n=2\) variables and \(m=2\) constraints has six intersections, illustrated in Figure 8.3. Three of them are actually corners \(P\), \(Q\), \(R\) of the feasible set. They are the vectors \((0,6)\), \((2,2)\), and \((6,0)\), One of them must be the optimal vector (unless the minimum cost is \(\multimap\)). The other three, including the origin, are fakes.

In general there are \((n+m)!/n!m!\) possible intersections. That counts the number of ways to choose \(n\) plane equations out of \(n+m\). The size of that binomial coefficient makes computing all corners totally impractical for large \(m\) and \(n\). It is the task of Phase 