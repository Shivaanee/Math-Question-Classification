38. 1. Find all functions that satisfy \(\frac{dy}{dx}=0\). 2. Choose a particular function that satisfies \(\frac{dy}{dx}=3\). 3. Find all functions that satisfy \(\frac{dy}{dx}=3\).
39. The cosine space \(\mathbf{F}_{3}\) contains all combinations \(y(x)=A\cos x+B\cos 2x+C\cos 3x\). Find a basis for the subspace that has \(y(0)=0\).
40. Find a basis for the space of functions that satisfy 1. \(\frac{dy}{dx}-2y=0\). 2. \(\frac{dy}{dx}-\frac{y}{x}=0\).
41. Suppose \(y_{1}(x)\), \(y_{2}(x)\), \(y_{3}(x)\) are three different functions of \(x\). The vector space they span could have dimension 1, 2, or 3. Give an example of \(y_{1}\), \(y_{2}\), \(y_{3}\) to show each possibility.
42. Find a basis for the space of polynomials \(p(x)\) of degree \(\leq 3\). Find a basis for the subspace with \(p(1)=0\).
43. Write the 3 by 3 identity matrix as a combination of the other five permutation matrices! Then show that those five matrices are linearly independent. (Assume a combi nation gives zero, and check entries to prove each term is zero.) The five permutations are a basis for the subspace of 3 by 3 matrices with row and column sums all equal.
44. _Review_: Which of the following are bases for \(\mathbf{R}^{3}\)? 1. \((1,2,0)\) and \((0,1,-1)\). 2. \((1,1,-1)\), \((2,3,4)\), \((4,1,-1)\), \((0,1,-1)\). 3. \((1,2,2)\), \((-1,2,1)\), \((0,8,0)\). 4. \((1,2,2)\), \((-1,2,1)\), \((0,8,6)\).
45. _Review_: Suppose \(A\) is 5 by 4 with rank 4. Show that \(Ax=b\) has no solution when the 5 by 5 matrix \([A\;\;b]\) is invertible. Show that \(Ax=b\) is solvable when \([A\;\;b]\) is singular.

### The Four Fundamental Subspaces

The previous section dealt with definitions rather than constructions. We know what a basis is, but not how to find one. Now, starting from an explicit description of a subspace, we would like to compute an explicit basis.

Subspaces can be described in two ways. First, we may be given a set of vectors that span the space. (_Example_: The columns span the column space.) Second, we may be 