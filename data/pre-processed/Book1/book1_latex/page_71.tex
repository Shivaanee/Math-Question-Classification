1. Find a nonzero solution \(x\) to \(Ax=0\). The matrix is 3 by 3. 2. Elimination keeps column 1 \(+\) column 2 = column 3. Explain why there is no third pivot.
27. Suppose \(A\) is invertible and you exchange its first two rows to reach \(B\). Is the new matrix \(B\) invertible? How would you find \(B^{-1}\) from \(A^{-1}\)?
28. If the product \(M=ABC\) of three square matrices is invertible, then \(A\), \(B\), \(C\) are invertible. Find a formula for \(B^{-1}\) that involves \(M^{-1}\) and \(A\) and \(C\).
29. Prove that a matrix with a column of zeros cannot have an inverse.
30. Multiply \([\begin{smallmatrix}a&b\\ c&d\end{smallmatrix}]\) times \([\begin{smallmatrix}d&-b\\ -c&a\end{smallmatrix}]\). What is the inverse of each matrix if \(ad\neq bc\)?
31. 1. What matrix \(E\) has the same effect as these three steps? Subtract row 1 from row 2, subtract row 1 from row 3, then subtract row 2 from row 3. 2. What single matrix \(L\) has the same effect as these three reverse steps? Add row 2 to row 3, add row 1 to row 3, then add row 1 to row 2.
32. Find the numbers \(a\) and \(b\) that give the inverse of \(\mathbf{5}*\mathsf{eye}(\mathbf{4})-\mathsf{ones}(\mathbf{4},\mathbf{4})\): \[\begin{bmatrix}4&-1&-1&-1\\ -1&4&-1&-1\\ -1&-1&4&-1\\ -1&-1&-1&4\end{bmatrix}^{-1}=\begin{bmatrix}a&b&b&b\\ b&a&b&b\\ b&b&a&b\\ b&b&b&a\end{bmatrix}.\] What are \(a\) and \(b\) in the inverse of \(\mathbf{6}*\mathsf{eye}(\mathbf{5})-\mathsf{ones}(\mathbf{5},\mathbf{5})\)?
33. Show that \(\mathsf{A}=\mathbf{4}*\mathsf{eye}(\mathbf{4})-\mathsf{ones}(\mathbf{4}, \mathbf{4})\) is _not_ invertible: Multiply \(\mathsf{A}*\mathsf{ones}(\mathbf{4},\mathbf{1})\).
34. There are sixteen 2 by 2 matrices whose entries are 1s and 0s. How many of them are invertible? Problems 35-39 are about the Gauss-Jordan method for calculating \(A^{-1}\).
35. Change \(I\) into \(A^{-1}\) as you reduce \(A\) to \(I\) (by row operations): \[\begin{bmatrix}A&I\end{bmatrix}=\begin{bmatrix}1&3&1&0\\ 2&7&0&1\end{bmatrix}\quad\text{and}\quad\begin{bmatrix}A&I\end{bmatrix}= \begin{bmatrix}1&4&1&0\\ 3&9&0&1\end{bmatrix}.\]
36. Follow the 3 by 3 text example but with plus signs in \(A\). Eliminate above and below the pivots to reduce \([A\;\;I]\) to \([I\;\;A^{-1}]\): \[\begin{bmatrix}\boldsymbol{A}&I\end{bmatrix}=\begin{bmatrix}\mathbf{2}& \mathbf{1}&\mathbf{0}&1&0&0\\ \mathbf{1}&\mathbf{2}&\mathbf{1}&0&1&0\\ \mathbf{0}&\mathbf{1}&\mathbf{2}&0&0&1\end{bmatrix}.\] 