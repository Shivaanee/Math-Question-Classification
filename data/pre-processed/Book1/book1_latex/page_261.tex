These determinants give the volumes--or areas, since we are in two dimensions--drawn in Figure 4.3. The parallelogram has unit base and unit height; its area is also 1.

## 4. A Formula for the Pivots

We can finally discover when elimination is possible without row exchanges. The key observation is that the first \(k\) pivots are completely determined by the submatrix \(A_{k}\) in the upper left corner of \(A\). _The remaining rows and columns of \(A\) have no effect on this corner of the problem_:

\[\begin{array}{c}\textbf{Elimination on }A\\ \textbf{includes}\\ \textbf{elimination on }A_{2}\end{array}\qquad A=\begin{bmatrix}\textbf{a}& \textbf{b}&e\\ \textbf{c}&\textbf{d}&f\\ g&h&i\end{bmatrix}\rightarrow\begin{bmatrix}\textbf{a}&\textbf{b}&e\\ \textbf{0}&(\textbf{ad}-\textbf{bc})/\textbf{a}&(af-ec)/a\\ g&h&i\end{bmatrix}.\]

Certainly the first pivot depended only on the first row and column, The second pivot \((ad-bc)/a\) depends only on the 2 by 2 corner submatrix \(A_{2}\). The rest of \(A\) does not enter until the third pivot. Actually it is not just the pivots, but the entire upper-left corners of \(L\), \(D\), and \(U\), that are determined by the upper-left corner of \(A\):

\[A=LDU=\begin{bmatrix}1&&\\ c/a&1&\\ *&*&1\end{bmatrix}\begin{bmatrix}a&\\ &(ad-bc)/a&\\ &*\end{bmatrix}\begin{bmatrix}1&b/a&*\\ &1&*\\ &&1\end{bmatrix}.\]

What we see in the first two rows and columns is exactly the factorization of the corner submatrix \(A_{2}\). This is a general rule if there are no row exchanges:

* If \(A\) is factored into \(LDU\), the upper left corners satisfy \(A_{k}=L_{k}D_{k}U_{k}\). For every \(k\), the submatrix \(A_{k}\) is going through a Gaussian elimination of its own.

The proof is to see that this corner can be settled first, before even looking at other eliminations. Or use the laws for _block multiplication_:

\[LDU=\begin{bmatrix}\textbf{L}_{\textbf{k}}&0\\ B&C\end{bmatrix}\begin{bmatrix}\textbf{D}_{\textbf{k}}&0\\ 0&E\end{bmatrix}\begin{bmatrix}\textbf{U}_{\textbf{k}}&F\\ 0&G\end{bmatrix}=\begin{bmatrix}\textbf{L}_{\textbf{k}}\textbf{D}_{\textbf{k} }\textbf{U}_{\textbf{k}}&L_{k}D_{k}F\\ BD_{k}U_{k}&BD_{k}F+CEG\end{bmatrix}.\]

Comparing the last matrix with \(A\), the corner \(L_{k}D_{k}U_{k}\) coincides with \(A_{k}\). Then:

\[\det A_{k}=\det L_{k}\det D_{k}\det U_{k}=\det D_{k}=d_{1}d_{2}\cdots d_{k}.\]

Figure 4.3: The areas of a unit square and a unit parallelogram are both 1.

 