

**14.**: Choose \(x=(x_{1},x_{2},x_{3},x_{4})\) in \(\mathbf{R}^{4}\). It has 24 rearrangements like \((x_{2},x_{1},x_{3},x_{4})\) and \((x_{4},x_{3},x_{1},x_{2})\). Those 24 vectors, including \(x\) itself, span a subspace \(\mathbf{S}\). Find specific vectors \(x\) so that the dimension of \(\mathbf{S}\) is: (a) 0, (b) 1, (c) 3, (d) 4.
**15.**: \(v+w\) and \(v-w\) are combinations of v and w. Write \(v\) and \(w\) as combinations of \(v+w\) and \(v-w\). The two pairs of vectors \(\underline{\phantom{\rule{0.0pt}{14.226378pt}}}\) the same space. When are they a basis for the same space?
**16.**: Decide whether or not the following vectors are linearly independent, by solving \(c_{1}v_{1}+c_{2}v_{2}+c_{3}v_{3}+c_{4}v_{4}=0\):

\[v_{1}=\begin{bmatrix}1\\ 1\\ 0\\ 0\end{bmatrix},\qquad v_{2}=\begin{bmatrix}1\\ 0\\ 1\\ 0\end{bmatrix},\qquad v_{3}=\begin{bmatrix}0\\ 0\\ 1\\ 1\end{bmatrix},\qquad v_{4}=\begin{bmatrix}0\\ 1\\ 0\\ 1\end{bmatrix}.\]

Decide also if they span \(\mathbf{R}^{4}\), by trying to solve \(c_{1}v_{1}+\cdots+c_{4}v_{4}=(0,0,0,1)\).
**17.**: Suppose the vectors to be tested for independence are placed into the rows instead of the columns of \(A\), How does the elimination process from \(A\) to \(U\) decide for or against independence?
**18.**: To decide whether \(b\) is in the sub space spanned by \(w_{1},\ldots,w_{n}\), let the vectors \(w\) be the columns of \(A\) and try to solve \(Ax=b\). What is the result for

1. \(w_{1}=(1,1,0)\), \(w_{2}=(2,2,1)\), \(w_{3}=(0,0,2)\), \(b=(3,4,5)\)?
2. \(w_{1}=(1,2,0)\), \(w_{2}=(2,5,0)\), \(w_{3}=(0,0,2)\), \(w_{4}=(0,0,0)\), and any \(b\)?

**Problems 19-37 are about the requirements for a basis.**
**19.**: If \(v_{1},\ldots,v_{n}\) are linearly independent, the space they span has dimension \(\underline{\phantom{\rule{0.0pt}{14.226378pt}}}\). These vectors are a \(\underline{\phantom{\rule{0.0pt}{14.226378pt}}}\) for that space. If the vectors are the columns of an \(m\) by \(n\) matrix, then \(m\) is \(\underline{\phantom{\rule{0.0pt}{14.226378pt}}}\) than \(n\).
**20.**: Find a basis for each of these subspaces of \(\mathbf{R}^{4}\):

1. All vectors whose components are equal.
2. All vectors whose components add to zero.
3. All vectors that are perpendicular to \((1,1,0,0)\) and \((1,0,1,1)\).
4. The column space (in \(\mathbf{R}^{2}\)) and nullspace (in \(\mathbf{R}^{5}\)) of \(U=\begin{bmatrix}1&0&1&0&1\\ 0&1&0&1&0\end{bmatrix}\).
**21.**: Find three different bases for the column space of \(U\) above. Then find two different bases for the row space of \(U\).
**22.**: Suppose \(v_{1},v_{2},\ldots,v_{6}\) are six vectors in \(\mathbf{R}^{4}\).

1. Those vectors (do)(do not)(might not) span \(\mathbf{R}^{4}\).

