

**15.**: If every row of \(A\) adds to zero, prove that \(\det A=0\). If every row adds to \(1\), prove that \(\det(A-I)=0\). Show by example that this does not imply \(\det A=1\).
**16.**: Find these \(4\) by \(4\) determinants by Gaussian elimination:

\[\det\begin{bmatrix}11&12&13&14\\ 21&22&23&24\\ 31&32&33&34\\ 41&42&43&44\end{bmatrix}\qquad\text{and}\qquad\det\begin{bmatrix}1&t&t^{2}&t^{ 3}\\ t&1&t&t^{2}\\ t^{2}&t&1&t\\ t^{3}&t^{2}&t&1\end{bmatrix}.\]
**17.**: Find the determinants of

\[A=\begin{bmatrix}4&2\\ 1&3\end{bmatrix},\qquad A^{-1}=\frac{1}{10}\begin{bmatrix}3&-2\\ -1&4\end{bmatrix},\qquad A-\lambda I=\begin{bmatrix}4-\lambda&2\\ 1&3-\lambda\end{bmatrix}.\]

For which values of \(\lambda\) is \(A-\lambda I\) a singular matrix?
**18.**: Evaluate \(\det A\) by reducing the matrix to triangular form (rules 5 and 7).

\[A=\begin{bmatrix}1&1&3\\ 0&4&6\\ 1&5&8\end{bmatrix},\qquad B=\begin{bmatrix}1&1&3\\ 0&4&6\\ 0&0&1\end{bmatrix},\qquad C=\begin{bmatrix}1&1&3\\ 0&4&6\\ 1&5&9\end{bmatrix}.\]

What are the determinants of \(B\), \(C\), \(AB\), \(A^{\mathrm{T}}A\), and \(C^{\mathrm{T}}\)?
**19.**: Suppose that \(CD=-DC\), and find the flaw in the following argument: Taking determinants gives \((\det C)(\det D)=-(\det D)(\det C)\), so either \(\det C=0\) or \(\det D=0\). Thus \(CD=-DC\) is only possible if \(C\) or \(D\) is singular.
**20.**: Do these matrices have determinant \(0\), \(1\), \(2\), or \(3\)?

\[A=\begin{bmatrix}0&0&1\\ 1&0&0\\ 0&1&0\end{bmatrix}\qquad B=\begin{bmatrix}0&1&1\\ 1&0&1\\ 1&1&0\end{bmatrix}\qquad C=\begin{bmatrix}1&1&1\\ 1&1&1\\ 1&1&1\end{bmatrix}.\]
**21.**: The inverse of a \(2\) by \(2\) matrix seems to have determinant \(=1\):

\[\det A^{-1}=\det\frac{1}{ad-bc}\begin{bmatrix}d&-b\\ -c&a\end{bmatrix}=\frac{ad-bc}{ad-bc}=1.\]

What is wrong with this calculation? What is the correct \(\det A^{-1}\)?
**Problems 22-28 use the rules to compute specific determinants.**
**22.**: Reduce \(A\) to \(U\) and find \(\det A=\) product of the pivots:

\[A=\begin{bmatrix}1&1&1\\ 1&2&2\\ 1&2&3\end{bmatrix}\qquad\text{and}\qquad A=\begin{bmatrix}1&2&3\\ 2&2&3\\ 3&3&3\end{bmatrix}.\]