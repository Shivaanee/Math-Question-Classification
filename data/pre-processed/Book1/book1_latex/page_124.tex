

**30.**: By locating the pivots, find a basis for the column space of

\[U=\begin{bmatrix}0&5&4&3\\ 0&0&2&1\\ 0&0&0&0\\ 0&0&0&0\end{bmatrix}.\]

Express each column that is not in the basis as a combination of the basic columns, Find also a matrix \(A\) with this echelon form \(U\), but a different column space.
**31.**: Find a counterexample to the following statement: If \(v_{1}\), \(v_{2}\), \(v_{3}\), \(v_{4}\) is a basis for the vector space \(\mathbf{R}^{4}\), and if \(\mathbf{W}\) is a subspace, then some subset of the \(v\)'s is a basis for \(\mathbf{W}\).
**32.**: Find the dimensions of these vector spaces:

1. The space of all vectors in \(\mathbf{R}^{4}\) whose components add to zero.
2. The nullspace of the 4 by 4 identity matrix.
3. The space of all 4 by 4 matrices.
**33.**: Suppose \(\mathbf{V}\) is known to have dimension \(k\). Prove that

1. any \(k\) independent vectors in \(\mathbf{V}\) form a basis;
2. any \(k\) vectors that span \(\mathbf{V}\) form a basis.

In other words, if the number of vectors is known to be correct, either of the two properties of a basis implies the other.
**34.**: Prove that if \(\mathbf{V}\) and \(\mathbf{W}\) are three-dimensional subspaces of \(\mathbf{R}^{5}\), then \(\mathbf{V}\) and \(\mathbf{W}\) must have a nonzero vector in common. _Hint_: Start with bases for the two subspaces, making six vectors in all.
**35.**: _True or false?_

1. If the columns of \(A\) are linearly independent, then \(Ax=b\) has exactly one solution for every \(b\).
2. A 5 by 7 matrix never has linearly independent columns,
**36.**: If \(A\) is a 64 by 17 matrix of rank 11, how many independent vectors satisfy \(Ax=0\)? How many independent vectors satisfy \(A^{\mathrm{T}}y=0\)?
**37.**: Find a basis for each of these subspaces of 3 by 3 matrices:

1. All diagonal matrices.
2. All symmetric matrices (\(A^{\mathrm{T}}=A\)).
3. All skew-symmetric matrices (\(A^{\mathrm{T}}=-A\)).
**Problems 38-42 are about spaces in which the "vectors" are functions.**