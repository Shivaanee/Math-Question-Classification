The second component of the product \(Ax\) is \(4u-6v+0w\), from the second row of \(A\). The matrix equation \(Ax=b\) is equivalent to the three simultaneous equations in equation (1).

_Row times column_ is fundamental to all matrix multiplications. From two vectors it produces a single number. This number is called the _inner product_ of the two vectors. In other words, the product of a 1 by \(n\) matrix (a _row vector_) and an \(n\) by 1 matrix (a _column vector_) is a 1 by 1 matrix:

\[\textbf{Inner product}\qquad\begin{bmatrix}2&1&1\end{bmatrix}\begin{bmatrix}1 \\ 1\\ 2\end{bmatrix}=\begin{bmatrix}2\cdot 1+1\cdot 1+1\cdot 2\end{bmatrix}=\begin{bmatrix}5 \end{bmatrix}.\]

This confirms that the proposed solution \(x=(1,1,2)\) does satisfy the first equation.

_There are two ways to multiply a matrix \(A\) and a vector \(x\)_. One way is _a row at a time_, Each row of \(A\) combines with \(x\) to give a component of \(Ax\). There are three inner products when \(A\) has three rows:

\[Ax\textbf{ by rows}\qquad\begin{bmatrix}1&1&6\\ 3&0&1\\ 1&1&4\end{bmatrix}\begin{bmatrix}2\\ 5\\ 0\end{bmatrix}=\begin{bmatrix}1\cdot 2+1\cdot 5+6\cdot 0\\ 3\cdot 2+0\cdot 5+3\cdot 0\\ 1\cdot 2+1\cdot 5+4\cdot 0\end{bmatrix}=\begin{bmatrix}7\\ 6\\ 7\end{bmatrix}.\] (4)

That is how \(Ax\) is usually explained, but the second way is equally important. In fact it is more important! It does the multiplication _a column at a time_. The product \(Ax\) is found all at once, as _a combination of the three columns of \(A\)_:

\[Ax\textbf{ by columns}\qquad 2\begin{bmatrix}1\\ 3\\ 1\end{bmatrix}+5\begin{bmatrix}1\\ 0\\ 1\end{bmatrix}+0\begin{bmatrix}6\\ 3\\ 4\end{bmatrix}=\begin{bmatrix}7\\ 6\\ 7\end{bmatrix}.\] (5)

The answer is twice column 1 plus 5 times column 2. It corresponds to the "column picture" of linear equations. If the right-hand side \(b\) has components 7, 6, 7, then the solution has components 2, 5, 0. Of course the row picture agrees with that (and we eventually have to do the same multiplications).

The column rule will be used over and over, and we repeat it for emphasis:

1A Every product \(Ax\) can be found using whole columns as in equation (5). Therefore \(Ax\) is _a combination of the columns of \(A\)_. The coefficients are the components of \(x\).

To multiply \(A\) times \(x\) in \(n\) dimensions, we need a notation for the individual entries in \(A\). _The entry in the ith row and \(jth\) column is always denoted by \(a_{ij}\)_. The first subscript gives the row number, and the second subscript indicates the column. (In equation (4), \(a_{21}\) is 3 and \(a_{13}\) is 6.) If \(A\) is an \(m\) by \(n\) matrix, then the index \(i\) goes from 1 to \(m\)--there are \(m\) rows--and the index \(j\) goes from 1 to \(n\). Altogether the matrix has \(mn\) entries, and \(a_{mn}\) is in the lower right corner.

 