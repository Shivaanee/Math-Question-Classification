

### Problem Set 5.1

1. Find the eigenvalues and eigenvectors of the matrix \(A=\left[\begin{smallmatrix}1&-1\\ 2&4\end{smallmatrix}\right]\). Verify that the trace equals the sum of the eigenvalues, and the determinant equals their product.
2. With the same matrix \(A\), solve the differential equation \(du/dt=Au\), \(u(0)=\left[\begin{smallmatrix}0\\ 6\end{smallmatrix}\right]\). What are the two pure exponential solutions?
3. If we shift to \(A-7I\), what are the eigenvalues and eigenvectors and how are they related to those of \(A\)? \[B=A-7I=\left[\begin{matrix}-6&-1\\ 2&-3\end{matrix}\right].\]
4. Solve \(du/dt=Pu\), when \(P\) is a projection: \[\frac{du}{dt}=\left[\begin{smallmatrix}1&\frac{1}{2}\\ \frac{1}{2}&\frac{1}{2}\end{smallmatrix}\right]u\qquad\text{with}\qquad u(0)= \left[\begin{matrix}5\\ 3\end{matrix}\right].\]

Part of \(u(0)\) increases exponentially while the nullspace part stays fixed.
5. Find the eigenvalues and eigenvectors of \[A=\left[\begin{matrix}3&4&2\\ 0&1&2\\ 0&0&0\end{matrix}\right]\qquad\text{and}\qquad B=\left[\begin{matrix}0&0&2\\ 0&2&0\\ 2&0&0\end{matrix}\right].\] Check that \(\lambda_{1}+\lambda_{2}+\lambda_{3}\) equals the trace and \(\lambda_{1}\lambda_{2}\lambda_{3}\) equals the determinant.
6. Give an example to show that the eigenvalues can be changed when a multiple of one row is subtracted from another. Why is a zero eigenvalue _not_ changed by the steps of elimination?
7. Suppose that \(\lambda\) is an eigenvalue of \(A\), and \(x\) is its eigenvector: \(Ax=\lambda x\). 1. Show that this same \(x\) is an eigenvector of \(B=A-7I\), and find the eigenvalue. This should confirm Exercise 3. 2. Assuming \(\lambda\neq 0\), show that \(x\) is also an eigenvector of \(A^{-1}\)--and find the eigenvalue.
8. Show that the determinant equals the product of the eigenvalues by imagining that the characteristic polynomial is factored into \[\det(A-\lambda I)=(\lambda_{1}-\lambda)(\lambda_{2}-\lambda)\cdots(\lambda_{n }-\lambda),\] (16) and making a clever choice of \(\lambda\).

