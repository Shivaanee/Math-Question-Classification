is

\[u(t)=\begin{bmatrix}1&1\\ 1&-1\end{bmatrix}\begin{bmatrix}e^{-t}\\ &e^{-3t}\end{bmatrix}\begin{bmatrix}c_{1}\\ c_{2}\end{bmatrix}=S\begin{bmatrix}e^{-t}\\ &e^{-3t}\end{bmatrix}S^{-1}u(0).\] (3)

_Here is the fundamental formula of this section_: \(Se^{\Lambda t}S^{-1}u(0)\) solves the differential equation, just as \(S\Lambda^{k}S^{-1}u_{0}\) solved the difference equation:

\[u(t)=Se^{\Lambda t}S^{-1}u(0)\quad\text{with}\quad\Lambda=\begin{bmatrix}-1\\ &-3\end{bmatrix}\quad\text{and}\quad e^{\Lambda t}=\begin{bmatrix}e^{-t}\\ &e^{-3t}\end{bmatrix}.\] (4)

There are two more things to be done with this example. One is to complete the mathematics, by giving a direct definition of the _exponential of a matrix_. The other is to give a physical interpretation of the equation and its solution. It is the kind of differential equation that has useful applications.

The exponential of a diagonal matrix \(\Lambda\) is easy; \(e^{\Lambda t}\) just has the \(n\) numbers \(e^{\lambda t}\) on the diagonal. For a general matrix \(A\), the natural idea is to imitate the power series \(e^{x}=1+x+x^{2}/2!+x^{3}/3!+\cdots\). If we replace \(x\) by \(At\) and \(1\) by \(I\), this sum is an \(n\) by \(n\) matrix:

\[\textbf{Matrix exponential}\qquad e^{\Lambda t}=I+At+\frac{(At)^{2}}{2!}+ \frac{(At)^{3}}{3!}+\cdots.\] (5)

The series always converges, and its sum \(e^{\Lambda t}\) has the right properties:

\[(e^{As})(e^{At})=(e^{A(s+t)}),\qquad(e^{At})(e^{-At})=I,\quad\text{and}\quad \frac{d}{dt}(e^{At})=Ae^{At}.\] (6)

From the last one, \(u(t)=e^{\Lambda t}u(0)\) solves the differential equation. This solution must be the same as the form \(Se^{\Lambda t}S^{-1}u(0)\) used for computation. To prove directly that those solutions agree, remember that each power \((S\Lambda S^{-1})^{k}\) telescopes into \(A^{k}=S\Lambda^{k}S^{-1}\) (because \(S^{-1}\) cancels \(S\)). The whole exponential is diagonalized by \(S\):

\[e^{\Lambda t} =I+S\Lambda S^{-1}t+\frac{S\Lambda^{2}S^{-1}t^{2}}{2!}+\frac{S \Lambda^{3}S^{-1}t^{3}}{3!}+\cdots\] \[=S\left(I+\Lambda t+\frac{(\Lambda t)^{2}}{2!}+\frac{(\Lambda t) ^{3}}{3!}+\cdots\right)S^{-1}=Se^{\Lambda t}S^{-1}.\]

**Example 1**.: In equation (1), the exponential of \(A=\begin{bmatrix}-2&1\\ &1&-2\ 