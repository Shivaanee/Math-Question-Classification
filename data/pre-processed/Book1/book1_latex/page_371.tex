for exponential solutions \(e^{i\omega t}x\):

\[Mu^{\prime\prime}+Au=0\qquad\text{becomes}\qquad M(i\omega)^{2}e^{i\omega t}x+Ae^ {i\omega t}x=0.\] (8)

Canceling \(e^{i\omega t}\), and writing \(\lambda\) for \(\omega^{2}\), this is an eigenvalue problem:

\[\text{\bf Generalized problem}\;Ax=\lambda Mx\qquad\begin{bmatrix}2&-1\\ -1&2\end{bmatrix}x=\lambda\begin{bmatrix}m_{1}&0\\ 0&m_{2}\end{bmatrix}x.\] (9)

There is a solution when \(A-\lambda M\) is singular. The special choice \(M=I\) brings back the usual \(\det(A-\lambda I)=0\). We work out \(\det(A-\lambda M)\) with \(m_{1}=1\) and \(m_{2}=2\):

\[\det\begin{bmatrix}2-\lambda&-1\\ -1&2-2\lambda\end{bmatrix}=2\lambda^{2}-6\lambda+3=0\quad\text{gives}\quad \lambda=\frac{3\pm\sqrt{3}}{2}.\]

For the eigenvector \(x_{1}(\sqrt{3}-1,1)\), the two masses oscillate together--but the first mass only moves as far as \(\sqrt{3}-1\approx.73\). In the fastest mode, the components of \(x_{2}=(1+\sqrt{3},-1)\) have opposite signs and the masses move in opposite directions. This time the smaller mass goes much further.

The underlying theory is easier to explain if \(M\) is split into \(R^{\mathrm{T}}R\). (\(M\) is assumed to be positive definite.) Then the substitution \(y=Rx\) changes

\[Ax=\lambda Mx=\lambda R^{\mathrm{T}}Rx\quad\text{into}\quad AR^{-1}y=\lambda R ^{\mathrm{T}}y.\]

Writing \(C\) for \(R^{-1}\), and multiplying through by \((R^{\mathrm{T}})^{-1}=C^{\mathrm{T}}\), this becomes a standard eigenvalue problem for the _single_ symmetric matrix \(C^{\mathrm{T}}AC\):

\[\text{\bf Equivalent problem}\qquad C^{\mathrm{T}}ACy=\lambda y.\] (10)

The eigenvalues \(\lambda_{j}\) are the same as for the original \(Ax=\lambda Mx\). and the eigenvectors are related by \(y_{j}=Rx_{j}\). The properties of \(C^{\mathrm{T}}AC\) lead directly to thc properties of \(Ax=\lambda Mx\), when \(A=A^{\mathrm{T}}\) and \(M\) is positive definite:

1. The eigenvalues for \(Ax=\lambda Mx\) are real, because \(C^{\mathrm{T}}AC\) is symmetric.
2. The \(\lambda\)'s have the same signs as the eigenvalues of \(A\), by the law of inertia.
3. \(C^{\mathrm{T}}AC\) has orthogonal eigenvectors \(y_{j}\). So the eigenvectors of \(Ax=\lambda Mx\) have \[\text{\bf``$M$-orthogonality''}\qquad x_{i}^{\mathrm{T}}Mx_{j}=x_{i}^{\mathrm{T} }R^{\mathrm{T}}Rx_{j}=y_{i}^{\mathrm{T}}y_{j}=0.\] (11)

\(A\) and \(M\) are being _simultaneously diagonalized_. If \(S\) has the \(x_{j}\) in its columns, then \(S^{\mathrm{T}}AS=\Lambda\) and \(S^{\mathrm{T}}MS=I\). This is a _congruence_ transformation, with \(S^{\mathrm{T}}\) on the left, and not a similarity transformation with \(S^{-1}\). The main point is easy to summarize: As long as \(M\) is positive definite, the generalized eigenvalue problem \(Ax=-\lambda Mx\) behaves exactly like \(Ax=\lambda x\).

 