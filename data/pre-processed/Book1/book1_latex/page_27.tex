1. What multiple $\ell$ of equation 1 should be subtracted from equation 2 ?
$$
\begin{aligned}
2 x+3 y & =1 \\
10 x+9 y & =11 .
\end{aligned}
$$

After this elimination step, write down the upper triangular system and circle the two pivots. The numbers 1 and 11 have no influence on those pivots.
2. Solve the triangular system of Problem 1 by back-substitution, $y$ before $x$. Verify that $x$ times $(2,10)$ plus $y$ times $(3,9)$ equals $(1,11)$. If the right-hand side changes to $(4,44)$, what is the new solution?
3. What multiple of equation 2 should be subtracted from equation 3 ?
$$
\begin{aligned}
& 2 x-4 y=6 \\
& -x+5 y=0 .
\end{aligned}
$$

After this elimination step, solve the triangular system. If the right-hand side changes to $(-6,0)$, what is the new solution?
4. What multiple $\ell$ of equation 1 should be subtracted from equation 2 ?
$$
\begin{aligned}
& a x+b y=f \\
& c x+d y=g .
\end{aligned}
$$

The first pivot is $a$ (assumed nonzero). Elimination produces what formula for the second pivot? What is $y$ ? The second pivot is missing when $a d=b c$.
5. Choose a right-hand side which gives no solution and another right-hand side which gives infinitely many solutions. What are two of those solutions?
$$
\begin{aligned}
& 3 x+2 y=10 \\
& 6 x+4 y=\square
\end{aligned} .
$$
6. Choose a coefficient $b$ that makes this system singular. Then choose a right-hand side $g$ that makes it solvable. Find two solutions in that singular case.
$$
\begin{aligned}
& 2 x+b y=16 \\
& 4 x+8 y=g .
\end{aligned}
$$
7. For which numbers $a$ does elimination break down (a) permanently, and (b) temporarily?
$$
\begin{aligned}
& a x+3 y=-3 \\
& 4 x+6 y=6 .
\end{aligned}
$$

Solve for $x$ and $y$ after fixing the second breakdown by a row exchange.
