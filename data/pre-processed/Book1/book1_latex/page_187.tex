The transpose \(A^{\rm T}\) can be defined by the following property: The inner product of \(Ax\) with \(y\) equals the inner product of \(x\) with \(A^{\rm T}y\). Formally, this simply means that

\[(Ax)^{\rm T}y=x^{\rm T}A^{\rm T}y=x^{\rm T}(A^{\rm T}y).\] (8)

This definition gives us another (better) way to verify the formula \((AB)^{\rm T}=B^{\rm T}A^{\rm T}\), Use equation (8) twice:

\[\mbox{\bf Move $A$ then move $B$}\qquad(ABx)^{\rm T}y=(Bx)^{\rm T}(A^{\rm T}Y)=x^{ \rm T}(B^{\rm T}A^{\rm T}y).\]

The transposes turn up in reverse order on the right side, just as the inverses do in the formula \((AB)^{-1}=B^{-1}A^{-1}\). We mention again that these two formulas meet to give the remarkable combination \((A^{-1})^{\rm T}=(A^{\rm T})^{-1}\).

## Problem Set 3.2

1. (a) Given any two positive numbers \(x\) and \(y\), choose the vector \(b\) equal to \((\sqrt{x},\sqrt{y})\), and choose \(a=(\sqrt{y},\sqrt{x})\). Apply the Schwarz inequality to compare the arithmetic mean \(\frac{1}{2}(x+y)\) with the geometric mean \(\sqrt{xy}\). (b) Suppose we start with a vector from the origin to the point \(x\), and then add a vector of length \(\|y\|\) connecting \(x\) to \(x+y\). The third side of the triangle goes from the origin to \(x+y\). _The triangle inequality asserts that this distance cannot be greater than the sum of the first two_: \[\|x+y\|\leq\|x\|+\|y\|.\] After squaring both sides, and expanding \((x+y)^{\rm T}(x+y)\), reduce this to the Schwarz inequality.
2. Verify that the length of the projection in Figure 3.7 is \(\|p\|=\|b\|\cos\theta\), using formula (5).
3. What multiple of \(a=(1,1,1)\) is closest to the point \(b=(2,4,4)\)? Find also the point closest to \(a\) on the line through \(b\).
4. Explain why the Schwarz inequality becomes an equality in the case that \(a\) and \(b\) lie on the same line through the origin, and only in that case. What if they lie on opposite sides of the origin?
5. In \(n\) dimensions, what angle does the vector \((1,1,\ldots,1)\) make with the coordinate axes? What is the projection matrix \(P\) onto that vector?
6. The Schwarz inequality has a one-line proof if \(a\) and \(b\) are normalized ahead of time to be unit vectors: \[|a^{\rm T}b|=\left|\sum a_{j}b_{j}\right|\leq\sum|a_{j}||b_{j}|\leq\sum\frac{ |a_{j}|^{2}+|b_{j}|^{2}}{2}=\frac{1}{2}+\frac{1}{2}=\|a\|\|b\|.\] 