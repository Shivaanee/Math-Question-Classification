This is the basic statement of the problem. Note that it is a first-order equation--no higher derivatives appear--and it is _linear_ in the unknowns, It also has _constant coefficients_; the matrix \(A\) is independent of time.

How do we find \(u(t)\)? If there were only one unknown instead of two, that question would be easy to answer. We would have a scalar instead of a vector equation:

\[\text{{Single equation}}\qquad\frac{du}{dt}=au\quad\text{with}\quad u=u(0)\; \text{ at }\;t=0.\] (3)

The solution to this equation is the one thing you need to know:

\[\text{{Pure exponential}}\qquad u(t)=e^{at}u(0).\] (4)

At the initial time \(t=0\), \(u\) equals \(u(0)\) because \(e^{0}=1\). The derivative of \(e^{at}\) has the required factor \(a\), so that \(du/dt=au\). Thus the initial condition and the equation are both satisfied.

Notice the behavior of \(u\) for large times. The equation is unstable if \(a>0\), neutrally stable if \(a=0\), or stable if \(a<0\); the factor \(e^{at}\) approaches infinity, remains bounded, or goes to zero. If \(a\) were a complex number, \(a=\alpha+i\beta\), then the same tests would be applied to the real part \(\alpha\). The complex part produces oscillations \(e^{i\beta t}=\cos\beta t+i\sin\beta t\). Decay or growth is governed by the factor \(e^{\alpha t}\).

So much for a single equation. We shall take a direct approach to systems, and look for solutions with the _same exponential dependence on \(t\)_ just found in the scalar case:

\[\begin{split} v(t)&=e^{\lambda t}y\\ w(t)&=e^{\lambda t}z\end{split}\] (5)

or in vector notation

\[u(t)=e^{\lambda t}x.\] (6)

This is the whole key to differential equations \(du/dt=Au\): _Look for pure exponential solutions_. Substituting \(v=e^{\lambda t}y\) and \(w=e^{\lambda t}z\) into the equation, we find

\[\begin{split}\lambda e^{\lambda t}y&=4e^{\lambda t }y-5e^{\lambda t}z\\ \lambda e^{\lambda t}z&=2e^{\lambda t}y-3e^{\lambda t }z.\end{split}\]

The factor \(e^{\lambda t}\) is common to every term, and can be removed. This cancellation is the reason for assuming the same exponent \(\lambda\) for both unknowns; it leaves

\[\begin{split}\text{{Eigenvalue problem}}\qquad\begin{split} 4y-5z&=\lambda y\\ 2y-3z&=\lambda z.\end{split}\] (7)

That is the eigenvalue equation. In matrix form it is \(Ax=\lambda x\). You can see it again if we use \(u=e^{\lambda t}x\)--a number \(e^{\lambda t}\) that grows or decays times a fixed vector \(x\). _Substituting into \(du/dt=Au\) gives \(\lambda e^{\lambda t}x=Ae^{\lambda t}x\). The cancellation of \(e^{\lambda t}\) produces_

\[\text{{Eigenvalue equation}}\qquad Ax=\lambda x.\] (8) 