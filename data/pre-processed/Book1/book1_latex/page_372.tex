

### Problem Set 6.2

**1.** For what range of numbers \(a\) and \(b\) are the matrices \(A\) and \(B\) positive definite?

\[A=\begin{bmatrix}a&2&2\\ 2&a&2\\ 2&2&a\end{bmatrix}\qquad B=\begin{bmatrix}1&2&4\\ 2&b&8\\ 4&8&7\end{bmatrix}.\]

**2.** Decide for or against the positive definiteness of

\[A=\begin{bmatrix}2&-1&-1\\ -1&2&-1\\ -1&-1&2\end{bmatrix},\qquad B=\begin{bmatrix}2&-1&-1\\ -1&2&1\\ -1&1&2\end{bmatrix},\qquad C=\begin{bmatrix}0&1&2\\ 1&0&1\\ 2&1&0\end{bmatrix}^{2}.\]

**3.** Construct an indefinite matrix with its largest entries on the main diagonal:

\[A=\begin{bmatrix}1&b&-b\\ b&1&b\\ -b&b&1\end{bmatrix}\quad\text{ with }|b|<1\text{ can have }\text{det}A<0.\]

**4.** Show from the eigenvalues that if \(A\) is positive definite, so is \(A^{2}\) and so is \(A^{-1}\).

**5.**_If \(A\) and \(B\) are positive definite, then \(A+B\) is positive definite_. Pivots and eigenvalues are not convenient for \(A+B\). Much better to prove \(x^{\rm T}(A+B)x>0\).

**6.** From the pivots, eigenvalues, and eigenvectors of \(A=\begin{bmatrix}5&4\\ 4&5\end{bmatrix}\), write \(A\) as \(R^{\rm T}R\) in three ways: \((L\sqrt{D})(\sqrt{D}L^{\rm T})\), \((Q\sqrt{\Lambda})(\sqrt{\Lambda}Q^{\rm T})\), and \((Q\sqrt{\Lambda}Q^{\rm T})(Q\sqrt{\Lambda}Q^{\rm T})\).

**7.** If \(A=Q\Lambda Q^{\rm T}\) is symmetric positive definite, then \(R=Q\sqrt{\Lambda}Q^{\rm T}\) is its _symmetric positive definite square root_. Why does \(R\) have positive eigenvalues? Compute \(R\) and verify \(R^{2}=A\) for

\[A=\begin{bmatrix}10&6\\ 6&10\end{bmatrix}\qquad\text{ and }\qquad A=\begin{bmatrix}10&-6\\ -6&10\end{bmatrix}.\]

**8.** If \(A\) is symmetric positive definite and \(C\) is nonsingular, prove that \(B=C^{\rm T}AC\) is also symmetric positive definite.

**9.** If \(A=R^{\rm T}R\) prove the generalized Schwarz inequality \(|x^{\rm T}Ay|^{2}\leq(x^{\rm T}Ax)(y^{\rm T}Ay)\).

**10.** The ellipse \(u^{2}+4v^{2}=1\) corresponds to \(A=\begin{bmatrix}1&0\\ 0&4\end{bmatrix}\). Write the eigenvalues and eigenvectors, and sketch the ellipse.

**11.** Reduce the equation \(3u^{2}-2\sqrt{2}uv+2v^{2}=1\) to a sum of squares by finding the eigenvalues of the corresponding \(A\), and sketch the ellipse.

