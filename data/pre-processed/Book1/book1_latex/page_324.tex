The most important special case is when \(r=1\). Then \(a+ib\) is \(e^{i\theta}=\cos\theta+i\sin\theta\). It falls on the _unit circle_ in the complex plane. As \(\theta\) varies from \(0\) to \(2\pi\), this number \(e^{i\theta}\) circles around zero at the constant radial distance \(|e^{i\theta}|=\sqrt{\cos^{2}\theta+\sin^{2}\theta}=1\).

**Example 1**.: \(x=3+4i\) times its conjugate \(\overline{x}=3-4i\) is the absolute value squared:

\[x\overline{x}=(3+4i)(3-4i)=25=|x|^{2}\quad\text{so}\quad r=|x|=5.\]

To divide by \(3+4i\), multiply numerator and denominator by its conjugate \(3-4i\):

\[\frac{2+i}{3+4i}=\frac{2+i}{3+4i}\frac{3-4i}{3-4i}=\frac{10-5i}{25}.\]

In polar coordinates, multiplication and division are easy:

* \(re^{i\theta}\) times \(Re^{i\alpha}\) has absolute value \(rR\) and angle \(\theta+\alpha\).
* \(re^{i\theta}\) divided by \(Re^{i\alpha}\) has absolute value \(r/R\) and angle \(\theta-\alpha\).

### Lengths and Transposes in the Complex Case

We return to linear algebra, and make the conversion from real to complex. By definition, _the complex vector space \(\mathbf{C}^{n}\) contains all vectors \(x\) with \(n\) complex components_:

\[\text{Complex vector}\qquad x=\begin{bmatrix}x_{1}\\ x_{2}\\ \vdots\\ x_{n}\end{bmatrix}\quad\text{ with components}\quad x_{j}=a_{j}+ib_{j}.\]

Vectors \(x\) and \(y\) are still added component by component. Scalar multiplication \(cx\) is now done with complex numbers \(c\). The vectors \(v_{1},\ldots,v_{k}\) are linearly _dependent_ if some nontrivial combination gives \(c_{1}v_{1}+\ldots+c_{k}v_{k}=0\); the \(c_{j}\) may now be complex. The unit coordinate vectors are still in \(\mathbf{C}^{n}\); they are still independent; and they still form a basis. Therefore \(\mathbf{C}^{n}\) is a complex vector space of dimension \(n\).

In the new definition of length, each \(x_{j}^{2}\) is replaced by its modulus \(|x_{j}|^{2}\):

\[\text{Length squared}\qquad\|x\|^{2}=|x_{1}|^{2}+\cdots+|x_{n}|^{2}.\] (4)

**Example 2**.: \(x=\begin{bmatrix}1\\ i\end{bmatrix}\quad\text{and}\quad\|x\|^{2}=2\); \(y=\begin{bmatrix}2+i\\ 2-4i\end{bmatrix}\quad\text{and}\quad\|y\|^{2}=25\).

For real vectors there was a close connection between the length and the inner product: \(\|x\|^{2}=x^{\mathrm{T}}x\). This connection we want to preserve. The inner product must be modified to match the new definition of length, and we _conjugate the first vector in the inner product_. Replacing \(x\) by \(\overline{x}\), _the inner product becomes_

\[\text{Inner product}\qquad\overline{x}^{\mathrm{T}}y=\overline{x}_{1}y_{1}+ \cdots+\overline{x}_{n}y_{n}.\] (5) 