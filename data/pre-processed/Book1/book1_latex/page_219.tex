10. If \(q_{1}\) and \(q_{2}\) are the outputs from Gram-Schmidt, what were the possible input vectors \(a\) and \(b\)?
11. Show that an orthogonal matrix that is upper triangular must be diagonal.
12. What multiple of \(a_{1}=\left[\begin{smallmatrix}1\\ 1\end{smallmatrix}\right]\) should be subtracted from \(a2=\left[\begin{smallmatrix}4\\ 0\end{smallmatrix}\right]\) to make the result orthogonal to \(a_{1}\)? Factor \(\left[\begin{smallmatrix}1&4\\ 1&0\end{smallmatrix}\right]\) into \(QR\) with orthonormal vectors in \(Q\).
13. Apply the Gram-Schmidt process to \[a=\begin{bmatrix}0\\ 0\\ 0\\ 1\end{bmatrix},\qquad b=\begin{bmatrix}0\\ 1\\ 1\end{bmatrix},\qquad c=\begin{bmatrix}1\\ 1\\ 1\end{bmatrix}\] and write the result in the form \(A=QR\).
14. From the nonorthogonal \(a\), \(b\), \(c\), find orthonormal vectors \(q_{1}\), \(q_{2}\), \(q_{3}\): \[a=\begin{bmatrix}1\\ 1\\ 0\end{bmatrix},\qquad b=\begin{bmatrix}1\\ 0\\ 1\end{bmatrix},\qquad c=\begin{bmatrix}0\\ 1\\ 1\end{bmatrix}.\]
15. Find an orthonormal set \(q_{1}\), \(q_{2}\), \(q_{3}\) for which \(q_{1}\), \(q_{2}\) span the column space of \[A=\begin{bmatrix}1&1\\ 2&-1\\ -2&4\end{bmatrix}.\] Which fundamental subspace contains \(q_{3}\)? What is the least-squares solution of \(Ax=b\) if \(b=[1\ \ 2\ \ 7]^{\mathrm{T}}\)?
16. Express the Gram-Schmidt orthogonalization of \(a_{1}\), \(a_{2}\) as \(A=QR\): \[a_{1}=\begin{bmatrix}1\\ 2\\ 2\end{bmatrix},\qquad a_{2}=\begin{bmatrix}1\\ 3\\ 1\end{bmatrix}.\] Given \(n\) vectors \(a_{i}\) with \(m\) components, what are the shapes of \(A\), \(Q\), and \(R\)?
17. With the same matrix \(A\) as in Problem 16, and with \(b=[1\ \ 1\ \ 1]^{\mathrm{T}}\), use \(A=QR\) to solve the least-squares problem \(Ax=b\).
18. If \(A=QR\), find a simple formula for the projection matrix \(P\) onto the column space of \(A\).
19. Show that these _modified Gram-Schmidt_ steps produce the same \(C\) as in equation (10): \[C^{*}=c-(q_{1}^{\mathrm{T}}c)q_{1}\qquad\text{and}\qquad C=C^{*}-(q_{2}^{ \mathrm{T}}C^{*})q_{2}.\] This is much more stable, to subtract the projections one at a time.

 