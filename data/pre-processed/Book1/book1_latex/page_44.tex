1. If \(A^{2}\) is defined then \(A\) is necessarily square. 2. If \(AB\) and \(BA\) are defined then \(A\) and \(B\) are square. 3. If \(AB\) and \(BA\) are defined then \(AB\) and \(BA\) are square. 4. If \(AB=B\) then \(A=I\).
**43.**: If \(A\) is \(m\) by \(n\), how many separate multiplications are involved when

1. \(A\) multiplies a vector \(x\) with \(n\) components? 2. \(A\) multiplies an \(n\) by \(p\) matrix \(B\)? Then \(AB\) is \(m\) by \(p\). 3. \(A\) multiplies itself to produce \(A^{2}\)? Here \(m=n\).
**44.**: To prove that \((AB)C=A(BC)\), use the column vectors \(b_{1},\ldots,b_{n}\) of \(B\). First suppose that \(C\) has only one column \(c\) with entries \(c_{1},\ldots,c_{n}\): \(AB\) has columns \(Ab_{1},\ldots,Ab_{n}\), and \(Bc\) has one column \(c_{1}b_{1}+\cdots+c_{n}b_{n}\). Then \((AB)c=c_{1}Ab_{1}+\cdots+c_{n}Ab_{n}\), equals \(A(c_{1}b_{1}+\cdots+c_{n}b_{n})=A(Bc)\). _Linearity_ gives equality of those two sums, and \((AB)c=A(Bc)\). The same is true for all other \(\underline{\underline{\underline{\underline{\underline{\underline{\ \underline{\ \underline{\ \underline{\ \ 