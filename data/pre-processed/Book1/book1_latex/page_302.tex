

### Problem Set 5.3

1. Prove that every third Fibonacci number in \(0,1,1.2,3,\ldots\) is even.
2. Bernadelli studied a beetle "which lives three years only. and propagates in as third year." They survive the first year with probability \(\frac{1}{2}\), and the second with probability \(\frac{1}{3}\), and then produce six females on the way out: \[\textbf{Beetle matrix}\qquad A=\begin{bmatrix}0&0&6\\ \frac{1}{2}&0&0\\ 0&\frac{1}{3}&0\end{bmatrix}.\] Show that \(A^{3}=I\), and follow the distribution of 3000 beetles for six years.
3. For the Fibonacci matrix \(A=\begin{bmatrix}1&1\\ 1&0\end{bmatrix}\) , compute \(A^{2}\), \(A^{3}\), and \(A^{4}\). Then use the text and a calculator to find \(F_{20}\).
4. Suppose each "Gibonacci" number \(G_{k+2}\) is the _average_ of the two previous numbers \(G_{k+1}\) and \(G_{k}\). Then \(G_{k+2}=\frac{1}{2}(G_{k+1}+G_{k})\): \[\begin{array}{l}G_{k+2}=\frac{1}{2}G_{k+1}+\frac{1}{2}G_{k}\\ G_{k+1}=G_{k+1}\end{array}\quad\text{ is }\quad\begin{bmatrix}G_{k+2}\\ G_{k+1}\end{bmatrix}=\begin{bmatrix}A\end{bmatrix}\begin{bmatrix}G_{k+1}\\ G_{k}\end{bmatrix}.\] 1. Find the eigenvalues and eigenvectors of \(A\). 2. Find the limit as \(n\to\infty\) of the matrices \(A^{n}=S\Lambda^{n}S^{-1}\). 3. If \(G_{0}=0\) and \(G_{1}=1\), show that the Gibonacci numbers approach \(\frac{2}{3}\).
5. Diagonalize the Fibonacci matrix by completing \(S^{-1}\): \[\begin{bmatrix}1&1\\ 1&0\end{bmatrix}=\begin{bmatrix}\lambda_{1}&\lambda_{2}\\ 1&1\end{bmatrix}\begin{bmatrix}\lambda_{1}&0\\ 0&\lambda_{2}\end{bmatrix}\begin{bmatrix}\\ \end{bmatrix}.\] Do the multiplication \(S\Lambda^{k}S^{-1}\begin{bmatrix}1\\ 0\end{bmatrix}\) to find its second component. This is the \(k\)th Fibonacci number \(F_{k}=(\lambda_{1}^{k}\lambda_{2}^{k})/(\lambda_{1}\lambda_{2})\).
6. The numbers \(\lambda_{1}^{k}\) and \(\lambda_{2}^{k}\) satisfy the Fibonacci rule \(F_{k+2}=F_{k+1}+F_{k}\): \[\lambda_{1}^{k+2}=\lambda_{1}^{k+1}+\lambda_{1}^{k}\qquad\text{and}\qquad \lambda_{2}^{k+2}=\lambda_{2}^{k+1}+\lambda_{2}^{k}.\] Prove this by using the original equation for the \(\lambda\)'s (multiply it by \(\lambda^{k}\)). Then any combination of \(\lambda_{1}^{k}\) and \(\lambda_{2}^{k}\) satisfies the rule. The combination \(F_{k}=(\lambda_{1}^{k}-\lambda_{2}^{k})/(\lambda_{1}-\lambda_{2})\) gives the right start of \(F_{0}=0\) and \(F_{1}=1\).

