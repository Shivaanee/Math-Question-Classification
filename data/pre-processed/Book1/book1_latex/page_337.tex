

**Example 1**.: \(A=\left[\begin{smallmatrix}1&0\\ 0&0\end{smallmatrix}\right]\) has eigenvalues \(1\) and \(0\). Each \(B\) is \(M^{-1}AM\):

\[\text{If }M =\left[\begin{matrix}1&b\\ 0&1\end{matrix}\right],\text{ then }B=\left[\begin{matrix}1&b\\ 0&0\end{matrix}\right]:\quad\text{ triangular with }\lambda=0\text{ and }0.\] \[\text{If }M =\left[\begin{matrix}1&1\\ -1&1\end{matrix}\right],\text{ then }B=\left[\begin{matrix}\frac{1}{2}&\frac{1}{2}\\ \frac{1}{2}&\frac{1}{2}\end{matrix}\right]:\quad\text{ projection with }\lambda=0\text{ and }0.\] \[\text{If }M =\left[\begin{matrix}a&b\\ c&d\end{matrix}\right],\text{ then }B=\text{an arbitrary matrix with }\lambda=0\text{ and }0.\]

In this case we can produce any \(B\) that has the correct eigenvalues. It is an easy case, because the eigenvalues \(1\) and \(0\) are distinct. The diagonal \(A\) was actually \(\Lambda\), the outstanding member of this family of similar matrices