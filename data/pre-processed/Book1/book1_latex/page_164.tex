Find \(H^{-1}\) and write \(v=(7,5,3,1)\) as a combination of the columns of \(H\).
48. Suppose we have two bases \(v_{1},\ldots,v_{n}\) and \(w_{1},\ldots,w_{n}\) for \(\mathbf{R}^{n}\). If a vector has coefficients \(b_{i}\) in one basis and \(c_{i}\) in the other basis, what is the change-of-basis matrix in \(b=Mc\)? Start from \[b_{1}v_{1}+\cdots+b_{n}v_{n}=Vb=c_{1}w_{1}+\cdots+c_{n}w_{n}=Wc.\] Your answer represents \(T(v)=v\) with input basis of \(v\)'s and output basis of \(w\)'s. Because of different bases, the matrix is not \(I\).
49. True or false: If we know \(T(v)\) for \(n\) different nonzero vectors in \(\mathbf{R}^{2}\), then we know \(T(v)\) for every vector in \(\mathbf{R}^{n}\).
50. (Recommended) Suppose all vectors \(x\) in the unit square \(0\leq x_{1}\leq 1\), \(0\leq x_{2}\leq 1\) are transformed to \(Ax\) (\(A\) is \(2\) by \(2\)). 1. What is the shape of the transformed region (all \(Ax\))? 2. For which matrices \(A\) is that region a square? 3. For which \(A\) is it a line? 4. For which \(A\) is the new area still \(1\)?

### Review Exercises

1.1 Find a basis for the following subspaces of \(\mathbf{R}^{4}\): 1. The vectors for which \(x1=2x_{4}\). 2. The vectors for which \(x_{1}+x_{2}+x_{3}=0\) and \(x_{3}+x_{4}=0\). 3. The subspace spanned by \((1,1,1,1)\), \((1,2,3,4)\), and \((2,3,4,5)\).
1.2 By giving a basis, describe a two-dimensional subspace of \(\mathbf{R}^{3}\) that contains none of the coordinate vectors \((1,0,0)\), \((0,1,0)\), \((0,0,1)\).
1.3 True or false, with counterexample if false: 1. If the vectors \(x_{1},\ldots,x_{m}\) span a subspace \(S\), then \(\dim S=m\). 2. The intersection of two subspaces of a vector space cannot be empty. 3. If \(Ax=Ay\), then \(x=y\). 4. The row space of \(A\) has a unique basis that can be computed by reducing \(A\) to echelon form. 5. If a square matrix \(A\) has independent columns, so does \(A^{2}\).

 