

**Problem Set 6.3, page 337**

1. \(A^{\mathrm{T}}A=\begin{bmatrix}5&20\\ 20&80\end{bmatrix}\) has only \(\sigma_{1}^{2}=85\) with \(v_{1}=\begin{bmatrix}1/\sqrt{17}\\ 4/\sqrt{17}\end{bmatrix}\), so \(v_{2}=\begin{bmatrix}4/\sqrt{17}\\ -1/\sqrt{17}\end{bmatrix}\).
3. \(A^{\mathrm{T}}A=\begin{bmatrix}2&1\\ 1&1\end{bmatrix}\) has eigenvalues \(\sigma_{1}^{2}=\dfrac{3+\sqrt{5}}{2}\) and \(\sigma_{2}^{2}=\dfrac{3-\sqrt{5}}{2}\). Since \(A=A^{\mathrm{T}}\), the eigenvectors of \(A^{\mathrm{T}}A\) are the same as for \(A\). Since \(\lambda_{2}=\frac{1}{2}(1-\sqrt{5})\) is _negative_, \(\sigma_{1}=\lambda_{1}\) but \(\sigma_{2}=-\lambda_{2}\). The unit eigenvectors are the same as in Section 6.2 for \(A\), except for the effect of this minus sign (because we need \(Av_{2}=\sigma_{2}u_{2}\)): \[u_{1}=v_{1}=\begin{bmatrix}\lambda_{1}/\sqrt{1+\lambda_{1}^{2}}\\ 1/\sqrt{1+\lambda_{1}^{2}}\end{bmatrix}\text{ and }u_{2}=-v_{2}=\begin{bmatrix} \lambda_{2}/\sqrt{1+\lambda_{2}^{2}}\\ 1/\sqrt{1+\lambda_{2}^{2}}\end{bmatrix}.\]
5. \(AA^{\mathrm{T}}=\begin{bmatrix}2&1\\ 1&2\end{bmatrix}\) has \(\sigma_{1}^{2}=3\) with \(u_{1}=\begin{bmatrix}1/\sqrt{2}\\ 1/\sqrt{2}\end{bmatrix}\) and \(\sigma_{2}^{2}=1\) with \(u_{2}=\begin{bmatrix}1/\sqrt{2}\\ -1/\sqrt{2}\end{bmatrix}\). \(A^{\mathrm{T}}A=\begin{bmatrix}1&1&0\\ 1&2&1\\ 0&1&1\end{bmatrix}\) has \(\sigma_{1}^{2}=3\) with \(v_{1}=\begin{bmatrix}1/\sqrt{6}\\ 2/\sqrt{6}\\ 1/\sqrt{6}\end{bmatrix}\), \(\sigma_{2}^{2}=1\) with \(v_{2}=\begin{bmatrix}1/\sqrt{2}\\ 0\\ -1/\sqrt{2}\end{bmatrix}\), and nullvector \(v_{3}=\begin{bmatrix}1/\sqrt{3}\\ -1/\sqrt{3}\end{bmatrix}\). Then \(\begin{bmatrix}1&1&0\\ 0&1&1\end{bmatrix}=[u_{1}\quad u_{2}]\begin{bmatrix}\sqrt{3}&0&0\\ 0&1&0\end{bmatrix}[v_{1}\quad v_{2}\quad v_{3}]^{\mathrm{T}}\).
7. \(A=12\,u\upsilon^{\mathrm{T}}\) has one singular value \(\sigma_{1}=12\).
9. Multiply \(U\,\Sigma\,V^{\mathrm{T}}\) using columns (of \(U\)) times rows (of \(\Sigma\,V^{\mathrm{T}}\)).
10. To make \(A\) singular, the smallest change sets its smallest singular value \(\sigma_{2}\) to zero.
11. The singular values of \(A+I\) are not \(\sigma_{j}+1\). They come from eigenvalues of \((A+I)^{\mathrm{T}}(A+I)\).
12. \(A^{+}=\begin{bmatrix}\frac{1}{4}\\ \frac{1}{4}\\ \frac{1}{4}\\ \frac{1}{4}\end{bmatrix}\), \(B=\begin{bmatrix}0&1\\ 1&0\end{bmatrix}\begin{bmatrix}1&0&0\\ 0&1&0\end{bmatrix}\begin{bmatrix}1&0&0\\ 0&1&0\end{bmatrix}\), \(B^{+}=\begin{bmatrix}0&1\\ 1&0\end{bmatrix}\), \(C^{+}=\begin{bmatrix}\frac{1}{2}&0\\ \frac{1}{2}&0\end{bmatrix}\). \(A^{+}\) is the right-inverse of \(A\); \(B^{+}\) is the left-inverse of \(B\).
13. \(A^{\mathrm{T}}A=\begin{bmatrix}10&6\\ 6&10\end{bmatrix}=\dfrac{1}{2}\begin{bmatrix}1&-1\\ 1&1\end{bmatrix}\begin{bmatrix}4&0\\ 0&16\end{bmatrix}\begin{bmatrix}1&1\\ -1&1\end{bmatrix}\), take square roots of 4 and 16 to obtain \(S=\dfrac{1}{2}\begin{bmatrix}1&-1\\ 1&1\end{bmatrix}\begin{bmatrix}2&0\\ 0&4\end{bmatrix}\begin{bmatrix}1&1\\ -1&1\end{bmatrix}=\begin{bmatrix}3&1\\ 1&3\end{bmatrix}\) and \(Q=AS^{-1}=\\ \dfrac{1}{\sqrt{10}}\begin{bmatrix}3&1\\ -1&3\end{bmatrix}\).
14. With independent columns, the row space is all of \(\mathbf{R}^{n}\); check \((A^{\mathrm{T}}A)A^{+}b=A^{\mathrm{T}}b\). 2. \(A^{\mathrm{T}}(AA^{\mathrm{T}})^{-1}b\) is in the row space because \(A^{\mathrm{T}}\) times any vector is in that space; now \((A^{\mathrm{T}}A)A^{+}b=A^{\mathrm{T}}AA^{\mathrm{T}}(AA^{\mathrm{T}})^{-1}b=A^ {\mathrm{T}}b\). Both cases give \(A^{\mathrm{T}}Ax^{+}=A^{\mathrm{T}}b\).

