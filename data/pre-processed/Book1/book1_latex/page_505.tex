3. Projection on \(a_{3}\): \((-2/3,\,1/3,\,-2/3)\); the sum is \(b\) itself; notice that \(a_{1}a_{1}^{\mathrm{T}},\,a_{2}a_{2}^{\mathrm{T}},\,a_{3}a_{3}^{\mathrm{T}}\) are projections onto three orthogonal directions. Their sum is projection onto the whole space and should be the identity.
5. \((I-2uu^{\mathrm{T}})^{\mathrm{T}}(I-2uu^{\mathrm{T}})=I-4uu^{\mathrm{T}}+4uu^{ \mathrm{T}}uu^{\mathrm{T}}=I\); \(Q=\begin{bmatrix}\frac{1}{2}&-\frac{1}{2}&\frac{1}{2}&\frac{1}{2}\\ -\frac{1}{2}&\frac{1}{2}&\frac{1}{2}&\frac{1}{2}\\ \frac{1}{2}&\frac{1}{2}&\frac{1}{2}&-\frac{1}{2}\\ \frac{1}{2}&\frac{1}{2}&-\frac{1}{2}&\frac{1}{2}\end{bmatrix}\).
7. \((x_{1}q_{1}+\cdots+x_{n}q_{n})^{\mathrm{T}}(x_{1}q_{1}+\cdots+x_{n}q_{n})=x_{1 }^{2}+\cdots+x_{n}^{2}\Rightarrow\|b\|^{2}=b^{\mathrm{T}}b=x_{1}^{2}+\cdots+x_ {n}^{2}\).
9. The combination closest to \(q_{3}\) is \(0q_{1}+0q_{2}\).
10. \(Q\) is upper triangular: column 1 has \(q_{11}=\pm 1\); by orthogonality column 2 must be (0, \(\pm 1\), 0, \(\ldots\)); by orthogonality column 3 is (0, 0, \(\pm 1,\ldots\)); and so on.
11. \(A=\begin{bmatrix}0&0&1\\ 0&1&1\\ 1&1&1\end{bmatrix}=\begin{bmatrix}0&0&1\\ 0&1&0\\ 1&0&0\end{bmatrix}\begin{bmatrix}1&1&1\\ 0&1&1\\ 0&0&1\end{bmatrix}=Q\,R\).
12. \(q_{1}=\begin{bmatrix}1/3\\ 2/3\\ -2/3\end{bmatrix}\), \(q_{2}=\begin{bmatrix}2/3\\ 1/3\\ 2/3\end{bmatrix}\), \(q_{3}=\begin{bmatrix}-2/3\\ 2/3\\ 1/3\end{bmatrix}\) is in the left nullspace; \(\widehat{x}=\begin{bmatrix}q_{1}^{\mathrm{T}}b\\ q_{2}^{\mathrm{T}}b\\ \end{bmatrix}=\begin{bmatrix}1\\ 2\end{bmatrix}\).
13. \(R\widehat{x}=Q^{\mathrm{T}}b\) gives \(\begin{bmatrix}3&3\\ 0&\sqrt{2}\end{bmatrix}\begin{bmatrix}\widehat{x}\\ \end{bmatrix}=\begin{bmatrix}5/3\\ 0\end{bmatrix}\) and \(\widehat{x}=\begin{bmatrix}5/9\\ 0\end{bmatrix}\).
14. \(C^{*}-\left(q_{2}^{\mathrm{T}}C^{*}\right)q_{2}\) is \(c-\left(q_{1}^{\mathrm{T}}c\right)q_{1}-\left(q_{2}^{\mathrm{T}}c\right)q_{2}\) because \(q_{2}^{\mathrm{T}}q_{1}=0\).
15. By orthogonality, the closest functions are \(0\sin 2x=0\) and \(0+0x=0\).
16. \(a_{0}=1/2\), \(a_{1}=0\), \(b_{1}=2/\pi\).
17. The closest line is \(y=1/3\) (horizontal since \((x,\,x^{2})=0\)).
18. \((1/\sqrt{2},\,-1/\sqrt{2},\,0,\,0)\), \((1/\sqrt{6},\,1/\sqrt{6},\,2/\sqrt{6},\,0)\), \((-1/2\sqrt{3},\,-1/2\sqrt{3},\,1/2\sqrt{3},\,-1/\sqrt{3})\).
19. \(A=a=(1,-1,\,0,\,0)\); \(B=b-p=\left(\frac{1}{2},\,\frac{1}{2},\,-1,\,0\right)\) ; \(C=c-p_{A}-p_{B}=\left(\frac{1}{3},\,\frac{1}{3},\,\frac{1}{3},\,-1\right)\) Notice the pattern in those orthogonal vectors \(A\), \(B\), \(C\). Next, \((1,\,1,\,1)/4\).
20. True. \(Qx=x_{1}q_{1}+x_{2}q_{2}\). \(\|Qx\|^{2}=x_{1}^{2}+x_{2}^{2}\) because \(q_{1}^{\mathrm{T}}q_{2}=0\).

### Problem Set 3.5, page 196

1. \(F^{2}=\begin{bmatrix}4&0&0&0\\ 0&0&0&4\\ 0&0&4&0\\ 0&4&0&0\end{bmatrix}\), \(F^{4}=\begin{bmatrix}16&0&0&0\\ 0&16&0&0\\ 0&0&16&0\\ 0&0&0&16\end{bmatrix}=4^{2}I\).
2. The submatrix is \(F_{3}\).
3. \(e^{ix}=-1\) for \(x=(2k+1)\pi\), \(e^{i\theta}=i\) for \(\theta=2k\pi+\pi/2\), \(k\) is integer.

 