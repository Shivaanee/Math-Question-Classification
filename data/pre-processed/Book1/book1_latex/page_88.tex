Normally our vectors belong to one of the spaces \({\bf R}^{n}\); they are ordinary column vectors. If \(x=(1,0,0,3)\), then \(2x\) (and also \(x+x\)) has components 2, 0, 0, 6. The formal definition allows other things to be "vectors"-provided that addition and scalar multiplication are all right. We give three examples:

1. _The infinite-dimensional space_\({\bf R}^{\infty}\). Its vectors have infinitely many components, as in \(x=(1,2,1,2,\ldots)\). The laws for \(x+y\) and \(cx\) stay unchanged.
2. _The space of 3 by 2 matrices_. In this case the "vectors" are matrices! We can add two matrices, and \(A+B=B+A\), and there is a zero matrix, and so on. This space is almost the same as \({\bf R}^{6}\). (The six components are arranged in a rectangle instead of a column.) Any choice of \(m\) and \(n\) would give, as a similar example, the vector space of all \(m\) by \(n\) matrices.
3. _The space of functions_\(f(x)\). Here we admit all functions \(f\) that are defined on a fixed interval, say \(0\leq x\leq 1\). The space includes \(f(x)=x^{2}\), \(g(x)=\sin x\ 