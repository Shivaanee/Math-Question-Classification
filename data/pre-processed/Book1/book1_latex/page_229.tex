\(0,\ldots,7\) in binary and _reverse the order of their bits_. The subscripts appear in "bit-reversed order" on the left side of the graph. Even numbers come before odd (numbers ending in 0 come before numbers ending in 1).

#### Problem Set 3.5

1. What are \(F^{2}\) and \(F^{4}\) for the 4 by 4 Fourier matrix \(F\)?
2. Find a permutation \(P\) of the columns of \(F\) that produces \(FP=\overline{F}\) (\(n\) by \(n\)), Combine with \(F\overline{F}=nI\) to find \(F^{2}\) and \(F^{4}\) for the \(n\) by \(n\) Fourier matrix.
3. If you form a 3 by 3 submatrix of the 6 by 6 matrix \(F_{6}\), keeping only the entries in its first, third, and fifth rows and columns, what is that submatrix?
4. Mark all the sixth roots of 1 in the complex plane. What is the primitive root \(w_{6}\)? (Find its real and imaginary part.) Which power of \(w_{6}\) is equal to \(1/w_{6}\)? What is \(1+w+w^{2}+w^{3}+w^{4}+w^{5}\)?
5. Find all solutions to the equation \(e^{ix}=-1\), and all solutions to \(e^{i\theta}=i\).
6. What are the square and the square root of \(w_{128}\), the primitive 128th root of 1?
7. Solve the 4 by 4 system (6) if the right-hand sides are \(y_{0}=2\), \(y_{1}=0\), \(y_{2}=2\), \(y_{3}=0\). In other words, solve \(F_{4}c=y\).
8. Solve the same system with \(y=(2,0,-2,0)\) by knowing \(F_{4}^{-1}\) and computing \(c=F_{4}^{-1}y\). Verify that \(c_{0}+c_{1}e^{ix}+c_{2}e^{2ix}+c_{3}e^{3ix}\) takes the values 2, 0, \(-2\), 0 at the points \(x=0,\pi/2,\pi,3\pi/2\).
9. 1. If \(y=(1,1,1,1)\), show that \(c=(1,0,0,0)\) satisfies \(F_{4}c=y\). 2. Now suppose \(y=(1,0,0,0)\), and find \(c\).

Figure 3.12: Flow graph for the Fast Fourier Transform with \(n=4\).

 