

**9.**: 1. \(a_{11}\quad\) (b) \(\ell_{i1}=a_{i1}/a_{11}\quad\) (c) new \(a_{ij}\) is \(a_{ij}-\frac{a_{i1}}{a_{11}}a_{1j}\) (d) second pivot \(a_{22}-\frac{a_{21}}{a_{11}}a_{12}\).
**11.**: The coefficients of rows of \(B\) are \(2\), \(1\), \(4\) from \(A\). The first row of \(AB\) is [6 - 3].
**13.**: \(A=\begin{bmatrix}0&1\\ -1&0\end{bmatrix}\), \(B=\begin{bmatrix}0&1\\ 0&0\end{bmatrix}\), \(C=\begin{bmatrix}0&1\\ 1&0\end{bmatrix}\), \(D=A\), \(E=F=\begin{bmatrix}1&-1\\ 1&-1\end{bmatrix}\).
**15.**: \(AB_{1}=B_{1}A\) gives \(b=c=0\). \(AB_{2}=B_{2}A\) gives \(a=d\). So \(A=aI\).
**17.**: \(A(A+B)+B(A+B)\), \((A+B)(B+A)\), \(A^{2}+AB+BA+B^{2}\) always equal \((A+B)^{2}\).
**19.**: \(\begin{bmatrix}a&b\\ c&d\end{bmatrix}\begin{bmatrix}p&q\\ r&s\end{bmatrix}=\begin{bmatrix}a\\ c\end{bmatrix}\begin{bmatrix}p&q\\ d\end{bmatrix}+\begin{bmatrix}b\\ d\end{bmatrix}\begin{bmatrix}r&s\end{bmatrix}=\begin{bmatrix}ap+br&aq+bs\\ cp+dr&cq+ds\end{bmatrix}\).
**21.**: \(A^{n}=A\); \(B^{n}=\begin{bmatrix}1&0\\ 0&(-1)^{n}\end{bmatrix}\), \(C=\begin{bmatrix}0&0\\ 0&0\end{bmatrix}=\text{zero matrix}\).
**23.**: \(E_{32}E_{21}b=(1,-5,-35)\) but \(E_{21}E_{32}b=(1,-5,0)\). Then row 3 feels no effect from row 1.
**25.**: Changing \(a_{33}\) from 7 to 11 will change the third pivot from 5 to 9. Changing \(a_{33}\) from 7 to 2 will change the pivot from 5 to _no pivot_.
**27.**: To reverse \(E_{31}\), _add_ 7 times row 1 to row 3. The matrix is \(R_{31}=\begin{bmatrix}1&0&0\\ 0&1&0\\ 7&0&1\end{bmatrix}\).
**29.**: \(E_{13}=\begin{bmatrix}1&0&1\\ 0&1&0\\ 0&0&1\end{bmatrix}\); \(\begin{bmatrix}1&0&1\\ 0&1&0\\ 1&0&1\end{bmatrix}\); \(E_{31}E_{13}=\begin{bmatrix}2&0&1\\ 0&1&0\\ 1&0&1\end{bmatrix}\). Test on the identity matrix!
**31.**: \(E_{21}\) has \(\ell_{21}=-\frac{1}{2}\), \(E_{32}\) has \(\ell_{32}=-\frac{2}{3}\), \(E_{43}\) has \(\ell_{43}=-\frac{3}{4}\). Otherwise the \(E\)'s match \(I\). \(a+\ b+\ c=\begin{array}{ccc}4&a=2\\ 3&a+2b+4c=\begin{array}{ccc}8&\text{gives}&b=1.\\ &a+3b+9c=14&&c=1\end{array}\).
**35.**: 1. Each column is \(E\) times a column of \(B\). 2. \(EB=\begin{bmatrix}1&0\\ 1&1\end{bmatrix}\) \(\begin{bmatrix}1&2&4\\ 1&2&4\end{bmatrix}=\begin{bmatrix}1&2&4\\ 2&4&8\end{bmatrix}\). Rows of \(EB\) are combinations of rows of \(B\), so they are multiples of [1 - 4].
**37.**: (row 3) \(\cdot\)\(x\) is \(\sum a_{3j}x_{j}\), and \((A^{2})_{11}=\text{(row 1)}\cdot\text{(column 1)}=\sum a_{1j}a_{j1}\).
**39.**: \(BA=3I\) is 5 by 5, \(AB=5I\) is 3 by 3, \(AB=5D\) is 3 by 1, \(ABD\): No, \(A(B+C)\): No.
**41.**: 1. \(B=4I\). 2. \(B=0\). 3. \(B=\begin{bmatrix}0&0&1\\ 0&1&0\\ 1&0&0\end{bmatrix}\).
**42.**: Every row of \(B\) is \(1,0,0\).
**43.**: (a) \(mn\) (every entry). 2. \(mnp\). 3. \(n^{3}\) (this is \(n^{2}\) dot products).
**45.**: \(\begin{bmatrix}1\\ 2\\ 2\end{bmatrix}\) \(\begin{bmatrix}3&3&0\end{bmatrix}+\begin{bmatrix}0\\ 4\\ 1\end{bmatrix}\) \(\begin{bmatrix}1&2&1\end{bmatrix}=\begin{bmatrix}3&3&0\\ 6&6&0\\ 6&6&0\end{bmatrix}+\begin{bmatrix}0&0&0\\ 4&8&4\\ 1&2&1\end{bmatrix}=\begin{bmatrix}3&3&0\\ 10&14&4\\ 7&8&1\end{bmatrix}\).

