\[\lambda_{1}\text{ and }\lambda_{2}=.7:\qquad A=S\Lambda S^{-1}=\begin{bmatrix} \frac{2}{3}&\frac{1}{3}\\ \frac{1}{3}&-\frac{1}{3}\end{bmatrix}\begin{bmatrix}1&\\ &.7\end{bmatrix}\begin{bmatrix}1&1\\ 1&-2\end{bmatrix}.\]

To find \(A^{k}\), and the distribution after \(k\) years, change \(S\Lambda S^{-1}\) to \(S\Lambda^{k}S^{-1}\):

\[\begin{bmatrix}y_{k}\\ z_{k}\end{bmatrix} =A^{k}\begin{bmatrix}y_{0}\\ z_{0}\end{bmatrix}=\begin{bmatrix}\frac{2}{3}&\frac{1}{3}\\ \frac{1}{3}&-\frac{1}{3}\end{bmatrix}\begin{bmatrix}1^{k}&\\ &.7^{k}\end{bmatrix}\begin{bmatrix}1&1\\ 1&-2\end{bmatrix}\begin{bmatrix}y_{0}\\ z_{0}\end{bmatrix}\] \[=(y_{0}+z_{0})\begin{bmatrix}\frac{2}{3}\\ \frac{1}{3}\end{bmatrix}+(y_{0}-2z_{0})(.7)^{k}\begin{bmatrix}\frac{1}{3}\\ -\frac{1}{3}\end{bmatrix}.\]

Those two terms are \(c_{1}\lambda_{1}^{k}x_{1}+c_{2}\lambda_{2}^{k}x_{2}\). The factor \(\lambda_{1}^{k}=1\) is hidden in the first term. In the long run, the other factor \((.7)^{k}\) becomes extremely small. _The solution approaches a limiting state_\(u_{\infty}=(y_{\infty},z_{\infty})\):

\[\text{Steady state}\qquad\begin{bmatrix}y_{\infty}\\ z_{\infty}\end{bmatrix}=(y_{0}+z_{0})\begin{bmatrix}\frac{2}{3}\\ \frac{1}{3}\end{bmatrix}.\]

The total population is still \(y_{0}+z_{0}\), but in the limit \(\frac{2}{3}\) of this population is outside California and \(\frac{1}{3}\) is inside. This is true no matter what the initial distribution may have been! If the year starts with \(\frac{2}{3}\) outside and \(\frac{1}{3}\) inside, then it ends the same way:

\[\begin{bmatrix}.9&.2\\ .1&.8\end{bmatrix}\begin{bmatrix}\frac{2}{3}\\ \frac{1}{3}\end{bmatrix}=\begin{bmatrix}\frac{2}{3}\\ \frac{1}{3}\end{bmatrix}.\qquad\text{or}\qquad Au_{\infty}=u_{\infty}.\]

_The steady state is the eigenvector of \(A\) corresponding to \(\lambda=1\)_. Multiplication by \(A\), from one time step to the next, leaves \(u_{\infty}\) unchanged.

The theory of Markov processes is illustrated by that California example:

* A Markov matrix \(A\) has all \(a_{ij}\geq 0\), with each column adding to 1.
* \(\lambda_{1}=1\) is an eigenvalue of \(A\).
* Its eigenvector \(x_{1}\) is nonnegative--and it is a steady state, since \(Ax_{1}=x_{1}\).
* The other eigenvalues satisfy \(\|\lambda_{i}\|\leq 1\).
* If \(A\) or any power of \(A\) has all _positive_ entries, these other \(|\lambda_{i}|\) are below 1. The solution \(A^{k}u_{0}\) approaches a multiple of \(x_{1}\)--which is the steady state \(u_{\infty}\).

To find the right multiple of \(x_{1}\), use the fact that the total population stays the same. If California started with all 90 million people out, it ended with 60 million out and 30 million in. It ends the same way if all 90 million were originally inside.

We note that many authors transpose the matrix so its _rows_ add to 1.

_Remark_.: Our description of a Markov process was deterministic: populations moved in fixed proportions. But if we look at a single individual, the fractions that move become 