49. \(A=\begin{bmatrix}1&1\\ 0&2\\ 0&3\end{bmatrix}\); \(B\) can't exist since two equations in three unknowns can't have one solution.
51. \(A\) has rank \(4-1=3\); the complete solution to \(Ax=0\) is \(x=(2,\,3,\,1,0)\). \[R=\left[\begin{array}{rrrr}1&0&-2&0\\ 0&1&-3&0\\ 0&0&0&1\end{array}\right]\text{ with }-2,\,-3\text{ in the free column.}\]
53. (a) False. (b) True. (c) True (only \(n\) columns). (d) True (only \(m\) rows).
55. \(U=\begin{bmatrix}0&1&1&1&1&1&1\\ 0&0&0&1&1&1&1\\ 0&0&0&0&1&1&1\\ 0&0&0&0&0&0&0\end{bmatrix}\) and \(R=\begin{bmatrix}0&1&1&0&0&1&1\\ 0&0&0&1&0&1&1\\ 0&0&0&0&1&1&1\\ 0&0&0&0&0&0&0\end{bmatrix}\) (\(R\) doesn't come from this \(U\)).
57. If column \(1=\) column \(5\), then \(x_{5}\) is a free variable. Its special solution is \((-1,\,0,\,0,\,1)\).
59. Column \(5\) is sure to have no pivot since it is a combination of earlier columns, and \(x_{5}\) is free. With four pivots in the other columns, the special solution is \((1,\,0,\,1,\,0,\,1)\). The nullspace contains all multiples of \((1,\,0,\,1,\,0,\,1)\) (a line in \(\mathbf{R}^{5}\)).
61. \(A=\begin{bmatrix}1&0&0&-4\\ 0&1&0&-3\\ 0&0&1&-2\end{bmatrix}\).
63. This construction is impossible: two pivot columns, two free variables, only three columns.
65. \(A=\begin{bmatrix}0&1\\ 0&0\end{bmatrix}\).
67. \(R\) is most likely to be \(I\); \(R\) is most likely to be \(I\) with fourth row of zeros.
69. Any zero rows come after these rows: \(R=[1\,\,-2\,\,-3]\), \(R=\begin{bmatrix}1&0&0\\ 0&1&0\end{bmatrix}\); \(R=I\).

**Problem Set 2.3**, page 98

1. \(\begin{bmatrix}1&1&1\\ 0&1&1\\ 0&0&1\end{bmatrix}\)\(\begin{bmatrix}c_{1}\\ c_{2}\\ c_{3}\end{bmatrix}=0\) gives \(c_{3}=c_{2}=c_{1}=0\). But \(v_{1}+v_{2}-4v_{3}+v_{4}=0\) (dependent).
3. If \(a=0\) then column \(1=0\); if \(d=0\) then \(b\)(column \(1\)) \(-\)\(a\)(column \(2\)) \(=0\); if \(f=0\) then all columns end in zero (all are perpendicular to (0, 0, 1), all in the \(xy\) plane, must be dependent).

 