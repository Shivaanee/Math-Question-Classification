9. The inner maximum is the larger of \(y_{1}\) and \(y_{2}\); \(x\) concentrates on that one. Subject to \(y_{1}+y_{2}=1\), the minimum of the _larger_\(y\) is \(\frac{1}{2}\). Notice \(A=I\).
11. \(Ax^{\ast}=\begin{bmatrix}\frac{1}{2}&\frac{1}{2}\end{bmatrix}^{\mathsf{T}}\) and \(yxx^{\ast}=\frac{1}{2}y_{1}+\frac{1}{2}y_{2}=\frac{1}{2}\) for all strategies of \(Y\); \(y^{\ast}A=\begin{bmatrix}\frac{1}{2}&\frac{1}{2}&-1&-1\end{bmatrix}\) and \(y^{\ast}Ax=\frac{1}{2}x_{1}+\frac{1}{2}x_{2}-x_{3}-x_{4}\), which cannot exceed \(\frac{1}{2}\); in between is \(y^{\ast}Ax^{\ast}=\frac{1}{2}\).
13. Value 0 (fair game). \(X\) chooses 2 or 3, \(y\) chooses odd or even: \(x^{\ast}=y^{\ast}=\left(\frac{1}{2},\,\frac{1}{2}\right)\).

### Problem Set A, page 420

1. [label=0., ref=0]
2. Largest \(\dim\left(\mathbf{S}\ \cap\ \mathbf{T}\right)=7\) when \(\mathbf{S}\subset\mathbf{T}\). 1. Smallest \(\dim\left(\mathbf{S}\ \cap\ \mathbf{T}\right)=2\). 3. Smallest \(\dim\left(\mathbf{S}+\mathbf{T}\right)=8\) when \(\mathbf{S}\subset\mathbf{T}\). 4. Largest \(\dim\left(\mathbf{S}+\mathbf{T}\right)=13\) (all of \(\mathbf{R}^{13}\)).
3. \(\mathbf{V}+\mathbf{W}\) and \(\mathbf{V}\cap\mathbf{W}\) contain \(\begin{bmatrix}a_{11}&a_{12}&a_{13}&a_{14}\\ a_{21}&a_{22}&a_{23}&a_{24}\\ 0&a_{32}&a_{33}&a_{34}\\ 0&0&a_{43}&a_{44}\end{bmatrix}\) and \(\begin{bmatrix}a_{11}&a_{12}&0&0\\ 0&a_{22}&a_{23}&0\\ 0&0&a_{23}&a_{34}\\ 0&0&0&a_{44}\end{bmatrix}\). \(\dim\left(\mathbf{V}+\mathbf{W}\right)=13\) and \(\dim\left(\mathbf{V}\cap\mathbf{W}\right)=7\); add to get \(20=\dim\mathbf{V}+\dim\mathbf{W}\).
4. The lines through \((1,1,1)\) and \((1,1,2)\) have \(\mathbf{V}\cap\mathbf{W}=\{0\}\).
5. One basis for \(\mathbf{V}+\mathbf{W}\) is \(v_{1}\), \(v_{2}\), \(w_{1}\); \(\dim\left(\mathbf{V}\cap\mathbf{W}\right)=1\) with basis \((0,\,1,\,-1,\,0)\).
6. The intersection of column spaces is the line through \(y=(6,\,3,\,6)\): \[y=\begin{bmatrix}1&5\\ 3&0\\ 2&4\end{bmatrix}\begin{bmatrix}1\\ 1\end{bmatrix}=\begin{bmatrix}3&0\\ 0&1\\ 0&2\end{bmatrix}\begin{bmatrix}2\\ 3\end{bmatrix}\text{ matches }[A&B]x=\begin{bmatrix}1&5&3&0\\ 3&0&0&1\\ 2&4&0&2\end{bmatrix}\begin{bmatrix}1\\ 1\\ -2\\ -3\end{bmatrix}=0.\] The column spaces have dimension 2. Their sum and intersection give \(3+1=2+2\).

11. \(F_{2}\otimes F_{2}=\begin{bmatrix}F_{2}&F_{2}\\ F_{2}&-F_{2}\end{bmatrix}=\begin{bmatrix}1&1&1&1\\ 1&-1&1&-1\\ 1&1&-1&-1\\ 1&-1&-1&1\end{bmatrix}\).
13. \(A_{\text{3D}}=(A_{\text{1D}}\otimes I\otimes I)+(I\otimes A_{\text{1D}}\otimes I )+(I\otimes I\otimes A_{\text{1D}})\).

### Problem Set B, page 427

1. [label=0., ref=0]
2. \(J=\begin{bmatrix}2&0\\ 0&0\end{bmatrix}\) (\(A\) is diagonalizable); \(J=\begin{bmatrix}0&1&0\\ 0&0&0\\ 0&0&0\end{bmatrix}\) (eigenvectors \((1,0,0)\) and \((2,-1,\,0)\)).
3. \(e^{Bt}=\begin{bmatrix}1&t&2t\\ 0&1&0\\ 0&0&0\end{bmatrix}=I+Bt\) since \(B^{2}=0\). Also \(e^{Jt}=I+Jt\).
4. \(J=\begin{bmatrix}1&0&0\\ 0&4&0\\ 0&0&6\end{bmatrix}\) (distinct eigenvalues); \(J=\begin{bmatrix}0&1\\ 0&0\end{bmatrix}\) (\(B\) has \(\lambda=0\), \(0\) but rank 1).

 