

**25.**: \(L\) is lower triangular and \(S\) is symmetric. Assume they are invertible:

\[L=\begin{bmatrix}a&0&0\\ b&c&0\\ d&e&f\end{bmatrix}\qquad S=\begin{bmatrix}a&b&d\\ b&c&e\\ d&e&f\end{bmatrix}.\]

1. Which three cofactors of \(L\) are zero? Then \(L^{-1}\) is lower triangular.
2. Which three pairs of cofactors of \(S\) are equal? Then \(S^{-1}\) is symmetric.
**26.**: For \(n=5\) the matrix \(C\) contains cofactors and each 4 by 4 cofactor contains terms and each term needs multiplications. Compare with \(5^{3}=125\) for the Gauss-Jordan computation of \(A^{-1}\).

**Problems 27-36 are about area and volume by determinants.**

**27.**:
1. Find the area of the parallelogram with edges \(v=(3,2)\) and \(w=(1,4)\).
2. Find the area of the triangle with sides \(v\), \(w\), and \(v+w\). Draw it.
3. Find the area of the triangle with sides \(v\), \(w\), and \(w-v\). Draw it.
**28.**: A box has edges from \((0,0,0)\) to \((3,1,1)\), \((1,3,1)\), and \((1,1,3)\). Find its volume and also find the area of each parallelogram face.
**29.**:
1. The corners of a triangle are \((2,1)\), \((3,4)\), and \((0,5)\). What is the area?
2. A new corner at \((-1,0)\) makes it lopsided (four sides). Find the area.
**30.**: The parallelogram with sides \((2,1)\) and \((2,3)\) has the same area as the parallelogram with sides \((2,2)\) and \((1,3)\). Find those areas from 2 by 2 determinants and say why they must be equal. (I can't see why from a picture. Please write to me if you do.)
**31.**: The Hadamard matrix \(H\) has orthogonal rows. The box is a hypercube!

\[\text{What is }\det H=\begin{vmatrix}1&1&1&1\\ 1&1&-1&-1\\ 1&-1&-1&1\\ 1&-1&1&-1\end{vmatrix}=\text{volume of a hypercube in }\mathbf{R}^{4}\text{?}\]
**32.**: If the columns of a 4 by 4 matrix have lengths \(L_{1}\), \(L_{2}\), \(L_{3}\), \(L_{4}\), what is the largest possible value for the determinant (based on volume)? If all entries are 1 or \(-1\), what are those lengths and the maximum determinant?
**33.**: Show by a picture how a rectangle with area \(x_{1}y_{2}\) minus a rectangle with area \(x_{2}y_{1}\) produces the area \(x_{1}y_{2}-x_{2}y_{1}\) of a parallelogram.
**34.**: When the edge vectors \(\boldsymbol{a}\), \(\boldsymbol{b}\), \(\boldsymbol{c}\) are perpendicular, the volume of the box is \(\|\boldsymbol{a}\|\) times \(\|\boldsymbol{b}\|\) times \(\|\boldsymbol{c}\|\). The matrix \(A^{\text{T}}A\) is \(\underline{\text{\text{\text{\text{\text{\text{\text{\text{\text{\text