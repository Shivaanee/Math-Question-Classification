up; approximation by polynomials has to be possible. The right idea is to switch to orthogonal axes (by Gram-Schmidt). We look for combinations of \(1\), \(x\), and \(x^{2}\) that _are_ orthogonal.

It is convenient to work with a symmetrically placed interval like \(-1\leq x\leq 1\), because this makes all the odd powers of \(x\) orthogonal to all the even powers:

\[(1,x)=\int_{-1}^{1}xdx=0,\qquad(x,x^{2})=\int_{-1}^{1}x^{3}dx=0.\]

Therefore the Gram-Schmidt process can begin by accepting \(v_{1}=1\) and \(v_{2}=x\) as the first two perpendicular axes. Since \((x,x^{2})=0\), it only has to correct the angle between \(1\) and \(x^{2}\). By formula (10), the third orthogonal polynomial is

\[\mbox{\bf Orthogonalize}\qquad v_{3}=x^{2}-\frac{(1,x^{2})}{(1,1)}1-\frac{(x,x ^{2})}{(x,x)}x=x^{2}-\frac{\int_{-1}^{1}x^{2}dx}{\int_{-1}^{1}1dx}=x^{2}-\frac {1}{3}.\]

The polynomials constructed in this way are called the _Legendre polynomials_ and they are orthogonal to each other over the interval \(-1\leq x\leq 1\).

\[\mbox{\bf Check}\qquad\left(1,x^{2}-\frac{1}{3}\right)=\int_{-1}^{1}\left(x^{2 }-\frac{1}{3}\right)dx=\left[\frac{x^{3}}{3}-\frac{x}{3}\right]_{-1}^{1}=0.\]

The closest polynomial of degree ten is now computable, without disaster, by projecting onto each of the first \(10\) (or \(11\)) Legendre polynomials.

**5. Best Straight Line.** Suppose we want to approximate \(y=x^{5}\) by a straight line \(C+Dx\) between \(x=0\) and \(x=1\). There are at least three ways of finding that line, and if you compare them the whole chapter might become clear!

1. Solve \([1\;\;x]\left[\begin{smallmatrix}C\\ D\end{smallmatrix}\right]=x^{5}\) by least squares. The equation \(A^{\mbox{\scriptsize T}}A\widehat{x}=A^{\mbox{\scriptsize T}}b\) is \[\begin{bmatrix}(1,1)&(1,x)\\ (x,1)&(x,x)\end{bmatrix}\begin{bmatrix}C\\ D\end{bmatrix}=\begin{bmatrix}(1,x^{5})\\ (x,x^{5})\end{bmatrix}\qquad\mbox{or}\qquad\begin{bmatrix}1&\frac{1}{2}\\ \frac{1}{2}&\frac{1}{3}\end{bmatrix}\begin{bmatrix}C\\ D\end{bmatrix}=\begin{bmatrix}\frac{1}{6}\\ \frac{1}{17}\end{bmatrix}.\]
2. Minimize \(E^{2}=\int_{0}^{1}(x^{5}-C-Dx)^{2}dx=\frac{1}{11}-\frac{2}{6}C-\frac{2}{7}D+ C^{2}+CD+\frac{1}{3}D^{2}\). The derivatives with respect to \(C\) and \(D\), after dividing by \(2\), bring back the normal equations of method \(1\) (and the solution is \(\widehat{C}=\frac{1}{6}-\frac{5}{14}\), \(\widehat{D}=\frac{5}{17}\)): \[-\frac{1}{6}+C+\frac{1}{2}D=0\qquad\mbox{and}\qquad-\frac{1}{7}+\frac{1}{2}C+ \frac{1}{3}D=0.\]
3. Apply Gram-Schmidt to replace \(x\) by \(x-(1,x)/(1,1)\). That is \(x-\frac{1}{2}\), which is orthogonal to \(1\). Now the one-dimensional projections add to the best line: \[C+Dx=\frac{(x^{5},1)}{(1,1)}1+\frac{(x^{5},x-\frac{1}{2})}{(x-\frac{1}{2},x- \frac{1}{2})}(x-\frac{1}{2})=\frac{1}{6}+\frac{5}{7}\left(x-\frac{1}{2}\right).\] 