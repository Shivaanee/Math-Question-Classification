

**10.**: Construct a homogeneous equation in three unknowns whose solutions are the linear combinations of the vectors \((1,1,2)\) and \((1,2,3)\). This is the reverse of the previous exercise, but the two problems are really the same.
**11.**: The fundamental theorem is often stated in the form of _Fredholm's alternative_: For any \(A\) and \(b\), one and only one of the following systems has a solution:

1. \(Ax=b\).
2. \(A^{\mathrm{T}}y=0\), \(y^{\mathrm{T}}b\neq 0\).

Either \(b\) is in the column space \(C(A)\) or there is a \(y\) in \(N(A^{\mathrm{T}})\) such that \(y^{\mathrm{T}}b\neq 0\). Show that it is contradictory for (i) and (ii) both to have solutions.
**12.**: Find a basis for the orthogonal complement of the row space of \(A\):

\[A=\begin{bmatrix}1&0&2\\ 1&1&4\end{bmatrix}.\]

Split \(x=(3,3,3)\) into a row space component \(x_{r}\) and a nullspace component \(x_{n}\).
**13.**: Illustrate the action of \(A^{\mathrm{T}}\) by a picture corresponding to Figure 3.4, sending \(C(A)\) back to the row space and the left nullspace to zero.
**14.**: Show that \(x-y\) is orthogonal to \(x+y\) if and only if \(\|x\|=\|y\|\).
**15.**: Find a matrix whose row space contains \((1,2,1)\) and whose nullspace contains \((1,-2,1)\), or prove that there is no such matrix.
**16.**: Find all vectors that are perpendicular to \((1,4,4,1)\) and \((2,9,8,2)\).
**17.**: If \(\mathbf{V}\) is the orthogonal complement of \(\mathbf{W}\) in \(\mathbf{R}^{n}\), is there a matrix with row space \(\mathbf{V}\) and nullspace \(\mathbf{W}\)? Starting with a basis for \(\mathbf{V}\), construct such a matrix.
**18.**: If \(\mathbf{S}=\{0\}\) is the subspace of \(\mathbf{R}^{4}\) containing only the zero vector, what is \(\mathbf{S}^{\perp}\)? If \(\mathbf{S}\) is spanned by \((0,0,0,1)\), what is \(\mathbf{S}^{\perp}\)? What is \((\mathbf{S}^{\perp})^{\perp}\)?
**19.**: _Why are these statements false_?

1. If \(\mathbf{V}\) is orthogonal to \(\mathbf{W}\), then \(\mathbf{V}^{\perp}\) is orthogonal to \(\mathbf{W}^{\perp}\).
2. \(\mathbf{V}\) orthogonal to \(\mathbf{W}\) and \(\mathbf{W}\) orthogonal to \(\mathbf{Z}\) makes \(\mathbf{V}\) orthogonal to \(\mathbf{Z}\).
**20.**: Let \(\mathbf{S}\) be a subspace of \(\mathbf{R}^{n}\). Explain what \((\mathbf{S}^{\perp})^{\perp}=\mathbf{S}\) means and why it is true.
**21.**: Let \(\mathbf{P}\) be the plane in \(\mathbf{R}^{2}\) with equation \(x+2y-z=0\). Find a vector perpendicular to \(\mathbf{P}\). What matrix has the plane \(\mathbf{P}\) as its nullspace, and what matrix has \(\mathbf{P}\) as its row space?
**22.**: Let \(\mathbf{S}\) be the subspace of \(\mathbf{R}^{4}\) containing all vectors with \(x_{1}+x_{2}+x_{3}+x_{4}=0\). Find a basis for the space \(\mathbf{S}^{\perp}\), containing all vectors orthogonal to \(\mathbf{S}\).

