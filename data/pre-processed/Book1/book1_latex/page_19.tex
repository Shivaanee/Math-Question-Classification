Another singular system, close to this one, has an **infinity of solutions**. When the 6 in the last equation becomes 7, the three equations combine to give \(0=0\). Now the third equation is the sum of the first two. In that case the three planes have a whole _line in common_ (Figure 1.5c). Changing the right sides will move the planes in Figure 1.5b parallel to themselves, and for \(b=(2,5,7)\) the figure is suddenly different. The lowest plane moved up to meet the others, and there is a line of solutions. Problem 1.5c is still singular, but now it suffers from _too many solutions_ instead of too few.

The extreme case is three parallel planes. For most right sides there is no solution (Figure 1.5d). For special right sides (like \(b=(0,0,0)\)!) there is a whole plane of solutions--because the three parallel planes move over to become the same.

What happens to the _column picture_ when the system is singular? it has to go wrong; the question is how, There are still three columns on the left side of the equations, and we try to combine them to produce \(b\). Stay with equation (3):

\[\begin{array}{ll}\mbox{\bf Singular case: Column picture}\\ \mbox{\bf Three columns in the same plane}\\ \mbox{\bf Solvable only for $b$ in that plane}\end{array}\qquad u\begin{bmatrix}1\\ 2\\ 3\end{bmatrix}+v\begin{bmatrix}1\\ 0\\ 1\end{bmatrix}+w\begin{bmatrix}1\\ 3\\ 4\end{bmatrix}=b.\] (4)

For \(b=(2,5,7)\) this was possible; for \(b=(2,5,6)\) it was not. The reason is that _those three columns lie in a plane_. Then every combination is also in the plane (which goes through the origin). If the vector \(b\) is not in that plane, no solution is possible (Figure 1.6). That is by far the most likely event; a singular system generally has no solution. But there is a chance that \(b\)_does_ lie in the plane of the columns. In that case there are too many solutions; the three columns can be combined in _infinitely many ways_ to produce \(b\). That column picture in Figure 1.6b corresponds to the row picture in Figure 1.5c.

How do we know that the three columns lie in the same plane? One answer is to find a combination of the columns that adds to zero. After some calculation, it is \(u=3\), \(v=1\), \(w=-2\). Three times column 1 equals column 2 plus twice column 3. Column 1 is in

Figure 1.6: Singular cases: \(b\) outside or inside the plane with all three columns.

 