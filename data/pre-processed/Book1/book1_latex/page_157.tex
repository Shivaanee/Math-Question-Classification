finishes with \(ABu\) in \(\mathbf{W}\). This "composition" \(AB\) is again a linear transformation (from \(\mathbf{U}\) to \(\mathbf{W}\)). Its matrix is the product of the individual matrices representing \(A\) and \(B\). For \(A_{\mathrm{diff}}A_{\mathrm{int}}\), the composite transformation was the identity (and \(A_{\mathrm{int}}A_{\mathrm{diff}}\) annihilated all constants). For rotations, the order of multiplication does not matter. Then \(\mathbf{U}=\mathbf{V}=\mathbf{W}\) is the \(x\)-\(y\) plane, and \(Q_{\theta}Q_{\varphi}\) is the same as \(Q_{\varphi}Q_{\theta}\). For a rotation and a reflection, the order makes a difference. _Technical note:_ To construct the matrices, we need bases for \(\mathbf{V}\) and \(\mathbf{W}\), and then for \(\mathbf{U}\) and \(\mathbf{V}\). By keeping the same basis for \(\mathbf{V}\), the product matrix goes correctly from the basis in \(\mathbf{U}\) to the basis in \(\mathbf{W}\). If we distinguish the transformation \(A\) from its matrix (call that \([A]\)), then the product rule \(2V\) becomes extremely concise: \([AB]=[A][B]\). The rule for multiplying matrices in Chapter 1 was totally determined by this requirement--it must match the product of linear transformations.
2. Projection Figure 2.10 also shows the projection of \((1,0)\) onto the \(\theta\)-line. The length of the projection is \(c=\cos\theta\). Notice that the _point_ of projection is not \((c,s)\), as I mistakenly thought; that vector has length 1 (it is the rotation), so we must multiply by \(c\). Similarly the projection of \((0,1)\) has length \(s\), and falls at \(s(c,s)=(cs,s^{2})\), That gives the second column of the projection matrix \(P\): \[\text{{Projection onto $\theta$-line}}\qquad P=\begin{bmatrix}c^{2}&cs\\ cs&s^{2}\end{bmatrix}.\] This matrix has no inverse, because the transformation has no inverse. Points on the perpendicular line are projected onto the origin; that line is the nullspace of \(P\). Points on the \(\theta\)-line are projected to themselves! Projecting twice is the same as projecting once, and \(P^{2}=P\): \[P^{2}=\begin{bmatrix}c^{2}&cs\\ cs&s^{2}\end{bmatrix}^{2}=\begin{bmatrix}c^{2}(c^{2}+s^{2})&cs(c^{2}+s^{2})\\ cs(c^{2}+s^{2})&s^{2}(c^{2}+s^{2})\end{bmatrix}=P.\]

Figure 2.10: Rotation through \(\theta\) (left). Projection onto the \(\theta\)-line (right).

 