

**17.**: Choose the third row of the "companion matrix"

\[A=\begin{bmatrix}0&1&0\\ 0&0&1\\ .&.&.\end{bmatrix}\]

so that its characteristic polynomial \(|A-\lambda I|\) is \(-\lambda^{3}+4\lambda^{2}+5\lambda+6\).
**18.**: Suppose \(A\) has eigenvalues 0, 3, 5 with independent eigenvectors \(u\), \(v\), \(w\).

1. [label=()]
2. Give a basis for the nullspace and a basis for the column space.
3. Find a particular solution to \(Ax=v+w\). Find all solutions.
4. Show that \(Ax=u\) has no solution. (If it had a solution, then \(\underline{\phantom{x}}\) would be in the column space.)
**19.**: The powers \(A^{k}\) of this matrix \(A\) approaches a limit as \(k\to\infty\):

\[A=\begin{bmatrix}.8&.3\\ .2&.7\end{bmatrix},\qquad A^{2}=\begin{bmatrix}.70&.45\\ .30&.55\end{bmatrix},\quad\text{and}\quad A^{\infty}=\begin{bmatrix}.6&.6\\ .4&.4\end{bmatrix}.\]

The matrix \(A^{2}\) is halfway between \(A\) and \(A^{\infty}\). Explain why \(A^{2}=\frac{1}{2}(A+A^{\infty})\) from the eigenvalues and eigenvectors of these three matrices.
**20.**: Find the eigenvalues and the eigenvectors of these two matrices:

\[A=\begin{bmatrix}1&4\\ 2&3\end{bmatrix}\qquad\text{and}\qquad A+I=\begin{bmatrix}2&4\\ 2&4\end{bmatrix}.\]

\(A+I\) has the \(\underline{\phantom{x}}\) eigenvectors as \(A\). Its eigenvalues are \(\underline{\phantom{x}}\) by 1.
**21.**: Compute the eigenvalues and eigenvectors of \(A\) and \(A^{-1}\):

\[A=\begin{bmatrix}0&2\\ 2&3\end{bmatrix}\qquad\text{and}\qquad A^{-1}=\begin{bmatrix}-3/4&1/2\\ 1/2&0\end{bmatrix}.\]

\(A^{-1}\) has the \(\underline{\phantom{x}}\) eigenvectors as \(A\). When \(A\) has eigenvalues \(\lambda_{1}\) and \(\lambda_{2}\), its inverse has eigenvalues \(\underline{\phantom{x}}\).
**22.**: Compute the eigenvalues and eigenvectors of \(A\) and \(A^{2}\):

\[A=\begin{bmatrix}-1&3\\ 2&0\end{bmatrix}\qquad\text{and}\qquad A^{2}=\begin{bmatrix}7&-3\\ -2&6\end{bmatrix}.\]

\(A^{2}\) has the same \(\underline{\phantom{x}}\) as \(A\). When \(A\) has eigenvalues \(\lambda_{1}\) and \(\lambda_{2}\), \(A^{2}\) has eigenvalues \(\underline{\phantom{x}}\).
**23.**:
1. [label=()]
2. If you know \(x\) is an eigenvector, the way to find \(\lambda\) is to \(\underline{\phantom{x}}\).

