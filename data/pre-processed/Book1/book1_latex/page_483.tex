Solutions to Selected Exotices

**Problem Set 1.2, page 9**

1. The lines intersect at \((x,\,y)=(3,1)\). Then \(3\)(column \(1\)) + \(1\)(column \(2\)) = \((4,\,4)\).
2. These "planes" intersect in a line in four-dimensional space. The fourth plane normally intersects that line in a point. An inconsistent equation like \(u+w=5\) leaves no solution (no intersection). .
3. The two points on the plane are \((1,\,0,\,0,\,0)\) and \((0,\,1,\,0,\,0)\).
4. Solvable for \((3,\,5,\,8)\) and \((1,\,2,\,3)\); not solvable for \(b=(3,\,5,\,7)\) or \(b=(1,\,2,\,2)\).
5. Column \(3=2\)(column \(2\)) - column \(1\). If \(b=(0,\,0,\,0)\), then \((u,\,v,\,w)=(c,\,-2c,\,c)\)
6. Both \(a=2\) and \(a=-2\) give a line of solutions. All other \(a\) give \(x=0,\,y=0\).
7. The row picture has two lines meeting at \((4,\,2)\). The column picture has \(4(1,\,1)\) + \(2(-2,\,1)\) = \(4\)(column \(1\)) + \(2\)(column \(2\)) = right-hand side \((0,\,6)\).
8. The row picture shows four _lines_. The column picture is in _four_-dimensional space. No solution unless the right-hand side is a combination of _the two columns_.
9. If \(x\), \(y\), \(z\) satisfy the first two equations, they also satisfy the third equation. The line \(\mathbf{L}\) of solutions contains \(v=(1,\,1,\,0)\), \(w=\left(\frac{1}{2},\,1,\,\frac{1}{2}\right)\), and \(u=\frac{1}{2}v+\frac{1}{2}w\), and all combinations \(cv+dw\) with \(c+d=1\).
10. Column \(3=\) column \(1\); solutions \((x,\,y,\,z)=(1,\,1,\,0)\) or \((0,\,1,\,1)\) and you can add any multiple of \((-1,\,0,\,1)\); \(b=(4,\,6,\,c)\) needs \(c=10\) for solvability.
11. The second plane and row \(2\) of the matrix and all columns of the matrix are changed. The solution is not changed.
12. \(u=0\), \(v=0\), \(w=1\), because \(1\)(column \(3\)) = \(b\).

**Problem Set 1.3, page 15**

1. Multiply by \(\ell=\frac{10}{2}=5\), and subtract to find \(2x+3y=1\) and \(-6y=6\). Pivots \(2,\,-6\).
2. Subtract \(-\frac{1}{2}\) times equation \(1\) (or add \(\frac{1}{2}\) times equation \(1\)). The new second equation is \(3y=3\). Then \(y=1\) and \(x=5\). If the right-hand side changes sign, so does the solution: \((x,\,y)=(-5,\,-1)\).
3. \(6x+4y\) is \(2\) times \(3x+2y\). There is no solution unless the right-hand side is \(2\cdot 10=20\). Then all points on the line \(3x+2y=10\) are solutions, including \((0,\,5)\) and \((4,\,-1)\).
4. If \(a=2\), elimination must fail. The equations have no solution. If \(a=0\), elimination stops for a row exchange. Then \(3y=-3\) gives \(y=-1\) and \(4x+6y=6\) gives \(x=3\).
5. \(6x-4y\) is \(2\) times \((3x-2y)\). Therefore, we need \(b_{2}=2b_{1}\). Then there will be infinitely many solutions. The columns \((3,\,6)\) and \((-2,\,-4)\) are on the same line.

 