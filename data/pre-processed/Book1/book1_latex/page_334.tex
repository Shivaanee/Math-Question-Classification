

**27.**: Prove that \(A^{\rm H}A\) is always a Hermitian matrix, Compute \(A^{\rm H}A\) and \(AA^{\rm H}\):

\[A=\begin{bmatrix}i&1&i\\ 1&i&i\end{bmatrix}.\]
**28.**: If \(Az=0\), then \(A^{\rm H}Az=0\). If \(A^{\rm H}Az=0\), multiply by \(z^{\rm H}\) to prove that \(Az=0\). The nullspaces of \(A\) and \(A^{\rm H}A\) are \(\underline{\phantom{\rm H}}\). \(A^{\rm H}A\) is an invertible Hermitian matrix when the nullspace of \(A\) contains only \(z=\underline{\phantom{\rm H}}\).
**29.**: When you multiply a Hermitian matrix by a real number \(c\), is \(cA\) still Hermitian? If \(c=i\), show that \(iA\) is skew-Hermitian. The 3 by 3 Hermitian matrices are a subspace, provided that the "scalars" are real numbers.
**30.**: Which classes of matrices does \(P\) belong to: orthogonal, invertible, Hermitian, unitary, factorizable into \(LU\), factorizable into \(QR\)?

\[P=\begin{bmatrix}0&1&0\\ 0&0&1\\ 1&0&0\end{bmatrix}.\]
**31.**: Compute \(P^{2}\), \(P^{3}\), and \(P^{100}\) in Problem 30. What are the eigenvalues of \(P\)?
**32.**: Find the unit eigenvectors of \(P\) in Problem 30, and put them into the columns of a unitary matrix \(U\). What property of \(P\) makes these eigenvectors orthogonal?
**33.**: Write down the 3 by 3 _circulant matrix_\(C=2I+5P+4P^{2}\). It has the same eigenvectors as \(P\) in Problem 30. Find its eigenvalues.
**34.**: If \(U\) is unitary and \(Q\) is a real orthogonal matrix, show that \(U^{-1}\) is unitary and also \(UQ\) is unitary. Start from \(U^{\rm H}U=I\) and \(Q^{\rm T}Q=I\).
**35.**: Diagonalize \(A\) (real \(\lambda\)'s) and \(K\) (imaginary \(\lambda\)'s) to reach \(U\Lambda U^{\rm H}\):

\[A=\begin{bmatrix}0&1-i\\ i+1&1\end{bmatrix}\qquad K=\begin{bmatrix}0&-1+i\\ 1+i&i\end{bmatrix}\]
**36.**: Diagonalize this orthogonal matrix to reach \(Q=U\Lambda U^{\rm H}\). Now all \(\lambda\)'s are \(\underline{\phantom{\rm H}}\):

\[Q=\begin{bmatrix}\cos\theta&-\sin\theta\\ \sin\theta&\cos\theta\end{bmatrix}.\]
**37.**: Diagonalize this unitary matrix \(V\) to reach \(V=U\Lambda U^{\rm H}\). Again all \(|\lambda|=1\):

\[V=\frac{1}{\sqrt{3}}\begin{bmatrix}1&1-i\\ 1+i&-1\end{bmatrix}.\]