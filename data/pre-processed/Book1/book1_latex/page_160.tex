

**7.**: On the space \({\bf P}_{3}\) of cubic polynomials, what matrix represents \(d^{2}/dt^{2}\)? Construct the 4 by 4 matrix from the standard basis \(1\), \(t\), \(t^{2}\), \(t^{3}\). Find its nullspace and column space. What do they mean in terms of polynomials?
**8.**: From the cubics \({\bf P}_{3}\) to the fourth-degree polynomials \({\bf P}_{4}\), what matrix represents multiplication by \(2+3t\)? The columns of the 5 by 4 matrix \(A\) come from applying the transformation to \(1\), \(t\), \(t^{2}\), \(t^{3}\).
**9.**: The solutions to the linear differential equation \(d^{2}u/dt^{2}=u\) form a vector space (since combinations of solutions are still solutions). Find two independent solutions, to give a basis for that solution space.
**10.**: With initial values \(u=x\) and \(du/dt=y\) at \(t=0\), what combination of basis vectors in Problem 9 solves \(u^{\prime\prime}=u\)? This transformation from initial values to solution is linear. What is its 2 by 2 matrix (using \(x=1\), \(y=0\) and \(x=0\), \(y=1\) as basis for \({\bf V}\), and your basis for \({\bf W}\))?
**11.**: Verify directly from \(c^{2}+s^{2}=1\) that reflection matrices satisfy \(H^{2}=1\).
**12.**: Suppose \(A\) is a linear transformation from the \(x\)-\(y\) plane to itself. Why does \(A^{-1}(x+y)=A^{-1}x+A^{-1}y\)? If \(A\) is represented by the matrix \(M\), explain why \(A^{-1}\) is represented by \(M^{-1}\).
**13.**: The product \((AB)C\) of linear transformations starts with a vector \(x\) and produces \(u=Cx\). Then rule 2V applies \(AB\) to \(u\) and reaches \((AB)Cx\).

1. Is this result the same as separately applying \(C\) then \(B\) then \(A\)?
2. Is the result the same as applying \(BC\) followed by \(A\)? Parentheses are unnecessary and the associative law \((AB)C=A(BC)\) holds for linear transformations. This is the best proof of the same law for matrices.
**14.**: Prove that \(T^{2}\) is a linear transformation if \(T\) is linear (from \({\bf R}^{3}\) to \({\bf R}^{3}\)).
**15.**: The space of all 2 by 2 matrices has the four basis "vectors"

\[\left[\begin{matrix}1&0\\ 0&0\end{matrix}\right],\qquad\left[\begin{matrix}0&1\\ 0&0\end{matrix}\right],\qquad\left[\begin{matrix}0&0\\ 1&0\end{matrix}\right],\qquad\left[\begin{matrix}0&0\\ 0&1\end{matrix}\right].\]

For the linear transformation of _transposing_, find its matrix \(A\) with respect to this basis. Why is \(A^{2}=I\)?
**16.**: Find the 4 by 4 cyclic permutation matrix: \((x_{1},x_{2},x_{3},x_{4})\) is transformed to \(Ax=(x_{2},x_{3},x_{4},x_{1})\). What is the effect of \(A^{2}\)? Show that \(A^{3}=A^{-1}\).
**17.**: Find the 4 by 3 matrix \(A\) that represents a _right shift_: \((x_{1},x_{2},x_{3})\) is transformed to \((0,x_{1},x_{2},x_{3})\). Find also the _left shift_ matrix \(B\) from \({\bf R}^{4}\) back to \({\bf R}^{3}\), transforming \((x_{1},x_{2},x_{3},x_{4})\) to \((x_{2},x_{3},x_{4})\). What are the products \(AB\) and \(BA\)?