\[(1,3,2)\quad\text{is odd so}\quad\left|\begin{array}{c}1\\ 1\end{array}\right|=-1\qquad(3,1,2)\quad\text{is even so}\quad\left|\begin{array} []{c}1\\ 1\end{array}\right|=1\]

\((1,3,2)\) requires one exchange and \((3,1,2)\) requires two exchanges to recover \((1,2,3)\). These are two of the six \(\pm\) signs. For \(n=2\), we only have \((1,2)\) and \((2,1)\):

\[\det A=a_{11}a_{22}\det\begin{bmatrix}1&0\\ 0&1\end{bmatrix}+a_{12}a_{21}\det\begin{bmatrix}0&1\\ 1&0\end{bmatrix}=a_{11}a_{22}-a_{12}a_{21}\quad(\text{or }ad-bc).\]

No one can claim that the big formula (6) is particularly simple. Nevertheless, it is possible to see why it has properties 1-3. For \(A=I\), every product of the \(a_{ij}\) will be zero, except for the column sequence \((1,2,\ldots,n)\). This term gives \(\det I=1\). Property 2 will be checked in the next section, because here we are most interested in property 3: The determinant should depend linearly on the first row \(a_{11},a_{12},\ldots,a_{1n}\).

Look at all the terms \(a_{1\alpha}a_{2\beta}\cdots a_{nv}\) involving \(a_{11}\). The first column is \(\alpha=1\). This leaves some permutation \((\beta,\ldots,v)\) of the remaining columns \((2,\ldots,n)\). We collect all these terms together as \(a_{11}C_{11}\), where the coefficient of \(a_{11}\) is a smaller determinant--_with row \(1\) and column \(1\) removed_:

\[\text{{Cofactor of $a_{11}$}}\qquad C_{11}=\sum(a_{2\beta}\cdots a_{nv})\det P= \det(\text{submatrix of $A$}).\] (7)

Similarly, the entry \(a_{12}\) is multiplied by some smaller determinant \(C_{12}\). Grouping all the terms that start with the same \(a_{1j}\), formula (6) becomes

\[\boxed{\text{{Cofactors along row 1}}}\qquad\det A=a_{11}C_{11}+a_{12}C_{12}+ \cdots+a_{1n}C_{1n}.\] (8)

This shows that \(\det A\) depends linearly on the entries \(a_{11},\ldots,a_{1n}\) of the first row.

**Example 2**.: For a 3 by 3 matrix, this way of collecting terms gives

\[\det A=a_{11}(a_{22}a_{33}-a_{23}a_{32})+a_{12}(a_{23}a_{31}-a_{21}a_{33})+a_ {13}(a_{21}a_{32}-a_{22}a_{31}).\] (9)

The _cofactors_\(C_{11}\), \(C_{12}\), \(C_{13}\) are the 2 by 2 determinants in parentheses.

### Expansion of \(\det A\) in Cofactors

We want one more formula for the determinant. If this meant starting again from scratch, it would be too much, But _the formula is already discovered--it is_ (8), _and the only point is to identify the cofactors \(C_{1j}\) that multiply \(a_{1j}\)._

We know that \(C_{1j}\) depends on rows \(2,\ldots,n\). Row 1 is already accounted for by \(a_{1j}\). Furthermore, \(a_{1j}\) also accounts for the \(j\)th column, so its cofactor \(C_{1j}\) must depend 