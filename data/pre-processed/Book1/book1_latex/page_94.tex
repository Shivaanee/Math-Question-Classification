* the nullspace of \(A\).
* the column space of \(A\).
* Show that the set of nonsingular 2 by 2 matrices is not a vector space. Show also that the set of _singular_ 2 by 2 matrices is not a vector space.
* The matrix \(A=\left[\begin{smallmatrix}2&-2\\ 2&-2\end{smallmatrix}\right]\) is a "vector" in the space \(\mathbf{M}\) of all 2 by 2 matrices. Write the zero vector in this space, the vector \(\frac{1}{2}A\), and the vector \(-A\). What matrices are in the smallest subspace containing \(A\)?
* Describe a subspace of \(\mathbf{M}\) that contains \(A=\left[\begin{smallmatrix}1&0\\ 0&0\end{smallmatrix}\right]\) but not \(B=\left[\begin{smallmatrix}0&0\\ 0&-1\end{smallmatrix}\right]\).
* If a subspace of \(\mathbf{M}\) contains \(A\) and \(B\), must it contain \(I\)?
* Describe a subspace of \(\mathbf{M}\) that contains no nonzero diagonal matrices.
* The functions \(f(x)=x^{2}\) and \(g(x)=5x\) are "vectors" in the vector space \(\mathbf{F}\) of all real functions. The combination \(3f(x)-4g(x)\) is the function \(h(x)=\raisebox{-1.5pt}{\includegraphics[]{fig/f(x).eps}}\). Which rule is broken if multiplying \(f(x)\) by \(c\) gives the function \(f(cx)\)?
* If the sum of the "vectors" \(f(x)\) and \(g(x)\) in \(\mathbf{F}\) is defined to be \(f(g(x))\), then the "zero vector" is \(g(x)=x\). Keep the usual scalar multiplication \(cf(x)\), and find two rules that are broken.
* Describe the smallest subspace of the 2 by 2 matrix space \(\mathbf{M}\) that contains
* \(\left[\begin{smallmatrix}1&0\\ 0&0\end{smallmatrix}\right]\) and \(\left[\begin{smallmatrix}0&1\\ 0&0\end{smallmatrix}\right]\).
* \(\left[\begin{smallmatrix}1&0\\ 0&0\end{smallmatrix}\right]\).
* \(\left[\begin{smallmatrix}1&1\\ 0&0\end{smallmatrix}\right]\).
* Let \(\mathbf{P}\) be the plane in \(\mathbf{R}^{3}\) with equation \(x+y-2z=4\). The origin \((0,0,0)\) is not in \(\mathbf{P}\)! Find two vectors in \(\mathbf{P}\) and check that their sum is not in \(\mathbf{P}\).
* \(\mathbf{P}_{0}\) is the plane through \((0,0,0)\) parallel to the plane \(\mathbf{P}\) in Problem 15. What is the equation for \(\mathbf{P}_{0}\)? Find two vectors in \(\mathbf{P}_{0}\) and check that their sum is in \(\mathbf{P}_{0}\).
* The four types of subspaces of \(\mathbf{R}^{3}\) are planes, lines, \(R^{3}\) itself, or \(\mathbf{Z}\) containing only \((0,0,0)\).
* Describe the three types of subspaces of \(\mathbf{R}^{2}\).
* Describe the five types of subspaces of \(\mathbf{R}^{4}\).
* The intersection of two planes through \((0,0,0)\) is probably a \(\raisebox{-1.5pt}{\includegraphics[]{fig/f(x).eps}}\) but it could be a \(\raisebox{-1.5pt}{\includegraphics[]{fig/f(x).eps}}\). It can't be the zero vector \(\mathbf{Z}\)!
* The intersection of a plane through \((0,0,0)\) with a line through \((0,0,0)\) is probably a \(\raisebox{-1.5pt}{\includegraphics[]{fig/f(x).eps}}\) but it could be a \(\raisebox{-1.5pt}{\includegraphics[]{fig/f(x).eps}}\).
 