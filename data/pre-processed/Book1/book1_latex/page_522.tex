
* All the pivots (without row exchanges) satisfy \(d_{i}<0\). (V) There is a matrix \(R\) with independent columns such that \(A=-R^{\mathrm{T}}R\).
* False (\(Q\) must contain eigenvectors of \(A\)); True (same eigenvalues as \(A\)); True (\(Q^{\mathrm{T}}AQ=Q^{-1}AQ\) is similar to \(A\)); True (eigenvalues of \(e^{-A}\) are \(e^{-\lambda}>0\)).
* Start from \(a_{jj}=\) (row \(j\) of \(R^{\mathrm{T}}\))(column \(j\) of \(R\)) = length squared of column \(j\) of \(R\). Then \(\det A=(\det R)^{2}=(\)volume of the \(R\) parallelepiped\()^{2}\leq\) product of the lengths squared of all the columns of \(R\). This product is \(a_{11}a_{22}\cdots a_{nn}\).
* \(A=\begin{bmatrix}2&-1&0\\ -1&2&-1\\ 0&-1&2\end{bmatrix}\) has pivots \(2\), \(\frac{3}{2}\), \(\frac{4}{3}\); \(A=\begin{bmatrix}2&-1&-1\\ -1&2&-1\\ -1&-1&2\end{bmatrix}\) is singular; \(A\begin{bmatrix}1\\ 1\\ 1\end{bmatrix}=\begin{bmatrix}0\\ 0\\ 0\end{bmatrix}\).
* \(x^{\mathrm{T}}Ax\) is not positive when \((x_{1},x_{2},x_{3})=(0,\,1,0)\) because of the zero on the diagonal.
* Positive definite requires positive determinant (also: all \(\lambda>0\)). (b) All projection matrices except \(I\) are singular. (c) The diagonal entries of \(D\) are its eigenvalues. (d) The negative definite matrix \(-I\) has \(\det=+1\) when \(n\) is even.
* \(\lambda_{1}=1/a^{2}\) and \(\lambda_{2}=1/b^{2}\), so \(a=1/\sqrt{\lambda_{1}}\) and \(b=1/\sqrt{\lambda_{2}}\). The ellipse \(9x^{2}+16y^{2}=1\) has axes with half-lengths \(a=\frac{1}{3}\) and \(b=\frac{1}{4}\).
* \(A=\begin{bmatrix}9&3\\ 3&5\end{bmatrix}=\begin{bmatrix}3&0\\ 1&2\end{bmatrix}\begin{bmatrix}3&1\\ 0&2\end{bmatrix}\); \(C=\begin{bmatrix}2&0\\ 4&3\end{bmatrix}\) has \(CC^{\mathrm{T}}=\begin{bmatrix}4&8\\ 8&25\end{bmatrix}\).
* \(ax^{2}+2bxy+cy^{2}=a\big{(}x+\frac{b}{a}y\big{)}^{2}+\frac{ac-b^{2}}{a}y^{2} ;2x^{2}+8xy+10y^{2}=2(x+2y)^{2}+2y^{2}\).
* \(x^{\mathrm{T}}Ax=2\big{(}x_{1}-\frac{1}{2}x_{2}-\frac{1}{2}x_{3}\big{)}^{2}+ \frac{3}{2}(x_{2}-x_{3})^{2};\,x^{\mathrm{T}}Bx=(x_{1}+x_{2}+x_{3})^{2}\). \(B\) has one pivot.
* \(A\) and \(C^{\mathrm{T}}AC\) have \(\lambda_{1}>0\), \(\lambda_{2}=0\). \(C\,(t)=t\,Q+(1-t)\,Q\,R\), \(Q=\begin{bmatrix}1&0\\ 0&-1\end{bmatrix}\), \(R=\begin{bmatrix}2&0\\ 0&1\end{bmatrix}\); \(C\) has one positive and one negative eigenvalue, but \(I\) has two positive eigenvalues.
* The pivots of \(A-\frac{1}{2}I\) are \(2.5\), \(5.9\), \(-0.81\), so one eigenvalue of \(A-\frac{1}{2}I\) is negative. Then \(A\) has an eigenvalue smaller than \(\frac{1}{2}\).
* \(\mathrm{rank}(C^{\mathrm{T}}AC)\leq\) rank \(A\), but also \(\mathrm{rank}(C^{\mathrm{T}}AC)\geq\) rank \(((C^{\mathrm{T}})^{-1}C^{\mathrm{T}}AC^{-1})=\) rank \(A\).
* No. If \(C\) is not square, \(C^{\mathrm{T}}AC\) is not the same size matrix as \(A\).
* \(\det\begin{bmatrix}6-4\lambda/18&-3-\lambda/18\\ -3-\lambda/18&6-4\lambda/18\end{bmatrix}=0\) gives \(\lambda_{1}=54\), \(\lambda_{2}=\frac{54}{5}\). Eigenvectors \(\begin{bmatrix}1\\ -1\end{bmatrix}\), \(\begin{bmatrix}1\\ 1\end{bmatrix}\).
* _Groups_: orthogonal matrices; \(e^{t\,A}\) for all \(t\); all matrices with \(\det=1\). If \(A\) is positive definite, the group of all powers \(A^{k}\) contains only positive definite matrices.

 