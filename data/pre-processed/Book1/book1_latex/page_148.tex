the data. These matrix equations and the corresponding differential equations are in our textbook _Introduction to Applied Mathematics_, and the new _Applied Mathematics and Scientific Computing_. (See www.wellesleycambridge.com.)

We end this chapter at that high point--the _formulation_ of a fundamental problem in applied mathematics. Often that requires more insight than the _solution_ of the problem. We solved linear equations in Chapter 1, as the first step in linear algebra. To set up the equations has required the deeper insight of Chapter 2. The contribution of mathematics, and of people, is not computation but intelligence.

### Problem Set 2.5

1. For the 3-node triangular graph in the figure following, write the 3 by 3 incidence matrix \(A\). Find a solution to \(Ax=0\) and describe all other vectors in the nullspace of \(A\). Find a solution to \(A^{\mathrm{T}}y=0\) and describe all other vectors in the left nullspace of \(A\).
2. For the same 3 by 3 matrix, show directly from the columns that every vector \(b\) in the column space will satisfy \(b_{1}+b_{2}-b_{3}=0\). Derive the same thing from the three rows--the equations in the system \(Ax=b\). What does that mean about potential differences around a loop?
3. Show directly from the rows that every vector \(f\) in the row space will satisfy \(f_{1}+f_{2}+f_{3}=0\). Derive the same thing from the three equations \(A^{\mathrm{T}}y=f\). What does that mean when the \(f\)'s are currents into the nodes?
4. Compute the 3 by 3 matrix \(A^{\mathrm{T}}A\), and show that it is symmetric but singular--what vectors are in its nullspace? Removing the last column of \(A\) (and last row of \(A^{\mathrm{T}}\)) leaves the 2 by 2 matrix in the upper left corner; show that it is _not_ singular.
5. Put the diagonal matrix \(C\) with entries \(c_{1}\), \(c_{2}\), \(c_{3}\) in the middle and compute \(A^{\mathrm{T}}CA\). Show again that the 2 by 2 matrix in the upper left corner is invertible.
6. Write the 6 by 4 incidence matrix \(A\) for the second graph in the figure. The vector \((1,1,1,1)\) is in the nullspace of \(A\), but now there will be \(m-n+1=3\) independent vectors that satisfy \(A^{\mathrm{T}}y=0\). Find three vectors \(y\) and _connect them to the loops in the graph_.

 