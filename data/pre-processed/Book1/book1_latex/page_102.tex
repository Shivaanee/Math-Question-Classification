satisfies this constraint, so it is forced on \(b!\)_ Geometrically, we shall see that the vector \((5,-2,1)\) is perpendicular to each column.

If \(b\) belongs to the column space, the solutions of \(Ax=b\) are easy to find. The last equation in \(Ux=c\) is \(0=0\). To the free variables \(v\) and \(y\), we may assign any values, as before. The pivot variables \(u\) and \(w\) are still determined by back-substitution. For a specific example with \(b_{3}-2b_{2}+5b_{1}=0\), choose \(b=(1,5,5)\):

\[Ax=b\qquad\begin{bmatrix}1&3&3&2\\ 2&6&9&7\\ -1&-3&3&4\end{bmatrix}\begin{bmatrix}u\\ v\\ w\\ y\end{bmatrix}=\begin{bmatrix}1\\ 5\\ 5\end{bmatrix}.\]

Forward elimination produces \(U\) on the left and \(c\) on the right:

\[Ux=c\qquad\begin{bmatrix}1&3&3&2\\ 0&0&3&3\\ 0&0&0&0\end{bmatrix}\begin{bmatrix}u\\ v\\ w\\ y\end{bmatrix}=\begin{bmatrix}1\\ 3\\ 0\end{bmatrix}.\]

The last equation is \(0=0\), as expected. Back-substitution gives

\[\begin{array}{r@{\qquad}l@{\qquad}l}3w+3y=3\qquad&\text{or}\qquad&w=1-y\\ u+3v+3w+2y=1\qquad&\text{or}\qquad&u=-2-3v+y.\end{array}\]

Again there is a double infinity of solutions: \(v\) and \(y\) are free, \(u\) and \(w\) are not:

\[\begin{array}{r@{\qquad}l}\textbf{Complete solution}\qquad&x=x_{p}+x_{n} \qquad&x=\begin{bmatrix}u\\ v\\ w\\ y\end{bmatrix}=\begin{bmatrix}-2\\ 0\\ 1\\ 0\end{bmatrix}+v\begin{bmatrix}-3\\ 1\\ 0\\ 0\end{bmatrix}+y\begin{bmatrix}1\\ 0\\ -1\\ 1\end{bmatrix}.\end{array}\] (4)

This has all solutions to \(Ax=0\), plus the new \(x_{p}=(-2,0,1,0)\). That \(x_{p}\) is _a particular solution_ to \(Ax=b\). The last two terms with \(v\) and \(y\) yield more solutions (because they satisfy \(Ax=0\)). _Every solution to \(Ax=b\) is the sum of one particular solution and a solution to \(Ax=0\)_:

\[x_{\textbf{complete}}=x_{\textbf{particular}}+x_{\textbf{nullspace}\textbf{ space}}\]

The particular solution in equation (4) comes from solving the equation _with all free variables set to zero_. That is the only new part, since the nullspace is already computed. When you multiply the highlighted equation by \(A\), you get \(Ax_{\text{complete}}=b+0\).

Geometrically, the solutions again fill a two-dimensional surface--but it is not a subspace. It does not contain \(x=0\). It is _parallel_ to the nullspace we had before, shifted by the particular solution \(x_{p}\) as in Figure 2.2. Equation (4) is a good way to write the answer:

1. Reduce \(Ax=b\) to \(Ux=c\).

 