\[\det(A-\lambda I)=\det\left|\begin{matrix}a-\lambda&b\\ c&d-\lambda\end{matrix}\right|=\lambda^{2}-(\text{trace})\lambda+\text{determinant}\]

\[\text{The eigenvalues are }\lambda=\frac{\text{trace}\pm\left[(\text{trace})^{2}-4 \det\right]^{1/2}}{2}.\]

Those two \(\lambda\)'s add up to the trace; Exercise 9 gives \(\sum\lambda_{i}=\text{trace}\) for all matrices.

### Eigshow

There is a MATLAB demo (just type eigshow), displaying the eigenvalue problem for a 2 by 2 matrix. It starts with the unit vector \(x=(1,0)\). _The mouse makes this vector move around the unit circle_. At the same time the screen shows \(Ax\), in color and also moving. Possibly \(Ax\) is ahead of \(x\). Possibly \(Ax\) is behind \(x\). _Sometimes \(Ax\) is parallel to \(x\)_. At that parallel moment, \(Ax=\lambda x\) (twice in the second figure).

\(x=(1,0)\)\(x=(0,1)\)\(x=(1,0 