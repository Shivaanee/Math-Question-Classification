That row exchange recovers $L U-$ but now $\ell_{31}=1$ and $\ell_{21}=2$ :
$$
P=\left[\begin{array}{lll}
1 & 0 & 0 \\
0 & 0 & 1 \\
0 & 1 & 0
\end{array}\right] \text { and } L=\left[\begin{array}{lll}
1 & 0 & 0 \\
2 & 1 & 0 \\
1 & 0 & 1
\end{array}\right] \text { and } P A=L U .
$$

In MATLAB, $\mathrm{A}([\mathrm{rk}]:)$ exchanges row $k$ with row $r$ below it (where the $k$ th pivot has been found). We update the matrices $L$ and $P$ the same way. At the start, $P=I$ and sign $=+1$ :
$$
\begin{aligned}
& \mathrm{A}([\mathrm{r} k],:)=\mathrm{A}([\mathrm{k} r],:) ; \\
& \mathrm{L}([\mathrm{rk}], 1: \mathrm{k}-1)=\mathrm{L}([\mathrm{k} r], 1: \mathrm{k}-1) ; \\
& \mathrm{P}([\mathrm{r} k],:)=\mathrm{P}([\mathrm{k}] \mathrm{r}],:) ; \\
& \text { sign = -sign }
\end{aligned}
$$

The "sign" of $P$ tells whether the number of row exchanges is even (sign $=+1$ ) or odd $(\operatorname{sign}=-1)$. A row exchange reverses sign. The final value of sign is the determinant of $P$ and it does not depend on the order of the row exchanges.

To summarize: A good elimination code saves $L$ and $U$ and $P$. Those matrices carry the information that originally came in $A$-and they carry it in a more usable form. $A x=$ $b$ reduces to two triangular systems. This is the practical equivalent of the calculation we do next-to find the inverse matrix $A^{-1}$ and the solution $x=A^{-1} b$.

Problem Set 1.5
1. When is an upper triangular matrix nonsingular (a full set of pivots)?
2. What multiple $\ell_{32}$ of row 2 of $A$ will elimination subtract from row 3 of $A$ ? Use the factored form
$$
A=\left[\begin{array}{lll}
1 & 0 & 0 \\
2 & 1 & 0 \\
1 & 4 & 1
\end{array}\right]\left[\begin{array}{lll}
5 & 7 & 8 \\
0 & 2 & 3 \\
0 & 0 & 6
\end{array}\right] .
$$

What will be the pivots? Will a row exchange be required?
3. Multiply the matrix $L=E^{-1} F^{-1} G^{-1}$ in equation (6) by $G F E$ in equation (3):
$$
\left[\begin{array}{ccc}
1 & 0 & 0 \\
2 & 1 & 0 \\
-1 & -1 & 1
\end{array}\right] \text { times }\left[\begin{array}{ccc}
1 & 0 & 0 \\
-2 & 1 & 0 \\
-1 & 1 & 1
\end{array}\right] .
$$

Multiply also in the opposite order. Why are the answers what they are?
