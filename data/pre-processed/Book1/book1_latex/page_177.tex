individual vectors. That honor belongs to \(A^{-1}\) if it exists--and it only exists if \(r=m=n\). We cannot ask \(A^{-1}\) to bring back a whole nullspace out of the zero vector.

When \(A^{-1}\) fails to exist, the best substitute is the _pseudoinverse_\(A^{+}\). This inverts \(A\) where that is possible: \(A^{+}Ax=x\) for \(x\) in the row space. On the left nullspace, nothing can be done: \(A^{+}y=0\). Thus \(A^{+}\) inverts \(A\) where it is invertible, and has the same rank \(r\). One formula for \(A^{+}\) depends on the _singular value decomposition_--for which we first need to know about eigenvalues.

**Problem Set 3.1**:
**1.**: Find the lengths and the inner product of \(x=(1,4,0,2)\) and \(y=(2,-2,1,3)\).
**2.**: Give an example in \(\mathbf{R}^{2}\) of linearly independent vectors that are not orthogonal. Also, give an example of orthogonal vectors that are not independent.
**3.**: Two lines in the plane are perpendicular when the product of their slopes is \(-1\). Apply this to the vectors \(x=(x_{1},x_{2})\) and \(y=(y_{1},y_{2})\), whose slopes are \(x_{2}/x_{1}\) and \(y_{2}/y_{1}\), to derive again the orthogonality condition \(x^{\mathrm{T}}y=0\).
**4.**: How do we know that the \(i\)th row of an invertible matrix \(B\) is orthogonal to the \(j\)th column of \(B^{-1}\), if \(i\neq j\)?
**5.**: Which pairs are orthogonal among the vectors \(v_{1}\), \(v_{2}\), \(v_{3}\), \(v_{4}\)?

\[v_{1}=\begin{bmatrix}1\\ 2\\ -2\\ 1\end{bmatrix},\qquad v_{2}=\begin{bmatrix}4\\ 0\\ 4\\ 0\end{bmatrix},\qquad v_{3}=\begin{bmatrix}1\\ -1\\ -1\\ -1\end{bmatrix},\qquad v_{4}=\begin{bmatrix}1\\ 1\\ 1\\ 1\end{bmatrix}.\]
**6.**: Find all vectors in \(\mathbf{R}^{3}\) that are orthogonal to \((1,1,1)\) and \((1,-1,0)\). Produce an orthonormal basis from these vectors (mutually orthogonal unit vectors).
**7.**: Find a vector \(x\) orthogonal to the row space of \(A\), and a vector \(y\) orthogonal to the column space, and a vector \(z\) orthogonal to the nullspace:

\[A=\begin{bmatrix}1&2&1\\ 2&4&3\\ 3&6&4\end{bmatrix}.\]
**8.**: If \(\mathbf{V}\) and \(\mathbf{W}\) are orthogonal subspaces, show that the only vector they have in common is the zero vector: \(\mathbf{V}\cap\mathbf{W}=\{0\}\).
**9.**: Find the orthogonal complement of the plane spanned by the vectors \((1,1,2)\) and \((1,2,3)\), by taking these to be the rows of \(A\) and solving \(Ax=0\). Remember that the complement is a whole line.

 