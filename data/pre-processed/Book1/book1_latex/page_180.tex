

**33.** Find the pieces \(x_{r}\) and \(x_{n}\), and draw Figure 3.4 properly, if

\[A=\begin{bmatrix}1&-1\\ 0&0\\ 0&0\end{bmatrix}\qquad\text{and}\qquad x=\begin{bmatrix}2\\ 0\end{bmatrix}.\]

**Problems 34-44 are about orthogonal subspaces.**

**34.** Put bases for the orthogonal subspaces \(\mathbf{V}\) and \(\mathbf{W}\) into the columns of matrices \(V\) and \(W\). _Why does \(V^{\mathrm{T}}W=zero\) matrix_? This matches \(v^{\mathrm{T}}W=0\) for vectors.

**35.** The floor and the wall are not orthogonal subspaces because they share a nonzero vector (along the line where they meet). Two planes in \(\mathbf{R}^{3}\) cannot be orthogonal! Find a vector in both column spaces \(C(A)\) and \(C(B)\):

\[A=\begin{bmatrix}1&2\\ 1&3\\ 1&2\end{bmatrix}\qquad\text{and}\qquad B=\begin{bmatrix}5&4\\ 6&3\\ 5&1\end{bmatrix}.\]

This will be a vector \(Ax\) and also \(B\widehat{x}\). Think 3 by 4 with the matrix \([A\;\;B]\).

**36.** Extend Problem 35 to a \(p\)-dimensional subspace \(\mathbf{V}\) and a \(q\)-dimensional subspace \(\mathbf{W}\) of \(\mathbf{R}^{n}\). What inequality on \(p+q\) guarantees that \(\mathbf{V}\) intersects \(\mathbf{W}\) in a nonzero vector? These subspaces cannot be orthogonal.

**37.** Prove that every \(y\) in \(N(A^{\mathrm{T}})\) is perpendicular to every \(Ax\) in the column space, using the matrix shorthand of equation (8). Start from \(A^{\mathrm{T}}y=0\).

**38.** If \(\mathbf{S}\) is the subspace of \(\mathbf{R}^{3}\) containing only the zero vector, what is \(\mathbf{S}^{\perp}\)? If \(\mathbf{S}\) is spanned by \((1,1,1)\), what is \(\mathbf{S}^{\perp}\)? If \(\mathbf{S}\) is spanned by \((2,0,0)\) and \((0,0,3)\), what is \(\mathbf{S}^{\perp}\)?

**39.** Suppose \(\mathbf{S}\) only contains \((1,5,1)\) and \((2,2,2)\) (not a subspace). Then \(\mathbf{S}^{\perp}\) is the nullspace of the matrix \(A=\underline{\phantom{\underline{\underline{\underline{\underline{\underline{ \underline{\underline{\underline{\underline{\underline{\underline{\underline{ \underline{\underline{\underline{\underline{ \underline{ \underline{        }}}}}}}}}}}}}}} \mathbf{S^{\perp}\) is a subspace even if \(\mathbf{S}\) is not.

**40.** Suppose \(\mathbf{L}\) is a one-dimensional subspace (a line) in \(\mathbf{R}^{3}\). Its orthogonal complement \(\mathbf{L}^{\perp}\) is the perpendicular to \(\mathbf{L}\). Then \((\mathbf{L}^{\perp})^{\perp}\) is a perpendicular to \(\mathbf{L}^{\perp}\). In fact \((\mathbf{L}^{\perp})^{\perp}\) is the same as \(\underline{\phantom{\underline{\underline{\underline{\underline{\underline{ \underline{\underline{\underline{\underline{\underline{\underline{\underline{ \underline{\underline{\underline{\underline{ \underline{ \underline{ \underline{ \underlineunderlineunderline { \underline \cdot}}}}}}}}}}}}}}}\).

** **41. Suppose \(\mathbf{V}\) is the whole space \(\mathbf{R}^{4}\). Then \(\mathbf{V}^{\perp}\) contains only the vector \(\underline{\phantom{\underline{\underline{\underline{\underline{\underline{ \underline{\underline