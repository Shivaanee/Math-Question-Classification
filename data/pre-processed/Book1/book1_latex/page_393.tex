

**5.**: For any symmetric matrix \(A\), compute the ratio \(R(x)\) for the special choice \(x=(1,\ldots,1)\). How is the sum of all entries \(a_{ij}\) related to \(\lambda_{1}\) and \(\lambda_{n}\)?
**6.**: With \(A=\left[\begin{smallmatrix}2&-1\\ -1&2\end{smallmatrix}\right]\), find a choice of \(x\) that gives a smaller \(R(x)\) than the bound \(\lambda_{1}\leq 2\) that comes from the diagonal entries. What is the minimum value of \(R(x)\)?
**7.**: If \(B\) is positive definite, show from the Rayleigh quotient that the smallest eigenvalue of \(A+B\) is larger than the smallest eigenvalue of \(A\).
**8.**: If \(\lambda_{1}\) and \(\mu_{1}\) are the smallest eigenvalues of \(A\) and \(B\), show that the smallest eigenvalue \(\theta_{1}\) of \(A+B\) is at least as large as \(\lambda_{1}+\mu_{1}\). (Try the corresponding eigenvector \(x\) in the Rayleigh quotients.)
**Note.**: Problems 7 and 8 are perhaps the most typical and most important results that come easily from Rayleigh's principle, but only with great difficulty from the eigenvalue equations themselves.
**9.**: If \(B\) is positive definite, show from the minimax principle (12) that the _second_ smallest eigenvalue is increased by adding \(B:\lambda_{2}(A+B)>\lambda_{2}(A)\).
**10.**: If you throw away _two_ rows and columns of \(A\), what inequalities do you expect between the smallest eigenvalue \(\mu\) of the new matrix and the original \(\lambda\)'s?
**11.**: Find the minimum values of

\[R(x)=\frac{x_{1}^{2}-x_{1}x_{2}+x_{2}^{2}}{x_{1}^{2}+x_{2}^{2}}\qquad\text{ and}\qquad R(x)=\frac{x_{1}^{2}-x_{1}x_{2}+x_{2}^{2}}{2x_{1}^{2}+x_{2}^{2}}.\]
**12.**: Prove from equation (11) that \(R(x)\) is never larger than the largest eigenvalue \(\lambda_{n}\).
**13.**: The minimax principle for \(\lambda_{j}\) involves \(j\)-dimensional subspaces \(S_{j}\):

\[\text{{Equivalent to equation (\ref{eq:eq:eq:eq:eq:eq:eq:eq:eq:eq:eq:eq:eq:eq:eq:eq: 