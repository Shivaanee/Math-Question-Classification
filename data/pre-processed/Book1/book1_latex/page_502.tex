35. \(Ax=B\widehat{x}\) means that \([\begin{array}{cc}A&B\end{array}]\left[\begin{array}{c}x\\ -\widehat{x}\end{array}\right]=0\). Three homogeneous equations in four unknowns always have a nonzero solution. Here \(x=(3,1)\) and \(\widehat{x}=(1,0)\), and \(Ax=B\widehat{x}=(5,6,5)\) is in both column spaces. Two planes in \(\mathbf{R}^{3}\) (through zero) must intersect in a line at least!
37. \(A^{\mathrm{T}}y=0\) gives \((Ax)^{\mathrm{T}}y=x^{\mathrm{T}}A^{\mathrm{T}}y=0\). Then \(y\perp Ax\) and \(N(A^{\mathrm{T}})\perp\boldsymbol{C}(A)\).
39. \(\mathbf{S}^{\perp}\) is the nullspace of \(A=\left[\begin{matrix}1&5&1\\ 2&2\end{matrix}\right]\). Therefore \(\mathbf{S}^{\perp}\) is a _subspace_ even if \(\mathbf{S}\) is not. \(\setminus\)
41. If \(\mathbf{V}\) is all of \(\mathbf{R}^{4}\), then \(\mathbf{V}^{\perp}\) contains only the _zero vector_. Then \((\mathbf{V}^{\perp})^{\perp}=\mathbf{R}^{4}=\mathbf{V}\).
43. \((1,1,1,1)\) is a basis for \(\mathbf{P}^{\perp}\). \(A=[1\quad 1\quad 1\quad 1]\) has the plane \(\mathbf{P}\) as its nullspace.
45. Column 1 of \(A^{-1}\) is orthogonal to the space spanned by the 2nd, ..., \(n\)th rows of \(A\).
47. \(A=\left[\begin{matrix}2&2&-1\\ -1&2&2\\ 2&-1&2\end{matrix}\right]\), \(A^{\mathrm{T}}A=9I\) is _diagonal_: \((A^{\mathrm{T}}A)_{ij}=(\mathrm{column}\ i\ \mathrm{of}\ A)\cdot(\mathrm{column}\ j)\).
49. \((\mathrm{a})\ (1,-1,0)\) is in both planes. Normal vectors are perpendicular, planes still intersect! \((\mathrm{b})\) Need _three_ orthogonal vectors to span the whole orthogonal complement in \(\mathbf{R}^{5}\). \((\mathrm{c})\) Lines can meet without being orthogonal.
51. When \(AB=0\), the column space of \(B\) is contained in the nullspace of \(A\). Therefore the dimension of \(\boldsymbol{C}(B)\leq\mathrm{dimension}\) of \(N(A)\). This means \(\mathrm{rank}(B)\leq 4\ -\ \mathrm{rank}(A)\).

Problem Set 3.2, page 157 .

1. \((x+y)/2\geq\sqrt{xy}\) (arithmetic mean \(\geq\) geometric mean of \(x\) and \(y\)). 2. \(\|x+y\|^{2}\leq(\|x\|+\|y\|)^{2}\) means that \((x+y)^{\mathrm{T}}(x+y)\leq\|x\|^{2}+2\|x\|\|y\|+\|y\|^{2}\). The left-hand side is \(x^{\mathrm{T}}x+2x^{\mathrm{T}}y+y^{\mathrm{T}}y\). After cancelling this is \(x^{\mathrm{T}}y\leq\|x\|\|y\|\).
3. \(p=(10/3,10/3,10/3)\); \((5/9,10/9,10/9)\).
4. \(\cos\theta=1/\sqrt{n}\) so \(\theta=\arccos(1/\sqrt{n})\); \(P=\left[\begin{matrix}1\\ \vdots\\ 1\end{matrix}\right][1/n\quad\cdots\quad 1/n]=\mathrm{all\ entries}\ \frac{1}{n}\).
5. Choose \(b=(1,\,\ldots,\,1)\); equality if \(a_{1}=\cdots=a_{n}\) (then \(a\) is parallel to \(b\)).
6. \(P^{2}=\frac{aa^{\mathrm{T}}aa^{\mathrm{T}}}{a^{\mathrm{T}}aa^{\mathrm{T}}a}= \frac{a(a^{\mathrm{T}}a)a^{\mathrm{T}}}{(a^{\mathrm{T}}a)(a^{\mathrm{T}}a)}= \frac{aa^{\mathrm{T}}}{a^{\mathrm{T}}a}=P\).

11. \(P=\left[\begin{matrix}\frac{1}{10}&\frac{3}{10}\\ \frac{3}{10}&\frac{9}{10}\end{matrix}\right]\); \(P_{2}=I-P_{1}=\left[\begin{matrix}\frac{9}{10}&-\frac{3}{10}\\ -\frac{3}{10}&\frac{1}{10}\end{matrix}\right]\). \((\mathrm{b})\)\(P_{1}+P_{2}=\left[\begin{matrix}1&0\\ 0&1\end{matrix}\right]\); \(P_{1}P_{2}=\left[\begin{matrix}0&0\\ 0&1\end{matrix}\right]\); \(P_{1}P_{2}=\left[\begin{matrix}0&0\\ 0&0\end{matrix}\right]\). The sum of the projections onto two perpendicular lines gives the vector itself. The projection onto one line and then a perpendicular line gives the zero vector.
13. Trace \(=\frac{a_{1}a_{1}}{a^{\mathrm{T}}a}+\cdots+\frac{a_{n}a_{n}}{a^{\mathrm{T}}a}= \frac{a^{\mathrm{T}}a}{a^{\mathrm{T}}a}=1\).

 