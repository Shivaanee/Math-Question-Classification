Its determinant is the product of its pivots. The numbers \(2,\ldots,n\) all cancel:

\[\det A=2\left(\frac{3}{2}\right)\left(\frac{4}{3}\right)\cdots\left(\frac{n+1}{ n}\right)=n+1.\]

MATLAB computes the determinant from the pivots. But concentrating all information into the pivots makes it impossible to figure out how a change in one entry would affect the determinant. We want to find an explicit expression for the determinant in terms of the \(n^{2}\) entries.

For \(n=2\), we will be proving that \(ad-bc\) is correct. For \(n=3\), the determinant formula is again pretty well known (it has six terms):

\[\begin{vmatrix}a_{11}&a_{12}&a_{13}\\ a_{21}&a_{22}&a_{23}\\ a_{31}&a_{32}&a_{33}\end{vmatrix}=\begin{array}{c}+a_{11}a_{22}a_{33}+a_{12} a_{23}a_{31}+a_{13}a_{21}a_{32}\\ -a_{11}a_{23}a_{32}-a_{12}a_{21}a_{33}-a_{13}a_{22}a_{31}.\end{array}\] (2)

Our goal is to derive these formulas directly from the defining properties 1-3 of \(\det A\). If we can handle \(n=2\) and \(n=3\) in an organized way, you will see the pattern.

To start, each row can be broken down into vectors in the coordinate directions:

\[\begin{bmatrix}a&b\end{bmatrix}=\begin{bmatrix}a&0\end{bmatrix}+\begin{bmatrix} 0&b\end{bmatrix}\qquad\text{and}\qquad\begin{bmatrix}c&d\end{bmatrix}= \begin{bmatrix}c&0\end{bmatrix}+\begin{bmatrix}0&d\end{bmatrix}.\]

Then we apply the property of linearity, first in row 1 and then in row 2:

\[\begin{array}{ll}\textbf{Separate into}&\begin{vmatrix}a&b\\ c&d\end{vmatrix}&=\begin{vmatrix}a&0\\ c&d\end{vmatrix}+\begin{vmatrix}0&b\\ c&d\end{vmatrix}\\ n^{n}=2^{2}\textbf{easy}&\\ \textbf{determinants}&=\begin{vmatrix}a&0\\ c&0\end{vmatrix}+\begin{vmatrix}a&0\\ 0&d\end{vmatrix}+\begin{vmatrix}0&b\\ c&0\end{vmatrix}+\begin{vmatrix}0&b\\ 0&d\end{vmatrix}.\end{array}\] (3)

Every row splits into \(n\) coordinate directions, so this expansion has \(n^{n}\) terms. Most of those terms (all but \(n!=n\) factorial) will be automatically zero. When two rows are in the same coordinate direction, one will be a multiple of the other, and

\[\begin{vmatrix}a&0\\ c&0\end{vmatrix}=0,\qquad\begin{vmatrix}0&b\\ 0&d\end{vmatrix}=0.\]

We pay attention _only when the rows point in different directions_. _The nonzero terms have to come in different columns_. Suppose the first row has a nonzero term in column \(\alpha\), the second row is nonzero in column \(\beta\), and finally the \(n\)th row in column \(v\). The column numbers \(\alpha,\beta,\ldots,v\) are all different. They are a reordering, or _permutation_, of 