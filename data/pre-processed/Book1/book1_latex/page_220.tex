

**20.**: In Hilbert space, find the length of the vector \(v=(1/\sqrt{2},1/\sqrt{4},1/\sqrt{8},\ldots)\) and the length of the function \(f(x)=e^{x}\) (over the interval \(0\leq x\leq 1\)). What is the inner product over this interval of \(e^{x}\) and \(e^{-x}\)?
**21.**: What is the closest function \(a\cos x+b\sin x\) to the function \(f(x)=\sin 2x\) on the interval from \(-\pi\) to \(\pi\)? What is the closest straight line \(c+dx\)?
**22.**: By setting the derivative to zero, find the value of \(b_{1}\) that minimizes

\[\|b_{1}\sin x-\cos x\|^{2}=\int_{0}^{2\pi}(b_{1}\sin x-\cos x)^{2}dx.\]

Compare with the Fourier coefficient \(b_{1}\).
**23.**: Find the Fourier coefficients \(a_{0}\), \(a_{1}\), \(b_{1}\) of the step function \(y(x)\), which equals 1 on the interval \(0\leq x\leq\pi\) and 0 on the remaining interval \(\pi<x<2\pi\):

\[a_{0}=\frac{(y,1)}{(1,1)}\qquad a_{1}=\frac{(y,\cos x)}{(\cos x,\cos x)} \qquad b_{1}=\frac{(y,\sin x)}{(\sin x,\sin x)}.\]
**24.**: Find the fourth Legendre polynomial. It is a cubic \(x^{3}+ax^{2}+bx+c\) that is orthogonal to 1, \(x\), and \(x^{2}-\frac{1}{3}\) over the interval \(-1\leq x\leq 1\).
**25.**: What is the closest straight line to the parabola \(y=x^{2}\) over \(-1\leq x\leq 1\)?
**26.**: In the Gram-Schmidt formula (10), verify that \(C\) is orthogonal to \(q_{1}\) and \(q_{2}\).
**27.**: Find an orthonormal basis for the subspace spanned by \(a_{1}=(1,-1,0,0)\), \(a_{2}=(0,1,-1,0)\), \(a_{3}=(0,0,1,-1)\).
**28.**: Apply Gram-Schmidt to \((1,-1,0)\), \((0,1,-1)\), and \((1,0,-1)\), to find an orthonormal basis on the plane \(x_{1}+x_{2}+x_{3}=0\). What is the dimension of this subspace, and how many nonzero vectors come out of Gram-Schmidt?
**29.**: (Recommended) Find orthogonal vectors \(A\), \(B\), \(C\) by Gram-Schmidt from \(a\), \(b\), \(c\):

\[a=(1,-1,0,0)\qquad b=(0,1,-1,0)\qquad c=(0,0,1,-1).\]

\(A\), \(B\), \(C\) and \(a\), \(b\), \(c\) are bases for the vectors perpendicular to \(d=(1,1,1,1)\).
**30.**: If \(A=QR\) then \(A^{\rm T}A=R^{\rm T}R=\underline{\hbox{ triangular times }}\underline{\hbox{ triangular. }}\underline{\hbox{ }}

