3. Why does no matrix transform \((2,6)\) to \((1,0)\) and \((1,3)\) to \((0,1)\)?
37. 1. What matrix \(M\) transforms \((1,0)\) and \((0,1)\) to \((r,t)\) and \((s,u)\)? 2. What matrix \(N\) transforms \((a,c)\) and \((b,d)\) to \((1,0)\) and \((0,1)\)? 3. What condition on \(a\), \(b\), \(c\), \(d\) will make part (b) impossible?
38. 1. How do \(M\) and \(N\) in Problem 37 yield the matrix that transforms \((a,c)\) to \((r,t)\) and \((b,d)\) to \((s,u)\)? 2. What matrix transforms \((2,5)\) to \((1,1)\) and \((1,3)\) to \((0,2)\)?
39. If you keep the same basis vectors but put them in a different order, the change-of-basis matrix \(M\) is a matrix. If you keep the basis vectors in order but change their lengths, \(M\) is a matrix.
40. The matrix that transforms \((1,0)\) and \((0,1)\) to \((1,4)\) and \((1,5)\) is \(M=\underline{\phantom{\rule{0.0pt}{1.0pt}}\phantom{\rule{0.0pt}{1.0pt}}}\). The combination \(a(1,4)+b(1,5)\) that equals \((1,0)\) has \((a,b)=(\phantom{\rule{0.0pt}{1.0pt}}\phantom{\rule{0.0pt}{1.0pt}}\phantom{\rule{0.0 pt}{1.0pt}}\phantom{\rule{0.0pt}{1.0pt}}\phantom{\rule{0.0pt}{1.0pt}}\phantom{\rule{0.0pt}{1.0pt}} \phantom{\rule{0.0pt}{1.0pt}}\phantom{\rule{0 