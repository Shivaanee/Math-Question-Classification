

**8.**: For which three numbers \(k\) does elimination break down? Which is fixed by a row exchange? In each case, is the number of solutions 0 or 1 or \(\infty\)?

\[\begin{array}{rclrcl}kx&+&3y&=&6\\ 3x&+&ky&=&-6.\end{array}\]
**9.**: What test on \(b_{1}\) and \(b_{2}\) decides whether these two equations allow a solution? How many solutions will they have? Draw the column picture.

\[\begin{array}{rclrcl}3x&-&2y&=&b_{1}\\ 6x&-&4y&=&b_{2}.\end{array}\]
**Problems 10-19 study elimination on 3 by 3 systems (and possible failure).**
**10.**: Reduce this system to upper triangular form by two row operations:

\[\begin{array}{rclrclrcl}2x&+&3y&+&z&=&8\\ 4x&+&7y&+&5z&=&20\\ &-&2y&+&2z&=&0.\end{array}\]

Circle the pivots. Solve by back-substitution for \(z\), \(y\), \(x\).
**11.**: Apply elimination (circle the pivots) and back-substitution to solve

\[\begin{array}{rclrclrcl}2x&-&3y&&=&3\\ 4x&-&5y&+&z&=&7\\ 2x&-&y&-&3z&=&5.\end{array}\]

List the three row operations: Subtract times row from row.
**12.**: Which number \(d\) forces a row exchange, and what is the triangular system (not singular) for that \(d\)? Which \(d\) makes this system singular (no third pivot)?

\[\begin{array}{rclrclrcl}2x&+&5y&+&z&=&0\\ 4x&+&dy&+&z&=&2\\ &&y&-&z&=&3.\end{array}\]
**13.**: Which number \(b\) leads later to a row exchange? Which \(b\) leads to a missing pivot? In that singular case find a nonzero solution \(x\), \(y\), \(z\).

\[\begin{array}{rclrclrclrcl}x&+&by&&=&0\\ x&-&2y&-&z&=&0\\ &&y&+&z&=&0.\end{array}\]
**14.**: (a) Construct a 3 by 3 system that needs two row exchanges to reach a triangular form and a solution.
**2.**: Construct a 3 by 3 system that needs a row exchange to keep going, but breaks down later.

