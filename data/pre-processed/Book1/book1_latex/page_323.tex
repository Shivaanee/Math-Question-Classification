Multiplying \(a+ib\) times \(c+id\) uses the rule that \(i^{2}=-1\):

\[\textbf{Multiplication}\qquad(a+ib)(c+id) =ac+ibc+iad+i^{2}bd\] \[=(ac-bd)+i(bc+ad).\]

The _complex conjugate_ of \(a+ib\) is the number \(a-ib\). _The sign of the imaginary part is reversed_. It is the mirror image across the real axis; any real number is its own conjugate, since \(b=0\). The conjugate is denoted by a bar or a star: \((a+ib)^{*}=\overline{a+ib}=a-ib\). It has three important properties:

1. The conjugate of a product equals the product of the conjugates: \[\overline{(a+ib)(c+id)}=(ac-bd)-i(bc+ad)=\overline{(a+ib)(c+id)}.\] (1)
2. The conjugate of a sum equals the sum of the conjugates: \[\overline{(a+c)+i(b+d)}=(a+c)-i(b+d)=\overline{(a+ib)}+\overline{(c+id)}.\]
3. Multiplying any \(a+ib\) by its conjugate \(a-ib\) produces a real number \(a^{2}+b^{2}\): \[\textbf{Absolute value}\qquad(a+ib)(a-ib)=a^{2}+b^{2}=r^{2}.\] (2) This distance \(r\) is the _absolute value_\(|a+ib|=\sqrt{a^{2}+b^{2}}\).

Finally, trigonometry connects the sides \(a\) and \(b\) to the hypotenuse \(r\) by \(a=r\cos\theta\) and \(b=r\sin\theta\). Combining these two equations moves us into polar coordinates:

\[\textbf{Polar form}\qquad a+ib=r(\cos\theta+i\sin\theta)=re^{i\theta}.\] (3) 