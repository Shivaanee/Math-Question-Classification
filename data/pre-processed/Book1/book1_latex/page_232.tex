

**3.10**: Which constant function is closest to \(y=x^{4}\) (in the least-squares sense) over the interval \(0\leq x\leq 1\)?
**3.11**: If \(Q\) is orthogonal, is the same true of \(Q^{3}\)?
**3.12**: Find all 3 by 3 orthogonal matrices whose entries are zeros and ones.
**3.13**: What multiple of \(a_{1}\) should be subtracted from \(a_{2}\), to make the result orthogonal to \(a_{1}\)? Sketch a figure.
**3.14**: Factor

\[\begin{bmatrix}\cos\theta&\sin\theta\\ \sin\theta&0\end{bmatrix}\]

into \(QR\), recognizing that the first column is already a unit vector.
**3.15**: If every entry in an orthogonal matrix is either \(\frac{1}{4}\) or \(-\frac{1}{4}\), how big is the matrix?
**3.16**: Suppose the vectors \(q_{1},\ldots,q_{n}\) are orthonormal. If \(b=c_{1}q_{1}+\cdots+c_{n}q_{n}\), give a formula for the first coefficient \(c_{1}\) in terms of \(b\) and the \(q\)'s.
**3.17**: What words describe the equation \(A^{\rm T}A\widehat{x}=A^{\rm T}b\), the vector \(p=A\widehat{x}=Pb\), and the matrix \(P=A(A^{\rm T}A)^{-1}A^{\rm T}\)?
**3.18**: If the orthonormal vectors \(q_{1}=(\frac{2}{3},\frac{2}{3},-\frac{1}{3})\) and \(q_{2}=(-\frac{1}{3},\frac{2}{3},\frac{2}{3})\) are the columns of \(Q\), what are the matrices \(Q^{\rm T}Q\) and \(QQ^{\rm T}\)? Show that \(QQ^{\rm T}\) is a projection matrix (onto the plane of \(q_{1}\) and \(q_{2}\)).
**3.19**: If \(v_{1},\ldots,v_{n}\) is an orthonormal basis for \({\bf R}^{n}\), show that \(v_{1}v_{1}^{\rm T}+\cdots+v_{n}v_{n}^{\rm T}=I\).
**3.20**: _True or false_: If the vectors \(x\) and \(y\) are orthogonal, and \(P\) is a projection, then \(Px\) and \(Py\) are orthogonal.
**3.21**: Try to fit a line \(b=C+Dt\) through the points \(b=0\), \(t=2\), and \(b=6\), \(t=2\), and show that the normal equations break down. Sketch all the optimal lines, minimizing the sum of squares of the two errors.
**3.22**: What point on the plane \(x+y-z=0\) is closest to \(b=(2,1,0)\)?
**3.23**: Find an orthonormal basis for \({\bf R}^{3}\) starting with the vector \((1,1,1)\).
**3.24**: CT scanners examine the patient from different directions and produce a matrix giving the densities of bone and tissue at each point. Mathematically, the problem is to recover a matrix from its projections. in the 2 by 2 case, can you recover the matrix \(A\) if you know the sum along each row and down each column?
**3.25**: Can you recover a 3 by 3 matrix if you know its row sums and column sums, and also the sums down the main diagonal and the four other parallel diagonals?