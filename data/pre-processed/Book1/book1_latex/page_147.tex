right-hand side accounts for the battery of strength \(b_{3}\) in edge 3. The block form has \(C^{-1}y+Ax=b\) above \(A^{\mathrm{T}}y=f\):

\[\begin{bmatrix}C^{-1}&A\\ A^{\mathrm{T}}&0\end{bmatrix}\begin{bmatrix}y\\ x\end{bmatrix}=\begin{bmatrix}R_{1}&&&&-1&1&0\\ R_{2}&&&&0&-1&1\\ &&R_{3}&&-1&0&1\\ &&&R_{4}&&0&0&-1\\ &&&R_{5}&-1&0&0\\ -1&0&-1&0&-1&&&\\ 1&-1&0&0&0&&\\ 0&1&1&-1&0&&&\end{bmatrix}\begin{bmatrix}y_{1}\\ y_{2}\\ y_{3}\\ y_{4}\\ y_{5}\\ x_{1}\\ x_{2}\\ x_{3}\end{bmatrix}=\begin{bmatrix}0\\ 0\\ b_{3}\\ 0\\ 0\\ f_{2}\\ 0\end{bmatrix}\]

The system is 8 by 8, with five currents and three potentials. Elimination of \(y\)'s reduces to the 3 by 3 system \(A^{\mathrm{T}}CAx=A^{\mathrm{T}}Cb-f\). The matrix \(A^{\mathrm{T}}CA\) contains the reciprocals \(c_{i}=1/R_{i}\) (because in elimination you divide by the pivots). We also show the fourth row and column, from the grounded node, outside the 3 by 3 matrix:

\[A^{\mathrm{T}}CA=\begin{bmatrix}c_{1}+c_{3}+c_{5}&-c_{1}&-c_{3}\\ -c_{1}&c_{1}+c_{2}&-c_{2}\\ -c_{3}&-c_{2}&c_{2}+c_{3}+c_{4}\end{bmatrix}-c_{5} \text{(node 1)}\] \[-c_{4} \text{(node 3)}\]

The first entry is \(1+1+1\), or \(c_{1}+c_{3}+c_{5}\) when \(C\) is included, because edges 1, 3, 5 touch node 1. The next diagonal entry is \(1+1\) or \(c_{1}+c_{2}\), from the edges touching node 2. Off the diagonal the \(c\)'s appear with minus signs. _The edges to the grounded node 4 belong in the fourth_ row and column, which are deleted when column 4 is removed from \(A\) (making \(A^{\mathrm{T}}CA\) invertible). The 4 by 4 matrix would have all rows and columns adding to zero, and \((1,1,1,1)\) would be in its nullspace.

Notice that \(A^{\mathrm{T}}CA\) is symmetric. It has positive pivots and it comes from the **basic framework of applied mathematics** illustrated in Figure 2.8.

In mechanics, \(x\) and \(y\) become displacements and stresses. In fluids, the unknowns are pressure and flow rate. In statistics, \(e\) is the error and \(x\) is the best least-squares fit to

Figure 2.8: The framework for equilibrium: sources \(b\) and \(f\), three steps to \(A^{\mathrm{T}}CA\).

 