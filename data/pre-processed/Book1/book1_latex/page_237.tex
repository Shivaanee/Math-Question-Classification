the best definition. Obviously, \(\det A\) will not be some extremely simple function of \(n^{2}\) variables; otherwise \(A^{-1}\) would be much easier to find than it actually is.

_The simple things about the determinant are not the explicit formulas, but the properties it possesses_. This suggests the natural place to begin. The determinant can be (and will be) defined by its three most basic properties: \(\det I=1\), **the sign is reversed by a row exchange, the determinant is linear in each row separately**. The problem is then to show, by systematically using these properties, how the determinant can be computed. This will bring us back to the product of the pivots.

Section 4.2 explains these three defining properties of the determinant, and their most important consequences. Section 4.3 gives two more formulas for the determinant--the "big formula" with \(n!\) terms, and a formula "by induction". In Section 4.4 the determinant is applied to find \(A^{-1}\). Then we compute \(x=A^{-1}b\) by _Cramer's rule_. And finally, in an optional remark on permutations, we show that whatever the order in which the properties are used, the result is always the same--the defining properties are self-consistent.

Here is a light-hearted question about permutations. _How many exchanges does it take to change VISA into AVIS_? Is this permutation odd or even?

### 4.2 Properties of the Determinant

This will be a pretty long list. Fortunately each rule is easy to understand, and even easier to illustrate, for a 2 by 2 example. Therefore we shall verify that the familiar definition in the 2 by 2 case,

\[\det\begin{bmatrix}a&b\\ c&d\end{bmatrix}=\begin{vmatrix}a&b\\ c&d\end{vmatrix}=ad-bc,\]

possesses every property in the list. (Notice the two accepted notations for the determinant, \(\det A\) and \(|A|\).) Properties 4-10 will be deduced from the previous ones. **Every property is a consequence of the first three**. We emphasize that the rules apply to square matrices _of any size_.

**1.**_The determinant of the identity matrix is \(1\)_.

\[\det I=1\qquad\begin{vmatrix}1&0\\ 0&1\end{vmatrix}=1\qquad\text{and}\qquad\begin{vmatrix}1&0&0\\ 0&1&0\\ 0&0&1\end{vmatrix}=1\qquad\text{and}\ldots\] 