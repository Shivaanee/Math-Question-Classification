

**8.** Show that starting from \(A_{0}=\left[\begin{smallmatrix}2&-1\\ -1&2\end{smallmatrix}\right]\), the unshifted \(QR\) algorithm produces only the modest improvement \(A_{1}=\frac{1}{5}\left[\begin{smallmatrix}14&-3\\ -3&6\end{smallmatrix}\right]\).
**9.** Apply to the following matrix \(A\) a single \(QR\) step with the shift \(\alpha=a_{22}\)--which in this case means without shift, since \(a_{22}=0\). Show that the off-diagonal entries go from \(\sin\theta\) to \(-sin^{3}\theta\), which is _cubic convergence_.

\[A=\begin{bmatrix}\cos\theta&\sin\theta\\ \sin\theta&0\end{bmatrix}.\]
**10.** Check that the tridiagonal \(A=\left[\begin{smallmatrix}0&1\\ 1&0\end{smallmatrix}\right]\) is left unchanged by the \(QR\) algorithm. It is one of the (rare) counterexamples to convergence (so we shift).
**11.** Show by induction that, without shifts, \((Q_{0}Q_{1}\cdots Q_{k})(R_{k}\cdots R_{1}R_{0})\) is exactly the \(QR\) factorization of \(A_{k+1}\). This identity connects \(QR\) to the power method and leads to an explanation of its convergence. If \(|\lambda_{1}|>|\lambda_{2}|>\cdots>|\lambda_{n}|\), these eigenvalues will gradually appear on the main diagonal.
**12.** Choose \(\sin\theta\) and \(\cos\theta\) in the rotation \(P\) to triangularize \(A\), and find \(R\):

\[P_{21}A=\begin{bmatrix}\cos\theta&-\sin\theta\\ \sin\theta&\cos\theta\end{bmatrix}\begin{bmatrix}1&-1\\ 3&5\end{bmatrix}=\begin{bmatrix}*&*\\ 0&*\end{bmatrix}=R.\]
**13.** Choose \(\sin\theta\) and \(\cos\theta\) to make \(P_{21}AP_{21}^{-1}\) triangular (same \(A\)). What are the eigenvalues?
**14.** When \(A\) is multiplied by \(P_{ij}\) (plane rotation), which entries are changed? When \(P_{ij}A\) is multiplied on the right by \(P_{ij}^{-1}\), which entries are changed now?
**15.** How many multiplications and how many additions are used to compute \(PA\)? (A careful organization of all the rotations gives \(\frac{2}{3}n^{3}\) multiplications and additions, the same as for \(QR\) by reflectors and twice as many as for \(LU\).)
**16.** (Turning a robot hand) A robot produces any 3 by 3 rotation \(A\) from plane rotations around the \(x\), \(y\), and \(z\) axes. If \(P_{32}P_{31}P_{21}A=I\), the three robot turns are in \(A=P_{21}^{-1}P_{31}^{-1}P_{32}^{-1}\). The three angles are _Euler angles_. Choose the first \(\theta\) so that

\[P_{21}A=\begin{bmatrix}\cos\theta&-\sin\theta&0\\ \sin\theta&\cos\theta&0\\ 0&0&1\end{bmatrix}\frac{1}{2}\begin{bmatrix}-1&2&2\\ 2&-1&2\\ 2&2&-1\end{bmatrix}\quad\text{ is zero in the $(2,1)$ position}.\]

### Iterative Methods for \(Ax=b\)

In contrast to eigenvalues, for which there was no choice, we do not absolutely need an iterative method to solve \(Ax=b\). Gaussian elimination will reach the solution \(x\) in