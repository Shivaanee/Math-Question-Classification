instead of \(0\), \(F(x)\) and all derivatives are computed at \(x_{0}\). Then \(x\) changes to \(x-x_{0}\) on the right-hand side.

The next section contains the tests to decide whether \(x^{\rm T}Ax\) is positive (the bowl goes up from \(x=0\)). Equivalently, **the tests decide whether the matrix \(A\) is positive definite**--which is the main goal of the chapter.

### Problem Set 6.1

1. The quadratic \(f=x^{2}+4xy+2y^{2}\) has a saddle point at the origin, despite the fact that its coefficients are positive. Write \(f\) as a _difference of two squares_.
2. Decide for or against the positive definiteness of these matrices, and write out the corresponding \(f=x^{\rm T}Ax\): (a) \[\begin{bmatrix}1&3\\ 3&5\end{bmatrix}.\qquad\text{(b)}\quad\begin{bmatrix}1&-1\\ -1&1\end{bmatrix}.\qquad\text{(c)}\quad\begin{bmatrix}2&3\\ 3&5\end{bmatrix}.\qquad\text{(d)}\quad\begin{bmatrix}-1&2\\ 2&-8\end{bmatrix}.\] The determinant in (b) is zero; along what line is \(f(x,y)=0\)?
3. If a 2 by 2 symmetric matrix passes the tests \(a>0\), \(ac>b^{2}\), solve the quadratic equation \(\det(A-\lambda I)=0\) and show that both eigenvalues are positive.
4. Decide between a minimum, maximum, or saddle point for the following functions. 1. \(F=-1+4(e^{x}-x)-5x\sin y+6y^{2}\) at the point \(x=y=0\). 2. \(F=(x^{2}-2x)\cos y\), with stationary point at \(x=1\), \(y=\pi\).
5. 1. For which numbers \(b\) is the matrix \(A=\left[\begin{smallmatrix}1&b\\ b&9\end{smallmatrix}\right]\) positive definite? 2. Factor \(A=LDL^{\rm T}\) when \(b\) is in the range for positive definiteness. 3. Find the minimum value of \(\frac{1}{2}(x^{2}+2bxy+9y^{2})-y\) for \(b\) in this range. 4. What is the minimum if \(b=3\)?
6. Suppose the positive coefficients \(a\) and \(c\) dominate \(b\) in the sense that \(a+c>2b\). Find an example that has \(ac<b^{2}\), so the matrix is not positive definite.
7. 1. What 3 by 3 symmetric matrices \(A_{1}\) and \(A_{2}\) correspond to \(f_{1}\) and \(f_{2}\)? \[\begin{split}& f_{1}=x_{1}^{2}+x_{2}^{2}+x_{3}^{2}-2x_{1}x_{2}-2x_{1 }x_{3}+2x_{2}x_{3}\\ & f_{2}=x_{1}^{2}+2x_{2}^{2}+11x_{3}^{2}-2x_{1}x_{2}-2x_{1}x_{3}-4x _{2}x_{3}.\end{split}\] 2. Show that \(f_{1}\) is a _single_ perfect square and not positive definite. Where is \(f_{1}\) equal to 0? 3. Factor \(A_{2}\) into \(LL^{\rm T}\), Write \(f_{2}=x^{\rm T}A_{2}x\) as a sum of three squares.
8. If \(A=\left[\begin{smallmatrix}a&b\\ b&c\end{smallmatrix}\right]\) is positive definite, test \(A^{-1}=\left[\begin{smallmatrix}p&q\\ q&r\end{smallmatrix}\right]\) for positive definiteness.

 