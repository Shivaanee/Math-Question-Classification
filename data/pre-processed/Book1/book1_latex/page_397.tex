This looks just like finite differences! It has led to a thousand discussions about the relation between these two methods. More complicated finite elements--polynomials of higher degree. defined on triangles or quadrilaterals for partial differential equations--also produce sparse matrices \(A\). You could think of finite elements as a systematic way to construct accurate difference equations on irregular meshes. The essential thing is the _simplicity_ of these piecewise polynomials. Inside every element, their slopes arc easy to find and to integrate.

The components \(b_{j}\) on the right side are new. Instead of just the value of \(f\) at \(x_{j}\), as for finite differences, they are now an average of \(f\) around that point: \(b_{j}=\int V_{j}fdx\). Then, in step 3, we solve the tridiagonal system \(Ay=b\), which gives the coefficients in the minimizing trial function \(U=y_{1}V_{1}+\cdots+y_{n}V_{n}\). Connecting all these heights \(y_{j}\) by a broken line, we have the approximate solution \(U(x)\).

**Example 1**.: \(u^{\prime\prime}=2\) with \(u(0)=u(1)=0\), and solution \(u(x)=x-x^{2}\).

The approximation will use three intervals and two hat functions, with \(h=\frac{1}{3}\). The matrix \(A\) is 2 by 2. The right side requires integration of the hat function times \(f(x)=2\). That produces twice the area \(\frac{1}{3}\) under the hat:

\[A=3\begin{bmatrix}2&-1\\ -1&2\end{bmatrix}\qquad\text{and}\qquad b=\begin{bmatrix}\frac{2}{3}\\ \frac{2}{3}\end{bmatrix}.\]

The solution to \(Ay=b\) is \(y=(\frac{2}{9},\frac{2}{9})\). The best \(U(x)\) is \(\frac{2}{9}V_{1}+\frac{2}{9}V_{2}\), which equals \(\frac{2}{9}\) at the mesh points. _This agrees with the exact solution \(u(x)=x-x^{2}=\frac{1}{3}-\frac{1}{9}\)_.

In a more complicated example, the approximation will not be exact at the nodes. But it is remarkably close. The underlying theory is explained in the author's book _An Analysis of the Finite Element Method_ (see www.wellesleycambridge.com) written jointly with George Fix. Other books give more detailed applications, and the subject of finite elements has become an important part of engineering education. It is treated in _Introduction to Applied Mathematics_, and also in my new book _Applied Mathematics and Scientific Computing_. There we discuss partial differential equations, where the method really comes into its own.

### Eigenvalue Problems

The Rayleigh-Ritz idea--to minimize over a finite-dimensional family of \(V\)'s in place of all admissible \(v\)'s--is also useful for eigenvalue problems. The true minimum of the Rayleigh quotient is the fundamental frequency \(\lambda_{1}\). Its approximate minimum \(\Lambda_{1}\) will be larger--because the class of trial functions is restricted to the \(V\)'s. This step was completely natural and inevitable: to apply the new finite element ideas to this long-established variational form of the eigenvalue problem.

The best example of an eigenvalue problem has \(u(x)=\sin\pi x\) and \(\lambda_{1}=\pi^{2}\):

\[\text{{Eigenfunction}}\ u(x)\qquad-u^{\prime\prime}=\lambda u,\quad\text{with} \quad u(0)=u(1)=0.\] 