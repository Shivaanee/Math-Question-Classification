If we keep \(x^{\rm T}Ax=1\), then \(R(x)\) is a minimum when \(x^{\rm T}x=\|x\|^{2}\) is as large as possible. We are looking for the point on the ellipsoid \(x^{\rm T}Ax=1\) farthest from the origin--the vector \(x\) of greatest length. From our earlier description of the ellipsoid, its longest axis points along the first eigenvector. So \(R(x)\) is a minimum at \(x_{1}\).

Algebraically, we can diagonalize the symmetric \(A\) by an orthogonal matrix: \(Q^{\rm T}AQ=\Lambda\). Then set \(x=Qy\) and the quotient becomes simple:

\[R(x)=\frac{(Qy)^{\rm T}A(Qy)}{(Qy)^{\rm T}(Qy)}=\frac{y^{\rm T}\Lambda y}{y^{ \rm T}y}=\frac{\lambda_{1}y_{1}^{2}+\cdots+\lambda_{n}y_{n}^{2}}{y_{1}^{2}+ \cdots+y_{n}^{2}}.\] (11)

The minimum of \(R\) is \(\lambda_{1}\), at the point where \(y_{1}=1\) and \(y_{2}=\cdots=y_{n}=0\):

\[\mbox{\bf At all points}\qquad\lambda_{1}(y_{1}^{2}+y_{2}^{2}+\cdots+y_{n}^{2}) \leq(\lambda_{1}

 