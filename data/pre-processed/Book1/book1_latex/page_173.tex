equations can be written out as rows of \(A\) multiplying \(x\):

\[\begin{array}{l}\textbf{Every row is}\\ \textbf{orthogonal to }x\end{array}\qquad Ax=\begin{bmatrix}\cdots&\textbf{row 1}& \cdots\\ \cdots&\text{row 2}&\cdots\\ \vdots&\vdots&\vdots\\ \cdots&\text{row }m&\cdots\end{bmatrix}\begin{bmatrix}x_{1}\\ x_{2}\\ \vdots\\ x_{n}\end{bmatrix}=\begin{bmatrix}\textbf{0}\\ 0\\ \vdots\\ 0\end{bmatrix}.\] (6)

The main point is already in the first equation: _row \(1\) is orthogonal to \(x\)_. Their inner product is zero; that is equation 1. Every right-hand side is zero, so \(x\) is orthogonal to every row. Therefore \(x\) is orthogonal to every _combination_ of the rows. Each \(x\) in the nullspace is orthogonal to each vector in the row space, so \(\boldsymbol{N}(A)\bot\boldsymbol{C}(A^{\mathrm{T}})\).

The other pair of orthogonal subspaces comes from \(A^{\mathrm{T}}y=0\), or \(y^{\mathrm{T}}A=0\):

\[y^{\mathrm{T}}A=\begin{bmatrix}y_{1}&\cdots&y_{m}\end{bmatrix}\begin{bmatrix} \textbf{c}&&\textbf{c}\\ \textbf{0}&&\textbf{o}\\ \textbf{l}&&1&\\ \textbf{u}&\cdots&\text{u}\\ \textbf{m}&&\text{m}\\ \textbf{n}&&\text{n}\\ \textbf{1}&&n\end{bmatrix}=\begin{bmatrix}\textbf{0}&\cdots&0\end{bmatrix}.\] (7)

The vector \(y\) is orthogonal to every column. The equation says so, from the zeros on the right-hand side. Therefore \(y\) is orthogonal to every combination of the columns. It is orthogonal to the column space, and it is a typical vector in the left nullspace: \(\boldsymbol{N}(A^{\mathrm{T}})\bot\boldsymbol{C}(A)\). This is the same as the first half of the theorem, with \(A\) replaced by \(A^{\mathrm{T}}\). 

Second Proof.: The contrast with this "coordinate-free proof" should be useful to the reader. It shows a more "abstract" method of reasoning. I wish I knew which proof is clearer, and more permanently understood.

If \(x\) is in the nullspace then \(Ax=0\). If \(v\) is in the row space, it is a combination of the rows: \(v=A^{\mathrm{T}}z\) for some vector \(z\). Now, in one line:

\[\textbf{Nullspace }\bot\textbf{Row space}\qquad v^{\mathrm{T}}x=(A^{\mathrm{T}}z)^{ \mathrm{T}}x=z^{\mathrm{T}}Ax=z^{\mathrm{T}}0=0.\] (8)

**Example 3**.: Suppose \(A\) has rank 1, so its row space and column space are lines:

\[\textbf{Rank-1 matrix}\qquad A=\begin{bmatrix}1&3\\ 2&6\\ 3&9\end{bmatrix}.\]

The rows are multiples of \((1,3)\). The nullspace contains \(x=(-3,1)\), which is orthogonal to all the rows. The nullspace and row space are perpendicular lines in \(\mathbf{R}^{2}\):

\[\begin{bmatrix}1&3\end{bmatrix}\begin{bmatrix}3\\ -1\end{bmatrix}=0\quad\text{and}\quad\begin{bmatrix}2&6\end{bmatrix} \begin{bmatrix}3\\ -1\end{bmatrix}=0\quad\text{and}\quad\begin{bmatrix}3&9\end{bmatrix} \begin{bmatrix}3\\ -1\end{bmatrix}=0.\] 