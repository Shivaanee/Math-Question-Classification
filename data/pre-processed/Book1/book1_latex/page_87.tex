

## Chapter 2 Vector Spaces

### 2.1 Vector Spaces and Subspaces

Elimination can simplify, one entry at a time, the linear system \(Ax=b\). Fortunately it also simplifies the theory. The basic questions of _existence_ and _uniqueness_--Is there one solution, or no solution, or an infinity of solutions?--are much easier to answer after elimination, We need to devote one more section to those questions, to find every solution for an \(m\) by \(n\) system. Then that circle of ideas will be complete.

But elimination produces only one kind of understanding of \(Ax=b\). Our chief object is to achieve a different and deeper understanding. This chapter may be more difficult than the first one. It goes to the heart of linear algebra.

For the concept of a _vector space_, we start immediately with the most important spaces. They are denoted by \(\mathbf{R}^{1},\mathbf{R}^{2},\mathbf{R}^{3},\ldots\); the space \(\mathbf{R}^{n}\) consists of _all column vectors with \(n\) components_. (We write \(\mathbf{R}\) because the components are real numbers.) \(\mathbf{R}^{2}\) is represented by the usual \(x\)-\(y\) plane; the two components of the vector become the \(x\) and \(y\) coordinates of the corresponding point. The three components of a vector in \(\mathbf{R}^{3}\) give a point in three-dimensional space. The one-dimensional space \(\mathbf{R}^{1}\) is a line.

The valuable thing for linear algebra is that the extension to \(n\) dimensions is so straightforward. For a vector in \(\mathbf{R}^{7}\) we just need the seven components, even if the geometry is hard to visualize. Within all vector spaces, two