We use parentheses and commas when the components are listed horizontally, and square brackets (with no commas) when a column vector is printed vertically. What really matters is _addition of vectors_ and _multiplication by a scalar_ (a number). In Figure 1.4a you see a vector addition, component by component:

\[\textbf{Vector addition}\qquad\begin{bmatrix}5\\ 0\\ 0\end{bmatrix}+\begin{bmatrix}0\\ -2\\ 0\end{bmatrix}+\begin{bmatrix}0\\ 0\\ 9\end{bmatrix}=\begin{bmatrix}5\\ -2\\ 9\end{bmatrix}.\]

In the right-hand figure there is a multiplication by \(2\) (and if it had been \(-2\) the vector would have gone in the reverse direction):

\[\textbf{Multiplication by scalars}\qquad 2\begin{bmatrix}1\\ 0\\ 2\end{bmatrix}=\begin{bmatrix}2\\ 0\\ 4\end{bmatrix},\quad-2\begin{bmatrix}1\\ 0\\ 2\end{bmatrix}=\begin{bmatrix}-2\\ 0\\ -4\end{bmatrix}.\]

Also in the right-hand figure is one of the central ideas of linear algebra. It uses _both_ of the basic operations; vectors are _multiplied by numbers and then added_. The result is called a _linear combination_, and this combination solves our equation:

\[\textbf{Linear combination}\qquad\textbf{1}\begin{bmatrix}2\\ 4\\ -2\end{bmatrix}+\textbf{1}\begin{bmatrix}1\\ -6\\ 7\end{bmatrix}+\textbf{2}\begin{bmatrix}1\\ 0\\ 2\end{bmatrix}=\begin{bmatrix}5\\ -2\\ 9\end{bmatrix}.\]

Equation (2) asked for multipliers \(u\), \(v\), \(w\) that produce the right side \(b\). Those numbers are \(u=\textbf{1}\), \(v=\textbf{1}\), \(w=\textbf{2}\). They give the correct combination of the columns. They also gave the point \((1,1,2)\) in the row picture (where the three planes intersect).

Figure 1.4: The column picture: linear combination of columns equals \(b\).

 