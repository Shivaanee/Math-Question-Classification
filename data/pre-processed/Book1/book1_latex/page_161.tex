

**18.**: In the vector space \(P_{3}\) of all \(p(x)=a_{0}+a_{1}x+a_{2}x^{2}+a_{3}x^{3}\), let \({\bf S}\) be the subset of polynomials with \(\int_{0}^{1}p(x)dx=0\). Verify that \({\bf S}\) is a subspace and find a basis.
**19.**: A _nonlinear_ transformation is invertible if \(T(x)=b\) has exactly one solution for every \(b\). The example \(T(x)=x^{2}\) is not invertible because \(x^{2}=b\) has two solutions for positive \(b\) and no solution for negative \(b\). Which of the following transformations (from the real numbers \({\bf R}^{1}\) to the real numbers \({\bf R}^{1}\)) are invertible? None are linear, not even (c).

\[\begin{array}{llll}\mbox{(a)}&T(x)=x^{3}.&\mbox{(b)}&T(x)=e^{x}.\\ \mbox{(c)}&T(x)=x+11.&\mbox{(d)}&T(x)=\cos x.\end{array}\]
**20.**: What is the axis and the rotation angle for the transformation that takes \((x_{1},x_{2},x_{3})\) into \((x_{2},x_{3},x_{1})\)?
**21.**: A linear transformation must leave the zero vector fixed: \(T(0)=0\). Prove this from \(T(v+w)=T(v)+T(w)\) by choosing \(w=\). Prove it also from the requirement \(T(cv)=cT(v)\) by choosing \(c=\).
**22.**: Which of these transformations is not linear? The input is \(v=(v_{1},v_{2})\).

\[\begin{array}{llll}\mbox{(a)}&T(v)=(v_{2},v_{1}).&\mbox{(b)}&T(v)=(v_{1},v_{ 1}).\\ \mbox{(c)}&T(v)=(0,v_{1}).&\mbox{(d)}&T(v)=(0,1).\end{array}\]
**23.**: If \(S\) and \(T\) are linear with \(S(v)=T(v)=v\), then \(S