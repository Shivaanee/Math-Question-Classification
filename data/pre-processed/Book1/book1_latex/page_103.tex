2. With free variables \(=0\), find a particular solution to \(Ax_{p}=b\) and \(Ux_{p}=c\).
3. Find the special solutions to \(Ax=0\) (or \(Ux=0\) or \(Rx=0\)). Each free variable, in turn, is 1. Then \(x=x_{p}+\) (any combination \(x_{n}\) of special solutions).

When the equation was \(Ax=0\), the particular solution was the zero vector! It fits the pattern, but \(x_{\text{particular}}=0\) was not written in equation (2). Now \(x_{p}\) is added to the nullspace solutions, as in equation (4).

Question: How does the reduced form \(R\) make this solution even clearer? You will see it in our example. Subtract equation 2 from equation 1, and then divide equation 2 by its pivot. On the left-hand side, this produces \(R\), as before. On the right-hand side, these operations change \(c=(1,3,0)\) to a new vector \(d=(-2,1,0)\):

\[\begin{array}{c}\text{\bf Reduced equation}\\ Rx=d\end{array}\qquad\begin{bmatrix}1&3&0&-1\\ 0&0&1&1\\ 0&0&0&0\end{bmatrix}\begin{bmatrix}u\\ v\\ w\\ y\end{bmatrix}=\begin{bmatrix}-2\\ 1\\ 0\end{bmatrix}.\] (5)

Our particular solution \(x_{p}\), (one choice out of many) has free variables \(v=y=0\). Columns 2 and 4 can be ignored. Then we immediately have \(u=-2\) and \(w=1\), exactly as in equation (4). **The entries of \(d\) go directly into \(x_{p}\)**. This is because the identity matrix is sitting in the pivot columns of \(R\)!

Let me summarize this section, before working a new example. Elimination reveals the pivot variables and free variables. _If there are \(r\) pivots, there are \(r\) pivot variables and \(n-r\) free variables_. That important number \(r\) will be given a name--it is the _rank of the matrix_.

**2D** Suppose elimination reduces \(Ax=b\) to \(Ux=c\) and \(Rx=d\), with \(r\) pivot rows and \(r\) pivot columns. **The rank of those matrices is \(r\)**. The last \(m-r\) rows of \(U\) and \(R\) are zero, so there is a solution only if the last \(m-r\) entries of \(c\) and \(d\) are also zero.

The complete solution is \(x=x_{p}+x_{n}\). One particular solution \(x_{p}\) has all free variables zero. Its pivot variables are the first \(r\) entries of \(d\), so \(Rx_{p}=d\).

The nullspace solutions \(x_{n}\) are combinations of \(n-r\) special solutions, with one free variable equal to 1. The pivot variables in that special solution can be found in the corresponding column of \(R\) (with sign reversed).

You see how the rank \(r\) is crucial. It counts the pivot rows in the "row space" and the pivot columns in the column space. There are \(n-r\) special solutions in the nullspace. There are \(m-r\) solvability conditions on \(b\) or \(c\) or \(d\).

 