

**5.**: (a) \(\begin{bmatrix}1&2&3\\ 3&1&2\\ 2&3&1\end{bmatrix}\to\begin{bmatrix}1&2&3\\ 0&-5&-7\\ 0&-1&-5\end{bmatrix}\)

\[\to\begin{bmatrix}1&2&3\\ 0&-5&-7\\ 0&0&-18/5\end{bmatrix}\!\!:\begin{array}{l}\text{invertible}\Rightarrow\text {independent columns}\\ \text{(could have used rows).}\end{array}\] (b) \(\begin{bmatrix}1&2&-3\\ -3&1&2\\ 2&-3&1\end{bmatrix}\to\begin{bmatrix}1&2&-3\\ 0&7&-7\\ 0&0&0\end{bmatrix}\!\!;\begin{array}{l}1\\ 1\end{bmatrix}=\begin{bmatrix}0\\ 0\\ 0\end{bmatrix}\!\!,\quad\text{columns add to 0}\\ \text{(could use rows).}\end{array}\)
**7.**: The sum \(v_{1}-v_{2}+v_{3}=0\) because \((w_{2}-w_{3})-(w_{1}-w_{3})+(w_{1}-w_{2})=0\).
**9.**: (a) The four vectors are the columns of a 3 by 4 matrix \(A\) with at least one free variable, so \(Ax=0\). (b) Dependent if \([v_{1}\;\;\;v_{2}]\) has rank 0 or 1. (c) \(0v_{1}+c(0,0,0)=0\) has a nonzero solution (take any \(c\neq 0\)).
**11.**: (a) Line in \(\mathbf{R}^{3}\). (b) Plane in \(\mathbf{R}^{3}\). (c) Plane in \(\mathbf{R}^{3}\). (d) All of \(\mathbf{R}^{3}\).
**13.**: All dimensions are 2. The row spaces of \(A\) and \(U\) are the same.
**15.**: \(v=\frac{1}{2}(v+w)+\frac{1}{2}(v\!\!\leftarrow\!w)\) and \(w=\frac{1}{2}(v+w)-\frac{1}{2}(v-w)\). The two pairs _span_ the same space. They are a basis when \(v\) and \(w\) are _independent_.
**17.**: If elimination produces one or more zero rows, the rows of \(A\) are linearly dependent;

for example in Problem 16 \(\begin{bmatrix}1&1&0&0\\ 1&0&1&0\\ 0&0&1&1\\ 0&1&0&1\end{bmatrix}\to\begin{bmatrix}1&1&0&0\\ 0&-1&1&0\\ 0&0&1&1\\ 0&0&1&1\end{bmatrix}\)

\[\to\begin{bmatrix}1&1&0&0\\ 0&-1&1&0\\ 0&0&1&1\\ 0&0&0&0\end{bmatrix}\!\!.\]
**19.**: The \(n\) independent vectors span a space of dimension \(n\). They are a _basis_ for that space. If they are the columns of \(A\) then \(m\) is _not less_ than \(n\) (\(m\geq n\)).
**21.**: \(\boldsymbol{C}(U)\): Any bases for \(\mathbf{R}^{2}\); \(\boldsymbol{N}(U)\): (row 1 and row 2) or (row 1 and row 1 + row 2).
**23.**: Independent columns \(\Rightarrow\) rank \(n\). Columns span \(\mathbf{R}^{m}\Rightarrow\) rank \(m\). Columns are basis for \(\mathbf{R}^{m}\Rightarrow\) rank \(=m=n\).
**25.**: (a) The only solution is \(x=0\) because _the columns are independent._ (b) \(Ax=b\) is solvable because _the columns span_\(\mathbf{R}^{5}\).
**27.**: Columns 1 and 2 are bases for the (different) column spaces of \(A\) and \(U\); rows 1 and 2 are bases for the (equal) row spaces; \((1,-1,1)\) is a basis for the (equal) nullspaces.
**29.**: rank\((A)=2\) if \(c=0\) and \(d=2\); rank\((B)=2\) except when \(c=d\) or \(c=-d\).
**31.**: Let \(v_{1}=(1,0,0,0)\), ..., \(v_{4}=(0,0,0,1)\) be the coordinate vectors. If \(\mathbf{W}\) is the line through \((1,2,3,4)\), none of the \(v\)'s are in \(\mathbf{W}\).
**33.**: (a) If it were not a basis, we could add more independent vectors, which would exceed the given dimension \(k\). (b) If it were not a basis, we could delete some vectors, leaving less than the given dimension \(k\).

