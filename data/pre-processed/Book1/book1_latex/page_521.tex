

**7.**: 1. \(A_{1}=\begin{bmatrix}1&-1&-1\\ -1&1&1\\ -1&1&1\end{bmatrix}\) and \(A_{2}=\begin{bmatrix}1&-1&-1\\ -1&2&-2\\ -1&-2&11\end{bmatrix}\).
2. \(f_{1}=(x_{1}-x_{2}-x_{3})^{2}=0\) when \(x_{1}-x_{2}-x_{3}=0\).
3. \(f_{2}=(x_{1}-x_{2}-x_{3})^{2}+(x_{2}-3x_{3})^{2}+x_{3}^{2}\); \(L=\begin{bmatrix}1&0&0\\ -1&1&0\\ -1&-3&1\end{bmatrix}\).
**9.**: \(A=\begin{bmatrix}3&6\\ 6&16\end{bmatrix}=\begin{bmatrix}1&0\\ 2&1\end{bmatrix}\begin{bmatrix}3&0\\ 0&4\end{bmatrix}\begin{bmatrix}1&2\\ 0&1\end{bmatrix}\); the coefficients of the squares are the pivots in \(D\), whereas the coefficients inside the squares are columns of \(L\).
**11.**: 1. Pivots are \(a\) and \(c-|b|^{2}/a\) and \(\det A=ac-|b|^{2}\). 2. Multiply \(|x_{2}|^{2}\) by \((c-|b|^{2}/a)\). 2. Now \(x^{\rm H}Ax\) is a sum of squares. 2. \(\det=-1\) (indefinite) and \(\det=+1\) (positive definite).
**13.**: \(a>1\) and \((a-1)(c-1)>b^{2}\). This means that \(A-I\) is positive definite.
**15.**: \(f(x,\,y)=x^{2}+4xy+9y^{2}=(x+2y)^{2}+5y^{2}\); \(f(x,\,y)=x^{2}+6xy+9y^{2}=(x+3y)^{2}\).
**17.**: \(x^{\rm T}A^{\rm T}Ax=(Ax)^{\rm T}(Ax)=\) length squared\(=0\) only if \(\dot{A}x=0\). Since \(A\) has independent columns, this only happens when \(x=0\).
**19.**: \(A=\begin{bmatrix}4&-4&8\\ -4&4&-8\\ 8&-8&16\end{bmatrix}\) has only one pivot\(=4\), rank\(=1\), eigenvalues \(24\), \(0\), \(0\), \(\det A=0\).
**21.**: \(ax^{2}+2bxy+cy^{2}\) has a saddle point at \((0,\,0)\) if \(ac<b^{2}\). The matrix is _indefinite_ (\(\lambda<0\) and \(\lambda>0\)).

**Problem Set 6.2, page 326**.:

1. \(A\) is positive definite for \(a>2\). \(B\) is never positive definite: notice \(\begin{bmatrix}1&4\\ 4&7\end{bmatrix}\).
2. \(\det A=-2b^{3}-3b^{2}+1\) is negative at (and near) \(b=\frac{2}{3}\).
3. If \(x^{\rm T}Ax>0\) and \(x^{\rm T}Bx>0\) for any \(x\neq 0\), then \(x^{\rm T}(A+B)x>0\); condition (I).
4. Positive \(\lambda\)'s because \(R\) is symmetric and \(\sqrt{\Lambda}>0\). \(R=\begin{bmatrix}3&1\\ 1&3\end{bmatrix}\); \(R=\begin{bmatrix}3&-1\\ -1&3\end{bmatrix}\).
5. \(|x^{\rm T}Ay|^{2}=|x^{\rm T}R^{\rm T}Ry|^{2}=|(Rx)^{\rm T}Ry|^{2}\leq(\)by the ordinary Schwarz inequality) \(\|Rx\|^{2}\|Ry\|^{2}=(x^{\rm T}R^{\rm T}Rx)(y^{\rm T}R^{\rm T}Ry)=(x^{\rm T}Ax) (y^{\rm T}Ay)\).
**11.**: \(A=\begin{bmatrix}3&-\sqrt{2}\\ -\sqrt{2}&2\end{bmatrix}\) has \(\lambda=1\) and \(4\), axes \(1\begin{bmatrix}1\\ \sqrt{2}\end{bmatrix}\) and \(\frac{1}{2}\begin{bmatrix}\sqrt{2}\\ -1\end{bmatrix}\) along eigenvectors.
**13.**: _Negative definite matrices_: 1. \(x^{\rm T}Ax<0\) for all nonzero vectors \(x\). 2. All the eigenvalues of \(A\) satisfy \(\lambda_{i}<0\). 3. 4. \(\det A_{1}<0\), \(\det A_{2}>0\), \(\det A_{3}<0\)