\(A^{\mathrm{T}}A\) **has the same nullspace as \(A\).**

Certainly if \(Ax=0\) then \(A^{\mathrm{T}}Ax=0\). Vectors \(x\) in the nullspace of \(A\) are also in the nullspace of \(A^{\mathrm{T}}A\). To go in the other direction, start by supposing that \(A^{\mathrm{T}}Ax=0\), and take the inner product with \(x\) to show that \(Ax=0\):

\[x^{\mathrm{T}}A^{\mathrm{T}}Ax=0,\quad\text{or}\quad\|Ax\|^{2}=0,\quad\text{or }\quad Ax=0.\]

The two nullspaces are identical. In particular, if \(A\) has independent columns (and only \(x=0\) is in its nullspace), then the same is true for \(A^{\mathrm{T}}A\):

**3M** If \(A\) has independent columns, then \(A^{\mathrm{T}}A\) is _square, symmetric_, and _invertible_.

We show later that \(A^{\mathrm{T}}A\) is also positive definite (all pivots and eigenvalues are positive).

This case is by far the most common and most important. Independence is not so hard in \(m\)-dimensional space if \(m>n\). We assume it in what follows.

### Projection Matrices

We have shown that the closest point to \(b\) is \(p=A(A^{\mathrm{T}}A)^{-1}A^{\mathrm{T}}b\). _This formula expresses in matrix terms the construction of a perpendicular line from \(b\) to the column space of \(A\)._ The matrix that gives \(p\) is a projection matrix, denoted by \(P\):

\[\text{{Projection matrix}}\qquad P=A(A^{\mathrm{T}}A)^{-1}A^{\mathrm{T}}.\] (4)

This matrix projects any vector \(\mathsf{b}\) 