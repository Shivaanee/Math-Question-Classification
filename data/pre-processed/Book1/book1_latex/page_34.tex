One subscript is enough for a vector. The \(j\)th component of \(x\) is denoted by \(x_{j}\). (The multiplication above had \(x_{1}=2\), \(x_{2}=5\), \(x_{3}=0\).) Normally \(x\) is written as a column vector--like an \(n\) by 1 matrix. But sometimes it is printed on a line, as in \(x=(2,5,0)\). The parentheses and commas emphasize that it is not a 1 by 3 matrix. It is a column vector, and it is just temporarily lying down.

To describe the product \(Ax\), we use the "_sigma_" symbol \(\Sigma\) for summation:

\[\text{{Sigma notation}}\qquad\text{The $i$th component of $Ax$ is }\sum_{j=1}^{n}a_{ij}x_{j}.\]

This sum takes us along the \(i\)th row of \(A\). The column index \(j\) takes each value from 1 to \(n\) and we add up the results--the sum is \(a_{i1}x_{1}+a_{i2}x_{2}+\cdots+a_{in}x_{n}\).

We see again that the length of the rows (the number of columns in \(A\)) must match the length of \(x\). **An \(m\) by \(n\) matrix multiplies an \(n\)-dimensional vector** (and produces an \(m\)-dimensional vector). Summations are simpler than writing everything out in full, but matrix notation is better. (Einstein used "tensor notation," in which a repeated index automatically means summation. He wrote \(a_{ij}x_{j}\) or even \(a_{i}^{j}x_{j}\), without the \(\Sigma\). Not being Einstein, we keep the \(\Sigma\).)

### The Matrix Form of One Elimination Step

So far we have a convenient shorthand \(Ax=b\) for the original system of equations. What about the operations that are carried out during elimination? In our example, the first step subtracted 2 times the first equation from the second. On the right-hand side, 2 times the first component of \(b\) was subtracted from the second component. _The same result is achieved if we multiply \(b\) by this elementary matrix_ (or _elimination matrix_):

\[\text{{Elementary matrix}}\qquad E=\begin{bmatrix}1&0&0\\ -\mathbf{2}&1&0\\ 0&0&1\end{bmatrix}.\]

This is verified just by obeying the rule for multiplying a matrix and a vector:

\[Eb=\begin{bmatrix}1&0&0\\ -\mathbf{2}&1&0\\ 0&0&1\end{bmatrix}\begin{bmatrix}5\\ -2\\ 9\end{bmatrix}=\begin{bmatrix}5\\ -\mathbf{12}\\ 9\end{bmatrix}.\]

The components 5 and 9 stay the same (because of the 1, 0, 0 and 0, 0, 1 in the rows of \(E\)). The new second component \(-12\) appeared after the first elimination step.

It is easy to describe the matrices like \(E\), which carry out the separate elimination steps. We also notice the "identity matrix," which does nothing at all.

1BThe _identity matrix I_, with 1s on the diagonal and 0s everywhere else, leaves every vector unchanged. The _elementary matrix_\(E_{ij}\) subtracts \(\ell\) times 