2. If you know \(\lambda\) is an eigenvalue, the way to find \(x\) is to .

**24.**: What do you do to \(Ax=\lambda x\), in order to prove (a), (b), and (c)?

1. \(\lambda^{2}\) is an eigenvalue of \(A^{2}\), as in Problem 22.
2. \(\lambda^{-1}\) is an eigenvalue of \(A^{-1}\), as in Problem 21.
3. \(\lambda+1\) is an eigenvalue of \(A+I\), as in Problem 20.

**25.**: From the unit vector \(u=\left(\frac{1}{6},\frac{1}{6},\frac{3}{6},\frac{5}{6}\right)\), construct the rank-1 projection matrix \(P=uu^{\mathrm{T}}\).

1. Show that \(Pu=u\). Then \(u\) is an eigenvector with \(\lambda=1\).
2. If \(v\) is perpendicular to \(u\) show that \(Pv=\) zero vector. Then \(\lambda=0\).
3. Find three independent eigenvectors of \(P\) all with eigenvalue \(\lambda=0\).

**26.**: Solve \(\det(Q-\lambda I)=0\) by the quadratic formula, to reach \(\lambda=\cos\theta\pm i\sin\theta\):

\[Q=\begin{bmatrix}\cos\theta&-\sin\theta\\ \sin\theta&\cos\theta\end{bmatrix}\quad\text{rotates the $xy$-plane by the angle $\theta$.}\]

Find the eigenvectors of \(Q\) by solving \((Q-\lambda I)x=0\). Use \(i^{2}=-1\).

**27.**: Every permutation matrix leaves \(x=(1,1,\ldots,1)\) unchanged. Then \(\lambda=1\). Find two more \(\lambda\)'s for these permutations:

\[P=\begin{bmatrix}0&1&0\\ 0&0&1\\ 1&0&0\end{bmatrix}\qquad\text{and}\qquad P=\begin{bmatrix}0&0&1\\ 0&1&0\\ 1&0&0\end{bmatrix}.\]

**28.**: If \(A\) has \(\lambda_{1}=4\) and \(\lambda_{2}=5\), then \(\det(A-\lambda I)=(\lambda-4)(\lambda-5)=\lambda^{2}-9\lambda+20\). Find three matrices that have trace \(a+d=9\), determinant 20, and \(\lambda=4,5\).

**29.**: A 3 by 3 matrix \(B\) is known to have eigenvalues 0, 1, 2, This information is enough to find three of these:

1. the rank of \(B\),
2. the determinant of \(B^{\mathrm{T}}B\),
3. the eigenvalues of \(B^{\mathrm{T}}B\), and
4. the eigenvalues of \((B+I)^{-1}\).

**30.**: Choose the second row of \(A=[\begin{smallmatrix}0&1\\ *&*\end{smallmatrix}]\) so that \(A\) has eigenvalues 4 and 7.

**31.**: Choose \(a\), \(b\), \(c\), so that \(\det(A-\lambda I)=9\lambda-\lambda^{3}\). Then the eigenvalues are \(-3\), 0, 3:

\[A=\begin{bmatrix}0&1&0\\ 0&0&1\\ a&b&c\end{bmatrix}.\] 