
31. If \(c=1\), \(R=\begin{bmatrix}1&1&2&2\\ 0&0&0&0\\ 0&0&0&0\end{bmatrix}\) has \(x_{2},x_{3},x_{4}\) free. If \(c\neq 1\), \(R=\begin{bmatrix}1&0&2&2\\ 0&1&0&0\\ 0&0&0&0\end{bmatrix}\) has \(x_{3}\), \(x_{4}\) free. Special solutions in \(N=\begin{bmatrix}-1&-2&-2\\ 1&0&0\\ 0&1&0\\ 0&0&1\end{bmatrix}\) (\(c=1\)) and \(N=\begin{bmatrix}-2&-2\\ 0&0\\ 1&0\\ 0&1\end{bmatrix}\) (\(c\neq 1\)). If \(c=1\), \(R=\begin{bmatrix}0&1\\ 0&0\end{bmatrix}\) has \(x_{1}\) free; if \(c=2\), \(R=\begin{bmatrix}1&-2\\ 0&0\end{bmatrix}\) has \(x_{2}\) free; \(R=I\) if \(c\neq 1,2\). Special-solutions in \(N=\begin{bmatrix}1\\ 0\end{bmatrix}\) (\(c=1\)) or \(N=\begin{bmatrix}2\\ 1\end{bmatrix}\) (\(c=2\)) or \(N=2\) by \(0\)_empty matrix_.
33. \(x_{\text{complete}}=\begin{bmatrix}-2\\ 0\\ 1\end{bmatrix}+x_{2}\begin{bmatrix}-3\\ 1\\ 0\end{bmatrix};x_{\text{complete}}=\begin{bmatrix}1/2\\ 0\\ 1/2\\ 0\end{bmatrix}+x_{2}\begin{bmatrix}-3\\ 1\\ 0\end{bmatrix}+x_{4}\begin{bmatrix}0\\ 0\\ -2\\ 1\end{bmatrix}\).
35. (a) Solvable if \(b_{2}=2b_{1}\) and \(3b_{1}-3b_{3}+b_{4}=0\). Then \(x=\begin{bmatrix}5b_{1}-2b_{3}\\ b_{3}-2b_{1}\end{bmatrix}\) (no free variables). (b) Solvable if \(b_{2}=2b_{1}\) and \(3b_{1}-3b_{3}+b_{4}=0\). Then \(x=\begin{bmatrix}5b_{1}-2b_{3}\\ b_{3}-2b_{1}\\ 0\end{bmatrix}+x_{3}\begin{bmatrix}-1\\ -1\\ 1\end{bmatrix}\).
37. A 1 by 3 system has at least two free variables.
39. (a) The particular solution \(x_{p}\) is always multiplied by \(1\). (b) Any solution can be \(x_{p}\). (c) \(\begin{bmatrix}3&3\\ 3&3\end{bmatrix}\)\(\begin{bmatrix}x\\ y\end{bmatrix}=\begin{bmatrix}6\\ 6\end{bmatrix}\). Then \(\begin{bmatrix}1\\ 1\end{bmatrix}\) is shorter (length \(\sqrt{2}\)) than \(\begin{bmatrix}2\\ 0\end{bmatrix}\). (d) The "homogeneous" solution in the nullspace is \(x_{n}=0\) when \(A\) is invertible.
41. Multiply \(x_{p}\) by \(2\), same \(x_{n}\); \(\begin{bmatrix}x\\ X\end{bmatrix}_{p}\) is \(\begin{bmatrix}x_{p}\\ 0\end{bmatrix}\), special solutions also include the columns of \(\begin{bmatrix}-I\\ I\end{bmatrix}\); \(x_{p}\) and the special solutions are not changed.
43. For \(A\), \(q=3\) gives rank \(1\), every other \(q\) gives rank \(2\). For \(B\), \(q=6\) gives rank \(1\), every other \(q\) gives rank \(2\).
45. (a) \(r<m\), always \(r\leq n\). (b) \(r=m\), \(r<n\). (c) \(r<m\), \(r=n\). (d) \(r=m=n\).
47. \(R=\begin{bmatrix}1& 