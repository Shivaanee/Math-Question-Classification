constant. (c) \(\lambda=0\) and \(\pm(\sqrt{a^{2}+b^{2}+c^{2}})i\). Skew-symmetric matrices have pure imaginary \(\lambda\)'s.
* \(u(t)=\frac{1}{2}\cos 2t\left[\begin{matrix}1\\ -1\end{matrix}\right]+\frac{1}{2}\cos\sqrt{6}t\left[\begin{matrix}1\\ 1\end{matrix}\right]\).
* \(Ax=\lambda Fx+\lambda^{2}x\) or \((A-\lambda F-\lambda^{2}I)x=0\).
* Eigenvalues are real when \((\text{trace})^{2}-4\det\geq 0\Rightarrow-4(-a^{2}-b^{2}+c^{2})\geq 0\Rightarrow a ^{2}+b^{2}\geq c^{2}\).
* \(u_{1}=e^{4t}\left[\begin{matrix}1\\ 0\end{matrix}\right]\), \(u_{2}=e^{t}\left[\begin{matrix}1\\ -1\end{matrix}\right]\). If \(u(0)=(5,-2)\), then \(u(t)=3e^{4t}\left[\begin{matrix}1\\ 0\end{matrix}\right]+2e^{t}\left[\begin{matrix}1\\ -1\end{matrix}\right]\).
* \(\left[\begin{matrix}y^{\prime}\\ y^{\prime\prime}\end{matrix}\right]=\left[\begin{matrix}0&1\\ 4&5\end{matrix}\right]\left[\begin{matrix}y\\ y^{\prime}\end{matrix}\right]\). Then \(\lambda=\frac{1}{2}(5\pm\sqrt{41})\).
* \(\lambda_{1}=0\) and \(\lambda_{2}=2\). Now \(v(t)=20+10e^{2t}\rightarrow\infty\) as \(t\rightarrow\infty\).
* \(A=\left[\begin{matrix}0&1\\ -9&6\end{matrix}\right]\) has trace \(6\), \(\det 9,\lambda=3\) and \(3\), with only one independent eigenvector (1, 3). That gives \(y=ce^{3t}\), \(y^{\prime}=3e^{3t}\). Also \(te^{3t}\) solves \(y^{\prime\prime}=6y^{\prime}-9y\).
* \(y(t)=\cos t\) starts at \(y(0)=1\) and \(y^{\prime}(0)=0\). The vector equation has \(u=(y,\,y^{\prime})=(\cos t,\,-\sin t)\).
* Substituting \(u=e^{ct}v\) gives \(ce^{ct}v=Ae^{ct}v-e^{ct}b\), or \((A-cI)v=b\), or \(v=(A-cI)^{-1}b=\) particular solution. If \(c\) is an eigenvalue, then \(A-cI\) is not invertible: this \(v\) fails.
* \(de^{At}/dt=A+A^{2}t+\frac{1}{2}A^{3}t^{2}+\frac{1}{6}A^{4}t^{3}+\cdots=A(I+At+ \frac{1}{2}A^{2}t^{2}+\frac{1}{6}A^{3}t^{3}+\cdots)=Ae^{At}\).
* The solution at time \(t+T\) is also \(e^{A(t+T)}u(0)\). Thus \(e^{At}\) times \(e^{AT}\) equals \(e^{A(t+T)}\).
* If \(A^{2}=A\) then \(e^{At}=I+At+\frac{1}{2}At^{2}+\frac{1}{6}At^{3}+\cdots=I+(e^{t}-1)A\) \(=\left[\begin{matrix}1&0\\ 0&1\end{matrix}\right]+\left[\begin{matrix}e^{t}-1&e^{t}-1\\ 0&0\end{matrix}\right]=\left[\begin{matrix}e^{t}&e^{t}-1\\ 0&1\end{matrix}\right]\).
* \(A=\left[\begin{matrix}1&1\\ 0&3\end{matrix}\right]=\left[\begin{matrix}1&1\\ 2&0\end{matrix}\right]\left[\begin{matrix}3&0\\ 0&1\end{matrix}\right]\left[\begin{matrix}0&\frac{1}{2}\\ 1&-\frac{1}{2}\end{matrix}\right]\), then \(e^{At}=\left[\begin{matrix}e^{t}&\frac{1}{2}(e^{3t}-e^{t})\\ 0&e^{3t}\end{matrix}\right]=I\) at \(t=0\).
* The inverse of \(e^{At}\) is \(e^{-At}\). (b) If \(Ax=\lambda x\) then \(e^{At}x=e^{\lambda t}x\) and \(e^{\lambda t}\neq 0\).
* \(\lambda=2\) and \(5\) with eigenvectors \(\left[\begin{matrix}2\\ 1\end{matrix}\right]\) and \(\left[\begin{matrix}1\\ 1\end{matrix}\right]\). Then \(A=S\Lambda S^{-1}=\left[\begin{matrix}-1&6\\ -3&8\end{matrix}\right]\).

### Problem Set 5.5, page 288

1. (b) sum \(=4+3i\); product \(=7+i\). (c) \(\overline{3+4i}=3-4i\); \(\overline{1-i}=1+i\); \(|3+4i|=5\); \(|1-i|=\sqrt{2}\). Both numbers lie _outside_ the unit circle.
2. \(\overline{x}=2-i\), \(x\overline{x}=5\), \(xy=-1+7i\), \(1/x=2/5-(1/5)i\), \(x/y=1/2-(1/2)i\); check that \(|xy|=\sqrt{50}=|x||y|\) and \(|1/x|=1/\sqrt{5}=1/|x|\).
3. (a) \(x^{2}=r^{2}e^{i2\theta}\), \(x^{-1}=(1/r)e^{-i\theta}\), \(\overline{x}=re^{-i\theta}\); \(x^{-1}=\overline{x}\) gives \(|x|^{2}=1\): on the unit circle.

 