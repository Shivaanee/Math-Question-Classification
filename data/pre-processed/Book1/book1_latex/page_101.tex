There must be infinitely many solutions, since any multiple \(cx\) will also satisfy \(A(cx)=0\). The nullspace contains the line through \(x\). And if there are additional free variables, the nullspace becomes more than just a line in \(n\)-dimensional space. _The nullspace has the same "dimension" as the number of free variables and special solutions_.

This central idea--the _dimension_ of a subspace--is made precise in the next section. We count the free variables for the nullspace. We count the pivot variables for the column space!

### Solving \(Ax=b\), \(Ux=c\), and \(Rx=d\)

The case \(b\neq 0\) is quite different from \(b=0\). The row operations on \(A\) must act also on the right-hand side (on \(b\)). We begin with letters \((b_{1},b_{2},b_{3})\) to find the solvability condition--for \(b\) to lie in the column space. Then we choose \(b=(1,5,5)\) and find all solutions \(x\).

For the original example \(Ax=b=(b_{1},b_{2},b_{3})\), apply to both sides the operations that led from \(A\) to \(U\). The result is an upper triangular system \(Ux=c\):

\[Ux=c\qquad\begin{bmatrix}1&3&3&2\\ 0&0&3&3\\ 0&0&0&0\end{bmatrix}\begin{bmatrix}u\\ v\\ v\\ y\end{bmatrix}=\begin{bmatrix}b_{1}\\ b_{2}-2b_{1}\\ b_{3}-2b_{2}+5b_{1}\end{bmatrix}.\] (3)

The vector \(c\) on the right-hand side, which appeared after the forward elimination steps, is just \(L^{-1}b\) as in the previous chapter. Start now with \(Ux=c\).

It is not clear that these equations have a solution. The third equation is very much in doubt, because its left-hand side is zero. _The equations are inconsistent unless \(b_{3}-2b_{2}+5b_{1}=0\)_. Even though there are more unknowns than equations, there may be no solution. We know another way of answering the same question: \(Ax=b\) can be solved if and only if \(b\) lies in the column space of \(A\). This subspace comes from the four columns of \(A\) (not of \(U\)!):

\[\begin{array}{c}\textbf{Columns of }A\\ \textbf{``span'' the}\\ \textbf{column space}\end{array}\qquad\begin{bmatrix}1\\ 2\\ -1\end{bmatrix},\quad\begin{bmatrix}3\\ 6\\ -3\end{bmatrix},\quad\begin{bmatrix}3\\ 9\\ 3\end{bmatrix},\quad\begin{bmatrix}2\\ 7\\ 4\end{bmatrix}.\]

Even though there are four vectors, their combinations only fill out a plane in three-dimensional space. Column 2 is three times column 1. The fourth column equals the third minus the first. _These dependent columns, the second and fourth, are exactly the ones without pivots_.

The column space \(C(A)\) can be described in two different ways. On the one hand, it is _the plane generated by columns 1 and 3_. The other columns lie in that plane, and contribute nothing new. Equivalently, it is the plane of all vectors \(b\) that satisfy \(b_{3}-2b_{2}+5b_{1}=0\); this is the constraint if the system is to be solvable. _Every column 