We must expect a violent change in the solution from ordinary changes in the data. Chapter 1 compared the equations \(Ax=b\) and \(Ax^{\prime}=b^{\prime}\):

\[\begin{array}{rclrclrclrcl}u&+&v&=&2\\ u&+&1.0001v&=&2\end{array}\qquad\qquad\begin{array}{rclrclrclrcl}v&=&2\\ u&+&1.0001v&=&2.0001.\end{array}\]

The right-hand sides are changed only by \(\|\delta b\|=.0001=10^{-4}\). At the same time, the solution goes from \(u=2\), \(v=0\) to \(u=v=1\). This is a relative error of

\[\frac{\|\delta x\|}{\|x\|}=\frac{\|(-1,1)\|}{\|(2,0)\|}=\frac{\sqrt{2}}{2}, \quad\mbox{ which equals }\quad 2\cdot 10^{4}\frac{\|\delta b\|}{\|b\|}.\]

Without having made any special choice of the perturbation, there was a relatively large change in the solution. Our \(x\) and 