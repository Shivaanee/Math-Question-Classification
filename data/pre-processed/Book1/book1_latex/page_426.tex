* [18] In Arnoldi, show that \(q_{2}\) is orthogonal to \(q_{1}\). The Arnoldi method is Gram-Schmidt orthogonalization applied to the Krylov matrix: \(K_{N}=Q_{N}R_{N}\). The eigenvalues of \(Q_{N}^{\mathrm{T}}AQ_{N}\) are often very close to those of \(A\), even for \(N\ll n\). The _Lanczos iteration_ is Arnoldi for symmetric matrices (all coded in ARPACK).
* [19] In conjugate gradients, show that \(r_{1}\) is orthogonal to \(r_{0}\) (orthogonal residuals), and \(p^{\mathrm{T}}Ap_{0}=0\) (search directions are \(A\)-orthogonal). The iteration solves \(Ax=b\) by minimizing the error \(e^{\mathrm{T}}Ae\) in the Krylov subspace. It is a fantastic algorithm.

 