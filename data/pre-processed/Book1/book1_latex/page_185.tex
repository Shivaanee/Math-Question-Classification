notice that ours came directly from the calculation of \(\|b-p\|^{2}\). This stays nonnegative when we introduce new possibilities for the lengths and inner products. The name of Cauchy is also attached to this inequality \(|a^{\mathrm{T}}b|\leq\|a\|\|b\|\), and the Russians refer to it as the Cauchy-Schwarz-Buniakowsky inequality! Mathematical historians seem to agree that Buniakowsky's claim is genuine.

One final observation about \(|a^{\mathrm{T}}b|\leq\|a\|\|b\|\). _Equality holds if and only if \(b\) is a multiple of \(a\)_. The angle is \(\theta=0^{\circ}\) or \(\theta=180^{\circ}\) and the cosine is \(1\) or \(-1\). In this case \(b\) is identical with its projection \(p\), and the distance between \(b\) and the line is zero.

**Example 1**.: Project \(b=(1,2,3)\) onto the line through \(a=(1,1,1)\) to get \(\widehat{x}\) and \(p\):

\[\widehat{x}=\frac{a^{\mathrm{T}}b}{a^{\mathrm{T}}a}=\frac{6}{3}=2.\]

The projection is \(p=\widehat{x}a=(2,2,2)\). The angle between \(a\) and \(b\) has

\[\cos 