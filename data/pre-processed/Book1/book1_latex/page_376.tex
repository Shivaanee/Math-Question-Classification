1. To prove that the \(p+q\) vectors \(x_{1},\ldots,x_{p}\), \(Cy_{1},\ldots,Cy_{q}\) are independent, assume that some combination gives zero: \[a_{1}x_{1}+\cdots+a_{p}x_{p}=b_{1}Cy_{1}+\cdots+b_{q}Cy_{q}\quad(=z,\;\text{say}).\] Show that \(z^{\text{T}}Az=\lambda_{1}a_{1}^{2}+\cdots+\lambda_{p}a_{p}^{2}\geq 0\) and \(z^{\text{T}}Az=\mu_{1}b_{1}^{2}+\cdots+\mu_{q}b_{q}^{2}\leq 0\). 2. Deduce that the \(a\)'s and \(b\)'s are zero (proving linear independence). From that deduce \(p+q\leq n\). 3. The same argument for the \(n-p\) negative \(\lambda\)'s and the \(n-q\) positive \(\mu\)'s gives \(n-p+n-q\leq n\). (We again assume no zero eigenvalues--which are handled separately). Show that \(p+q=n\), so the number \(p\) of positive \(\lambda\)'s equals the number \(n-q\) of positive \(\mu\)'s--which is the law of inertia.
**37.**: If \(C\) is nonsingular, show that \(A\) and \(C^{\text{T}}AC\) have the same rank. Thus they have the same number of zero eigenvalues.
**38.**: Find by experiment the number of positive, negative, and zero eigenvalues of

\[A=\begin{bmatrix}I&B\\ B^{\text{T}}&0\end{bmatrix}\]

when the block \(B\) (of order \(\frac{1}{2}n\)) is nonsingular.
**39.**: Do \(A\) and \(C^{\text{T}}AC\) always satisfy the law of inertia when \(C\) is not square?
**40.**: In equation (9) with \(m_{1}=1\) and \(m_{2}=2\), verify that the normal modes are \(M\)-orthogonal: \(x_{1}^{\text{T}}Mx_{2}=0\).
**41.**: Find the eigenvalues and eigenvectors of \(Ax=\lambda Mx\):

\[\begin{bmatrix}6&-3\\ -3&6\end{bmatrix}x=\frac{\lambda}{18}\begin{bmatrix}4&1\\ 1&4\end{bmatrix}x.\]
**42.**: If the symmetric matrices \(A\) and \(M\) are indefinite, \(Ax=\lambda Mx\) might not have real eigenvalues. Construct a 2 by 2 example.
**43.**: A _group_ of nonsingular matrices includes \(AB\) and \(A^{-1}\) if it includes \(A\) and \(B\). "Products and inverses stay in the group." Which of these sets are groups? _Positive definite symmetric matrices \(A\), orthogonal matrices \(Q\), all exponentials \(e^{tA}\) of a fixed matrix \(A\), matrices \(P\) with positive eigenvalues, matrices \(D\) with determinant \(1\)_. Invent a group containing only positive definite matrices.

 