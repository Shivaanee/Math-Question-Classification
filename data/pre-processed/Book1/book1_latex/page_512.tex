9. \(\operatorname{trace}(AB)=\operatorname{trace}(BA)=aq+bs+cr+dt\). Then \(\operatorname{trace}(AB-BA)=0\) (always). So \(AB-BA=I\) is impossible for matrices, since \(I\) does not have trace zero.
11. (a) True; \(\det A=2\neq 0\). (b) False; \(\begin{bmatrix}1&1&1\\ 0&1&1\\ 0&0&2\end{bmatrix}\). (c) False; \(\begin{bmatrix}1&0&0\\ 0&1&0\\ 0&0&2\end{bmatrix}\) is diagonal!
13. \(A=\begin{bmatrix}1&1\\ 1&-1\end{bmatrix}\begin{bmatrix}9&0\\ 0&1\end{bmatrix}\begin{bmatrix}1&1\\ 1&-1\end{bmatrix}^{-1}\); \(\begin{bmatrix}2&1\\ 1&2\end{bmatrix}\); four square roots.
15. \(\begin{bmatrix}1&2\\ 0&3\end{bmatrix}=\begin{bmatrix}1&1\\ 0&1\end{bmatrix}\begin{bmatrix}1&0\\ 0&3\end{bmatrix}\begin{bmatrix}1&-1\\ 0&1\end{bmatrix}\); \(\begin{bmatrix}1&1\\ 2&2\end{bmatrix}=\begin{bmatrix}1&1\\ -1&2\end{bmatrix}\begin{bmatrix}0&0\\ 0&3\end{bmatrix}\begin{bmatrix}\frac{2}{3}&-\frac{1}{3}\\ \frac{1}{3}&\frac{1}{3}\end{bmatrix}\).
17. \(A=\begin{bmatrix}1&1\\ 0&1\end{bmatrix}\begin{bmatrix}2&0\\ 0&5\end{bmatrix}\begin{bmatrix}1&-1\\ 0&1\end{bmatrix}=\begin{bmatrix}2&3\\ 0&5\end{bmatrix}\).
19. (a) False: don't know \(\lambda\)'s. (b) True. (c) True. (d) False: need eigenvectors of \(S\)!
21. The columns of \(S\) are multiples of (2, 1) and (0, 1) in either order. Same for \(A^{-1}\).
23. \(A\) and \(B\) have \(\lambda_{1}=1\) and \(\lambda_{2}=1\). \(A+B\) has \(\lambda_{1}=1\), \(\lambda_{2}=3\). Eigenvalues of \(A+B\)_are not equal_ to eigenvalues of \(A\) plus eigenvalues of \(B\).
25. (a) True. (b) False. (c) False (\(A\) might have 2 or 3 independent eigenvectors).
27. \(A=\begin{bmatrix}8&3\\ -3&2\end{bmatrix}\) (or other), \(A=\begin{bmatrix}9&4\\ -4&1\end{bmatrix}\), \(A=\begin{bmatrix}10&5\\ -5&0\end{bmatrix}\); only eigenvectors are (\(c\), \(-c\)).
29. \(SA^{k}S^{-1}\) approaches zero if and only if every \(|\lambda|<1\); \(B^{k}\to 0\) from \(\lambda=.9\) and \(\lambda=.3\). .
31. \(\Lambda=\begin{bmatrix}.9&0\\ 0&.3\end{bmatrix}\), \(S=\begin{bmatrix}3&-3\\ 1&1\end{bmatrix}\); \(B^{10}\begin{bmatrix}3\\ 1\end{bmatrix}=(.9)^{10}\begin{bmatrix}3\\ 1\end{bmatrix}\), \(B^{10}\begin{bmatrix}3\\ -1\end{bmatrix}=(.3)^{10}\begin{bmatrix}3\\ -1\end{bmatrix}\), \(B^{10}\begin{bmatrix}6\\ 0\end{bmatrix}=\) sum of those two.
33. \(B^{k}=\begin{bmatrix}1&1\\ 0&-1\end{bmatrix}\begin{bmatrix}3&0\\ 0&2\end{bmatrix}^{k}\begin{bmatrix}1&1\\ 0&-1\end{bmatrix}=\begin{bmatrix}3^{k}&3^{k}-2^{k}\\ 0&2^{k}\end{bmatrix}\).
35. trace \(AB=(aq+bs)+(cr+dt)=(qa+rc)+(sb+td)=\operatorname{trace}BA\). Proof for diagonalizable case: the trace of \(S\Lambda S^{-1}\) is the trace of \((\Lambda S^{-1})S=\Lambda\), which is _the sum of the \(\lambda\)'s_.
37. The \(A\)'s form a subspace, since \(cA\) and \(A_{1}+A_{2}\) have the same \(S\). When \(S=I\), the \(A\)'s give the subspace of diagonal matrices. Dimension 4.
39. Two problems: The nullspace and column space can overlap, so \(x\) could be in both. There may not be \(r\) independent eigenvectors in the column space.
41. \(A=\begin{bmatrix}1&1\\ 1&0\end{bmatrix}\) has \(A^{2}=\begin{bmatrix}2&1\\ 1&1\end{bmatrix}\), and \(A^{2}-A-I=\operatorname{zero}\) matrix confirms Cayley-Hamilton.

 