final factors \(L\) and \(U\)--is an essential foundation for the theory. I hope you will enjoy this book and this course.

### The Geometry of Linear Equations

The way to understand this subject is by example. We begin with two extremely humble equations, recognizing that you could solve them without a course in linear algebra. Nevertheless I hope you will give Gauss a chance:

\[2x - y = 1\] \[x + y = 5.\]

We can look at that system _by rows_ or _by columns_. We want to see them both.

The first approach concentrates on the separate equations (the _rows_). That is the most familiar, and in two dimensions we can do it quickly. The equation \(2x-y=1\) is represented by a _straight line_ in the \(x\)-\(y\) plane. The line goes through the points \(x=1\), \(y=1\) and \(x=\frac{1}{2}\), \(y=0\) (and also through \((2,3)\) and all intermediate points). The second equation \(x+y=5\) produces a second line (Figure 1.2a). Its slope is \(dy/dx=-1\) and it crosses the first line at the solution.

The point of intersection lies on both lines. It is the only solution to both equations. That point \(x=2\) and \(y=3\) will soon be found by "elimination."

The second approach looks at the _columns_ of the linear system. The two separate equations are really _one vector equation_:

\[\textbf{Column form}\qquad x\begin{bmatrix}2\\ 1\end{bmatrix}+y\begin{bmatrix}-1\\ 1\end{bmatrix}=\begin{bmatrix}1\\ 5\end{bmatrix}.\]

Figure 1.2: Row picture (two lines) and column picture (combine columns).

 