(b) \(\left[\begin{array}{cccc}0&0&0&1\\ 0&0&1&0\\ 0&-1&0&0\\ -1&0&0&0\end{array}\right]\) has \(\det=1\).
15. Adding every column of \(A\) to the first column makes it a zero column, so \(\det A=0\). If every row of \(A\) adds to 1, then every row of \(A-I\) adds to 0 and \(\det(A-I)=0\). But \(\det A\) need not be 1: \(A=\left[\begin{array}{rr}\frac{1}{2}&\frac{1}{2}\\ \frac{1}{2}&\frac{1}{2}\end{array}\right]\) has \(\det(A-I)=0\), but \(\det A=0\neq 1\).
17. \(\det(A)=10\), \(\det(A^{-1})=\frac{1}{10}\), \(\det(A-\lambda I)=\lambda^{2}-7\lambda+10=0\) for \(\lambda=5\) and \(\lambda=2\).
19. Taking determinants gives \((\det C)(\det D)=(-1)^{a}(\det D)(\det C)\). For \(n\) even the reasoning fails (because \((-1)^{n}=+1\)) and the conclusion is wrong.
21. \(\det(A^{-1})=\det\left[\begin{array}{cc}\frac{d}{ad-bc}&\frac{-b}{ad-bc}\\ \frac{-c}{ad-bc}&\frac{a}{ad-bc}\end{array}\right]=\frac{ad-bc}{(ad-bc)^{2}}= \frac{1}{ad-bc}\).
23. Determinant \(=36\) and determinant \(=5\).
25. \(\det(L)=1\), \(\det(U)=-6\), \(\det(A)=-6\), \(\det(U^{-1}L^{-1})=-\frac{1}{6}\), and \(\det(U^{-1}L^{-1}A)=1\).
27. Row 3 \(-\) row 2 \(=\) row 2 \(-\) row 1 so \(A\) is singular.
29. \(A\) is rectangular so \(\det(A^{\mathrm{T}}A)\neq(\det A^{\mathrm{T}})(\det A)\): these are not defined.
31. The Hilbert determinants are \(1\), \(8\times 10^{-2}\), \(4.6\times 10^{-4}\), \(1.6\times 10^{-7}\), \(3.7\times 10^{-12}\), \(5.4\times 10^{-18}\), \(4.8\times 10^{-25}\), \(2.7\times 10^{-33}\), \(9.7\times 10^{-43}\), \(2.2\times 10^{-53}\). Pivots are ratios of determinants, so the tenth pivot is near \(10^{-53}/10^{-43}=10^{-10}\): very small.
33. The largest determinants of 0-1 matrices for \(n=1\), \(2\), \(\dots\), are \(1\), \(1\), \(2\), \(3\), \(5\), \(9\), \(32\), \(56\), \(144\), \(320\), on the web at _www.mathworld.wolfram.com/HadamardsMaximum DeterminantProblem.html_ and also in the "On-Line Encyclopedia of Integer Sequences": _www.research.att.com_. With \(-1\)s and \(1\)s, the largest 4 by 4 determinant (see Hadamard in index) is 16.
35. \(\det(I+M)=1+a+b+c+d\). Subtract row 4 from rows 1, 2, and 3. Then subtract \(a(\mathrm{row}\ 1)+b(\mathrm{row}\ 2)+c(\mathrm{row}\ 3)\) from row 4. This leaves a triangular matrix with 1, 1, 1, and \(1+a+b+c+d\) on its diagonal.

### Problem Set 4.3, page 215

1. [label=0.,ref=0]
2. \(a_{12}a_{21}a_{34}a_{43}=1\); _even_, so \(\det A=1\). 3. \(b_{13}b_{22}b_{31}b_{14}=18\); _odd_, so \(\det B=-18\).
3. \(\mathrm{True}\) (product rule). 2. \(\mathrm{False}\) (all \(1\)s). 4. \(\mathrm{False}\) (\(\det[1\ \ \ 1\ \ 0;\ 0\ \ 1\ \ 1;\ 1\ \ 0\ \ 1]=2\)).
4. The 1, 1 cofactor is \(F_{n-1}\). The 1, 2 cofactor has a 1 in column 1, with cofactor \(F_{n-2}\). Multiply by \((-1)^{1+2}\) and also \(-1\) to find \(F_{n}=F_{n-1}+F_{n-2}\). So the determinants are Fibonacci numbers, except \(F_{n}\) is the usual \(F_{n-1}\).

 