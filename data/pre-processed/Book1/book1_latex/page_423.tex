entries of the true \(L\) and \(U\) are set to zero while factoring \(A\). It is called _incomplete LU_ and it can be terrific.

We cannot close without mentioning the _conjugate gradient method_, which looked dead hut is suddenly very much alive (Problem 33 gives the steps). It is direct rather than iterative, but unlike elimination, it can be stopped part way. And needless to say, a completely new idea may still appear and win. But it seems fair to say that it was the change from .99 to .75 that revolutionized the solution of \(Ax=b\).

#### Problem Set 7.4

**1.**: This matrix has eigenvalues \(2-\sqrt{2}\), \(2\), and \(2+\sqrt{2}\):

\[A=\begin{bmatrix}2&-1&0\\ -1&2&-1\\ 0&-1&2\end{bmatrix}.\]

Find the Jacobi matrix \(D^{-1}(-L-U)\) and the Gauss-Seidel matrix \((D+L)^{-1}(-U)\) and their eigenvalues, and the numbers \(\omega_{\text{opt}}\) and \(\lambda_{\text{max}}\) for SOR.
**2.**: For this \(n\) by \(n\) matrix, describe the Jacobi matrix \(J=D^{-1}(-L-U)\):

\[A=\begin{bmatrix}2&-1&&\\ -1&\cdot&\cdot&\\ &\cdot&\cdot&-1\\ &&-1&2\end{bmatrix}.\]

Show that the vector \(x_{1}=(\sin\pi h,\sin 2\pi h,\ldots,\sin n\pi h)\) is an eigenvector of \(J\) with eigenvalue \(\lambda_{1}=\cos\pi h=\cos\pi/(n+1)\).
**3.**: In Problem 2, show that \(x_{k}=(\sin k\pi h,\sin 2k\pi h,\ldots,\sin nk\pi h)\) is an eigenvector of \(A\). Multiply \(x_{k}\) by \(A\) to find the corresponding eigenvalue \(\alpha_{k}\). Verify that in the 3 by 3 case these eigenvalues are \(2-\sqrt{2}\), \(2\), \(2+\sqrt{2}\).

_Note._ The eigenvalues of the Jacobi matrix \(J=\frac{1}{2}(-L-U)=I-\frac{1}{2}A\) are \(\lambda_{k}=1-\frac{1}{2}\alpha_{k}=\cos k\pi h\). They occur in plus-minus pairs and \(\lambda_{\text{max}}\) is \(\cos\pi h\).

 