With \(x_{5}\) entering the basis, \(x_{1}\) or \(x_{2}\) must leave. In the first equation, increase \(x_{5}\) and decrease \(x_{1}\) while keeping \(x_{1}+2x_{5}=8\). Then \(x_{1}\) will be down to zero when \(x_{5}\) reaches \(4\). The second equation keeps \(x_{2}+3x_{5}=9\). Here \(x_{5}\) can only increase as far as \(3\). To go further would make \(x_{2}\) negative, so _the leaving variable is \(x_{2}\)_. The new corner has \(x=(2,0,0,0,3)\). _The cost is down to \(-9\)._

Quick WayIn \(Ax=b\), the right sides divided by the coefficients of the entering variable are \(\frac{8}{2}\) and \(\frac{9}{3}\). The smallest ratio \(\frac{9}{3}\) tells which variable hits zero first, and must leave. We consider only positive ratios, because if the coefficient of \(x_{5}\) were \(-3\), then increasing \(x_{5}\) would actually _increase \(x_{2}\)_. (At \(x_{5}=10\) the second equation would give \(x_{2}=39\).) _The ratio \(\frac{9}{3}\) says that the second variable leaves_. It also gives \(x_{5}=3\).

If all coefficients of \(x_{5}\) had been negative, this would be an _unbounded_ case: we can make \(x_{5}\) arbitrarily large, and bring the cost down toward \(-\infty\).

The current step ends at the new corner \(x=(2,0,0,0,3)\). The next step will only be easy if the basic variables \(x_{1}\) and \(x_{5}\) stand by themselves (as \(x_{1}\) and \(x_{2}\) originally did). Therefore, we "pivot" by substituting \(x_{5}=\frac{1}{3}(9-x_{2}-x_{3})\) into the cost function and the first equation. The new problem, starting from the new corner, is:

\[\begin{array}{c}\mbox{Minimize the cost}\qquad 7x_{3}-x_{4}-(9-x_{2}-x_{3})=x_{2}+8x_{3}-x_{4 }-9\\ \mbox{with constraints}\qquad\qquad\qquad\qquad x_{1}-\frac{2}{3}x_{2}+\frac{1}{3}x_{ 3}+6x_{4}\qquad=2\\ \qquad\qquad\qquad\qquad\frac{1}{3}x_{2}+\frac{1}{3}x_{3}\qquad\qquad+x_{5}=3. \end{array}\]

The next step is now easy. The only negative coefficient \(-1\) in the cost makes \(x_{4}\) the entering variable. The ratios of \(\frac{2}{6}\) and \(\frac{3}{6}\), the right sides divided by the \(x_{4}\) column, make \(x_{1}\) the leaving variable. The new corner is \(x^{*}=(0,0,0,\frac{1}{3},3)\). The new cost \(-9\frac{1}{3}\) is the minimum.

In a large problem, a departing variable might reenter the basis later on. But the cost keeps going down--except in a degenerate case--so the \(m\) basic variables can't be the same as before. No corner is ever revisited! The simplex method must end at the optimal corner (or at \(-\infty\) if the cost turns out to be unbounded). What is remarkable is the speed at which \(x^{*}\) is found.

SummaryThe cost coefficients \(7\), \(-1\), \(-3\) at the first corner and \(1\), \(8\), \(-1\) at the second corner decided the entering variables. (These numbers go into \(r\), the crucial vector defined below. When they are all positive we stop.) The ratios decided the leaving variables.

Remark on DegeneracyA corner is _degenerate_ if more than the usual \(n\) components of \(x\) are zero. More than \(n\) planes pass through the corner, so a basic variable happens to vanish. The ratios that determine the leaving variable will include zeros, and the basis might change without actually moving from the corner. In theory, we could stay at a corner and cycle forever in the choice of basis.

 