idea is simple, the applications can be complicated. For problems on this scale, the one undebatable point is their cost--I am afraid a billion dollars would be a conservative estimate of the expense so far. I hope some readers will be vigorous enough to master the finite element method and put it to good use.

### Trial Functions

Starting from the classical _Rayleigh-Ritz principle_, I will introduce the new idea of finite elements. The equation can be \(-u^{\prime\prime}=f(x)\) with boundary conditions \(u(0)=u(1)=0\). This problem is _infinite-dimensional_ (the vector \(b\) is replaced by a function \(f\), and the matrix \(A\) becomes \(-d^{2}/dx^{2}\)). We can write down the energy whose minimum is required, replacing inner products \(v^{\rm T}f\) by integrals of \(v(x)f(x)\):

\[\mbox{\bf Total energy}\qquad P(v)=\frac{1}{2}v^{\rm T}Av-v^{\rm T}f=\frac{1}{2} \int_{0}^{1}v(x)(-v^{\prime\prime}(x))dx-\int_{0}^{1}v(x)f(x)dx.\] (1)

\(P(v)\) is to be minimized over all functions \(v(x)\) that satisfy \(v(0)=v(1)=0\). _The function that gives the minimum will be the solution_\(u(x)\). The differential equation has been converted to a minimum principle, and it only remains to integrate by parts:

\[\int_{0}^{1}v(-v^{\prime\prime})dx=\int_{0}^{1}(v^{\prime})^{2}dx-[vv^{\prime }]_{x=0}^{x=1}\quad\mbox{so}\quad P(v)=\int_{0}^{1}\left[\frac{1}{2}(v^{ \prime}(x))^{2}+v(x)f(x)\right]dx.\]

The term \(vv^{\prime}\) is zero at both limits, because \(v\) is. Now \(\int(v^{\prime}(x))^{2}dx\) is _positive_ like \(x^{\rm T}Ax\). We are guaranteed a minimum.

To compute the minimum exactly is equivalent to solving the differential equation exactly. _The Rayleigh-Ritz principle produces an \(n\)-dimensional problem by choosing only \(n\) trial functions \(V_{1}(x),\ldots,V_{n}(x)\)_. From all combinations \(V=y_{1}V_{1}(x)+\cdots+y_{n}V_{n}(x)\), we look for the particular combination (call it \(U\)) that minimizes \(P(V)\). This is the key idea, to minimize over a subspace of \(V\)'s instead of over all possible \(v(x)\). The function that gives the minimum is \(U(x)\). We hope and expect that \(U(x)\) is near the correct \(u(x)\).

Substituting \(V\) for \(v\), the quadratic turns into

\[P(V)=\frac{1}{2}\int_{0}^{1}\big{(}y_{1}V_{1}^{\prime}(x)+\cdots+y_{n}V_{n}^{ \prime}(x)\big{)}^{2}dx-\int_{0}^{1}\big{(}y_{1}V_{1}(x)+\cdots+y_{n}V_{n}(x) \big{)}f(x)dx.\] (2)

The trial functions \(V\) are chosen in advance. That is the key step! The unknowns \(y_{1},\ldots,y_{n}\) go into a vector \(y\). Then \(P(V)=\frac{1}{2}y^{\rm T}Ay-y^{\rm T}b\) is recognized as one of the quadratics we are accustomed to. The matrix entries \(A_{ij}\) are \(\int V_{i}^{\prime}V_{j}^{\prime}dx=\mbox{coefficient of }y_{i}y_{j}\). The components \(b_{j}\) are \(\int V_{j}fdx\). We can certainly find the minimum of \(\frac{1}{2}y^{\rm T}Ay-y^{\rm T}b\) by solving \(Ay=b\). Therefore the Rayleigh-Ritz method has three steps:

1. _Choose the trial functions \(V_{1},\ldots,V_{n}\)._
2. _Compute the coefficients \(A_{ij}\) and \(b_{j}\)._ 