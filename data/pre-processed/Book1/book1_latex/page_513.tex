43. By **5F**, \(B\) has the same eigenvectors (1, 0) and (0, 1) as \(A\), so \(B\) is also diagonal. The equations \(AB-BA=\begin{bmatrix}a&b\\ 2c&2d\end{bmatrix}-\begin{bmatrix}a&2b\\ c&2d\end{bmatrix}=\begin{bmatrix}0&0\\ 0&0\end{bmatrix}\) are \(-b=0\) and \(c=0\): rank 2.
45. \(A\) has \(\lambda_{1}=1\) and \(\lambda_{2}=.4\) with \(x_{1}=(1,2)\) and \(x_{2}=(1,-1)\). \(A^{\infty}\) has \(\lambda_{1}=1\) and \(\lambda_{2}=0\) (same eigenvectors). \(A^{100}\) has \(\lambda_{1}=1\) and \(\lambda_{2}=(.4)^{100}\), which is near zero. So \(A^{100}\) is very near \(A^{\infty}\).

**Problem Set 5.3**, page 262

1. The Fibonacci numbers start even, odd, odd. Then \(odd+odd=even\). The next two are odd (from odd \(+\) even and even \(+\) odd). Then repeat \(odd+odd=even\).
3. \(A^{2}=\begin{bmatrix}2&1\\ 1&1\end{bmatrix}\), \(A^{3}=\begin{bmatrix}3&2\\ 2&1\end{bmatrix}\), \(A^{4}=\begin{bmatrix}5&3\\ 3&2\end{bmatrix}\); \(F_{20}=6765\).
5. \(A=S\Lambda S^{-1}=\begin{bmatrix}1&1\\ 1&0\end{bmatrix}=\dfrac{1}{\lambda_{1}-\lambda_{2}}\begin{bmatrix}\lambda_{1}& \lambda_{2}\\ 1&1\end{bmatrix}\begin{bmatrix}\lambda_{1}&0\\ 0&\lambda_{2}\end{bmatrix}\begin{bmatrix}1&-\lambda_{2}\\ -1&\lambda_{1}\end{bmatrix}\) (notice \(S^{-1}\)). \(S\Lambda^{k}S^{-1}=\dfrac{1}{\lambda_{1}-\lambda_{2}}\begin{bmatrix}\lambda_ {1}&\lambda_{2}\\ 1&1\end{bmatrix}\begin{bmatrix}\lambda_{1}^{k}&0\\ 0&\lambda_{2}^{k}\end{bmatrix}\begin{bmatrix}1&-\lambda_{2}\\ -1&\lambda_{1}\end{bmatrix}\begin{bmatrix}1\\ 0\end{bmatrix}=\begin{bmatrix}------\\ (\lambda_{1}^{k}-\lambda_{2}^{k})/(\lambda_{1}-\lambda_{2})\end{bmatrix}\)
7. Direct addition \(L_{k}+L_{k+1}\) gives \(L_{0}\), \(\ldots\), \(L_{10}\) as 2, 1, 3, 4, 7, 11, 18, 29, 47, 76, 123. My calculator gives \(\lambda_{1}^{10}=(1.618\ldots)^{10}=122.991\ldots\), which rounds off to \(L_{10}=123\).
9. The Markov transition matrix is \(\begin{bmatrix}\frac{7}{12}&\frac{1}{6}&0\\ \frac{1}{6}&\frac{1}{2}&0\\ \frac{1}{4}&\frac{1}{3}&1\end{bmatrix}\). Fractions \(\frac{7}{12}\), \(\frac{1}{2}\), 1 don't move.
11. (a) \(\lambda=0\), \((1,1,-2)\). (b) \(\lambda=1\) and \(-0.2\). (c) limit \((3,4,4)=\) eigenvector for \(\lambda=1\).
13. (a) \(0\leq a\leq 1\) \(0\leq b\leq 1\). (b) \(u_{k}=\begin{bmatrix}b/(1-a)&1\\ 1&-1\end{bmatrix}\begin{bmatrix}1^{k}&0\\ 0&(a-b)^{k}\end{bmatrix}\begin{bmatrix}b/(1-a)&1\\ 1&-1\end{bmatrix}^{-1}\begin{bmatrix}1\\ 1\end{bmatrix}\) \[=\begin{bmatrix}\dfrac{2b}{b-a+1}-\dfrac{1-a-b}{b-a+1}(a-b)^{k}\\ \dfrac{2(1-a)}{b-a+1}-\dfrac{1-a-b}{b-a+1}(a-b)^{k}\end{bmatrix}.\] (c) \(u_{k}\to\begin{bmatrix}\dfrac{2b}{b-a+1}\\ \dfrac{2(1-a)}{b-a+1}\end{bmatrix}\) if \(|a-b|<1\); \(a=1/3\) \(b=-1/3\) not Markov.
15. The components of \(Ax\) add to \(x_{1}+x_{2}+x_{3}\) (each column adds to 1 and nobody is lost). The components of \(\lambda x\) add to \(\lambda(x_{1}+x_{2}+x_{3})\). If \(\lambda\neq 1\), \(x_{1}+x_{2}+x_{3}\) must be zero.
17. \(\begin{bmatrix}\alpha&\alpha\\ \alpha&\alpha\end{bmatrix}\) is unstable for \(|\alpha|>1/2\), and stable for \(|\alpha|<1/2\). Neutral for \(\alpha=\pm 1/2\).

 