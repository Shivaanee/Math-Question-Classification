The problem is _to find the combination of the column vectors on the left side that produces the vector on the right side_. Those vectors \((2,1)\) and \((-1,1)\) are represented by the bold lines in Figure 1.2b. The unknowns are the numbers \(x\) and \(y\) that multiply the column vectors. The whole idea can be seen in that figure, where 2 times column 1 is added to 3 times column 2. Geometrically this produces a famous parallelogram. Algebraically it produces the correct vector \((1,5)\), on the right side of our equations. The column picture confirms that \(x=2\) and \(y=3\).

More time could be spent on that example, but I would rather move forward to \(n=3\). Three equations are still manageable, and they have much more variety:

\[\begin{array}{ccccccc}&2u&+&v&+&w&=&5\\ &4u&-&6v&&=&-2\\ &-2u&+&7v&+&2w&=&9.\end{array}\] (1)

Again we can study the rows or the columns, and we start with the rows. Each equation describes a _plane_ in three dimensions. The first plane is \(2u+v+w=5\), and it is sketched in Figure 1.3. It contains the points \((\frac{5}{2},0,0)\) and \((0,5,0)\) and \((0,0,5)\). It is determined by any three of its points--provided they do not lie on a line.

_Changing 5 to 10, the plane \(2u+v+w=10\) would be parallel to this one._ It contains \((5,0,0)\) and \((0,10,0)\) and \((0,0,10)\), twice as far from the origin--which is the center point \(u=0\), \(v=0\), \(w=0\). Changing the right side moves the plane parallel to itself, and the plane \(2u+v+w=0\)

Figure 1.3: The row picture: three intersecting planes from three linear equations.

 