The \((2,1)\) entry in this product is \(a_{11}\sin\theta+a_{21}\cos\theta\), and we choose the angle \(\theta\) that makes this combination zero. The next rotation \(P_{32}\) is chosen in a similar way, to remove the \((3,2)\) entry of \(P_{32}P_{21}A_{k}\). After \(n-1\) rotations, we have \(R_{0}\):

\[\textbf{Triangular factor}\qquad R_{k}=P_{n\,n-1}\cdots P_{32}P_{21}A_{k}.\] (8)

Books on numerical linear algebra give more information about this remarkable algorithm in scientific computing. We mention one more method--_Arnoldi_ in ARPACK--for large sparse matrices. It orthogonalizes the Krylov sequence \(x,Ax,A^{2}x,\ldots\) by Gram-Schmidt. If you need the eigenvalues of a large matrix, don't use \(\det(A-\lambda I)!\)

### Problem Set 7.3

1. For the matrix \(A=\left[\begin{smallmatrix}2&-1\\ -1&2\end{smallmatrix}\right]\) with eigenvalues \(\lambda_{1}=1\) and \(\lambda_{2}=3\), apply the power method \(u_{k+1}=Au_{k}\) three times to the initial guess \(u_{0}=\left[\begin{smallmatrix}1\\ 0\end{smallmatrix}\right]\). What is the limiting vector \(u_{\infty}\)?
2. For the same \(A\) and the initial guess \(u_{0}=\left[\begin{smallmatrix}3\\ 4\end{smallmatrix}\right]\), compare three inverse power steps to one shifted step with \(\alpha=u_{0}^{\mathrm{T}}Au_{0}/u_{0}^{\mathrm{T}}u_{0}\): \[u_{k+1}=A^{-1}u_{k}=\frac{1}{3}\begin{bmatrix}2&1\\ 1&2\end{bmatrix}u_{k}\qquad\text{or}\qquad u=(A-\alpha I)^{-1}u_{0}.\] The limiting vector \(u_{\infty}\) is now a multiple of the other eigenvector \((1,1)\).
3. Explain why \(|\lambda_{n}/\lambda_{n-1}|\) controls the convergence of the usual power method. Construct a matrix \(A\) for which this method _does not converge_.
4. The Markov matrix \(A=\left[\begin{smallmatrix}.9&.3\\ .1&.7\end{smallmatrix}\right]\) has \(\lambda=1\) and \(.6\), and the power method \(u_{k}=A^{k}u_{0}\) converges to \(\left[\begin{smallmatrix}.75\\ .25\end{smallmatrix}\right]\). Find the eigenvectors of \(A^{-1}\). What does the inverse power method \(u_{-k}=A^{-k}u_{0}\) converge to (after you multiply by \(.6^{k}\))?
5. Show that for any two different vectors of the same length, \(\|x\|=\|y\|\), the Householder transformation with \(v=x-y\) gives \(Hx=y\) and \(Hy=x\).
6. Compute \(\sigma=\|x\|\), \(v=x+\sigma z\), and \(H=I-2vv^{\mathrm{T}}/v^{\mathrm{T}}v\), Verify \(Hx=-\sigma z\): \[x=\begin{bmatrix}3\\ 4\end{bmatrix}\qquad\text{and}\qquad z=\begin{bmatrix}1\\ 0\end{bmatrix}.\]
7. Using Problem 6, find the tridiagonal \(HAH^{-1}\) that is similar to \[A=\begin{bmatrix}1&3&4\\ 3&1&0\\ 4&0&0\end{bmatrix}\] 