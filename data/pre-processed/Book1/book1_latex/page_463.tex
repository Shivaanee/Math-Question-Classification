

### Matrix Games

The most general "\(m\) by \(n\) matrix game" is exactly like our example. \(X\) has \(n\) possible moves (columns of \(A\)). \(Y\) chooses from the \(m\) rows. The entry \(a_{ij}\) is the payment when \(X\) chooses column \(j\) and \(Y\) chooses row \(i\). A negative entry means a payment to \(Y\). This is a _zero-sum game_. Whatever one player loses, the other wins.

\(X\) is free to choose any mixed strategy \(x=(x_{1},\ldots,x_{n})\). These \(x_{1}\) give the frequencies for the \(n\) columns and they add to \(1\). At every turn \(X\) uses a random device to produce strategy \(i\) with frequency \(x_{i}\). \(Y\) chooses a vector \(y=(y_{1},\ldots,y_{m})\), also with \(y_{i}\geq 0\) and \(\sum y_{i}=1\), which gives the frequencies for selecting rows.

A single play of the game is random. On the average, the combination of column \(j\) for \(X\) and row \(i\) for \(Y\) will turn up with probability \(x_{i}y_{i}\). When it does come up, the payoff is \(a_{ij}\). The expected payoff to \(X\) from this combination is \(a_{ij}x_{j}y_{i}\), and _the total expected payoff from each play of the same game is \(\sum\sum a_{ij}xjy_{i}=yAx\)_:

\[yAx=\begin{bmatrix}y_{1}&\cdots&y_{m}\end{bmatrix}\begin{bmatrix}a_{11}&a_{12 }&\cdots&a_{1n}\\ \vdots&\vdots&&\vdots\\ a_{m1}&a_{m2}&\cdots&a_{mn}\end{bmatrix}\begin{bmatrix}x_{1}\\ x_{2}\\ \vdots\\ x_{n}\end{bmatrix}=a_{11}x_{1}y_{1}+\cdots+a_{mn}x_{n}y_{m}\]

It is this payoff \(yAx\) that \(X\) wants to maximize and \(Y\) wants to minimize.

**Example 1**.: Suppose \(A\) is the \(n\) by \(n\) identity matrix, \(A=I\). The expected payoff becomes \(yIx=x_{1}y_{1}+\cdots+x_{n}y_{n}\). \(X\) is hoping to hit on the same choice as \(Y\), to win \(a_{ii}=\$1\). \(Y\) is trying to evade \(X\), to pay \(a_{ij}=\$0\). If \(X\) chooses any column more often than another, \(Y\) can escape more often. _The optimal mixture is \(x^{*}=(1/n,1/n,\ldots,1/n)\)_. Similarly \(Y\) cannot overemphasize any row--the optimal mixture is \(y^{*}=(1/n,1/n,\ldots,1/n)\). The probability that both will choose strategy \(i\) is \((1/n)^{2}\), and the sum over \(i\) is the expected payoff to \(X\). The total value of the game is \(n\) times \((1/n)^{2}\), or \(1/n\):

\[y^{*}Ax^{*}=\begin{bmatrix}1/n&\cdots&1/n\end{bmatrix}\begin{bmatrix}1&&\\ &\ddots&\\ &&1\end{bmatrix}\begin{bmatrix}1/n\\ \vdots\\ 1/n\end{bmatrix}=\begin{pmatrix}1\\ n\end{pmatrix}^{2}+\cdots+\begin{pmatrix}1\\ n\end{pmatrix}^{2}=\frac{1}{n}.\]

As \(n\) increases, \(Y\) has a better chance to escape. The value \(1/n\) goes down.

The symmetric matrix \(A=I\) did not make the game fair. _A skew-symmetric matrix_, \(A^{\mathrm{T}}=-A\), _means a completely fair game_. Then a choice of strategy \(j\) by \(X\) and \(i\) by \(Y\) wins \(a_{ij}\) for \(X\), and a choice of \(j\) by \(Y\) and \(i\) by \(X\) wins the same amount for \(Y\) (because \(a_{ji}=-a_{ij}\)). The optimal strategies \(x^{*}\) and \(y^{*}\) must be the same, and the expected payoff must be \(y^{*}Ax^{*}=0\). The value of the game, when \(A^{\mathrm{T}}=-A\), is zero. But the strategy is still to be found.

