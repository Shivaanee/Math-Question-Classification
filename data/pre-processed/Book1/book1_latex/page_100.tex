There is a "double infinity" of solutions, with \(v\) and \(y\) free and independent. The complete solution is a combination of two **special solutions**:

\[\begin{array}{ll}\textbf{Nullspace contains}&x=\begin{bmatrix}-3v+y\\ v\\ -y\\ y\end{bmatrix}=v\begin{bmatrix}-\textbf{3}\\ \textbf{1}\\ \textbf{0}\\ \textbf{0}\end{bmatrix}+y\begin{bmatrix}\textbf{1}\\ \textbf{0}\\ -\textbf{1}\\ \textbf{1}\end{bmatrix}.\end{array}\] (2)

Please look again at this complete solution to \(Rx=0\) and \(Ax=0\). The special solution \((-3,1,0,0)\) has free variables \(v=1\), \(y=0\). The other special solution \((1,0,-1,1)\) has \(v=0\) and \(y=1\). _All solutions are linear combinations of these two_. The best way to find all solutions to \(Ax=0\) is from the special solutions:

1. After reaching \(Rx=0\), identify the pivot variables and free variables.
2. Give one free variable the value 1, set the other free variables to 0, and solve \(Rx=0\) for the pivot variables. This \(x\) is a special solution.
3. Every free variable produces its own "special solution" by step 2. The combinations of special solutions form the nullspace--all solutions to \(Ax=0\).

Within the four-dimensional space of all possible vectors \(x\), the solutions to \(Ax=0\) form a _two-dimensional subspace_--the nullspace of \(A\), In the example, \(\boldsymbol{N}(A)\) is generated by the special vectors \((-3,1,0,0)\) and \((1,0,-1,1)\). The combinations of these two vectors produce the whole nullspace.

Here is a little trick. The special solutions are especially easy from \(R\). The numbers 3 and 0 and \(-1\) and 1 lie in the "nonpivot columns" of \(R\). **Reverse their signs to find the pivot variables** (not free) **in the special solutions**. I will put the two special solutions from equation (2) into a nullspace matrix \(N\), so you see this neat pattern:

\[\begin{array}{ll}\textbf{Nullspace matrix}&\\ \textbf{(columns are}&\\ \textbf{special solutions)}&N=\begin{bmatrix}-\textbf{3}&\textbf{1}\\ 1&0\\ \textbf{0}&-\textbf{1}\\ 0&1\end{bmatrix}&\begin{array}{l}\text{not free}\\ \text{not free}\\ \text{not free}\end{array}\end{array}\]

The free variables have values 1 and 0. When the free columns moved to the right-hand side of equation (2), their coefficients 3 and 0 and \(-1\) and 1 switched sign. That determined the pivot variables in the special solutions (the columns of \(N\)).

This is the place to recognize one extremely important theorem. Suppose a matrix has more columns than rows, \(n>m\). Since \(m\) rows can hold at most \(m\) pivots, _there must be at least \(n-m\) free variables_. There will be even more free variables if some rows of \(R\) reduce to zero; but no matter what, at least one variable must be free. This free variable can be assigned any value, leading to the following conclusion:

**2C** If \(Ax=0\) has more unknowns than equations (\(n>m\)), it has at least one special solution: There are more solutions than the trivial \(x=0\).

 