

**37.**: A 3 by 3 determinant has three products "down to the right" and three "down to the left" with minus signs. Compute the six terms in the figure to find \(D\). Then explain without determinants why this matrix is or is not invertible:
**38.**: For \(A_{4}\) in Problem 6, five of the \(4!=24\) terms in the big formula (6) are nonzero. Find those five terms to show that \(D_{4}=-1\).
**39.**: For the 4 by 4 tridiagonal matrix (entries \(-1\), \(2\), \(-1\)), find the five terms in the big formula that give \(\det A=16-4-4-4+1\).
**40.**: Find the determinant of this cyclic \(P\) by cofactors of row 1. How many exchanges reorder 4, 1, 2, 3 into 1, 2, 3, 4? Is \(|P^{2}|=+1\) or \(-1\)?

\[P=\begin{bmatrix}0&0&0&1\\ 1&0&0&0\\ 0&1&0&0\\ 0&0&1&0\end{bmatrix}\qquad P^{2}=\begin{bmatrix}0&0&1&0\\ 0&0&0&1\\ 1&0&0&0\\ 0&1&0&0\end{bmatrix}=\begin{bmatrix}0&I\\ I&0\end{bmatrix}.\]
**41.**: A=\(2*\)eye(n)\(-\)diag(ones(n\(-1\), \(1\)),\(1\))\(-\)diag(ones(n\(-1\), \(1\)),\(-1\)) is the \(-1\), \(2\), \(-1\) matrix. Change \(A(1,1)\) to \(1\) so \(\det A=1\). Predict the entries of \(A^{-1}\) based on \(n=3\) and test the prediction for \(n=4\).
**42.**: (MATLAB) The \(-1\), \(2\), \(-1\) matrices have determinant \(n+1\). Compute \((n+1)A^{-1}\) for \(n=3\) and \(4\), and verify your guess for \(n=5\). (Inverses of tridiagonal matrices have the rank-1 form \(uv^{\rm T}\) above the diagonal.)
**43.**: All **Pascal matrices** have determinant 1. If I subtract 1 from the \(n\), \(n\) entry, why does the determinant become zero? (Use rule 3 or a cofactor.)

\[\det\begin{bmatrix}1&1&1&1\\ 1&2&3&4\\ 1&3&6&10\\ 1&4&10&\mathbf{20}\end{bmatrix}=1\text{ (known)}\qquad\det\begin{bmatrix}1&1&1&1\\ 1&2&3&4\\ 1&3&6&10\\ 1&4&10&\mathbf{19}\end{bmatrix}=\mathbf{0}\text{ (explain)}.\]

### Applications of Determinants

This section follows through on four major applications: _inverse of \(A\), solving \(Ax=b\), volumes of boxes_, and _pivots_. They are among the key computations in linear algebra