9. The cost to be minimized is \(1000x+2000y+3000z+1500u+3000v+3700w\). The amounts \(x\), \(y\), \(z\) to Chicago and \(u\), \(v\), \(w\) to New England satisfy \(x+u=1\),000,000; \(y+v=1\),000,000; \(z+w=1\),000,000; \(x+y+z=800\),000; \(u+v+w=2\),200,000.

#### Problem Set 8.2, page 391

1. At present \(x_{4}=4\) and \(x_{5}=2\) are in the basis, and the cost is zero. The entering variable should be \(x_{3}\), to reduce the cost. The leaving variable should be \(x_{5}\), since \(2/1\) is less than \(4/1\). With \(x_{3}\) and \(x_{4}\) in the basis, the constraints give \(x_{3}=2\), \(x_{4}=2\), and the cost is now \(x_{1}+x_{2}-x_{3}=-2\).
3. The "reduced costs" are \(r=[1\quad 1]\), so change is not good and the corner is optimal.
4. At \(P\), \(r=[-5\quad 3]\); then at \(Q\), \(r=\begin{bmatrix}5&-1\end{bmatrix}\); \(R\) is optimal because \(r\geq 0\).
5. For a maximum problem the stopping test becomes \(r\leq 0\). If this fails, and the \(i\)th component is the largest, then that column of \(N\) enters the basis; the rule **8C** for the vector leaving the basis is the same.
6. \(BE=B[\cdots v\cdots]=[\cdots u\cdots]\), since \(Bv=u\). So \(E\) is the correct matrix.
7. If \(Ax\succeq 0\), then \(Px=x-A^{\top}(AA^{\top})^{-1}Ax=x\).

#### Problem Set 8.3, page 399

1. Maximize \(4y_{1}+11y_{2}\), with \(y_{1}\geq 0\), \(y_{2}\geq 0\), \(2y_{1}+y_{2}\leq 1\), \(3y_{2}\leq 1\); the primal has \(x_{1}^{\ast}=2\), \(x_{2}^{\ast}=3\), the dual has \(y_{1}^{\ast}=\frac{1}{3}\), \(y_{2}^{\ast}=\frac{1}{3}\), \(cost=5\).
2. The dual maximizes \(yb\), with \(y\geq c\) 