

**36.**: Which vectors \((b_{1},b_{2},b_{3})\) are in the column space of \(A\)? Which combinations of the rows of \(A\) give zero?

\[\text{(a)}\quad A=\begin{bmatrix}1&2&1\\ 2&6&3\\ 0&2&5\end{bmatrix}\qquad\qquad\text{(b)}\quad A=\begin{bmatrix}1&1&1\\ 1&2&4\\ 2&4&8\end{bmatrix}.\]
**37.**: Why can't a 1 by 3 system have \(x_{p}=(2,4,0)\) and \(x_{n}=\) any multiple of \((1,1,1)\)?
**38.**: 1. If \(Ax=b\) has two solutions \(x_{1}\) and \(x_{2}\), find two solutions to \(Ax=0\). 2. Then find another solution to \(Ax=b\).
**39.**: Explain why all these statements are false:

1. The complete solution is any linear combination of \(x_{p}\) and \(x_{n}\). 2. A system \(Ax=b\) has at most one particular solution. 3. The solution \(x_{p}\) with all free variables zero is the shortest solution (minimum length \(\|x\|\)). (Find a 2 by 2 counterexample.) 4. If \(A\) is invertible there is no solution \(x_{n}\) in the nullspace.
**40.**: Suppose column 5 of \(U\) has no pivot. Then \(x_{5}\) is a variable. The zero vector (is) (is not) the only solution to \(Ax=0\). If \(Ax=b\) has a solution, then it has solutions.
**41.**: If you know \(x_{p}\) (free variables \(=0\)) and all special solutions for \(Ax=b\), find \(x_{p}\) and all special solutions for these systems:

\[Ax=2b\qquad\begin{bmatrix}A&A\end{bmatrix}\begin{bmatrix}x\\ X\end{bmatrix}=b\qquad\begin{bmatrix}A\\ A\end{bmatrix}\begin{bmatrix}x\end{bmatrix}=\begin{bmatrix}b\\ b\end{bmatrix}.\]
**42.**: If \(Ax=b\) has infinitely many solutions, why is it impossible for \(Ax=B\) (new right-hand side) to have only one solution? Could \(Ax=B\) have no solution?
**43.**: Choose the number \(q\) so that (if possible) the ranks are (a) 1, (b) 2, (c) 3:

\[A=\begin{bmatrix}6&4&2\\ -3&-2&-1\\ 9&6&q\end{bmatrix}\qquad\text{and}\qquad B=\begin{bmatrix}3&1&3\\ q&2&q\end{bmatrix}.\]
**44.**: Give examples of matrices \(A\) for which the number of solutions to \(Ax=b\) is

1. 0 or 1, depending on \(b\).
2. \(\infty\), regardless of \(b\).
3. 0 or \(\infty\), depending on \(b\).
4. 1, regardless of \(b\).

