3. If \(\mathbf{S}\) and \(\mathbf{T}\) are subspaces of \(\mathbf{R}^{5}\), their intersection \(\mathbf{S}\cap\mathbf{T}\) (vectors in both subspaces) is a subspace of \(\mathbf{R}^{5}\). _Check the requirements on \(x+y\) and \(cx\)._
**19.**: Suppose \(\mathbf{P}\) is a plane through \((0,0,0)\) and \(\mathbf{L}\) is a line through \((0,0,0)\). The smallest vector space containing both \(\mathbf{P}\) and \(\mathbf{L}\) is either or .
**20.**: True or false for \(\mathbf{M}=\) all 3 by 3 matrices (check addition using an example)?

1. The skew-symmetric matrices in \(\mathbf{M}\) (with \(A^{\mathrm{T}}=-A\)) form a subspace.
2. The unsymmetric matrices in \(\mathbf{M}\) (with \(A^{\mathrm{T}}\neq A\)) form a subspace.
3. The matrices that have \((1,1,1)\) in their nullspace form a subspace.

**Problems 21-30 are about column spaces \(C(A)\) and the equation \(Ax=b\).**

**21.**: Describe the column spaces (lines or planes) of these particular matrices:

\[A=\begin{bmatrix}1&2\\ 0&0\\ 0&0\end{bmatrix}\quad\text{and}\quad B=\begin{bmatrix}1&0\\ 0&2\\ 0&0\end{bmatrix}\quad\text{and}\quad C=\begin{bmatrix}1&0\\ 2&0\\ 0&0\end{bmatrix}.\]
**22.**: For which right-hand sides (find a condition on \(b_{1}\), \(b_{2}\), \(b_{3}\)) are these systems solvable?

\[\text{(a)}\quad\begin{bmatrix}1&4&2\\ 2&8&4\\ -1&-4&-2\end{bmatrix}\begin{bmatrix}x_{1}\\ x_{2}\\ x_{3}\end{bmatrix}=\begin{bmatrix}b_{1}\\ b_{2}\\ b_{3}\end{bmatrix}.\qquad\text{(b)}\quad\begin{bmatrix}1&4\\ 2&9\\ -1&-4\end{bmatrix}\begin{bmatrix}x_{1}\\ x_{2}\end{bmatrix}=\begin{bmatrix}b_{1}\\ b_{2}\\ b_{3}\end{bmatrix}.\]
**23.**: Adding row 1 of \(A\) to row 2 produces \(B\). Adding column 1 to column 2 produces \(C\). A combination of the columns of \(\underline{\phantom{\text{\phantom{\text{\phantom{\text{\phantom{\text{\phantom{ \text{\phantom{\text{\text 