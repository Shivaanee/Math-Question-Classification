Then find a second pair of orthonormal eigenvectors \(x_{1}\), \(x_{2}\) for \(\lambda=0\). 2. Verify that \(P=x_{1}x_{1}^{\mathrm{T}}+x_{2}x_{2}^{\mathrm{T}}\) is the same for both pairs.
**17.**: Prove that every _unitary_ matrix \(A\) is diagonalizable, in two steps:

1. If \(A\) is unitary, and \(U\) is too, then so is \(T=U^{-1}AU\). 2. An upper triangular \(T\) that is unitary must be diagonal. Thus \(T=\Lambda\). Any unitary matrix \(A\) (distinct eigenvalues or not) has a complete set of orthonormal eigenvectors. All eigenvalues satisfy \(|\lambda|=1\).
**18.**: Find a normal matrix (\(NN^{\mathrm{H}}=N^{\mathrm{H}}N\)) that is not Hermitian, skew-Hermitian, unitary, or diagonal. Show that all permutation matrices are normal.
**19.**: Suppose \(T\) is a 3 by 3 upper triangular matrix, with entries \(t_{ij}\). Compare the entries of \(TT^{\mathrm{H}}\) and \(T^{\mathrm{H}}T\), and show that if they are equal, then \(T\) must be diagonal. All normal triangular matrices are diagonal.
**20.**: If \(N\) is normal, show that \(\|Nx\|=\|N^{\mathrm{H}}x\|\) for every vector \(x\). Deduce that the \(i\)th row of \(N\) has the same length as the \(i\)th column. _Note_: If \(N\) is also upper triangular, this leads again to the conclusion that it must be diagonal.
**21.**: Prove that a matrix with orthonormal eigenvectors must be normal, as claimed in **5T**: If \(U^{-1}NU=A\), or \(N=U\Lambda U^{\mathrm{H}}\), then \(NN^{\mathrm{H}}=N^{\mathrm{H}}N\).
**22.**: Find a unitary \(U\) and triangular \(T\) so that \(U^{-1}AU=T\), for

\[A=\begin{bmatrix}5&-3\\ 4&-2\end{bmatrix}\qquad\text{and}\qquad A=\begin{bmatrix}0&1&0\\ 0&0&0\\ 1&0&0\end{bmatrix}.\]
**23.**: If \(A\) has eigenvalues 0, 1, 2, what are the eigenvalues of \(A(A-I)(A-2I)\)?
**24.**:
1. Show by direct multiplication that every triangular matrix \(T\), say 3 by 3, satisfies its own characteristic equation: \((T-\lambda_{1}I)(T-\lambda_{2}I)(T-\lambda_{3}I)=0\).
2. Substituting \(U^{-1}AU\) for \(T\), deduce the famous _Cayley-Hamilton theorem: Every matrix satisfies its own characteristic equation_. For 3 by 3 this is \((A-\lambda_{1}I)(A-\lambda_{2}I)(A-\lambda_{3}I)=0\).
**25.**: The characteristic polynomial of \(A=\begin{bmatrix}a&b\\ c&d\end{bmatrix}\) is \(\lambda^{2}-(a+d)\lambda+(ad-bc)\). By direct substitution, verify Cayley-Hamilton: \(A^{2}-(a+d)A+(ad-bc)I=0\).
**26.**: If \(a_{ij}=1\) above the main diagonal and \(a_{ij}=0\) elsewhere, find the Jordan form (say 4 by 4) by finding all the eigenvectors.
**27.**: Show, by trying for an \(M\) and failing, that no two of the three Jordan forms in equation (8) are similar: \(J_{1}\neq M^{-1}J_{2}M\), \(J_{1}\neq M^{-1}J_{3}M\), and \(J_{2}\neq M^{-1}J_{3}M\).

 