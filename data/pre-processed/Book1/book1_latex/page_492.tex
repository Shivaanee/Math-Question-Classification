

**5.**: Broken rules: (a) 7, 8 (b) 1 (c) 1, 2, 8.
**7.**: (b), (d), (e) are subspaces. Can't multiply by \(-1\) in (a) and (c). Can't add in (f).
**9.**: The sum of two nonsingular matrices may be singular (\(A+(-A)\)). The sum of two singular matrices may be nonsingular.
**11.**: (a) One possibility: The matrices \(cA\) form a subspace not containing \(B\).
**(b)**: Yes: the subspace must contain \(A-B=I\).
**(c)**: The subspace of matrices whose main diagonal is all zero.
**13.**: If \((f+g)(x)\) is the usual \(f(g(x))\), then \((g+f)x\) is \(g(f(x))\), which is different. In rule 2 both sides are \(f(g(h(x)))\). Rule 4 is broken because there might be no inverse function \(f^{-1}(x)\) such that \(f(f^{-1}(x))=x\). If the inverse function exists, it will be the vector \(-f\).
**15.**: The sum of (4, 0, 0) and (0, 4, 0) is not on the plane; it has \(x+y-2z=8\).
**17.**: (a) The subspaces of \(\mathbf{R}^{2}\) are \(\mathbf{R}^{2}\) itself, lines through \((0,\,0)\), and the point \((0,\,0)\).
**(b)**: The subspaces of \(\mathbf{R}^{4}\) are \(\mathbf{R}^{4}\) itself, three-dimensional planes \(n\cdot v=0\), two-dimensional subspaces (\(n_{1}\cdot v=0\) and \(n_{2}\cdot v=0\)), one-dimensional lines through \((0,\,0,\,0,\,0)\), and \((0,\,0,\,0,\,0)\) alone.
**19.**: The smallest subspace containing \(\mathbf{P}\) and \(\mathbf{L}\) is either \(\mathbf{P}\) or \(\mathbf{R}^{3}\).
**21.**: The column space of \(A\) is the \(x\)-axis = all vectors (\(x,\,0,\,0\)). The column space of \(B\) is the \(x\)-\(y\) plane = all vectors (\(x,\,y,\,0\)). The column space of \(C\) is the line of vectors (\(x,\,2x,\,0\)).
**23.**: A combination of the columns of \(C\) is also a combination of the columns of \(A\) (same column space; \(B\) has a different column space).
**25.**: The extra column \(b\) enlarges the column space, unless \(b\) is _already in_ that space:

\[[A\;\;b]=\begin{bmatrix}1&0&1\\ 0&0&1\end{bmatrix}\quad\text{(larger column space)}\]

\[\begin{bmatrix}1&0&1\\ 0&1&1\end{bmatrix}\quad\text{($b$ already in column space)}\]

\[\begin{bmatrix}1&0&1\\ 0&1&1\end{bmatrix}\quad\text{($Ax=b$ has a solution)}.\]
**27.**: Column space \(=\mathbf{R}^{8}\). Every \(b\) is a combination of the columns, since \(Ax=b\) is solvable.
**29.**: \(A=\begin{bmatrix}1&1&0\\ 1&0&0\\ 0&1&0\end{bmatrix}\) or \(\begin{bmatrix}1&1&2\\ 1&0&1\\ 0&1&1\end{bmatrix}\); \(A=\begin{bmatrix}1&2&0\\ 2&4&0\\ 3&6&0\end{bmatrix}\) (columns on 1 line).
**31.**: \(\mathbf{R}^{2}\) contains vectors with _two_ components--they don't belong to \(\mathbf{R}^{3}\).

### Problem Set 2.2, page 85

1. \(x+y+z=1,x+y+z=0\). Changing 1 to 0, (\(x,\,y,\,z\)) = \(c(-1,\,1,\,0)+d(-1,\,0,\,1)\).
**3.**: Echelon form \(U=\begin{bmatrix}0&1&0&3\\ 0&0&0&0\end{bmatrix}\); free variables \(x_{1},x_{3},x_{4}\); special solutions \((1,\,0,\,0,\,0)\), \((0,\,0,\,1,\,0)\), and \((0,\,-3,\,0,\,1)\). Consistent when \(b_{2}=2b_{1}\). Complete solution \((0,\,b_{1},\,0,\,0)\) plus any combination of special solutions.

