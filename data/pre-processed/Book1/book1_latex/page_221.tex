1. Using those multipliers in \(A\), show that column 1 of \(A\) and \(B=\text{column 2}-\frac{1}{2}\)(column 1) and \(C=\text{column 3}-\frac{2}{3}\)(column 2) are orthogonal. 2. Check that \(\|\text{column 1}\|^{2}=2\), \(\|B\|^{2}=\frac{3}{2}\), and \(\|C\|^{2}=\frac{4}{3}\), using the pivots.
**31.**: True or false (give an example in either case): 1. \(Q^{-1}\) is an orthogonal matrix when \(Q\) is an orthogonal matrix. 2. If \(Q\) (3 by 2) has orthonormal columns then \(\|Qx\|\) always equals \(\|x\|\).
**32.**: 1. Find a basis for the subspace \(\mathbf{S}\) in \(\mathbf{R}^{4}\) spanned by all solutions of \[x_{1}+x_{2}+x_{3}-x_{4}=0.\] 2. Find a basis for the orthogonal complement \(\mathbf{S}^{\perp}\). 3. Find \(b_{1}\) in \(\mathbf{S}\) and \(b_{2}\) in \(\mathbf{S}^{\perp}\) so that \(b_{1}+b_{2}=b=(1,1,1,1)\).

 