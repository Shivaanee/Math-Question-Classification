gives a family of parallel lines. The minimum cost comes when the first line intersects the feasible set. That intersection occurs at \(B\), where \(x^{*}=0\) and \(y^{*}=2\); the minimum cost is \(2x^{*}+3y^{*}=6\). The vector \((0,2)\) is _feasible_ because it lies in the feasible set, it is _optimal_ because it minimizes the cost function, and the minimum cost 6 is the _value_ of the program. We denote optimal vectors by an asterisk.

_The optimal vector occurs at a corner of the feasible set_. This is guaranteed by the geometry, because the lines that give the cost function (or the planes, when we get to more unknowns) move steadily up until they intersect the feasible set. The first contact must occur along its boundary! The "simplex method" will go from one corner of the feasible set to the next until it finds the corner with lowest cost. In contrast, "interior point methods" approach that optimal solution from _inside_ the feasible set.

_Note_. With a different cost function, the intersection might not be just a single point. If the cost happened to be \(x+2y\), the whole edge between \(B\) and \(A\) would be optimal. The minimum cost is \(x^{*}+2y^{*}\), which equals 4 for all these optimal vectors. On our feasible set, the maximum problem would have no solution! The cost can go arbitrarily high and the maximum cost is infinite.

Every linear programming problem falls into one of three possible categories:

1. The feasible set is _empty_.
2. The cost function is _unbounded_ on the feasible set.
3. The cost reaches its _minimum_ (or maximum) on the feasible set: _the good case_.

The empty and unbounded cases should be very uncommon for a genuine problem in economics or engineering. We expect a solution.

Figure 8.2: The feasible set with flat sides, and the costs \(2x+3y\), touching at \(B\).

 