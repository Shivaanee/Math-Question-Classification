For linear programming, the important alternatives come when the constraints are inequalities. When is the feasible set empty (no \(x\))?

\(\mathbf{8J}\)\(Ax\geq b\) has a solution \(x\geq 0\)\(or\) there is a \(y\leq 0\) with \(yA\geq 0\) and \(yb<0\).

Proof.: The slack variables \(w=Ax-b\) change \(Ax\geq b\) into an equation. Use 8I:

\[\begin{array}{ccc}\textbf{First alternative}&\left[A\right.&-I\right]\left[ \begin{matrix}x\\ w\end{matrix}\right]=b&\text{for some }&\left[\begin{matrix}x\\ w\end{matrix}\right]\geq 0.\\ \\ \textbf{Second alternative}&y\left[A\right.&-I\right]\geq\left[ \begin{matrix}0&0\end{matrix}\right]&\text{for some }y\text{ with }&yb<0.\\ \end{array}\]

It is this result that leads to a "nonconstructive proof" of the duality theorem.

## Problem Set 8.3

**1.**: What is the dual of the following problem: Minimize \(x_{1}+x_{2}\), subject to \(x_{1}\geq 0\), \(x_{2}\geq 0\), \(2x_{1}\geq 4\), \(x_{1}+3x_{2}\geq 11\)? Find the solution to both this problem and its dual, and verify that minimum equals maximum.
**2.**: What is the dual of the following problem: Maximize \(y_{2}\) subject to \(y_{1}\geq 0\), \(y_{2}\geq 0\), \(y_{1}+y_{2}\leq 3\)? Solve both this problem and its dual.
**3.**: Suppose \(A\) is the identity matrix (so that \(m=n\)), and the vectors \(b\) and \(c\) are nonnegative. Explain why \(x^{*}=b\) is optimal in the minimum problem, find \(y^{*}\) in the maximum problem, and verify that the two values are the same. If the first component of \(b\) is negative, what are \(x^{*}\) and \(y^{*}\)?
**4.**: Construct a 1 by 1 example in which \(Ax\geq b\), \(x\geq 0\) is unfeasible, and the dual problem is unbounded.
**5.**: Starting with the 2 by 2 matrix \(A=\left[\begin{smallmatrix}1&0\\ 0&-1\end{smallmatrix}\right]\), choose \(b\) and \(c\) so that both of the feasible sets \(Ax\geq b\), \(x\geq 0\) and \(yA\leq c\), \(y\geq 0\) are empty.
**6.**: If all entries of \(A\), \(b\), and \(c\) are positive, show that both the primal and the dual are feasible.
**7.**: Show that \(x=(1,1,1,0)\) and \(y=(1,1,0,1)\) are feasible in the primal and dual, with

\[A=\begin{bmatrix}0&0&1&0\\ 0&1&0&0\\ 1&1&1&1\\ 1&0&0&1\end{bmatrix},\qquad b=\begin{bmatrix}1\\ 1\\ 1\\ 1\end{bmatrix},\qquad c=\begin{bmatrix}1\\ 1\\ 1\\ 3\end{bmatrix}.\]

Then, after computing \(cx\) and \(yb\), explain how you know they are optimal.

 