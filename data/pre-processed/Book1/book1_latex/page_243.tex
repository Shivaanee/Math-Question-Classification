* the "reverse-triangular" matrix that results from row exchanges, \[M=\begin{bmatrix}0&0&0&2\\ 0&0&2&6\\ 0&1&2&2\\ 4&4&8&8\end{bmatrix}.\]
* Show how rule 6 (\(\det=0\) if a row is zero) comes directly from rules 2 and 3.
* Suppose you do two row operations _at once_, going from \[\begin{bmatrix}a&b\\ c&d\end{bmatrix}\qquad\text{to}\qquad\begin{bmatrix}a-mc&b-md\\ c-\ell a&d-\ell b\end{bmatrix}.\] Find the determinant of the new matrix, by rule 3 or by direct calculation.
* If \(Q\) is an orthogonal matrix, so that \(Q^{\mathrm{T}}Q=I\), prove that \(\det Q\) equals \(+1\) or \(-1\). What kind of box is formed from the rows (or columns) of \(Q\)?
* Prove again that \(\det Q=1\) or \(-1\) using only the Product rule. If \(|\det Q|>1\) then \(\det Q^{n}\) blows up. How do you know this can't happen to \(Q^{n}\)?
* Use row operations to verify that the 3 by 3 "Vandermonde determinant" is \[\det\begin{bmatrix}1&a&a^{2}\\ 1&b&b^{2}\\ 1&c&c^{2}\end{bmatrix}=(b-a)(c-a)(c-b).\]
* A skew-symmetric matrix satisfies \(K^{\mathrm{T}}=-K\), as in \[K=\begin{bmatrix}0&a&b\\ -a&0&c\\ -b&-c&0\end{bmatrix}.\] In the 3 by 3 case, why is \(\det(-K)=(-1)^{3}\det K\)? On the other hand \(\det K^{\mathrm{T}}=\det K\) (always). Deduce that the determinant must be zero.
* Write down a 4 by 4 skew-symmetric matrix with \(\det K\)_not_ zero.
* True or false, with reason if true and counterexample if false:
* If \(A\) and \(B\) are identical except that \(b_{11}=2a_{11}\), then \(\det B=2\det A\).
* The determinant is the product of the pivots.
* If \(A\) is invertible and \(B\) is singular, then \(A+B\) is invertible.
* If \(A\) is invertible and \(B\) is singular, then \(AB\) is singular.
* The determinant of \(AB-BA\) is zero.
 