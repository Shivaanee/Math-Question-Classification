

**21.**: Removing zero rows of \(U\) leaves \(A=\underline{LU}\), where the \(r\) columns or \(\underline{L}\) span the column space of \(A\) and the \(r\) rows of \(\underline{U}\) span the row space. Then \(A^{+}\) has the explicit formula \(\underline{U}^{\mathrm{T}}(\underline{U}\ \underline{U}^{\mathrm{T}})^{-1}( \underline{L}^{\mathrm{T}}\underline{L})^{-1}\underline{L}^{\mathrm{T}}\).

Why is \(A^{+}b\) in the row space with \(\underline{U}^{\mathrm{T}}\) at the front? Why does \(A^{\mathrm{T}}AA^{+}b=A^{\mathrm{T}}b\), so that \(x^{+}=A^{+}b\) satisfies the normal equation as it should?
**22.**: Explain why \(AA^{+}\) and \(A^{+}A\) are projection matrices (and therefore symmetric). What fundamental subspaces do they project onto?

### 6.4 Minimum Principles

In this section we escape for the first time from linear equations. The unknown \(x\) will not be given as the solution to \(Ax=b\) or \(Ax=\lambda x\). Instead, the vector \(x\) will be determined by a minimum principle.

It is astonishing how many natural laws can be expressed as minimum principles. Just the fact that heavy liquids sink to the bottom is a consequence of minimizing their potential energy. And when you sit on a chair or lie on a bed, the springs adjust themselves so that the energy is minimized. A straw in a glass of water looks bent because light reaches your eye as quickly as possible. Certainly there are more highbrow examples: The fundamental principle of structural engineering is the minimization of total energy.1

Footnote 1: I am convinced that plants and people also develop in accordance with minimum principles. Perhaps civilization is based on a law of least action. There must be new laws (and minimum principles) to be found in the social sciences and life sciences.

We have to say immediately that these "energies" are nothing but _positive definite quadratic functions_. And the derivative of a quadratic is linear. We get back to the familiar linear equations, when we set the first derivatives to zero. Our first goal in this section is _to find the minimum principle that is equivalent to \(Ax=b\), and the minimization equivalent to \(Ax=\lambda x\)_. We will be doing in finite dimensions exactly what the theory of optimization does in a continuous problem, where "first derivatives \(=0\)" gives a differential equation. In every problem, we are free to solve the linear equation or minimize the quadratic.

The first step is straightforward: We want to find the "parabola" \(P(x)\) whose minimum occurs when \(Ax=b\). If \(A\) is just a scalar, that is easy to do:

\[\text{The graph of}\quad P(x)=\frac{1}{2}Ax^{2}-bx\quad\text{has zero slope when}\quad\frac{dP}{dx}=Ax-b=0.\]

This point \(x=A^{-1}b\) will be a minimum if \(A\) is positive. Then the parabola \(P(x)\) opens upward (Figure 6.4). In more dimensions this parabola turns into a parabolic bowl (a paraboloid). To assure a minimum of \(P(x)\), not a maximum or a saddle point, \(A\) must be positive definite!