_We have a right triangle when that sum of cross-product terms \(x_{i}y_{i}\) is zero_:

\[\textbf{Orthogonal vectors}\qquad x^{\mathrm{T}}y=x_{1}y_{1}+\cdots+x_{n}y_{n}=0.\] (3)

This sum is \(x^{\mathrm{T}}y=\sum x_{i}y_{i}=y^{\mathrm{T}}x\), the row vector \(x^{\mathrm{T}}\) times the column vector \(y\):

\[\textbf{Inner product}\qquad x^{\mathrm{T}}y=\begin{bmatrix}x_{1}&\cdots&x_{n} \end{bmatrix}\begin{bmatrix}y_{1}\\ \vdots\\ y_{n}\end{bmatrix}=x_{1}y_{1}+\cdots+x_{n}y_{n}.\] (4)

This number is sometimes called the scalar product or dot product, and denoted by \((x,y)\) or \(x\cdot y\). We will use the name _inner product_ and keep the notation \(x^{\mathrm{T}}y\).

**3A** The inner product \(x^{\mathrm{T}}y\) is zero if and only if \(x\) and \(y\) are orthogonal vectors.

If \(x^{\mathrm{T}}y>0\), their angle is less than \(90^{\circ}\). If \(x^{\mathrm{T}}y<0\), their angle is greater than \(90^{\circ}\).

The length squared is the inner product of \(x\) with itself: \(x^{\mathrm{T}}x=x_{1}^{2}+\cdots+x_{n}^{2}=\|x\|^{2}\).

 